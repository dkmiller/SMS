% !TEX root = sms.tex

\section{How to count rings and fields I}\label{sec:bhargava-ii}
\thanksauthor{Manjul Bhargava}





In this lecture, we'll use the parameterization of cubic rings discussed in 
\autoref{sec:wood-i} to count cubic rings. Recall that there is a bijection 
\[
  \{\text{cubic rings}/\sim\} \iso \{\text{int.~bin.~cubic forms}\} /\generallinear_2(\dZ) 
\]
which preserves discriminant. So counting cubic rings by discriminant is 
equivalent to counting cubic forms by discriminant. The corresponding problem 
for binary quadratic forms was solved by Gauss and Lipschitz (and first done 
rigorously by Mertens and Siegel). 





\subsection{Lipschitz principle and generalizations}

Lipschitz's principle is that if $R$ is some region in the Euclidean plane with 
area $T$, then the number of lattice points in $R$ is approximately $T$. 
Better, 
\[
  \# \{\text{lattice points in }R\} = T + O(T^{1/2}) .
\]
The implied constant in $O(T^{1/2})$ will depend on $R$. Here we assume the 
region $R$ is ``homogeneously expanding'' via homothety. This principle, though 
completely elementary, was sufficient to count binary quadratic forms. 

Davenport realized that in order to attack binary cubic forms, one needs a 
version of the Lipschitz principle where the region $R$ is fixed (not 
necessarily expanding homogeneously). 

\begin{theo}[Davenport] 
Let $R$ be a bounded semi-algebraic region in $\dR^n$ defined by at most $k$ 
inequalities each of degree at most $\ell$. Then the number of lattice points 
in $R$ is $\volume(R)+O_{k,\ell}(\max\{\volume(\overline R),1\})$, where 
$\volume(\overline R)$ denotes the greatest $d$-dimensional volume of a 
projection of $R$ onto a $d$-dimensional coordinate hyperplane, 
$1\leqslant d\leqslant n-1$. 
\end{theo}
\begin{proof}
See the papers \cite{d51,d64}. 
\end{proof}

The following is Davenport's cubic version of Gauss' formula for the number of 
binary quadratic forms. 

\begin{theo}[Davenport]
Let $H(D)$ be the number of irreducible integer binary cubic forms, up to 
$\generallinear_2(\dZ)$-equivalence, having discriminant $D$. Then 
\begin{align*}
  \sum_{0<D<X} H(D) &= \frac{\pi^2}{72} X + O(X^{15/16}) \\
  \sum_{0<-D<X} H(D) &= \frac{\pi^2}{24} X + O(X^{15/16}) .
\end{align*}
\end{theo}

The proof uses Davenport's refined Lipschitz principle, along with analysis of 
the cusps using explicit inequalities. 

Attempts to mimic Davenport's methods to count quartic forms fail because the 
inequalities involved are far too complicated to be analyzed explicitly. We 
will reprove Davenport's theorem using Davenport's principle, but without 
having to write down explicit inequalities. 





\subsection{Proof of Davenport's theorem}

Let $V$ be the space of binary cubic forms, $G=\generallinear(2)$. So 
$V(\dR)$ is the real vector space of binary cubic forms, and $V(\dZ)$ is the 
lattice of integer binary cubic forms. Similarly for $G(\dR)$ and $G(\dZ)$. 
We have a representation $G\to \generallinear(V)$ defined over 
$\spectrum(\dZ)$. 

Let $E$ be a fundamental domain for the action of $G(\dZ)$ on $V(\dR)$. We can 
write 
\begin{align*}
  \sum_{0<D<X} H(D) 
    &= \# \{x\in E\cap V(\dZ)^\irreducible:0<|\discriminant(x)| < X\} \\
    &= \# \{x\in E\cap V^+(\dZ)^\irreducible:|\discriminant(x)|<X\} .
\end{align*}
We start by constructing any fundamental domains $E$. Fix $v\in V^+(\dR)$, and 
let $\cF$ be a fundamental domain for the left action of $G(\dZ)$ on $G(\dR)$. 
Then the multiset $\cF v=\{x=g v:g\in \cF\}$ is the union of six fundamental 
domains for the action of $G(\dZ)$ on $V(\dR)$. Indeed, 
\[
  \left(G(\dZ)\backslash G(\dR)\right) \cdot \left(G(\dR) \backslash V^+(\dR)\right) = G(\dZ)\backslash V^+(\dR) .
\]
A brief consideration of stabilizers yields the number of fundamental domains. 

A nice $\cF$ to take was defined by Gauss: 
\[
  \cF = \{g\in \generallinear_2(\dR):g\cdot i \in \Omega\},
\]
where $\generallinear_2(\dR)$ acts on the upper half-plane 
$\fH=\{z\in \dC:\Im z>0\}$ as usual, and 
\[
  \Omega=\left\{z\in \fH:-\frac 1 2 \leqslant \Re z\leqslant \frac 1 2\text{ and }|z|\geqslant 1\right\} .
\]
Davenport's fundamental domain for $G(\dZ)$ on $V(\dR)$ occurs when 
$v=x^2 y - x y^2$. Now we apply the Taski-Seidenberg theorem to conclude that 
$\cF v$ is semi-algebraic. This was generalized by Borel and Harish-Chandra in 
\cite{bh62} to arbitrary reductive groups over number fields. Note that 
$\cF = N' A' K \Lambda$, where 
\begin{align*}
  N' &= \left\{\begin{pmatrix} 1 \\ n & 1 \end{pmatrix} : |n|\leqslant \frac 1 2\right\} \\
  A' &= \left\{\begin{pmatrix} t \\ & t^{-1}\end{pmatrix} : t\geqslant \frac{\sqrt[4] 3}{\sqrt 2}\right\} \\
  K &= \left\{\begin{pmatrix}\cos \theta & \sin \theta \\ -\sin \theta & \cos \theta\end{pmatrix} : 0\leqslant \theta < 2\pi\right\} \\
  \Lambda &= \{\lambda\in \dR^\times:\lambda>0\} .
\end{align*}
Let $B$ be a compact set in $V^+(\dR)$ that is the closure of a non-empty open 
set, on which $\discriminant\geqslant 1$. We will allow $v$ to vary in $B$. 

Next, we need estimates on reducibility. 

\begin{lemm}
Let $R_X(v)=\cF v\cap \{|\discriminant |<X\}$. Then the number of reducible 
integral forms $a x^3 + \cdots + d y^3$ in $R_X(v)$ with $a\ne 0$ is 
$O(X^{3/4+\epsilon})$. 
\end{lemm}
\begin{proof}
In $R_X(v)$, we have $a=O(X^{1/4})$, $b=O(X^{1/4})$, $a b c=O(X^{3/4})$, 
$a b d=O(X^{3/4})$, \ldots. If $d=0$, at most $O(X^{3/4+\epsilon})$ such 
forms in $R_X(v)$. If $d\ne 0$, then $r x+s y\mid f(x,y)=a x^3+\cdots + d y^3$, 
which implies $r\mid a,s\mid d$. Fixing $a,b,d$ (for which there are 
$O(X^{3/4+\epsilon})$ choices) there are $O(X^\epsilon)$ choices for $r$ and 
$s$. The fact that $f(-s,r)=0$ determines $c$. The result follows. 
\end{proof}

Finally we average. Recall that $|\discriminant(v)|^{-1}\, dv$ is the unique 
(up to scalar) $G(\dR)$-invariant measure on $V(\dR)$. This lets us compute  
\begin{align*}
  N^+(X) 
    &= \sum_{0<D<X} H(D) \\
    &= \frac{\displaystyle\int_B \#\{x\in \cF v\cap V(\dZ)^\irreducible:|\discriminant(x)|<X\}\cdot |\discriminant(v)|^{-1}\, dv}{6\displaystyle\int_B|\discriminant(v)|^{-1}\, dv} \\
    &= \frac{1}{M} \int_\cF\#\{x\in g B\cap V(\dZ)^\irreducible:|\discriminant(x)|<X\}\, dg \\
    &= \frac 1 6 \volume(R_X(v)) + O(X^{5/6}) ,
\end{align*}
the last equality coming from a uniform application of Davenport's inequality. 
Note that we have proved a stronger version of Davenport's result, namely one 
with an error term of $O(X^{5/6})$ instead of $O(X^{15/16})$. In \cite{bst13}, 
a more careful version of this proof shows that there is a second term of the 
form $c X^{5/6}+O(X^{3/4+\epsilon})$. So the error term is in many ways as good 
as possible. 




