% !TEX root = sms.tex

\section{Topological and algebra-geometric methods over function fields II}\label{sec:ellenberg-ii}
\thanksauthor{Jordan Ellenberg}





In \autoref{sec:ellenberg-i}, we finished by looking at the equidistribution of 
the M\"obius function. In this lecture, we'll show how the underlying geometry 
gives some hints as to why this is the case. 





\subsection{M\"obius function and big monodromy}

Recall: we found that ``average of $\mu(f)=0$'' comes down to 
$\# Y_n(\dF_q) = q^n+o(q^n)$, where $Y_n$ parameterizes pairs $(f,y)$ with $y$ 
a square root of $\Delta(f)$. The scheme $Y_n$ is a branched double cover of 
$\dA^n=\symmetric^n\dA^1$, the space of monic integral polynomials of degree 
$n$.  For this to work, we need $Y_n$ to be irreducible. 

On the level of function fields (that is, generic points) the extension 
corresponding to $Y_n \to \dA^n$ is the quadratic extension 
\[
  K = k(a_1,\dots,a_n) \hookrightarrow K(\sqrt\Delta) .
\]
This quadratic extension is given by a map $G_K=\galois(\bar K/K)\to \dZ/2$. 
The extension is a field precisely when $G_K\to \dZ/2$ is nontrivial. That is, 
$G_K\to \dZ/2$ surjectivs if and only if $K(\sqrt\Delta)$ is a fied, if and 
only if $\Delta$ is \emph{not} a square in $K=k(a_1,\dots,a_n)$. The image of 
$G_K\to \dZ/2$ is the monodromy group, so this is a ``large monodromy'' result. 

\begin{enonce}{Conjecture}[Cohen-Lenstra]
Let $\ell$ be an odd prime and $E_{r,\ell,N}$ be the expected value of 
\[
  \surjection\left(\class\left(\dQ(\sqrt{-d})\right), (\dZ/\ell)^{\oplus r}\right) 
\]
for $d$ random in $[N,2 N]$. Then $\lim_{N\to\infty} E_{r,\ell,N} = 1$. 
\end{enonce}

For example, when $r=1$, we have $\surjection(A,\dZ/\ell) = \#A[\ell]-1$. We 
will try to give a function-field analogue of the Cohen-Lenstra heuristic. 





\subsection{Cohen-Lenstra over function fields}

The analogue of $\dQ(\sqrt{-d})$ for $d\in [N,2 N]$ is 
$\dF_q(t,\sqrt f)$ for $\deg f=n$. The analogue of $-d<0$ is to require the 
extension $\dF_q(t)(\sqrt f)$ to be ramified at infinity. This happens exactly 
when $n$ is odd. So we're looking at ``hyperelliptic curves of odd degree.'' 
The analogue of an ideal on a curve is a divisor, and the analogue of the class 
group is the Jacobian, namely 
\[
  \class\left(\dF_q(t)(\sqrt f)\right) \simeq \jacobian(C_f)(\dF_q) .
\]
where $C_f$ is the curve $y^2=f(t)$. For simplicity, write $J_f$ instead of 
$\jacobian(C_f)$. Note that $J_f$ is a $g$-dimensional abelian variety, and 
$J_f[\ell](\overline{\dF_q}) \simeq (\dZ/\ell)^{2 g}$. Here $g=(n-1)/2$. 

Define a new space $\configuration_1^n(\ell)$, which parameterizes pairs 
$(f,P)$, where $f$ is squarefree of degree $n$ and 
$P\in J_f[\ell]\smallsetminus 0$. So 
$\pi:\configuration_1^n(\ell) \to \configuration^n$ is an \'etale cover of degree 
$\ell^{2g}-1$. Define $E_{q,\ell,r,n}$ to be the expected value of 
$\#\surjection\left(J_f(\dF_q),(\dZ/\ell)^{\oplus r}\right)$. Cohen-Lenstra 
would suggest that $\lim_{n\to \infty} E_{q,\ell,r,n} = 1$. 

When $r=1$, this is 
\begin{align*}
  \expected_f\left(\# J_f(\dF_q)[\ell]-1\right) 
    &= \expected_f\left(\pi^{-1}(f)(\dF_q)\right) \\
    &= \frac{\#\configuration_1^n(\ell)(\dF_q))}{\#\configuration^n(\dF_q)} \\
    &= \frac{\#\configuration_1^n(\ell)(\dF_q)}{q^n-q^{n-1}} .
\end{align*}
So Cohen-Lenstra suggests that $\#\configuration_1^n(\ell)(\dF_q)\sim q^n$ as 
$n\to \infty$. Note that to get 
\[
  \lim_{q\to \infty} \frac{\#\configuration_1^n(\ell)(\dF_q)}{\#\configuration^n(\dF_q)} = 1,
\]
we would need $\configuration_1^n(\ell)$ to be irreducible. This is known to 
be true, and is proven via a monodromy computation. 

Once again, let $K=\dF_q(a_1,\dots,a_n)$, the function field of 
$\configuration^n$ over $\dF_q$. Let $L$ be the function field of 
$\configuration_1^n(\ell)$ over $\dF_q$. Then $L$ is $K$ adjoined an 
$\ell$-torsion point of $J_f$, where $f=t^n+a_1 t^{n-1} + \cdots + a_n$. 

A curve $C_f$ over $K$ gives rise to a Galois representation 
\[
  \rho:G_K=\galois(\bar K/K) \to \generalsymplectic_{2 g}(\dZ/\ell) = \generalsymplectic(J_f[\ell](K^s),e),
\]
where $e$ denotes the Weil pairing. Our question is: is $L$ a field? 
Equivalently, does the monodromy group $\rho(G_K)$ act transitively on 
$(\dZ/\ell)^{2 g}\smallsetminus 0$? It's a good exercise to show that 
$\generalsymplectic_n(\dF_q)$ acts transitively on $\dF_q^n\smallsetminus 0$ 
for any finite field $\dF_q$. 

\begin{theo}[J-K Yu, 1995]
For $K$ as above, $\rho(G_K) = \generalsymplectic_{2 g}(\dF_\ell)$. 
\end{theo}

This is a nice ``big monodromy theorem.'' That is, the monodromy group is as 
large as it can be. 





\subsection{What happens when \texorpdfstring{$r=2$}{r=2}}

We could define $\configuration_2^n(\ell)$, which parameterizes triples 
$(f,P,Q)$, where $f$ is above and $P,Q$ are linearly independent points on 
$J_f[\ell]$. Then 
\[
  \frac{\#\configuration_2^n(\ell)(\dF_q)}{\#\configuration^n(\dF_q)} = E_{q,2,\ell,n} ,
\]
because $\#\surjection(J_f(\dF_q),(\dZ/\ell)^2)$ is (by duality for finite 
abelian groups) the number of injections $(\dZ/\ell)^2 \to J_f(\dF_q)$. 
Unfortunately, the space $\configuration_2^n(\ell)$ is not irrereducible. The 
components of $\configuration_2^n(\ell)$ are naturally identified with the 
orbits of $\galois(\bar K/K\overline{\dF_q})$ on the set of injections 
$(\dZ/\ell)^2 \hookrightarrow (\dZ/\ell)^{2g}$. Even when $g=1$, this action is 
not transitive. Let $V_0=\dF_\ell^2$ and 
$\iota:V_0\hookrightarrow\dF_\ell^{2g}$. The Weil pairing $\omega$ pulls back 
to a pairing on $V_0$. This pairing is preserved by the monodromy group. The 
quantity $\omega(P,Q)$ is an invariant. 

So in fact, $\configuration_2^n(\ell)$ has $\ell$ components, parameterized by 
$\langle P,Q\rangle$ via the map 
$(f,P,Q)\mapsto \langle P,Q\rangle\in \mu_\ell$. How does 
$G_{\dF_q} = \galois(\overline{\dF_q}/\dF_q)$ act on the components? It (the 
Frobenius at $q$) just multiplies $\langle P,Q\rangle$ by $q$. So the 
number of $\dF_q$-rational components of $\configuration_2^n(\ell)$ is the 
number of elements $x\in \dZ/\ell$ such that $q x=x$. This depends strongly on 
$q$. It is 1 if $q\not\equiv 1\pmod\ell$, but $\ell$ when $q\equiv 1\pmod\ell$.
That is, when $\mu_\ell\cap \dF_q=1$ there is one component, and 
$\ell$ components if $\mu_\ell\subset \dF_q$. 

The following is work of Derek Garton. You can compute the ``modified 
Cohen-Lenstra distribution,'' whose moments match those given by 
\[
  E_{q,r,\ell,n} = \#\{\text{$\dF_q$-rational components of $\configuration_r^n(\ell)$}\} ,
\]
a geometrically motivated repair of Cohen-Lenstra in the presence of extra 
roots of unity. 





\subsection{Selmer groups}

The inspiration here is the beautiful paper \cite{j02}. What does the 3-Selmer 
group of a random $E/\dF_q(t)$ look like? If $E:y^2=f_t(x)$, let $\cE$ be the 
elliptic surface $y^2=f(t,x)$. It turns out that 
$\selmer_3(E/\dF_q(t))$ ``is'' $\h^2(\cE_{\overline{\dF_q}},\dZ/3)(\dF_q)$. You 
can play the same game, expressing the average number of nontrivial elements of 
$\selmer_p(E)$ as $\frac{\#Y_n(\dF_q)}{\#X_n(\dF_q)}$, where $Y_n\to X_n$ is a 
cover corresponding to the action of $G_K$ on 
$\h^2(\cE_{\overline{\dF_q}},\dZ/\ell)$. This cohomology group has a canonical 
symmetric pairing, so ``big monodromy'' would mean $\rho(G_K)$ is the entire 
orthogonal group. Given a big monodromy theorem, the number of components would 
be the number of orbits of $\orthogonal_d(\dZ/\ell)$ acting on 
$(\dZ/\ell)^d$. It is $\ell+1$. 

It seems like a proof that ``in the large $q$ limit,'' 
$\average(\#\selmer_\ell)=\ell+1$. 





\subsection{From \texorpdfstring{$q\to \infty$}{q to infty} to the fixed \texorpdfstring{$q$}{q} regime}

The reason we know that $\# X_n(\dF_q)=q^n+o(q^n)$ when $X_n$ irreducible is 
the Weil bounds. What's really going on is the Grothendieck-Lefschetz trace 
formula 
\[
  \# X(\dF_q) = \sum_{i=0}^{2\dim X} (-1)^i \trace\left(\frobenius_q,\h_c^i(X_{\overline\dF_q},\dQ_\ell)\right) .
\]
The Weil bounds tell you that as $q\to \infty$, the $i=0$ term contributes 
$100\%$ of this sum. But $\h^0$ is just the vector space generated by connected 
components. 

If we want to let $q$ stay fixed, we have to show that the $\h^i$ contribute 
nothing to start with. The easiest way for this to be true is for the higher 
$\h^i$ to vanish. This fits into the general idea that ``families of moduli 
spaces have vanishing higher cohomology,'' i.e.~cohomological stability 
results. I have recent joint work with Vanketesh and Westerland on this. 




