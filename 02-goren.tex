% !TEX root = sms.tex

\section{Algebraic groups, representation theory, and invariant theory}
\thanksauthor{Eyal Goren}





This lecture will consist mostly of a review of the basic terminology, as 
well as a little bit of Galois cohomology. An ``official version'' of the 
notes can be found online at 
\url{http://www.math.mcgill.ca/goren/AlgberaicGroups.SMS2014.pdf}. 





\subsection{Algebraic groups}

For us, a \emph{linear algebraic group} is a Zariski-closed subgroup of 
$\generallinear_N(\bar k)$ for some integer $N\geqslant 1$, where $k$ is a 
fixed field of characteristic zero, and $\bar k$ is the algebraic closure of 
$k$. Good general references for linear algebraic groups are the books 
\cite{b91,gw09,h75,s09}. 

\begin{enonce}[remark]{Example}
The main example of a linear algebraic group is $G=\generallinear_N$. It 
contains several standard subgroups:
\begin{align*}
  B &= \begin{pmatrix} \ast & \cdots & \ast \\ & \ddots & \vdots \\ & & \ast \end{pmatrix} && \text{``standard Borel''} \\
  U &= \begin{pmatrix} 1 & \cdots & \ast \\ & \ddots & \vdots \\ & & 1\end{pmatrix} && \text{``unipotent radical of $B$''} \\
  T &= \begin{pmatrix} \ast \\ & \ddots \\ & & \ast \end{pmatrix} && \text{``maximal torus''}
\end{align*}
\end{enonce}

\begin{enonce}[remark]{Example}
Let $q$ be a symmetric bilinear form corresponding to the matrix 
$(q_{i j})_{1\leqslant i,j\leqslant N}$. Let 
\[
  \specialorthogonal(q) = \{g\in \generallinear_N: g q \transpose g = q\text{ and }\det g=1\} .
\]
If $q$ is the form 
$q(x_1,\dots,x_{2n+1}) = \frac 1 2 (x_1 x_{2n+1} + x_2 x_{2n} + \cdots + x_{n+1}^2$, 
then a maximal torus consists ``anti-triangular'' matrices with 
$(t_1,\dots,t_n,1,t_n^{-1},\dots,t_1^{-1})$. 
\end{enonce}

An affine group $G$ is determined by its coordinate ring $\bar k[G]$. The 
group operations $m:G\times G\to G$, $i:G\to G$, $e:1\to G$ correspond via 
the Yoneda lemma to $m^\ast:\bar k[G]\to \bar k[G]\otimes \bar k[G]$, 
$i^\ast:\bar k[G] \to \bar k[G]$, $e^\ast:\bar k[G] \to \bar k$. These give 
$\bar k[G]$ the structure of a Hopf algebra. 

A \emph{homomorphism} $f:G\to H$ of algebraic is a morphism of varieties that 
respects the group structures. It corresponds to a ring 
$f^\ast:\bar k[H] \to \bar k[G]$ that respects the comultiplication.

A \emph{character} of $G$ is a homomorphism 
$f:G\to \generallinear_1 = \dG_\multiplicative$. This corresponds to a 
Hopf-algebra homomorphism $\phi:\bar k[t^{\pm 1}] \to \bar k[G]$. If we let $f$ 
be the image of $t$ under this map, then the fact that $\phi$ respects 
comultiplication tells us that $m^\ast(f) = f\otimes f$. We call such 
elements \emph{grouplike}. Let $\character^\ast(G)$ be the group of characters 
of $G$; we have seen that $\character^\ast(G)$ is in bijection with the set 
of grouplike elements of $\bar k[G]$. 

If $g\in G$, put $\inner g:G\to G$ for the action of $g$ by inner 
automorphisms, i.e.\ $\inner g(x) = g x g^{-1}$. So we have a homomorphism 
$\inner:G\to \automorphism G$. 





\subsection{Non-abelian cohomology}

Let $G$ be a topological group acting continuously on a discrete group $M$. 
Define 
\begin{align*}
  \h^0(G,M) &= M^G = \{m\in M:g m=m\text{ for all }g\in G\} \\
  \h^1(G,M) &= \{\zeta:G\to M\text{ such that }\zeta(a b) = \zeta(a)\cdot a \zeta(b)\} / \sim 
\end{align*}
where $\zeta\sim \xi$ if there exists $m\in M$ such that 
$\zeta(a) = m^{-1} \xi(a) \cdot a m$ for all $a\in G$. The set $\h^0(G,M)$ is 
naturally a group, but $\h^1(G,M)$ is only a pointed set. If 
$0 \to A \to B \to C \to 0$ is an exact sequence of $G$-groups, we get a long 
exact sequence 
\[
  0 \to A^G \to B^G \to C^G \xrightarrow\delta \h^1(G,A) \to \h^1(G,B) \to \h^1(G,C) \to \h^2(G,A) \to \cdots 
\]
with the last map only existing if $A$ is central in $B$. A good source for all 
of this is \cite{s79}. 

We define $\delta$ directly. Given $c\in C^G$, lift $c$ to $b\in B$, and let 
$\zeta=\delta(c)$ be $\zeta(g)=b^{-1} \cdot g b$. One can check that the class 
of $\zeta$ in $\h^1(G,A)$ is well-defined. 





\subsection{Forms}

Suppose $G$ is defined over $k$. A \emph{$k$-form} of $G$ is an algebraic group 
$H$ over $k$ together with an isomorphism $f:G_{\bar k} \iso H_{\bar k}$. Let 
$\Gamma=\galois(\bar k/k)$. For all $\sigma\in G$, we have 
$\sigma f:G_{\bar k}\iso H_{\bar k}$. Then 
$f^{-1}\circ \sigma f\in \automorphism_{\bar k}(G)$. An easy exercise in the 
definitions shows that this is a cocycle. In fact, we have 

\begin{theo}
There is a natural isomorphism of pointed sets 
\[
  \{\text{$k$-forms of $G$}\}/\sim \iso \h^1(\Gamma,\automorphism_{\bar k} G) .
\]
\end{theo}

If $H$ corresponds to $\zeta:G\to M$, then $H(k)=G(\bar k)^\Gamma$, where 
$\Gamma$ now acts by $\tau \cdot g = \zeta(\tau)(\tau(g))$. 

\begin{enonce}[remark]{Example}[compact forms]
Let $\Gamma=\galois(\dC/\dR)=\langle c\rangle$. Let 
$\theta(g) = \transpose{\bar g}^{-1}$ be the \emph{Cartan involution}. The 
cocycle $\zeta$ given by $\zeta(c)=\theta$ corresponds to the real form 
$\unitary_N$ of $\generallinear_N(\dC)$. It is defined by 
$\unitary_N(\dR) = \{g\in \generallinear_N(\dC):\theta g=g\}$. 
\end{enonce}

\begin{theo}
Any (connected) reductive algebraic group $G$ over $\dR$ has a unique compact 
form. 
\end{theo}

All of the groups $\generallinear_n$, $\speciallinear_n$, $\specialorthogonal_n$, 
$\symplectic_{2 n}$,\ldots are reductive. 

\begin{enonce}[remark]{Example}
If $G=\dG_\multiplicative$, the compact form is 
$T(\dR)=\{z\in \dC^\times:z \bar z=1\}$. We have 
\[
  T\simeq \specialorthogonal(2) = \left\{\begin{pmatrix} a & b \\ -b & a \end{pmatrix} : a^2+b^2=1\right\} ,
\]
via $\begin{pmatrix} a & b\\ -b & a \end{pmatrix} \mapsto a+b i$. 
\end{enonce}

If $M$ is a $\Gamma$-module, we often put $\h^i(k,M) = \h^i(\Gamma,M)$. 
Also, if $G$ is an algebraic group defined over $k$, we put 
$\h^i(k,G) = \h^i(\Gamma,G(\bar k))$. 

\begin{enonce}[remark]{Example}
Start with the exact sequence 
$1 \to \dG_\multiplicative \to \generallinear_N \to \projectivegenerallinear_N \to 1$. 
The long exact sequence in cohomology is 
\[
  1 \to k^\times \to \generallinear_N(k) \to \projectivegenerallinear_N(\bar k)^\Gamma \to \h^1(k,\dG_\multiplicative) \to \h^1(k,\generallinear_N) \to \h^2(k,\dG_\multiplicative) .
\]
The famous \emph{Hilbert Theorem 90} tells us that 
$\h^1(k,\generallinear_N) = 1$ for all $N\geqslant 1$, so we get 
\[
  \projectivegenerallinear_N(\bar k)^\Gamma = \projectivegenerallinear_N(k) = \generallinear_N(k)/\dG_\multiplicative(k) .
\]
Moreover, $\h^1(k,\projectivegenerallinear_N)\hookrightarrow \h^2(k,\dG_\multiplicative)$. We call 
$\brauer(k)=\h^2(k,\dG_\multiplicative)$ the \emph{Brauer group} of $k$. 
Since 
$\projectivegenerallinear_N(\bar k) = \automorphism_{\bar k\text{-}\mathsf{Alg}}(M_N(\bar k))$, 
we see that $\h^1(k,\projectivegenerallinear_N)$ classifies $k$-forms of 
$M_N(\bar k)$, i.e.~central simple algebras over $K$ of rank $N^2$. 
\end{enonce}





\subsection{Jordan decomposition}

Any $g\in \generallinear_N(\bar k)$ has a unique decomposition 
$g=g_\simple g_\unipotent$, where $g_\simple$ is simple (i.e.~diagonalizable), 
$g_\unipotent$ is unipotent, and 
$g_\simple g_\unipotent = g_\unipotent g_\simple$. One has 
$(g_\unipotent - 1)^N=0$. For example, in the two-dimensional case, a matrix 
$\begin{pmatrix} t_1 & u \\ & t_2\end{pmatrix}$ is already diagonalizable if 
$t_1\ne t_2$, or $u=0$. If neither of those occur, we write it as 
\[
  \begin{pmatrix} t & u \\ & t \end{pmatrix} = \begin{pmatrix} t \\ & t\end{pmatrix} \begin{pmatrix} 1 & u/t \\ & 1 \end{pmatrix} .
\]
The Jordan decomposition enjoys very strong rigidity properties. Namely, 
if $g\in G\subset \generallinear_N$, then also $g_\simple\in G$ and 
$g_\unipotent \in G$. If $f:G\to H$ is a homomorpism of algebraic groups, then 
we have $f(g_\simple) = f(g)_\simple$ and $f(g_\unipotent) = f(g)_\unipotent$. 





\subsection{Tori}

A \emph{torus} $T$ is a form of $\dG_\multiplicative^N$ for some $N$. Tori of 
rank $N$ over $k$ are classified by 
$\h^1(\Gamma,\automorphism_{\bar k}(\dG_\multiplicative^N)) = \hom(\Gamma,\generallinear_N(\dZ))/\text{conj}$. 

If $T$ is a orus, then its group of characters 
$\character^\ast(T_{\bar k}) \simeq \character^\ast(\dG_\multiplicative^N) = \hom(\dG_\multiplicative^N,\dG_\multiplicative) = \dZ^N$ 
has a continuous action of $\Gamma$, via 
$\sigma \chi(g) = \sigma(\chi(\sigma^{-1}(g)))$. Tori are completely classified 
by this action. 

\begin{theo}
The functor $\character^\ast$ induces an anti-equivalence of categories 
\[
  \{\text{tori over }k\} \iso \{\text{finite free $\dZ$-modules with continuous $\Gamma$-action}\} .
\]
\end{theo}

A linear action of a torus $T$, namely $f:T\to \generallinear_N$, is 
simultaneously diagonalizable (every element of $T$ is semi-simple). This 
follows from rigidity properties of the Jordan Decomposition. 

All maximal tori in a linear algebraic group $G$ are conjugate. The common 
dimension of these tori is called the \emph{rank} of $G$, and written 
$\rank_{\bar k}(G)$. 

\begin{enonce}[remark]{Example}
The standard torus of diagonal matrices $T\subset\generallinear_N$ is 
maximal, so $\rank(\generallinear_N) = N$. 
\end{enonce}





\subsection{Solvable groups}

An algebraic group $G$ is called \emph{solvable} if it is solvable ``in the 
usual sense.'' In other words, there exists a filtration 
$1=G_0\subset G_1\subset \cdots \subset G_l = G$ such that each 
$G_i$ a normal algebraic subgroup of $G_{i+1}$, and each 
$G_{i+1}/G_i$ is abelian. The standard Borel $B\subset \generallinear_N$ is 
solvable. In fact, $B$ is a maximal solvable subgroup. 

\begin{theo}[Kolchin-Lie]
If $G\subset \generallinear_N$ is solvable, then $G$ can be conjugated 
into the standard Borel $B\subset \generallinear_N$. 
\end{theo}
\begin{proof}
One uses the fact that if $G$ acts on a projective space, then it has a fixed 
point. The standard representation $G\to \generallinear_N$ gives an action of 
$G$ on $\dP^{N-1}$, and a fixed point for $G$ in $\dP^{N-1}$ gives a line fixed 
by the action of $G$. 
\end{proof}

\begin{theo}[Borel]
If a solvable group $G$ acts on a proper variety, then $G$ has a fixed point. 
\end{theo}

For a group $G$, let $\rad(G)$ be the maximal connected normal solvable 
subgroup of $G$, and let $\rad_\unipotent(G)$ be the maximal connected normal 
unipotent subgroup of $G$. We say that $G$ is \emph{semisimple} if 
$\rad(G)=1$, and \emph{reductive} if $\rad_\unipotent(G)=1$. Clearly 
semisimple groups are reductive. The groups 
$\speciallinear_n$, $\specialorthogonal_n$, $\symplectic_n$ are semisimple 
and $\generallinear_n$, $\generalsymplectic_{2 n}$, $\generalspin_n$ are 
reductive. 

For any $G$, the quotient $G/\rad_\unipotent(G)$ is reductive, and 
$G/\rad(G)$ is semisimple. If $G$ is reductive, then $G/Z(G)$ is 
semisimple. A \emph{Levi subgroup} of a group $G$ is a subgroup $H$ such 
that $G=H\ltimes \rad_\unipotent (G)$. Such an $H$ will be a maximal 
reductive subgroup of $G$. 

A maximal connected solvable subgroup of $G$ is called a \emph{Borel subgroup}. 
The group $B$ of upper-triangular matrices is a borel subgroup of 
$\generallinear(n)$. Every torus is contained in a Borel subgroup, and if $G$ 
is a reductive group, then all Borel subgroups of $G$ are conjugate. 

A group $G$ over $\dC$ is reductive if and only if every representation 
$\rho:G\to \generallinear_N$ is semi-simple (a direct sum of irreducible 
representations). Alternatively, the ring $\dC[\rho(G)]$ should be semi-simple. 

\begin{enonce}[remark]{Example}
The group $\simeq U_1 = \begin{pmatrix} 1 & \ast \\ & 1 \end{pmatrix}$ 
is not reductive. 
\end{enonce}





\subsection{Parabolic subgroups}

A subgroup $P$ of a connected algebraic group $G$ is called \emph{parabolic} if 
the quotient $G/P$ is projective. The basic theorem is that $P$ is parabolic if 
and only if $P$ contains a Borel subgroup $B$. So in $\generallinear_N$, a 
subgroup is parabolic if it contains a conjugate of the subgroup of 
upper-triangular matrices. 

\begin{enonce}[remark]{Example}
Let $k$ be an algebraically closed field. Recall that a \emph{flag} in 
$k^n$ is a collection of subspaces 
$F=(0\subsetneq F_1\subsetneq \cdots \subsetneq F_a=k^n)$. The \emph{type} of 
$F$ is $\boldsymbol d=(\dim F_i)_i$. The space of type $\boldsymbol d$ is a 
projective variety $\flag_{\boldsymbol d}$ on which $\generallinear_n$ acts 
transitively. Let $P$ be a stabilizer of a flag. Then 
$G/P\simeq \flag_{\boldsymbol d}$ and $P$ is parabolic. For example, if 
$F_i$ is the span of $\{e_1,\dots,e_{d_i}\}$, then 
\[
  P = \begin{pmatrix} 1_{h_1} \end{pmatrix} [finish]
\]
\end{enonce}

[finish]




