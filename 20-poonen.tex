% !TEX root = sms.tex

\section{Selmer groups and heuristics I}\label{sec:poonen-ii}
\thanksauthor{Bjorn Poonen}





The results of this lecture are joint work with Eric Rains. See the paper 
\cite{pr12} for details. 





\subsection{Introduction}

For simplicity, we'll work over $\dQ$, but most of these results work over any 
number field, and even over any global field. Let 
$G=\galois(\overline\dQ/\dQ)$, $v$ be a place of $\dQ$, and 
\[
  \dQ_v = \begin{cases}\dR & v=\infty \\ \dQ_p & v=p\end{cases} 
\]
Let $\dA=\prod_v' (\dQ_v,\dZ_v)$ be the restricted direct product of the 
$\dQ_v$. That is, 
\[
  \dA = \left\{(a_v)\in \prod_v \dQ_v:a_v\in \dZ_v\text{ all but finitely many }v\right\}
\]
The ring of adeles $\dA$ is locally compact. 

\begin{theo}[Mordell]
Let $E$ be an elliptic curve over $\dQ$. Then $E(\dQ)$ is finitely generated. 
\end{theo}
\begin{proof}
This is essentially the only known. First, show that $E(\dQ)/n$ is finite for 
some $n\geqslant 2$. Then use height functions to conclude that $E(\dQ)$ is 
finitely generated. 
\end{proof}

The only known proof of finiteness of $E(\dQ)/n$ passes through the finiteness 
of Selmer groups. Start with the exact sequence 
\[\xymatrix{
  0 \ar[r] 
    & E[n] \ar[r] 
    & E \ar[r]^-n 
    & E \ar[r] 
    & 0 ,
}\]
which we think of as an exact sequence of \'etale sheaves over 
$\spectrum(\dQ)$, i.e.~$G_\dQ$-modules. Taking global sections 
($G$-invariants), we get a long exact sequence 
\[
  0 \to E[n](\dQ) \to E(\dQ) \xrightarrow n E(\dQ) \to \h^1(\dQ,E[n]) \to \h^1(\dQ,E) \to \cdots
\]
We write $\h^1(\dQ,E[n])$ for either of the two (naturally isomorphic) groups 
\[
  \h^1(G_\dQ,E[n](\overline\dQ))\qquad \h^1(k_\textnormal{\'et},E[n]).
\]
This gives us an exact sequence 
\[
  0 \to E(\dQ)/n \to \h^1(\dQ,E[n]) \to \h^1(\dQ,E) \to 0 .
\]
Unfortunately, $\h^1(\dQ,E[n])$ is (provably) infinite. But we can try to pin 
down the image of $E(\dQ)/n$ inside $\h^1(\dQ,E[n])$. Working at each place 
$v$, we get a commutative diagram 
\[\xymatrix{
  0 \ar[r] 
    & E(\dQ)/n \ar[r] \ar[d] 
    & \h^1(\dQ,E[n]) \ar[r] \ar[d]^-\beta 
    & \h^1(\dQ,E) \ar[r] \ar[d]^-\gamma 
    & 0 \\
  0 \ar[r] 
    & E(\dA)/n \ar[r]^-\alpha 
    & \h^1(\dA,E[n]) \ar[r] 
    & \h^1(\dA,E) \ar[r] 
    & 0 .
}\]
Here we write $\h^1(\dA,E[n])$ for either $\h^1(\dA_\textnormal{\'et},E[n])$ 
or $\prod_v' \h^1(\dQ_v,E[n])$ with respect to the $\h^1(\dZ_v,\cE[n])$, where 
$\cE$ is the N\'eron model for $E$. The two groups are isomorphic -- see 
\cite{mathoverflow} for details. 
Define 
\begin{align*}
  \selmer_n E &= \beta^{-1}(\image \alpha) \\
  \sha(E) &= \ker(\gamma) .
\end{align*}
The group $\selmer_n E$ is finite and, in principle, computable. As an 
exercise, show that there is a natural exact sequence 
\[
  0 \to E(\dQ)/n \to \selmer_n E \to \sha(E)[n] \to 0 .
\]

For an elliptic curve $E:y^2=x^3+A x+B$ with $A,B\in \dZ$ minimal, put 
$h(E)=\max(|A|^3,|B|^2)$. Let $\cE$ be the set of isomorphism classes of 
elliptic curves over $\dQ$. For $X\in \dR$, put 
$\cE_{<X}=\{E\in \cE:h(E)<X\}$. 

\begin{defi}
For $S\subset \cE$, define 
\[
  \probability(S) = \lim_{X\to \infty} \frac{\#(S\cap \cE_{<X})}{\# \cE_{<X}} .
\]
\end{defi}

Given a prime $p$, what is $\probability(\dim\selmer_p E=s)$? In \cite{h94}, 
Heath-Brown proved that as $E$ ranges over quadratic twists of $y^2=x^3-x$, 
we have 
\[
  \probability(\dim \selmer_2(E)-2=s) = \prod_{j\geqslant 0} \frac{1}{1+2^{-j}} \prod_{j=1}^s \frac{2}{2^j-1} .
\]
This is not one of the standard (e.g.~Bernoulli or Poisson) distributions. 
The distribution turns out to model ``random maximal isotropic subspaces of a 
quadratic space.'' 





\subsection{Maximal isotropic subspaces}

Let $V=\dF_p^{2n}$, and let 
\[
  Q(\boldsymbol x,\boldsymbol y) = Q(x_1,\dots,x_n,y_1,\dots,y_n) = x_1 y_1 + \cdots + x_n y_n .
\]
The pair $(V,Q)$ is a hyperbolic quadratic space over $\dF_p$. To any quadratic 
space we can associate a symmetric bilinear form 
$\langle \cdot,\cdot\rangle:V\times V\to \dF_p$ by 
\[
  \langle v,w\rangle = Q(v+w)-Q(v)-Q(w) .
\]
Given a subspace $Z\subset V$, define the ``orthogonal complement'' $Z^\bot$ as 
\[
  Z^\bot = \{v\in V:\langle v,z\rangle = 0\text{ for all }z\in Z\} .
\]
It's not actually a complement, because it could be that $Z\cap Z^\bot \ne 0$. 
We do have $V^\bot = 0$. If $\dim Z=m$, then $\dim Z^\bot = 2 n-m$. 

We call a subspace $Z\subset V$ \emph{isotropic} if $Q|_Z=0$. We say $Z$ is 
\emph{maximal isotropic} (or \emph{Lagrangian}) if $Q|_Z=0$ and $Z=Z^\bot$. The 
second condition forces $\dim Z=n$. 

\begin{example}
The subspace $\{(x_1,\dots,x_n,0,\dots,0)\}\subset V$ is maximal isotropic. 
\end{example}

Define the \emph{orthogonal Grassmannian} 
$\orthogonalgrassmannian_n\subset \grassmannian_{n,2 n}$ by 
\[
  \orthogonalgrassmannian_n(\dF_p) = \{\text{maximal isotropic }Z\subset V\} .
\]
Choose $Z,W\in \orthogonalgrassmannian_n(\dF_p)$ uniformly at random. We get 
a random variable $\dim_{\dF_p}(Z\cap W)$. 

\begin{conjecture}[Poonen, Rains]
As $E$ ranges over all elliptic curves over $\dQ$, 
\[
  \probability(\dim \selmer_p E=s) = \lim_{n\to \infty} \probability(\dim(Z\cap W)=s) .
\]
\end{conjecture}

Some basic combinatorics and linear algebra show that the predicted 
distribution is agrees with the theorem of Heath-Browns when $p=2$. 





\subsection{Selmer groups and maximal isotropic subspaces}

Why should we expect this conjecture be true? Is there anything in the 
arithmetic of elliptic curves that would lead us to expect Selmer groups to 
behave like intersections of maximal isotropic subspaces? 

Our goal is to show that $\selmer_p E$ actually is the intersection of two 
maximal isotropic subspaces of an infinite-dimensional quadratic space. 

For the rest of this lecture, for simplicity assume $p$ is an odd prime. Then 
to choose a quadratic function on an $\dF_p$-vector space $V$ is equivalent 
to choosing a bilinear form $\langle \cdot,\cdot\rangle$ on $V$. If we have 
$\langle\cdot,\cdot\rangle$, we set 
\[
  Q(x) = \frac 1 2 \langle x,x\rangle.
\]





\subsection{Local fields}

Let $E$ be an elliptic curve over $\dQ_v$. Let $V_v = \h^1(\dQ_v,E[p])$. This 
is a finite-dimensional $\dF_p$-vector space. To prove this, set 
$L=\dQ_v(E[p])\supset \dQ_v(\mu_p)$. The extension $L/\dQ_v$ is finite Galois. 
We get a short exact sequence 
\[\xymatrix{
  0 \ar[r] 
    & \h^1(\galois(L/\dQ_v),E[p]) \ar[r]^-{\mathrm{inf}} 
    & \h^1(\dQ_v,E[p]) \ar[r]^-{\mathrm{res}} 
    & \h^1(L,E[p]) .
}\]
To prove that $\h^1(\dQ_v,E[p])$ is finite-dimensional, it is sufficient to 
prove that the groups on the ends of this sequence are finite. 
The group on the left is obviously finite. The group on the right is 
$\hom_\mathrm{cts}(G_L,\dZ/p)^{\oplus 2}$. Since $\dZ/p$ is abelian, local 
class field theory tells us that $\hom_\mathrm{cts}(G_L,\dZ/p)$ is 
$\hom_\mathrm{cts}(L^\times / p,\dZ/p)$, a finite group. Actually, since 
$\mu_p\subset L$, we didn't need class field theory -- we could have just used 
Kummer theory. 

The Weil pairing 
$e:E[p]\times E[p] \to \boldsymbol\mu_p\subset \dG_\multiplicative$ induces via 
the cup-product a pairing 
\[
  \langle \cdot,\cdot\rangle:V_v\times V_v \to \h^2(\dQ_v,E[p]\otimes E[p]) \to \h^2(\dQ_v,\dG_\multiplicative) = \brauer(\dQ_v)\hookrightarrow \dQ/\dZ .
\]
We will write $Q$ for the induced quadratic map $Q_v:V_v\to \dR/\dZ$. Let 
$W_v = \image(E(\dQ_v)/p \hookrightarrow \h^1(\dQ_v,E[p]))$; it turns out that 
$W_v=\h^1(\dZ_v,\cE[p])$, where $\cE$ is the N\'eron model for $E$ over 
$\dZ_v$. The following theorem is proved in \cite{n02} using Tate local 
duality. 

\begin{theo}[O'Neil]
The subspace $W_v\subset V_v$ is maximal isotropic with respect to $Q_v$. 
\end{theo}





\subsection{Global fields}

Let $E$ be an elliptic curve over $\dQ$. For each place $v$, let 
$V_v=\h^1(\dQ_v,E[p]) \supset W_v$. Define 
\begin{align*}
  V &= \prod' (V_v,W_v) = \h^1(\dA,E[p]) \\
  Q &= \sum Q_v :V\to \dR/\dZ .
\end{align*}
Then $(V,Q)$ is a ``quadratic locally compact group.'' Recall our diagram: 
\[\xymatrix{
  \selmer_p E \ar@{^{(}->}[r] 
    & \h^1(\dQ,E[p]) \ar[d]^-\beta \\
  E(\dA)/p \ar[r]^-\alpha 
    & \h^1(\dA,E[p]) 
}\]
in which $\selmer_p E = \beta^{-1}(\image\alpha)$. 

\begin{theo}
1. $\image(\alpha)$ and $\image(\beta)$ are maximal isotropic. 

2. $\beta$ is injective. 

3. $\image(\alpha)\cap \image(\beta) = \beta(\selmer_p E)\simeq \selmer_p(E)$
\end{theo}
\begin{proof}[Ingredients of proof]
1. use the fact that $\image(\alpha) = \prod W_v$ and each $W_v$ is 
isotropic. The fact that $\image(\beta)$ is maximal isotropic follows from the 
9-term Poitou-Tate exact sequence in global duality and the Brauer reciprocity 
law. 

2. Use the \v Cebotarev density theorem, and the fact that the Sylow 
$p$-subgroups 
$\begin{pmatrix} 1 & \ast \\ & 1 \end{pmatrix}\subset \generallinear_2(\dF_p)$ 
is cyclic. 

3. This follows from 1, 2, and the definition of $\selmer_p E$. The 
group $\h^1(\dA,E[p])$ is self-dual via our pairing 
$\langle \cdot,\cdot\rangle$. 
\end{proof}

So $\selmer_p(E)$ ``is'' the intersection of the two maximal isotropic subspaces 
$E(\dA)/p$ and $\h^1(\dQ,E[p])$ of $\h^1(\dA,E[p])$. The theorem is false for 
abelian varieties, and for $p^2$-torsion. 





\subsection{Passing to \texorpdfstring{$p^\infty$}{pinfty}-torsion}

Recall that there is a canonical exact sequence 
\[
  0 \to E(\dQ)/n \to \selmer_n E \to \sha(E)[n] \to 0 
\]
for all integers $n\geqslant 1$. Take $n=p^e$ and pass to the direct limit. We 
get a short exact sequence 
\begin{equation*}\tag{$\ast$}\label{eq:sel-p}
  0 \to E(\dQ)\otimes \dQ_p/\dZ_p \to \selmer_{p^\infty} E \to \sha(E)[p^\infty] \to 0 .
\end{equation*}
These are all $\dZ_p$-modules with finitely-generated Pontryagin dual. (That 
is, this is an exact sequence of co-finitely generated $\dZ_p$-modules.) So 
each is of the form $(\dQ_p/\dZ_p)^s\oplus T$, where $T$ is a finite abelian 
$p$-group. We call $s$ the \emph{corank} of the group. Conjecturally, 
$\sha(E)$ is finite, so $\sha(E)[p^\infty]$ has corank $0$. It would follow 
that $\rank E$ is the $p$-corank of $\selmer_{p^\infty}(E)$. In 
\autoref{sec:poonen-iii}, we'll give a probabilistic model for the sequence 
\eqref{eq:sel-p}. 




