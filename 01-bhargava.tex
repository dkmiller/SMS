% !TEX root = sms.tex

\section{Introduction and perspective}
\thanksauthor{Manjul Bhargava}





\subsection{Motivation}

The main question we are interested in is: given a class $\cC$ of objects ``of 
arithmetic interest,'' how many objects are there in $\cC$, up to isomorphism, 
having bounded invariants? 

\begin{enonce}[remark]{Example}
The following are the main examples we're interested in: 
\begin{center}
\begin{tabular}{c|c}
$\cC$ & invariant \\ \hline
number fields of given degree & discriminant \\
class group elements of number fields of given degree & '' \\
rational points on curves & height \\
elliptic curves weighted by rank & '' \\
$n$-Selmer elements of Jacobians of curves & '' 
\end{tabular}
\end{center}
All of these will be defined precisely later on. 
\end{enonce}

Given such a class of objects of arithmetic interest, how they are distributed 
(asymptotically) with respect to their basic invariants? Beyond the cases of 
degree $2$ number fields and genus $0$ curves, little was known at the 
beginning of the 20th century. 





\subsection{Strategy}

Direct methods of counting arithmetic objects generally fail except in the 
``easy'' cases of degree $2$ number fields and genus $0$ curves. The modern 
approach uses representation theory. We try to find a map 
\[
  (\cC/\sim) \hookrightarrow G(\dZ)\backslash V(\dZ) ,
\]
where $G$ is an algebraic group and $V$ is a representation of $G$, both 
defined over $\dZ$. More precisely, one finds such a map that sends the 
invariants of objects in $\cC$ to the ring of fundamental polynomial 
invariants of the action of $G$ on $V$. Good choices of such maps often come 
from algebraic geometry, but we have to work out the theory over $\dZ$. 

\begin{enonce}[remark]{Example}[Gauss]
In his \emph{Disquisitiones}, Gauss constructs a map 
\[
  \left\{\begin{array}{c}\text{ideal classes of (orders } \\ \text{in) quadratic fields with} \\ \text{non-square discriminant}\end{array}\right\}\bigr/\simeq 
  \iso 
  \speciallinear_2(\dZ)\bigl\backslash \left\{\begin{array}{c}\text{integer binary quadratic} \\ \text{forms }a x^2+b x y+c y^2\end{array} \right\} 
\]
The map sends the discriminant of such an ideal class to $b^2-4 a c$ (the 
discriminant of the quadratic form, which is the unique polynomial invariant 
of quadratic forms). If $I=\langle \alpha,\beta\rangle$, then 
the corresponding quadratic form is $\norm(\alpha x+\beta x)/\norm(I)$. 
\end{enonce}

\begin{enonce}[remark]{Example}[Levi, Delone-Faddeev] % 20s, 40s [Granville talk]
Recall that a \emph{cubic ring} is a (commutative, unital) ring structure 
on $\dZ^3$. They constructed a map 
\[
  \{\text{cubic rings}\}/\sim \iso \generallinear_2(\dZ)\bigl\backslash\left\{\begin{array}{c}\text{integer binary cubic forms} \\ ax^3+b x^2 y+c x y^2+d y^3\end{array}\right\} ,
\]
which ``preserves discriminant.'' The discriminant of a binary cubic form is 
$b c^2-4 a c^3 - 4 b^3 d - 27 a^2 d^2 + 18 a b c d$. If 
$R=\langle 1,\alpha,\beta\rangle$ is a cubic ring, the associated binary cubic 
form is 
$\sqrt{\discriminant(\alpha x+\beta y)/\discriminant(R)} = [R:\dZ[\alpha x+\beta y]]$. 
See [Bhargava thesis, Wood thesis]. 
\end{enonce}

\begin{enonce}[remark]{Example} % [Wood II]
There are similar maps 
\[
  \{\text{quartic (resp.~quintic) fields}\}/\sim \to \{\text{representations\ldots}\} .
\]
These are also discriminant-preserving (in all these cases, the discriminant on 
the right is the unique polynomial invariant). 
\end{enonce}

\begin{enonce}[remark]{Example}[Birch, Swinnerton-Dyer] % Shankar
There is a map 
\[
  \left\{\begin{array}{c}\sigma\in E(\dQ)/2:E\text{ of the } \\ \text{form }y^2=x^3+A x+B\end{array}\right\} \hookrightarrow\generallinear_2(\dQ)\backslash \left\{\begin{array}{c}\text{integer binary} \\ \text{quartic forms}\end{array}\right\} .
\]
The program \texttt{mwrank} created by Cremona uses this. In fact, this map 
factors through the $2$-Selmer group $\selmer_2(E)$. Write $E_{A,B}$ for the 
elliptic curve $y^2=x^3+A x+B$. This map sends $E_{A,B}$ to the fundamental 
invariants $I,J$ (of degree $2$, $3$ respectively) of binary quartics. 
\end{enonce}

\begin{enonce}[remark]{Example}[Cremona-Fisher-Stoll]
For $n\in \{3,4,5\}$, there is a map 
\[
  \{\sigma \in E(\dQ)/n:E\text{ of the form }E_{A,B}\} \hookrightarrow G(\dZ)\backslash V(\dZ) ,
\]
where $A,B$ are send to the fundamental invariants $I,J$ on the right-hand 
side. 
\end{enonce}

\begin{enonce}[remark]{Example} % [Gross I, Poonen IV]
There is a non-injective, but still useful map 
\[
  \left\{\begin{array}{c}\text{rational points on odd hyperelliptic} \\ \text{curves }y^2=x^{2 g+1} + a_1 x^{2 g} + \cdots + a_{2g+1} \end{array} \right\} \to \specialorthogonal_{2 g+1}(\dZ) \backslash \symmetric^2(\dZ^{2 g+1}) .
\]
This sends the $a_i$ to the invariants on the right-hand side. 
\end{enonce}

\begin{enonce}[remark]{Example} % [Gross II, Ho] -- also see Jack Thorne, [Wood III]
There is a (non-injective) map 
\[
  \left\{\begin{array}{c}\text{rational points on even hyperelliptic} \\ \text{curves } z^2=a_0 x^{2 g+2} + \cdots + a_{2 g+2} y^{2 g+2} \end{array} \right\} \to \text{product of SL's} \backslash\cdots ,
\]
sending the $a_i$ to invariants. 
\end{enonce}

Once we've found our map from arithmetic objects to orbits, the question 
becomes: how many orbits of $G(\dZ)$ on $V(\dZ)$ are there having bounded 
invariants? Gauss worked this out for binary quadratic forms. Let $h(D)$ be 
the number of $\speciallinear_2(\dZ)$-orbits of integer binary quadratic forms of 
discriminant $D$. 

\begin{theo}[Gauss, Lipschitz, Mertens]
\[
  \sum_{0<-D<X} h(D) \sim \frac{\pi}{18} X^{3/2} .
\]
\end{theo}
\begin{proof}
Gauss shows that every integer binary quadratic form $a x^2+b x y+c y^2$ 
with $D=b^2-4 a c<0$ has a unique $\speciallinear_2(\dZ)$-equivalent form 
satisfying $|b|<a\leqslant c$ or $0<b=a\leqslant c$. 

We apply geometry of numbers to the counting problem 
\[
  \sum_{0<-D<X} h(D) \sim \# \{(a,b,c):0<4 a c-b^2<X\text{ and }|b|<a\leqslant c\} .
\]
Gauss's conjecture tells us that the number of points on the right is 
asymptotically the volume of the region. Proving this is tricky! In fact, the 
conjecture is false in general. If we consider the region 
$R=\{(a,b,c):0\leqslant 4 a c-b^2<X\text{ and }|b|\leqslant a \leqslant c\}$, 
then $R\cap \dZ^3$ contains infinitely points. 
\end{proof}

One can attack the lattice-point counting problem explicitly by writing the 
count as a triple sum $\sum_{a,b,c}$, approximating the sum by a triple 
integral, and keeping track of error terms. This is how Lipschitz and Mertens 
proved the result. Working through this is a good exercise. Davenport developed 
some general principles for bounded regions. He used the principle to reprove 
Gauss' count of binary quadratics, and extended the argument to a count of 
binary cubic forms. This requires knowing explicit inequalities for the region. 

A third approach uses zeta functions (or more generally $L$-functions). 
% [Granville, Taniguchi, Thorne]
Siegel first applied this to binary quaratic forms. Goldfeld-Hoffstein, 
Shintani, and Datskovsky extended these methods. 

% [Shankar]
There is a hybrid method: average over a compact continuum of fundamental 
domains. It doesn't need explicit inequalities, but still uses elementary 
geometry of numbers. The method does adapt to situations where there are more 
than one invariant. For examples, it works on all the above examples. In 
particular, it gives a count of quartic and quintic fields, boundedness of 
average rank of elliptic curves, and produces lots of hyperelliptic curves with 
few rational points. 

What if we replace $\dQ$ with another base field, like a number field or 
function field? Over a function field, one can use algebro-geometric and 
topological methods. Boundedness of average rank was proved by de Jong. 
Ellenberg has proved many other results of this type. Also, the ``hybrid 
method'' works over an arbitrary global field. % [Wang]




