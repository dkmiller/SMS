% !TEX root = sms.tex

\section{Exercises}





[list of TAs]





\subsection{Some cohomological computations for the representation \texorpdfstring{$V=\symmetric_2(n)\oplus \symmetric_2(n)$}{V=Sym2(n)+Sym2(n)} of \texorpdfstring{$G=\speciallinear_n$}{G=SLn} and \texorpdfstring{$H=\speciallinear_n/\boldsymbol\mu_2$}{H=SL2/mu2}}

Suppose $G$ is a reductive group with a representation $V$ over a field $k$. 
et $V\gq G=\spectrum k[V]^G$ denote the canonicl quotient. Let 
$f\in (V\gq G)(k)$ be a rational invariant and suppose $G(k^s)$ acts 
transitively on $V_f(k^s)$ with abelian stabilizers. In Gross' talk, we 
learned two obstructions for the existence of a rational element $v\in V(k)$ 
with invariant $f$. In this worksheet, we will make some computations regarding 
these obstructions when the representation $V$ is the space 
$\symmetric_2(n)\oplus\symmetric_2(n)$ of pairs of symmetric bilinear forms and 
when the reductive group is either $\speciallinear_n$ or 
$\speciallinear_n/\boldsymbol\mu_2$ with the action given by 
$g\cdot (A,B)=(\transpose g A g, \transpose g B g)$. 


\subsubsection{Warm up}

Consider the conjugation action of $G=\generallinear(W)$ on $V=\End(W)$. One 
can obtain invariants by taking the coefficients $c_1,\dots,c_n$ of the 
characteristic polynomial. 

\begin{enonce*}[remark]{Exercises}
Show that the ring of polynomial invariants $k[V]^G=k[c_1,\cdots,c_n]$ via the 
following steps (or however you want to). 
\begin{enumerate}
  \item Show that for any $c_1,\dots,c_n$ in $k^s$, there exists some 
    $T\in V(k^s)$ with characteristic polynomial 
    \[
      \det(x\cdot 1-T) = x^{2n+1} + c_1 x^{2n} + \cdots + c_{2n+1} .
    \]
    This shows that there is no relation among the invariants. 
  \item Show that for any $c_1,\dots,c_n$ in $k^s$ such that 
    $f(x)=x^{2n+1} + c_1 x^{2n} + \cdots + c_{2n+1}$ has no repeated roots and 
    for any $T,T'\in V_f(k^s)$, there exists some $g\in G(k^s)$ such that 
    $g T g^{-1} = T'$. This shows that there are no other invariants. 
\end{enumerate}
\end{enonce*}

Next we consider the conjugation action of the subgroup $H=\speciallinear(W)$ 
on $V=\End(W)$. Let $T\in V(k)$ be a regular semisimple operator, that is its 
characteristic polynomial $f(x)$ has no repeated factors. Let $L=k[x]/f(x)$ be 
the associated $k$-vector space of dimension $n$. 

\begin{enonce*}[remark]{Exercises}
\begin{enumerate}
  \item Show that the stabilizer $H_T$ of $T$ is isomorphic to 
    $(\weilres_{L/k}\dG_\multiplicative)^{\norm=1}$, the kernel of the norm 
    map $\weilres_{L/k}\dG_\multiplicative\to\dG_\multiplicative$. 
  \item Show that $\h^1(k,H_T)  \simeq k^\times/\norm(L^\times)$ by taking 
    Galois cohomology of the short exact sequence 
    \[
      1 \to H_f \to \weilres_{L/k}\dG_\multiplicative \to \dG_\multiplicative \to 1 .
    \]
    Note Shapiro's lemma implies that $\h^1(k,\weilres_{L/K} M) = \h^1(L,M)$ 
    for any $\galois(L^s/L)$-module $M$. 
  \item Using the same idea, show that 
    $\h^1\left(k,(\weilres_{L/k}\boldsymbol\mu_2)^{\norm=1}\right) \simeq (L^\times/2)^{\norm=\square}$. 
\end{enumerate}
\end{enonce*}




