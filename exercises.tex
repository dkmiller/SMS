% !TEX root = sms.tex

\section{Exercises}





Those directing the problem sessions were (in alphabetical order): 
Jordan Ellenberg, Wei Ho, Jennifer Park, Arul Shankar, Frank Thorne, and 
Jerry Wang. 





\subsection{The parameterization of cubic, quartic, and quintic rings}


\subsubsection{Directly from Wood's cubic rings lecture}
\begin{enumerate}[\indent a)]
  \item Prove that the inverse maps $R\mapsto \discriminant(R)$ and 
    $D\mapsto \dZ[\tau]/\left(\tau^2-D\tau+\frac{D^2-D}{4}\right)$ induce a 
    bijection between the set of quadratic rings (up to isomorphism) and 
    \[
      \{D\in \dZ:D\equiv 0,1\pmod 4\} . 
    \]
  \item In the Delone-Faddeev equations 
    \begin{align*}
      \omega\theta &= n \\
      \omega^2 &= m-b\omega + a\theta \\
      \theta^2 &= \ell-d\omega + c\theta 
    \end{align*}
    prove that associativity is equivalent to the equations 
    \begin{align*}
      n &= -a d \\
      \ell &= -b d \\
      m &= - a c
    \end{align*}
  \item Wood mentioned that if you write $+b$ and $+d$ in place of $-b$ and 
    $-d$ above, the correspondence comes out slightly wrong. Try it and see 
    what happens. 
  \item Orders in cubic number fields correspond to irreducible cubic forms 
    $f(x,y)$ and the number field can be recovered as 
    $\dQ[\theta]/f(\theta,1)$. What happens if you substitute $f(1,\theta)$ for 
    $f(\theta,1)$. (What \emph{must} happen?)
  \item For a cubic form $f$, prove that the functions on its vanishing set 
    $V_f$ determine a cubic ring, which is the same ring obtained by the 
    Delone-Faddeev correspondence. (Describe any special conditions, 
    e.g.~$f\ne 0$, which are necessary in your proof.) 
\end{enumerate}


\subsubsection{Other exercises concerning cubic rings}

We give more exercises for cubic than for quartic or quintic rings. Note that 
most or all of these exercises are interesting for all three parameterizations 
being discussed. You are \emph{strongly encouraged} to extrapolate problems 
from one section to another! What is the same, and what is different? 
\begin{enumerate}[\indent a)]
  \item A good way to get started is to compute lots of examples of the 
    Delone-Faddeev correspondence. (If you don't do any of the other 
    exercises, you should probably do at least this, and the quartic and 
    quintic analogues!) What binary cubic form $f$ corresponds to the cubic 
    ring $\dZ^3$? To $\dZ[\sqrt[3] n]$? Conversely, what cubic rings 
    corresponds to the cubic form $u^3-u v^2+v^3$? To $u(u-v)(u+v)$? To $u^3$? 
    To $0$? Work out these, as well as other examples of your own invention, 
    and compute all of their discriminants. 
  \item Another good way to get started is to work out the details of the 
    Delone-Faddeev and Davenport-Heilbronn correspondences. The exposition 
    given in \cite[\S 2]{bst13} leaves many small details to be checked by the 
    reader. Pick your favorite lemma or proposition and work out the proof in 
    more detail than given in the paper. 
  \item The Delone-Faddeev correspondence is very interesting over $\dF_p$. 
    Assuming for simplicity that $p\ne 2,3$, determine all of the cubic rings 
    over $\dF_p$ as well as the $\generallinear_2(\dF_p)$-equivalence classes 
    of cubic forms over $\dF_p$. How many equivalence classes are there? On the 
    cubic forms side, how large is each $\generallinear_2(\dF_p)$-equivalence 
    class, and how big is each of the corresponding stabilizer groups? If you 
    reduce an integral binary cubic form modulo $p$, what is the relationship 
    between the cubic ring over $\dZ$ and the cubic ring over $\dF_p$?
\end{enumerate}





\subsection{Some cohomological computations for the representation \texorpdfstring{$V=\symmetric_2(n)\oplus \symmetric_2(n)$}{V=Sym2(n)+Sym2(n)} of \texorpdfstring{$G=\speciallinear_n$}{G=SLn} and \texorpdfstring{$H=\speciallinear_n/\boldsymbol\mu_2$}{H=SL2/mu2}}

Suppose $G$ is a reductive group with a representation $V$ over a field $k$. 
et $V\gq G=\spectrum k[V]^G$ denote the canonicl quotient. Let 
$f\in (V\gq G)(k)$ be a rational invariant and suppose $G(k^s)$ acts 
transitively on $V_f(k^s)$ with abelian stabilizers. In Gross' talk, we 
learned two obstructions for the existence of a rational element $v\in V(k)$ 
with invariant $f$. In this worksheet, we will make some computations regarding 
these obstructions when the representation $V$ is the space 
$\symmetric_2(n)\oplus\symmetric_2(n)$ of pairs of symmetric bilinear forms and 
when the reductive group is either $\speciallinear_n$ or 
$\speciallinear_n/\boldsymbol\mu_2$ with the action given by 
$g\cdot (A,B)=(\transpose g A g, \transpose g B g)$. 


\subsubsection{Warm up}

Consider the conjugation action of $G=\generallinear(W)$ on $V=\End(W)$. One 
can obtain invariants by taking the coefficients $c_1,\dots,c_n$ of the 
characteristic polynomial. 

\begin{enonce*}[remark]{Exercise}
Show that the ring of polynomial invariants $k[V]^G=k[c_1,\cdots,c_n]$ via the 
following steps (or however you want to). 
\begin{enumerate}
  \item Show that for any $c_1,\dots,c_n$ in $k^s$, there exists some 
    $T\in V(k^s)$ with characteristic polynomial 
    \[
      \det(x\cdot 1-T) = x^{2n+1} + c_1 x^{2n} + \cdots + c_{2n+1} .
    \]
    This shows that there is no relation among the invariants. 
  \item Show that for any $c_1,\dots,c_n$ in $k^s$ such that 
    $f(x)=x^{2n+1} + c_1 x^{2n} + \cdots + c_{2n+1}$ has no repeated roots and 
    for any $T,T'\in V_f(k^s)$, there exists some $g\in G(k^s)$ such that 
    $g T g^{-1} = T'$. This shows that there are no other invariants. 
\end{enumerate}
\end{enonce*}

Next we consider the conjugation action of the subgroup $H=\speciallinear(W)$ 
on $V=\End(W)$. Let $T\in V(k)$ be a regular semisimple operator, that is its 
characteristic polynomial $f(x)$ has no repeated factors. Let $L=k[x]/f(x)$ be 
the associated $k$-vector space of dimension $n$. 

\begin{enonce*}[remark]{Exercise}
\begin{enumerate}
  \item Show that the stabilizer $H_T$ of $T$ is isomorphic to 
    $(\weilres_{L/k}\dG_\multiplicative)^{\norm=1}$, the kernel of the norm 
    map $\weilres_{L/k}\dG_\multiplicative\to\dG_\multiplicative$. 
  \item Show that $\h^1(k,H_T)  \simeq k^\times/\norm(L^\times)$ by taking 
    Galois cohomology of the short exact sequence 
    \[
      1 \to H_f \to \weilres_{L/k}\dG_\multiplicative \to \dG_\multiplicative \to 1 .
    \]
    Note Shapiro's lemma implies that $\h^1(k,\weilres_{L/K} M) = \h^1(L,M)$ 
    for any $\galois(L^s/L)$-module $M$. 
  \item Using the same idea, show that 
    $\h^1\left(k,(\weilres_{L/k}\boldsymbol\mu_2)^{\norm=1}\right) \simeq (L^\times/2)^{\norm=\square}$. 
\end{enumerate}
\end{enonce*}


\subsubsection{Cohomological obstructions}

The reference for his section is \cite[\S 2]{bgw13}. As shown in 
\autoref{sec:gross-ii}, the abelian groups $G_v$ for $v\in V_f(k^s)$ descend to 
a commutative group scheme $G_f$ over $k$ unique up to unique isomorphism. In 
other words, there are canonical isomorphisms $\iota_v:G_f(k^s)\iso G_v(k^s)$ 
for any $v\in V_f(k^s)$ such that for any $h\in G(k^s)$, $b\in G_f(k^s)$, 
$\sigma\in \galois(k^s/k)$, 
\begin{align*}
  \iota_{h v}(b) &= h \iota_v(b) h^{-1} \\
  \prescript{\sigma}{}{\iota_v(b)} &= \iota_{\prescript{\sigma}{}{v}}(\prescript{\sigma}{}{b}) .
\end{align*}
For any $v\in V_f(k^s)$ and any $\sigma\in \galois(k^s/k)$, choose $g_\sigma$ 
such that $g_\sigma \prescript{\sigma}{}{v}=v$. Define 
\[
  d_{\sigma,\tau} = \iota_v^{-1}(g_\sigma g_\tau g_{\sigma\tau}^{-1}) \in G_f(k^s) .
\]

\begin{enonce*}[remark]{Exercise}
\begin{enumerate}
  \item Show that $(d_{\sigma,\tau})$ is a 2-cocycle whose image $d_f$ in 
    $\h^2(k,G_f)$ does not depend on the choice of $g_\sigma$. That is, show 
    that for any $\sigma,\tau,\mu\in \galois(k^s/k)$, 
    \[
      \prescript{\sigma}{}{d_{\tau,\mu}} d_{\sigma,\tau\mu} = d_{\sigma\tau,\mu} d_{\sigma,\tau} .
    \]
    and that if each $g_\sigma$ is changed to $h_\sigma g_\sigma$ for some 
    $h_\sigma$ in $G_v$, then 
    \[
      d_{\sigma,\tau}' = d_{\sigma,\tau} \iota_v^{-1} (h_\sigma) (\prescript{\sigma}{}{\iota_v}^{-1}(h_\tau)) (\iota_v^{-1}(h_{\sigma\tau})^{-1} .
    \]
  \item Show that the 2-cochain $(d_{\sigma,\tau})$ does not depend on the 
    choice of $v\in V_f(k^s)$. 
\end{enumerate}
\end{enonce*}

Given a class $c\in \h^1(k,G)$, one can obtain a pure inner form $G^c$ of $G$ 
and a representation $V^c$ of $G^c$ as follows. Suppose $c$ is given by the 
1-cocycle $(c_\sigma)$ with values in $G(k^s)$. Then $G^c(k^s) = G(k^s)$ with 
action 
\begin{equation}\label{eq:gross-2}
  \sigma(h) = c_\sigma \prescript{\sigma}{}{h} c_\sigma^{-1} .
\end{equation}
If we compose the cocycle $c$ with values in $G(k^s)$ with the homomorphism 
$\rho:G\to \generallinear(V)$, we obtain a cocycle $\rho(c)$ with values in 
$\generallinear(V)(k^s)$. By the generalization of Hilbert's Theorem 90, we 
have $\h^1(k,\generallinear(V))=1$. Hence there is an element $g$ in 
$\generallinear(V)(k^s)$, well-defined up to left multiplication by 
$\generallinear(V)(k)$, such that 
$\rho(c_\sigma)=g^{-1}\prescript{\sigma}{}{g}$ for all $\sigma$ in 
$\galois(k^s/k)$. We use the element $g$ to define a twisted representation of 
the group $G^c$ on the vector space $V$ over $k$. The homomorphism 
\[
  \rho_g:G^c(k^s) \to \generallinear(V)(k^s) 
\]
defined by $\rho_g(h)=g \rho(h) g^{-1}$ commutes with the respective Galois 
actions, so defines a representation over $k$. We emphasize that the Galois 
action on $G^c(k^s)$ is defined as in \eqref{eq:gross-2}, whereas the Galois 
action on $\generallinear(V)(k^s)$ is the usual action. We write $V^c$ for the 
representation $\rho_g$ of $G^c$. For any $f\in (V\gq G)(k^s)$, we write 
\[
  V_f^c(k) = (k)\cap g V_f(k^s) .
\]

\begin{enonce*}[remark]{Exercise}
\begin{enumerate}
  \item Show that the isomorphism class of $G^c$ does not depend on the choice 
    of the representative $(c_\tau)$. 
  \item Show that the isomorphism class of the representation $V^c$ of $G^c$ 
    does not depend on the choice of the element $g$. 
  \item Observe that the group scheme $G_f$ over $k$ does not depend on the 
    twist $c\in \h^1(k,G)$. Show that the class $d_f\in \h^2(k,G_f)$ does not 
    depend on the twist $c\in \h^1(k,G)$. 
\end{enumerate}
\end{enonce*}

\begin{enonce*}{Theorem}
Let $G$ be a reductive group with representation $V$. Suppose there exists 
$v\in V(k)$ with invariant $f\in (V\gq G)(k)$ and stabilizer $G_v$ such that 
$G(k^s)$ acts transitively on $V_f(k^s)$. Then there is a bijection between the 
set of $G^c(k)$-orbits on $V_f^c(k)$ and the fiber $\gamma^{-1}(c)$ of the map 
\[
  \gamma:\h^1(k,G_v) \to \h^1(k,G) 
\]
above the class $c\in \h^1(k,G)$. In particular, the image of $\h^1(k,G_v)$ in 
$\h^1(k,G)$ determines the set of pure inner forms of $G$ for which the 
$k$-rational invariant $f$ lifts to a $k$-rational orbit of $G^c$ on $V^c$. 
\end{enonce*}

\begin{enonce*}[remark]{Exercise}
Prove the theorem via the following steps (or however you want to). 
\begin{enumerate}
  \item Fix some $c\in \h^1(k,G)$. Show that $V_f^c(k)$ is nonempty if and only 
    if $c$ is in the image of $\gamma$. 
  \item Suppose $V_f^c(k)$ is nonempty and take $w\in V_f^c(k)$. Then there is a 
    bijection between $G^c(k)\backslash V_f^c(k)$ and $\ker(\gamma_c)$, where 
    $\gamma_c$ is the natural map of sets $\h^1(G_w^c)\to \h^1(k,G^c)$. Show 
    that there is a bijection between $\gamma^{-1}(c)$ and $\ker(\gamma_c)$. 
\end{enumerate}
\end{enonce*}

\begin{enonce*}{Theorem}
Suppose that $f$ is a rational invariant, and that $G(k^s)$ acts transitively 
on $V_f(k^s)$ with abelian stabilizers. Then $d_f=0$ in $\h^2(k,G_f)$ if and 
only if there exists a pure inner form $G^c$ of $G$ such that $V_f^c(k)$ is 
nonempty. That is, the condition $d_f=0$ is necessary and sufficient for the 
existence of rational orbits for some pure inner twist of $G$. In particular, 
when $\h^1(k,G)=1$, the condition $d_f=0$ in $\h^2(k,G_f)$ is necessary and 
sufficient for the existence of rational orbits of $G(k)$ on $V_f(k)$. 
\end{enonce*}

\begin{enonce*}[remark]{Exercise}
$\Leftarrow$ is trivial. Prove $\Rightarrow$ via the following steps (or 
however you want to). 
\begin{enumerate}
  \item Show that there exists a 1-cochain $(e_\sigma)$ with values in 
    $G_v(k^s)$ such that $c=(e_\sigma g_\sigma)$ is a 1-cocycle. 
  \item Show that $V_f^c(k)$ is nonempty. 
\end{enumerate}
\end{enonce*}


\subsubsection{The representation $V=\symmetric_2(n)\oplus \symmetric_2(n)$ of $G=\speciallinear_n$ and $H=\speciallinear_n/\boldsymbol\mu_2$}

The ring $k[V]^G$ of polynomial invariants is freely generated by the 
coefficients of the invariant binary form $f(x,y)=(-1)^{n(n-1)/2}\det(A x-B y)$ 
for $(A,B)\in V$. Fix some binary $n$-ic form 
$f(x,y) = f_0 x^n + \cdots + f_n y^n$ with coefficients in $k$ and suppose that 
$f_0$ and $\Delta(f)$ are nonzero. Let $L$ denote the \'etale extension 
$k[x]/f(x,1)$. Then $G(k^s)$ acts transitively on $V_f(k^s)$ with abelian 
stabilizers. Moreover, we have 
\begin{align*}
  G_f &\simeq (\weilres_{L/k}\boldsymbol\mu_2)^{\norm=1} , \\
  H_f &\simeq (\weilres_{L/k}\boldsymbol\mu_2)^{\norm=1}/\boldsymbol\mu_2 .
\end{align*}
The groups $G_f$ and $H_f$ fit inside short exact sequences 
\begin{align*}
  1 \to (\weilres_{L/k}\boldsymbol\mu_2)^{\norm=1} &\to \weilres_{L/k}\boldsymbol\mu_2 \xrightarrow{\norm} \boldsymbol\mu_2 \to 1 \\
  1 \to \boldsymbol\mu_2 \to (\weilres_{L/k}\boldsymbol\mu_2)^{\norm=1} &\to (\weilres_{L/k}\boldsymbol\mu_2)^{\norm=1}/\boldsymbol\mu_2 \to 1 .
\end{align*}
Taking Galois cohomology gives long exact sequences 
\begin{align*}
  L^\times / 2 &\xrightarrow{\norm} k^\times / 2 \xrightarrow{\delta_0} \h^2(k,G_f) , \\
  \h^1(k,H_f) &\xrightarrow{\delta} \h^2(k,\boldsymbol\mu_2) \xrightarrow\alpha h^2(k,G_f) .
\end{align*}
Let $d_f^G$ (resp.~$d_f^H$) denote the corresponding classes in $\h^2(k,G_f)$ 
(resp.~$\h^2(k,H_F)$) that obstruct the existence of a rational lift for some 
pure inner form of $G$ (resp.~$H$). For the following exercise, see 
\cite[\S 4.5]{bgw13}. 

\begin{enonce*}[remark]{Exercise}
\begin{enumerate}
  \item Show that $d_f^G=\delta_0(f_0)$. Since $\h^1(k,G)=1$, we see that 
    $V_f(k)$ is nonempty if and only if $f_0\in (k^\times)^2 \norm(L^2)$. 
  \item Show that $d_f^H$ is the image of $d_f^G$ under the natural map 
    $\h^2(k,G_f) \to \h^2(k,H_f)$. 
  \item Suppose now that $d_f^H=0$. We would like to know for which pure inner 
    forms of $H$ do there exist rational orbits with invariant $f$. 
     \begin{enumerate}
       \item Show that $d_f^G\in \h^2(k,G_f)$ is the image of some 
         $d\in \h^2(k,\boldsymbol\mu_2)$ under $\alpha$ where $d$ lies in the 
         image of the map $\delta_2:\h^1(k,H) \hookrightarrow \h^2(k,\boldsymbol\mu_2)$ 
         obtained from the short exact sequence $1\to \boldsymbol\mu_2 \to G\to H\to 1$. 
       \item Let $c$ be an element of $\h^1(k,H)$ such that $\delta_2(c)=d$. 
         Show that $v_f^c(k)$ is nonempty. 
       \item Show that the pure inner forms of $H$ for which rational orbits 
         exist with invariant $f$ correspond to classes $c\in \h^1(k,H)$ such 
         that $\alpha(\delta_2(c))=d_f^G$. 
     \end{enumerate}
\end{enumerate}
\end{enonce*}




