\documentclass{article}

\usepackage{amsmath,amssymb,amsthm,enumerate,extarrows,mathrsfs,mathtools,sms,stmaryrd}
\usepackage[colorlinks=true]{hyperref}
\usepackage[all]{xy}
\newtheorem{conjecture}[subsubsection]{Conjecture}
\newtheorem{coro}[subsubsection]{Corollary}
\newtheorem{fact}[subsubsection]{Fact}
\newtheorem{heuristic}[subsubsection]{Heuristic}
\newtheorem{lemm}[subsubsection]{Lemma}
\newtheorem{prediction}[subsubsection]{Prediction}
\newtheorem{principle}[subsubsection]{Principle}
\newtheorem{prop}[subsubsection]{Proposition}
\newtheorem{question}[subsubsection]{Question}
\newtheorem*{theo*}{Theorem}
\newtheorem{theo}[subsubsection]{Theorem}
\theoremstyle{definition}
\newtheorem{defi}[subsubsection]{Definition}
\newtheorem{example}[subsubsection]{Example}
\newtheorem*{exercise}{Exercise}


\SetSymbolFont{stmry}{bold}{U}{stmry}{m}{n}

\title{Counting arithmetic objects}
\date{June 23 -- July 4, 2014}
\author{Notes taken by Daniel Miller}

\begin{document}
\setcounter{tocdepth}{1}
\maketitle
\tableofcontents





% !TEX root = sms.tex

\section{Introduction and perspective}\label{sec:bhargava-i}
\thanksauthor{Manjul Bhargava}





\subsection{Motivation}

The main question we are interested in is: given a class $\cC$ of objects ``of 
arithmetic interest,'' how many objects are there in $\cC$, up to isomorphism, 
having bounded invariants? 

\begin{enonce}[remark]{Example}
The following are the main examples we're interested in: 
\begin{center}
\begin{tabular}{c|c}
$\cC$ & invariant \\ \hline
number fields of given degree & discriminant \\
class group elements of number fields of given degree & '' \\
rational points on curves & height \\
elliptic curves weighted by rank & '' \\
$n$-Selmer elements of Jacobians of curves & '' 
\end{tabular}
\end{center}
All of these will be defined precisely later on. 
\end{enonce}

Given such a class of objects of arithmetic interest, how they are distributed 
(asymptotically) with respect to their basic invariants? Beyond the cases of 
degree $2$ number fields and genus $0$ curves, little was known at the 
beginning of the 20th century. 





\subsection{Strategy}

Direct methods of counting arithmetic objects generally fail except in the 
``easy'' cases of degree $2$ number fields and genus $0$ curves. The modern 
approach uses representation theory. We try to find a map 
\[
  \cC/\simeq \hookrightarrow G(\dZ)\backslash V(\dZ) ,
\]
where $G$ is an algebraic group and $V$ is a representation of $G$, both 
defined over $\dZ$. More precisely, one finds such a map that sends the 
invariants of objects in $\cC$ to the ring of fundamental polynomial 
invariants of the action of $G$ on $V$. Good choices of such maps often come 
from algebraic geometry, but we have to work out the theory over $\dZ$. 

\begin{enonce}[remark]{Example}[Gauss]
In his \emph{Disquisitiones}, Gauss constructs a map 
\[
  \left\{\begin{array}{c}\text{ideal classes of (orders } \\ \text{in) quadratic fields with} \\ \text{non-square discriminant}\end{array}\right\}\bigr/\simeq 
  \iso 
  \speciallinear_2(\dZ)\bigl\backslash \left\{\begin{array}{c}\text{integer binary quadratic} \\ \text{forms }a x^2+b x y+c y^2\end{array} \right\} 
\]
The map sends the discriminant of such an ideal class to $b^2-4 a c$ (the 
discriminant of the quadratic form, which is the unique polynomial invariant 
of quadratic forms). If $I=\langle \alpha,\beta\rangle$, then 
the corresponding quadratic form is $\norm(\alpha x+\beta x)/\norm(I)$. 
\end{enonce}

\begin{enonce}[remark]{Example}[Levi, Delone-Faddeev] % 20s, 40s [Granville talk]
Recall that a \emph{cubic ring} is a (commutative, unital) ring structure 
on $\dZ^3$. They constructed a map 
\[
  \{\text{cubic rings}\}/\simeq \iso \generallinear_2(\dZ)\bigl\backslash\left\{\begin{array}{c}\text{integer binary cubic forms} \\ ax^3+b x^2 y+c x y^2+d y^3\end{array}\right\} ,
\]
which ``preserves discriminant.'' The discriminant of a binary cubic form is 
$b c^2-4 a c^3 - 4 b^3 d - 27 a^2 d^2 + 18 a b c d$. If 
$R=\langle 1,\alpha,\beta\rangle$ is a cubic ring, the associated binary cubic 
form is 
$\sqrt{\discriminant(\alpha x+\beta y)/\discriminant(R)} = [R:\dZ[\alpha x+\beta y]]$. 
See \cite{df64} for Delone-Faddeev's original approach. Bhargava's thesis 
\cite{b01} and Wood's thesis \cite{w09} develop things further. See 
\autoref{sec:granville-i} for more details. 
\end{enonce}

\begin{enonce}[remark]{Example} % [Wood II]
There are similar maps 
\[
  \{\text{quartic (resp.~quintic) fields}\}/\sim \to \{\text{representations\ldots}\} .
\]
These are also discriminant-preserving (in all these cases, the discriminant on 
the right is the unique polynomial invariant). See \autoref{sec:wood-ii} for 
more details. 
\end{enonce}

\begin{enonce}[remark]{Example}[Birch, Swinnerton-Dyer] % Shankar
There is a map 
\[
  \left\{\begin{array}{c}\sigma\in E(\dQ)/2:E\text{ of the } \\ \text{form }y^2=x^3+A x+B\end{array}\right\} \hookrightarrow\generallinear_2(\dQ)\backslash \left\{\begin{array}{c}\text{integer binary} \\ \text{quartic forms}\end{array}\right\} .
\]
The program \texttt{mwrank} created by Cremona uses this. In fact, this map 
factors through the $2$-Selmer group $\selmer_2(E)$. Write $E_{A,B}$ for the 
elliptic curve $y^2=x^3+A x+B$. This map sends $E_{A,B}$ to the fundamental 
invariants $I,J$ (of degree $2$, $3$ respectively) of binary quartics. See 
\autoref{sec:shankar-i} and \autoref{sec:shankar-ii} for more. 
\end{enonce}

\begin{enonce}[remark]{Example}[Cremona-Fisher-Stoll]
For $n\in \{3,4,5\}$, there is a map 
\[
  \{\sigma \in E(\dQ)/n:E\text{ of the form }E_{A,B}\} \hookrightarrow G(\dZ)\backslash V(\dZ) ,
\]
where $A,B$ are send to the fundamental invariants $I,J$ on the right-hand 
side. 
\end{enonce}

\begin{enonce}[remark]{Example} % [Gross I, Poonen IV]
There is a non-injective, but still useful map 
\[
  \left\{\begin{array}{c}\text{rational points on odd hyperelliptic} \\ \text{curves }y^2=x^{2 g+1} + a_1 x^{2 g} + \cdots + a_{2g+1} \end{array} \right\} \to \specialorthogonal_{2 g+1}(\dZ) \backslash \symmetric^2(\dZ^{2 g+1}) .
\]
This sends the $a_i$ to the invariants on the right-hand side. See 
\autoref{sec:gross-i} and \autoref{sec:poonen-iv} for more. 
\end{enonce}

\begin{enonce}[remark]{Example} % [Gross II, Ho] -- also see Jack Thorne, [Wood III]
There is a (non-injective) map 
\[
  \left\{\begin{array}{c}\text{rational points on even hyperelliptic} \\ \text{curves } z^2=a_0 x^{2 g+2} + \cdots + a_{2 g+2} y^{2 g+2} \end{array} \right\} \to \text{product of SL's} \backslash\cdots ,
\]
sending the $a_i$ to invariants. See \autoref{sec:gross-ii}, 
\autoref{sec:ho}, \autoref{sec:thorne}, and \autoref{sec:wood-iii} for more. 
\end{enonce}

Once we've found our map from arithmetic objects to orbits, the question 
becomes: how many orbits of $G(\dZ)$ on $V(\dZ)$ are there having bounded 
invariants? Gauss worked this out for binary quadratic forms. Let $h(D)$ be 
the number of $\speciallinear_2(\dZ)$-orbits of integer binary quadratic forms of 
discriminant $D$. 

\begin{theo}[Gauss, Lipschitz, Mertens]
\[
  \sum_{0<-D<X} h(D) \sim \frac{\pi}{18} X^{3/2} .
\]
\end{theo}
\begin{proof}
Gauss shows that every integer binary quadratic form $a x^2+b x y+c y^2$ 
with $D=b^2-4 a c<0$ has a unique $\speciallinear_2(\dZ)$-equivalent form 
satisfying $|b|<a\leqslant c$ or $0<b=a\leqslant c$. 

We apply geometry of numbers to the counting problem 
\[
  \sum_{0<-D<X} h(D) \sim \# \{(a,b,c):0<4 a c-b^2<X\text{ and }|b|<a\leqslant c\} .
\]
Gauss's conjecture tells us that the number of points on the right is 
asymptotically the volume of the region. Proving this is tricky! In fact, the 
conjecture is false in general. If we consider the region 
$R=\{(a,b,c):0\leqslant 4 a c-b^2<X\text{ and }|b|\leqslant a \leqslant c\}$, 
then $R\cap \dZ^3$ contains infinitely points. 
\end{proof}

One can attack the lattice-point counting problem explicitly by writing the 
count as a triple sum $\sum_{a,b,c}$, approximating the sum by a triple 
integral, and keeping track of error terms. This is how Lipschitz and Mertens 
proved the result. Working through this is a good exercise. Davenport developed 
some general principles for bounded regions. He used the principle to reprove 
Gauss' count of binary quadratics, and extended the argument to a count of 
binary cubic forms. This requires knowing explicit inequalities for the region. 

A third approach uses zeta functions (or more generally $L$-functions). 
% [Granville, Taniguchi, Thorne]
Siegel first applied this to binary quaratic forms. Goldfeld-Hoffstein, 
Shintani, and Datskovsky extended these methods. See 
\autoref{sec:granville-ii}, \autoref{sec:taniguchi}, and 
\autoref{sec:thorne} for details. 

% [Shankar]
There is a hybrid method: average over a compact continuum of fundamental 
domains. It doesn't need explicit inequalities, but still uses elementary 
geometry of numbers. The method does adapt to situations where there are more 
than one invariant. For examples, it works on all the above examples. In 
particular, it gives a count of quartic and quintic fields, boundedness of 
average rank of elliptic curves, and produces lots of hyperelliptic curves with 
few rational points. See \autoref{sec:shankar-i} and \autoref{sec:shankar-ii} 
for details. 

What if we replace $\dQ$ with another base field, like a number field or 
function field? Over a function field, one can use algebro-geometric and 
topological methods. Boundedness of average rank was proved by de Jong. 
Ellenberg has proved many other results of this type. Also, the ``hybrid 
method'' works over an arbitrary global field, as is worked out in 
\autoref{sec:wang-ii}. 





% !TEX root = sms.tex

\section{Basics of binary quadratic forms and Gauss composition}\label{sec:granville-i}
\thanksauthor{Andrew Granville}

An official Beamer version of these notes can be found online at 
\url{http://www.crm.umontreal.ca/sms/2014/pdf/AlgTalkSlides.pdf}. 





% !TEX root = sms.tex

\section{Algebraic groups, representation theory, and invariant theory}
\thanksauthor{Eyal Goren}





This lecture will consist mostly of a review of the basic terminology, as 
well as a little bit of Galois cohomology. An ``official version'' of the 
notes can be found online at 
\url{http://www.math.mcgill.ca/goren/AlgberaicGroups.SMS2014.pdf}. 





\subsection{Algebraic groups}

For us, a \emph{linear algebraic group} is a Zariski-closed subgroup of 
$\generallinear_N(\bar k)$ for some integer $N\geqslant 1$, where $k$ is a 
fixed field of characteristic zero, and $\bar k$ is the algebraic closure of 
$k$. Good general references for linear algebraic groups are the books 
\cite{b91,gw09,h75,s09}. 

\begin{example}
The main example of a linear algebraic group is $G=\generallinear_N$. It 
contains several standard subgroups:
\begin{align*}
  B &= \begin{pmatrix} \ast & \cdots & \ast \\ & \ddots & \vdots \\ & & \ast \end{pmatrix} && \text{``standard Borel''} \\
  U &= \begin{pmatrix} 1 & \cdots & \ast \\ & \ddots & \vdots \\ & & 1\end{pmatrix} && \text{``unipotent radical of $B$''} \\
  T &= \begin{pmatrix} \ast \\ & \ddots \\ & & \ast \end{pmatrix} && \text{``maximal torus''}
\end{align*}
\end{example}

\begin{example}
Let $q$ be a symmetric bilinear form corresponding to the matrix 
$(q_{i j})_{1\leqslant i,j\leqslant N}$. Let 
\[
  \specialorthogonal(q) = \{g\in \generallinear_N: g q \transpose g = q\text{ and }\det g=1\} .
\]
If $q$ is the form 
$q(x_1,\dots,x_{2n+1}) = \frac 1 2 (x_1 x_{2n+1} + x_2 x_{2n} + \cdots + x_{n+1}^2$, 
then a maximal torus consists ``anti-triangular'' matrices with 
$(t_1,\dots,t_n,1,t_n^{-1},\dots,t_1^{-1})$. 
\end{example}

An affine group $G$ is determined by its coordinate ring $\bar k[G]$. The 
group operations $m:G\times G\to G$, $i:G\to G$, $e:1\to G$ correspond via 
the Yoneda lemma to $m^\ast:\bar k[G]\to \bar k[G]\otimes \bar k[G]$, 
$i^\ast:\bar k[G] \to \bar k[G]$, $e^\ast:\bar k[G] \to \bar k$. These give 
$\bar k[G]$ the structure of a Hopf algebra. 

A \emph{homomorphism} $f:G\to H$ of algebraic is a morphism of varieties that 
respects the group structures. It corresponds to a ring 
$f^\ast:\bar k[H] \to \bar k[G]$ that respects the comultiplication.

A \emph{character} of $G$ is a homomorphism 
$f:G\to \generallinear_1 = \dG_\multiplicative$. This corresponds to a 
Hopf-algebra homomorphism $\phi:\bar k[t^{\pm 1}] \to \bar k[G]$. If we let $f$ 
be the image of $t$ under this map, then the fact that $\phi$ respects 
comultiplication tells us that $m^\ast(f) = f\otimes f$. We call such 
elements \emph{grouplike}. Let $\character^\ast(G)$ be the group of characters 
of $G$; we have seen that $\character^\ast(G)$ is in bijection with the set 
of grouplike elements of $\bar k[G]$. 

If $g\in G$, put $\inner g:G\to G$ for the action of $g$ by inner 
automorphisms, i.e.\ $\inner g(x) = g x g^{-1}$. So we have a homomorphism 
$\inner:G\to \automorphism G$. 





\subsection{Non-abelian cohomology}

Let $G$ be a topological group acting continuously on a discrete group $M$. 
Define 
\begin{align*}
  \h^0(G,M) &= M^G = \{m\in M:g m=m\text{ for all }g\in G\} \\
  \h^1(G,M) &= \{\zeta:G\to M\text{ such that }\zeta(a b) = \zeta(a)\cdot a \zeta(b)\} / \sim 
\end{align*}
where $\zeta\sim \xi$ if there exists $m\in M$ such that 
$\zeta(a) = m^{-1} \xi(a) \cdot a m$ for all $a\in G$. The set $\h^0(G,M)$ is 
naturally a group, but $\h^1(G,M)$ is only a pointed set. If 
$0 \to A \to B \to C \to 0$ is an exact sequence of $G$-groups, we get a long 
exact sequence 
\[
  0 \to A^G \to B^G \to C^G \xrightarrow\delta \h^1(G,A) \to \h^1(G,B) \to \h^1(G,C) \to \h^2(G,A) \to \cdots 
\]
with the last map only existing if $A$ is central in $B$. A good source for all 
of this is \cite{s79}. 

We define $\delta$ directly. Given $c\in C^G$, lift $c$ to $b\in B$, and let 
$\zeta=\delta(c)$ be $\zeta(g)=b^{-1} \cdot g b$. One can check that the class 
of $\zeta$ in $\h^1(G,A)$ is well-defined. 





\subsection{Forms}

Suppose $G$ is defined over $k$. A \emph{$k$-form} of $G$ is an algebraic group 
$H$ over $k$ together with an isomorphism $f:G_{\bar k} \iso H_{\bar k}$. Let 
$\Gamma=\galois(\bar k/k)$. For all $\sigma\in G$, we have 
$\sigma f:G_{\bar k}\iso H_{\bar k}$. Then 
$f^{-1}\circ \sigma f\in \automorphism_{\bar k}(G)$. An easy exercise in the 
definitions shows that this is a cocycle. In fact, we have 

\begin{theo}
There is a natural isomorphism of pointed sets 
\[
  \{\text{$k$-forms of $G$}\}/\sim \iso \h^1(\Gamma,\automorphism_{\bar k} G) .
\]
\end{theo}

If $H$ corresponds to $\zeta:G\to M$, then $H(k)=G(\bar k)^\Gamma$, where 
$\Gamma$ now acts by $\tau \cdot g = \zeta(\tau)(\tau(g))$. 

\begin{example}[compact forms]
Let $\Gamma=\galois(\dC/\dR)=\langle c\rangle$. Let 
$\theta(g) = \transpose{\bar g}^{-1}$ be the \emph{Cartan involution}. The 
cocycle $\zeta$ given by $\zeta(c)=\theta$ corresponds to the real form 
$\unitary_N$ of $\generallinear_N(\dC)$. It is defined by 
$\unitary_N(\dR) = \{g\in \generallinear_N(\dC):\theta g=g\}$. 
\end{example}

\begin{theo}
Any (connected) reductive algebraic group $G$ over $\dR$ has a unique compact 
form. 
\end{theo}

All of the groups $\generallinear_n$, $\speciallinear_n$, $\specialorthogonal_n$, 
$\symplectic_{2 n}$,\ldots are reductive. 

\begin{example}
If $G=\dG_\multiplicative$, the compact form is 
$T(\dR)=\{z\in \dC^\times:z \bar z=1\}$. We have 
\[
  T\simeq \specialorthogonal(2) = \left\{\begin{pmatrix} a & b \\ -b & a \end{pmatrix} : a^2+b^2=1\right\} ,
\]
via $\begin{pmatrix} a & b\\ -b & a \end{pmatrix} \mapsto a+b i$. 
\end{example}

If $M$ is a $\Gamma$-module, we often put $\h^i(k,M) = \h^i(\Gamma,M)$. 
Also, if $G$ is an algebraic group defined over $k$, we put 
$\h^i(k,G) = \h^i(\Gamma,G(\bar k))$. 

\begin{example}
Start with the exact sequence 
$1 \to \dG_\multiplicative \to \generallinear_N \to \projectivegenerallinear_N \to 1$. 
The long exact sequence in cohomology is 
\[
  1 \to k^\times \to \generallinear_N(k) \to \projectivegenerallinear_N(\bar k)^\Gamma \to \h^1(k,\dG_\multiplicative) \to \h^1(k,\generallinear_N) \to \h^2(k,\dG_\multiplicative) .
\]
The famous \emph{Hilbert Theorem 90} tells us that 
$\h^1(k,\generallinear_N) = 1$ for all $N\geqslant 1$, so we get 
\[
  \projectivegenerallinear_N(\bar k)^\Gamma = \projectivegenerallinear_N(k) = \generallinear_N(k)/\dG_\multiplicative(k) .
\]
Moreover, $\h^1(k,\projectivegenerallinear_N)\hookrightarrow \h^2(k,\dG_\multiplicative)$. We call 
$\brauer(k)=\h^2(k,\dG_\multiplicative)$ the \emph{Brauer group} of $k$. 
Since 
$\projectivegenerallinear_N(\bar k) = \automorphism_{\bar k\text{-}\mathsf{Alg}}(M_N(\bar k))$, 
we see that $\h^1(k,\projectivegenerallinear_N)$ classifies $k$-forms of 
$M_N(\bar k)$, i.e.~central simple algebras over $K$ of rank $N^2$. 
\end{example}





\subsection{Jordan decomposition}

Any $g\in \generallinear_N(\bar k)$ has a unique decomposition 
$g=g_\simple g_\unipotent$, where $g_\simple$ is simple (i.e.~diagonalizable), 
$g_\unipotent$ is unipotent, and 
$g_\simple g_\unipotent = g_\unipotent g_\simple$. One has 
$(g_\unipotent - 1)^N=0$. For example, in the two-dimensional case, a matrix 
$\begin{pmatrix} t_1 & u \\ & t_2\end{pmatrix}$ is already diagonalizable if 
$t_1\ne t_2$, or $u=0$. If neither of those occur, we write it as 
\[
  \begin{pmatrix} t & u \\ & t \end{pmatrix} = \begin{pmatrix} t \\ & t\end{pmatrix} \begin{pmatrix} 1 & u/t \\ & 1 \end{pmatrix} .
\]
The Jordan decomposition enjoys very strong rigidity properties. Namely, 
if $g\in G\subset \generallinear_N$, then also $g_\simple\in G$ and 
$g_\unipotent \in G$. If $f:G\to H$ is a homomorpism of algebraic groups, then 
we have $f(g_\simple) = f(g)_\simple$ and $f(g_\unipotent) = f(g)_\unipotent$. 





\subsection{Tori}

A \emph{torus} $T$ is a form of $\dG_\multiplicative^N$ for some $N$. Tori of 
rank $N$ over $k$ are classified by 
$\h^1(\Gamma,\automorphism_{\bar k}(\dG_\multiplicative^N)) = \hom(\Gamma,\generallinear_N(\dZ))/\text{conj}$. 

If $T$ is a orus, then its group of characters 
$\character^\ast(T_{\bar k}) \simeq \character^\ast(\dG_\multiplicative^N) = \hom(\dG_\multiplicative^N,\dG_\multiplicative) = \dZ^N$ 
has a continuous action of $\Gamma$, via 
$\sigma \chi(g) = \sigma(\chi(\sigma^{-1}(g)))$. Tori are completely classified 
by this action. 

\begin{theo}
The functor $\character^\ast$ induces an anti-equivalence of categories 
\[
  \{\text{tori over }k\} \iso \{\text{finite free $\dZ$-modules with continuous $\Gamma$-action}\} .
\]
\end{theo}

A linear action of a torus $T$, namely $f:T\to \generallinear_N$, is 
simultaneously diagonalizable (every element of $T$ is semi-simple). This 
follows from rigidity properties of the Jordan Decomposition. 

All maximal tori in a linear algebraic group $G$ are conjugate. The common 
dimension of these tori is called the \emph{rank} of $G$, and written 
$\rank_{\bar k}(G)$. 

\begin{example}
The standard torus of diagonal matrices $T\subset\generallinear_N$ is 
maximal, so $\rank(\generallinear_N) = N$. 
\end{example}





\subsection{Solvable groups}

An algebraic group $G$ is called \emph{solvable} if it is solvable ``in the 
usual sense.'' In other words, there exists a filtration 
$1=G_0\subset G_1\subset \cdots \subset G_l = G$ such that each 
$G_i$ a normal algebraic subgroup of $G_{i+1}$, and each 
$G_{i+1}/G_i$ is abelian. The standard Borel $B\subset \generallinear_N$ is 
solvable. In fact, $B$ is a maximal solvable subgroup. 

\begin{theo}[Kolchin-Lie]
If $G\subset \generallinear_N$ is solvable, then $G$ can be conjugated 
into the standard Borel $B\subset \generallinear_N$. 
\end{theo}
\begin{proof}
One uses the fact that if $G$ acts on a projective space, then it has a fixed 
point. The standard representation $G\to \generallinear_N$ gives an action of 
$G$ on $\dP^{N-1}$, and a fixed point for $G$ in $\dP^{N-1}$ gives a line fixed 
by the action of $G$. 
\end{proof}

\begin{theo}[Borel]
If a solvable group $G$ acts on a proper variety, then $G$ has a fixed point. 
\end{theo}

For a group $G$, let $\rad(G)$ be the maximal connected normal solvable 
subgroup of $G$, and let $\rad_\unipotent(G)$ be the maximal connected normal 
unipotent subgroup of $G$. We say that $G$ is \emph{semisimple} if 
$\rad(G)=1$, and \emph{reductive} if $\rad_\unipotent(G)=1$. Clearly 
semisimple groups are reductive. The groups 
$\speciallinear_n$, $\specialorthogonal_n$, $\symplectic_n$ are semisimple 
and $\generallinear_n$, $\generalsymplectic_{2 n}$, $\generalspin_n$ are 
reductive. 

For any $G$, the quotient $G/\rad_\unipotent(G)$ is reductive, and 
$G/\rad(G)$ is semisimple. If $G$ is reductive, then $G/Z(G)$ is 
semisimple. A \emph{Levi subgroup} of a group $G$ is a subgroup $H$ such 
that $G=H\ltimes \rad_\unipotent (G)$. Such an $H$ will be a maximal 
reductive subgroup of $G$. 

A maximal connected solvable subgroup of $G$ is called a \emph{Borel subgroup}. 
The group $B$ of upper-triangular matrices is a borel subgroup of 
$\generallinear(n)$. Every torus is contained in a Borel subgroup, and if $G$ 
is a reductive group, then all Borel subgroups of $G$ are conjugate. 

A group $G$ over $\dC$ is reductive if and only if every representation 
$\rho:G\to \generallinear_N$ is semi-simple (a direct sum of irreducible 
representations). Alternatively, the ring $\dC[\rho(G)]$ should be semi-simple. 

\begin{example}
The group $\simeq U_1 = \begin{pmatrix} 1 & \ast \\ & 1 \end{pmatrix}$ 
is not reductive. 
\end{example}





\subsection{Parabolic subgroups}

A subgroup $P$ of a connected algebraic group $G$ is called \emph{parabolic} if 
the quotient $G/P$ is projective. The basic theorem is that $P$ is parabolic if 
and only if $P$ contains a Borel subgroup $B$. So in $\generallinear_N$, a 
subgroup is parabolic if it contains a conjugate of the subgroup of 
upper-triangular matrices. 

\begin{example}
Let $k$ be an algebraically closed field. Recall that a \emph{flag} in 
$k^n$ is a collection of subspaces 
$F=(0\subsetneq F_1\subsetneq \cdots \subsetneq F_a=k^n)$. The \emph{type} of 
$F$ is $\boldsymbol d=(\dim F_i)_i$. The space of type $\boldsymbol d$ is a 
projective variety $\flag_{\boldsymbol d}$ on which $\generallinear_n$ acts 
transitively. Let $P$ be a stabilizer of a flag. Then 
$G/P\simeq \flag_{\boldsymbol d}$ and $P$ is parabolic. For example, if 
$F_i$ is the span of $\{e_1,\dots,e_{d_i}\}$, then 
\[
  P = \begin{pmatrix} 1_{h_1} \end{pmatrix} [finish]
\]
\end{example}

[finish]





% !TEX root = sms.tex

\section{Basic algebraic number theory}\label{sec:3}
\thanksauthor{Eknath Ghate}





Essentially, this lecture will try to cover two semester-long courses 
(algebraic number theory and class field theory) in an hour. Hopefully, we'll 
focus on the theory of Hilbert class fields, and later on complex 
multiplication. 





\subsection{Number fields}

A \emph{number field} $K$ is a finite field extension of $\dQ$. An important 
invariant of $K$ is its \emph{class group}, $\class_K=I_K/P_K$, where 
$I_K$ is the group of all fractional ideals in $K$ and $P_K$ is the group of 
principal fractional ideals. 

\begin{theo}
The group $\class_K$ is finite. 
\end{theo}

So we can define the \emph{class number} $h_K$ of $K$ to be the cardinality 
of $\class_K$. The class number $h_K=1$ if and only if $\cO_K$ is a principal 
ideal domain. The famous \emph{Dirichlet unit theorem} says that $\cO_K^\times$ 
is a finitely-generated abelian group, and gives a formula for 
$\rank\cO_K^\times$ in terms of the number of real and complex places of $K$. 

We would like to relate $h_K$ with quadratic forms. Consider towers of 
fields 
\[\xymatrix@=0.5cm{
  & L \\
  F \ar@{-}[ur] & & K \ar@{-}[ul] \\
  & \dQ \ar@{-}[ul]^-2 \ar@{-}[ur]_-3 \ar@{-}[uu]^-{S_3} 
}\]
where $L/F$ is an unramified (cubic) extension. Then counting the number of 
cubic fields $K$ that are nowhere ramified with $|\discriminant K|<X$ is 
equivalent to summing $\# h_3(F)$ for $|\discriminant F|<X$. The average of 
$h_3(F)$ is $\frac 4 3$ for $F$ real quadratic, and $2$ for $F$ imaginary 
quadratic. By the ``average'' we mean, for example, 
\[
  \sum_{0<|\discriminant F|<X} h_3(F) \sim \frac 4 3 X 
\]
in the case of real quadratic $F$. The point here is that averaging class 
numbers is equivalent to counting certain types of fields. 





\subsection{Starting point of class field theory}

Class field theory is, in general, the study of abelian extensions of a field 
$k$. Let $K$ be a number field, and let $H/K$ be the maximal unramified 
abelian extension of $K$. One calls $H$ the \emph{Hilbert class field} of 
$K$. It is known that $H$ is a number field. Moreover, the \emph{Artin map} 
$I_K \to \galois(H/K)$ determined by $\fp\mapsto \frobenius_\fp = (\fp,H/K)$ 
induces an isomorphism $\class_K\iso \galois(H/K)$. Recall that the 
\emph{Frobenius element} $\frobenius_\fp\in \galois(H/K)$ is characterized by 
$\frobenius_\fp(x)\equiv x^{N_{K/\dQ}(\fp)}\pmod\fP$ for all $\fP\mid \fp$ in 
$H$. 

It is easy to show that $\fp$ splits compltely in $H$ if and only if $\fp$ is 
principal. Somewhat harder is the principal ideal theorem: 

\begin{theo}
Every ideal of $K$ becomes principal in $H$. 
\end{theo}
\begin{proof}
Let $G=\galois(K^\mathrm{ur}/K)$, and let $G_1\subset G$ correspond to the 
extension $H/K$. We have a commutative diagram: 
\[\xymatrix{
  \class_H \ar[r]^-{\artin} 
    & G_1^\abelian \\
  \class_K \ar[u] \ar[r]^-{\artin}
    & G^\abelian \ar[u]_-V 
}\]
where $V$ is the \emph{transfer map}. Since $G_1$ corresponds to $H$, 
$V=0$, whence $\class_K\to \class_H$ is the trivial map. The result 
follows. 
\end{proof}

If $K=\dQ$, then $H=\dQ$. 

Assume from now on that $K=\dQ(\sqrt d)$ is an imaginary quadratic field 
($d<0$). There are nine possible values of $d<0$ for which 
$h_{\dQ(\sqrt d)}=1$, namely 
\[
  -1,-2,-3,-7,-11,19,-43,-67,-163 .
\]
It is known that $h_K\to \infty$ as $d\to -\infty$. 

\begin{enonce}[remark]{Example}
Let $K=\dQ(\sqrt{-23})$. Then $H/K$ is cyclic of degree $3$. It is known 
that $H=K(\alpha)$, where $\alpha^3-\alpha+1=0$. 
\end{enonce}





\subsection{Complex multiplication}

One might ask if for any number field $K$, there is an explicit way of 
finding $\alpha\in \overline\dQ$ such that $H=K(\alpha)$. The theory of complex 
multiplication describes how to do this explicitely whenever $K$ is 
imaginary quadratic. The main theorem is the following: 

\begin{theo}
Let $E$ be an elliptic curve with CM by $\cO_K$. Then 
$H=K(j(E))$. 
\end{theo}

Recall that $E$ has \emph{complex multiplication} by $\cO_K$ if 
$\End E \simeq \cO_K$. If $E+y^2=4 x^3-g_2 X-g_3$, then its 
\emph{$j$-invariant} is given by 
\[
  j(E) = 1728 \frac{g_2^3}{g_2^3 - g_3^2} .
\]
Alternatively, if $E=\dC/\langle 1,\tau\rangle$, then 
$j(E) = \frac 1 q + 744 + 196884 q^2 + \cdots$, where $q=e^{2\pi i \tau}$. 

The rest of this lecture will be a sketch of a proof of this theorem. 





\subsection{Elliptic curves from fractional ideals}

Let $\cE_\dC(K)$ be the set of isomorphism classes of elliptic curves over 
$\dC$ with complex multiplication by $\cO_K$. There is a bijection 
$\class_K \iso \cE_\dC(K)$. Given $\fa\subset \cO_K$, we have a canonical 
embedding $\fa\hookrightarrow \dC$. Send $[\fa]$ to the elliptic curve 
$\dC/\fa$. This gives us a simply transitive action of $\cO_K$ on 
$\cE_\dC(K)$ via $[\fa]\cdot (\dC/\fb) = \dC/(\fa^{-1} \fb)$. 





\subsection{Fields of definition}

It is known that CM elliptic curves have rational models. In other words, 
the natural map $\cE_{\overline\dQ}(K) \to \cE_\dC(K)$ induced by an 
embedding $\overline\dQ\hookrightarrow\dC$ is a bijection. We know this 
because $E$ is \emph{always} defined over $\dQ(j(E))$. But when $E$ is CM, 
$j(E)\in \overline\dQ$, so $E$ is defined over $\overline\dQ$. Indeed, if 
$\sigma\in \automorphism\dC$, then $j(E^\sigma)$ also has CM by $K$. So the 
orbit of $E$ under $\automorphism(\dC)$ lies in $\cE_\dC(K)$, a finite set. 
This tells us that $\cE_{\overline\dQ}(K) \twoheadrightarrow \cE_\dC(K)$. We 
leave injectivity as an exercise. 

Because of this, we will write $\cE(K)$ instead of $\cE_{\overline\dQ}(K)$ or 
$\cE_\dC(K)$. 





\subsection{Towards \texorpdfstring{$H$}{H}}

Fix $E\in \cE(K)$. For each $\sigma\in \galois(\overline\dQ/K)$, there is a 
unique $[\fa]\in \class_K$ such that $E^\sigma = [\fa]\cdot E$. We define a map 
$F:\galois(\overline\dQ/K) \to \class_K$ by $F(\sigma) = [\fa]$. 

\begin{prop}
1. $F$ does not depend on the choice of $E$. 

2. $F$ is a homomorphism. 
\end{prop}
\begin{proof}
1. This is subtle. 

2. Let $\sigma,\tau\in \galois(\overline\dQ/K)$. Then 
\begin{align*}
  F(\sigma\tau) E 
    &\simeq E^{\sigma\tau} \\
    &= (E^\sigma)^\tau && \text{part 1}\\
    &= (F(\sigma) E)^\tau \\
    &= F(\tau)(F(\sigma) E) \\
    &= (F(\sigma)F(\tau)) E && \text{$\class_K$ is abelian}
\end{align*}
\end{proof}

Let $L$ be the fixed field of $\ker(F)$. We claim that 
$L=K(j(E))$. Indeed, 
\begin{align*}
  \galois(\overline\dQ/L)
    &=\{\sigma\in \galois(\overline\dQ/K):E^\sigma \simeq F(\sigma) E \simeq E\} \\
    &= \{\sigma:j(E)^\sigma = j(E)\} .
\end{align*}

Note that $F$ is an injection $\galois(L/K)\hookrightarrow \class_K$. So 
$L/K$ is abelian. We will show that it is unramified of the right degree. 

Let $\fm$ be the conductor of $L/K$. It is the greatest common divisor of all 
$\fm\subset \cO_K$ such that $K_{\fm,1}\subset \ker(\artin_{L/K})$, where 
$K_{\fm,1} = \langle (\alpha):\alpha\simeq 1\pmod\fm\rangle$. One checks that 
the composite 
\[\xymatrix{
  I_K^\fm \ar[r]^-{\artin} 
    & \galois(L/K) \ar[r]^-F 
    & \class_K 
}\]
is the ``identity map'' $\fa\mapsto [\fa]$. It follows that $F$ is surjective. 
So $F:\galois(L/K)\iso \class_K$, and $L=K(j(E))$. 

If $F(((\alpha),L/K)) = 1$ then $(\alpha)\in I_K^\fm$ is principal. Indeed, it 
suffices to show $((\alpha),L/K)=1$, and this follows from the injectivity of $F$. 
Since $\alpha$ was arbitrary, $\fm=1$, so $L/K$ is unramified. The rest is a 
simple dimension argument. 

\begin{theo}
If $E$ is CM, then $j(E)$ is an algebraic integer. 
\end{theo}

As a corollary, one has the surprizing fact that $e^{\pi\sqrt{163}}$ is 
very close to an integer. The main theorem of complex multiplication has an 
analogue for ray class fields. One gets 
$\operatorname{RCF}(\fm) = K(j(E),h(E[\fm]))$, where $\fm\subset \cO_K$ and $h$ 
is the Weber function. 





% !TEX root = sms.tex

\section{Geometric properties of curves}
\thanksauthor{Henri Darmon}




Here we treat those properties of curves which hold over an arbitrary field. 
Later on, in \autoref{sec:7}, we will specialize to number fields. 





\subsection{Motivation}

Throughout, $k$ is a field. For simplicity we assume $k$ has characteristic 
zero. 

\begin{defi}
A \emph{curve} over a field $k$ is a smooth geometrically connected variety of 
dimension one over $k$. 
\end{defi}

Concretely, we think of equations like 
\begin{align*}
  1 &= x^2 - D y^2 \\
  y^2 &= x^3 + a x+b \\
  z^n &= x^n + y^n .
\end{align*}
The difference between the general definition and these concrete 
examples should be seen as analogous to the difference between the notion of 
an ``abstract vector space'' and concrete examples $\dR^n$. 

The key tool for passing from an abstract curve to a concrete representation is 
the \emph{Riemann-Roch Theorem}. Let $X$ be a proper curve. Zariski-open 
subsets of $X$ are of the form $U=X\smallsetminus \{p_1,\dots,p_s\}$, where 
the $p_i\in X(\bar k)$ and $\{p_1,\dots,p_s\}$ is stable under the action of 
$G_k=\galois(\bar k/k)$. We have a sheaf $\sO=\sO_X$ of ``regular functions'' 
on $X$. For $U\subset X$, the ring $\sO(U)$ consists of all regular functions 
$U\to \dA^1$. 

Our goal is to understand $\sO(U)$ as a ring. Ideally, we would like to 
write $\sO(U)=k[f_1,\dots,f_n]/(p_1,\dots,p_m)$. In other words, we want 
sections $f_1,\dots,f_n\in \sO(U)$ such that $(f_1,\dots,f_n)$ induces an 
embedding $U\hookrightarrow \dA^n$. We certainly can't do this with $U=X$, 
because $\sO(X)=k$. If $k=\dC$, this fact is known as Liouville's theorem, but 
it holds for arbitrary $k$. 





\subsection{Crude form of Riemann-Roch}

We assume there is a point $\infty\in X(k)$. Put 
$U=X\smallsetminus \{\infty\}$. Define $\sO(U;n\infty)$ to be the set of 
functions $f\in \sO(U)$ such that $v_\infty(f)\geqslant -n$. This gives us a 
filtration $\sO(X)\subset \sO(U;\infty)\subset \sO(U;2\infty)\subset \cdots$ 
and $\bigcup \sO(U,n\infty) = \sO(U)$. Moreover, 
each successive quotient is at most one-dimensional, so 
$\dim \sO(U,n\infty) \leqslant n+1$. The Riemann-Roch Theorem gives a lower 
bound for $\dim \sO(U,n\infty)$. 

\begin{theo}[Riemann-Roch; crude form]
Let $X$ be a proper curve over $k$ and $\infty\in X(k)$. Then there is an 
integer $g\geqslant 0$, depending only on $X$, such that 
\[
  \dim \sO(U;n\infty) \geqslant n+1-g 
\]
with equality if $n\gg 0$. 
\end{theo}
\begin{proof}[Idea of proof]
Choose a local parameter $t$ at $\infty$. Define the \emph{principal part} 
$\principalpart_\infty:\sO(U;n\infty) \to t^{-n} k[t] / k[t]$ in the obvious 
way. What are the obstructions to producing $f$ with given principal part at 
$\infty$? The only obstruction comes from the Residue Theorem stated below. As 
a corollary, if $\omega\in \Omega^1(X)$ is a global regular differential, and 
if $f\in \sO(U)$, then $\residue_\infty(f\omega) = 0$. So global regular 
differentials provide obstructions to constructing $f$. Define 
$\residue_\infty:t^{-n} k[t]/k[t] \to \Omega^1(X)^\vee$ by 
$\residue_\infty(f)(\omega) = \residue_\infty(f\omega)$. We have a 5-term 
sequence 
\begin{equation*}
  0 \to k \to \sO(U;n\infty) \xrightarrow{\principalpart_\infty} t^{-n} k[t]/t \xrightarrow{\residue_\infty} \Omega^1(X)^\vee \to \Omega^1(X;-n\infty)^\vee \to 0.
\end{equation*}
The last term needs explanation. Define $\Omega^1(X;-n\infty)$ to be the set of 
$\omega\in \Omega^1(X)$ such that $v_\infty(\omega) \geqslant n$. The inclusion 
$\Omega^1(X;-n\infty)\hookrightarrow \Omega^1(X)$ induces the surjection in the 
sequence. 

One can check that the sequence is exact. For $n\gg 0$, 
$\Omega^1(X;-n\infty)=0$. We have in fact proved the ``more precise form'' 
below. 
\end{proof}

\begin{theo}[residue theorem]
If $\omega$ is a meromorphic differential on $X$, then 
$\sum_{x\in X} \residue_x(\omega) = 0$. 
\end{theo}

Let's recall what this means At each $x\in X(\bar k)$, choose a 
uniformizing parameter $t$ at $x$. One can write locally 
$\omega = (a_{-m} t^{-m} + \cdots) dt$; put $\residue_x(\omega) = a_{-1}$. 
Surprizingly, this does not depend on our choice of $t$. 

\begin{theo}[Riemann-Roch; more precise form]
Let $g=\dim \Omega^1(X)$. Then 
$\dim \sO(U;n\infty) - \dim\Omega^1(X;-n\infty) = n+1 - g$. 
\end{theo}

We call the integer $g$ the \emph{genus} of $X$. 





\subsection{Some vocabulary}

A \emph{divisor} of $X$ is a formal finite linear combination of points in 
$X(\bar k)$ with integer coefficients. So a typical divisor looks like 
\[
  D = \sum_{x\in X(\bar k)} n_x\cdot x ,
\]
where each $n_x\in \dZ$ and $n_x=0$ for all but finitely many $x$. Write 
$\divisors(X_{\bar k})$ for the (abelian group) of divisors on $X$, and 
write $\divisors(X)=\divisors(X_{\bar k})^{G_k}$. 

Recall the field of \emph{rational functions} on $X$ is 
$k(X)=\varinjlim_U \sO(U)$. Define $\divisor:k(X)^\times \to \divisors(X)$ 
by 
\[
  \divisor(f) = \sum_{x\in X(\bar k)} v_x(f)\cdot x .
\]
Divisors of the form $\divisor(f)$ are called \emph{principal divisors}. 
If $D_1,D_2$ are divisors, write $D_1\geqslant D_2$ if 
$n_x(D_1)\geqslant n_x(D_2)$ for all $x\in X(\bar k)$. For an arbitrary 
divisor $D$, define 
\[
  \sL(D) = \{f\in k(X):\divisor(f)\geqslant - D\} .
\]
Our space $\sO(U;n\infty)$ earlier is just $\sL(n\infty)$. Define the space of 
\emph{meromorphic divisors} by 
$\Omega_\meromorphic^1(X)=\varinjlim_U \Omega^1(U)$; this is a one-dimensional 
$k(X)$-vector space. Choose nonzero $\omega\in \Omega_\meromorphic^1(X)$. We 
call $K=\divisor(\omega)$ the \emph{canonical divisor}. It does not depend on 
the choice of $\omega$. 

It turns out that $\Omega^1(X)$ can be identified with $\sL(K)$. Indeed, 
we have a natural isomorphism $\sL(K) \iso \Omega^1(X)$ defined by 
$f\mapsto f\omega$. 

Finally, if $D$ is a divisor, one often writes $\ell(D)$ for 
$\dim \sL(D)$. We can now state the final form of the Riemann-Roch Theorem. 

\begin{theo}[Riemann-Roch]
For all divisors $D\in \divisors(X)$, we have 
$\ell(D)-\ell(K-D) = \deg D + 1-g$. 
\end{theo}





\subsection{Consequences of Riemann-Roch}

We could set $D=0$. Then the theorem specializes to 
$1-\ell(K)=1-g$, so $\ell(K)=g$. Since $\ell(K)=\dim \Omega^1(X)$, this 
recovers our definition of the genus of $X$. 

We could set $D=K$. Then the theorem tells us that 
$\ell(K)-1 = \deg K+1-g$. We already know $\ell(K)=g$, so 
$g-1=\deg K+1-g$, so $\deg K=2g - 2$. In other words, the number of zeros of 
a non-zero $\omega\in \Omega^1(X)$ is $2 g-2$. 

\begin{enonce}[remark]{Example}[$g=0$]
If $X(k)\ne \varnothing$, choose a point $\infty\in X(k)$. Then Riemann-Roch 
says $\ell(n\infty) = n+1$. In particular, $\sL(\infty) = k\oplus k t$, 
$\sL(2\infty) = k\oplus k t\oplus k t^2$, and in egneral 
$\sL(n\infty) = k\oplus \cdots \oplus k t^n$. So 
$\sO(U)=k[t]$, whence $U\simeq \dA^1$ and $X\simeq \dP^1$. 
Even if $X(k)=\varnothing$, $X$ has a rational divisor of degree $2$, namely 
$-K$. It must of be of the form $-K=p+p'$ for $p,p'$ conjugates in a quadratic 
extension of $k(X)$. We know that $\sL(-K)=k\oplus k u\oplus k v$, and that 
$\sL(-2 K)$ is spanned by $\{1,u,v,u v,v^2,u^2\}$. But 
$\ell(-2 K)=5$, so we must have a linear relation 
$a+b u+c v+d u v+e v^2 + f u^2 + 0$. In particular, all curves of genus zero 
are conics. 
\end{enonce}

\begin{enonce}[remark]{Example}[elliptic curves]
A curve of genus $1$ with chosen $\infty\in X(k)$ is called an \emph{elliptic 
curve}. Since $\dim \Omega^1(X)=1$, there is (up to homothety) a unique 
non-vanishing regular divisor $\omega\in \Omega^1(X)$. This gives an 
isomorphism $\sO_X\iso \Omega_X^1$. Apply Riemann-Roch to the spaces 
$\sL(n\cdot \infty)$. We get 
\begin{center}
\begin{tabular}{c|cl}
$n$ & $\ell(n\cdot \infty)$ & generators of $\sL(n\cdot\infty)$ \\ \hline
1 & 1 & $\{1\}$ \\
2 & 2 & $\{1,x\}$ \\
3 & 3 & $\{1,x,y\}$ \\
4 & 4 & $\{1,x,y,x^2\}$ \\
5 & 5 & $\{1,x,y,x^2,x y\}$ \\
6 & 6 & $\{1,x,y,x^2,x y,y^2, x^3\}$
\end{tabular}
\end{center}
Since $\ell(6\cdot \infty)=6$, we must have $y^2-x^3\in \sL(5\infty)$, 
so we get a relation 
\[
  y^2 + a_1 x y + a_3 y = x^3 + a_2 x^2 + a_4 x + a_6 .
\]
Now elementary algebra gets rid of $a_1$ if $2$ is invertible, and reduces 
further to an equation 
\[
  y^2 = x^3 + a x+b 
\]
if $3$ is also invertible. 
\end{enonce}

If $X$ is a curve of genus one with $X(k)=\varnothing$, the simplest case is 
when $X$ has a rational divisor of degree $2$. This is true for elements of 
$\selmer_2(E)$. We have 

\begin{theo}
$X$ has an equation of the form 
\[
  y^2 = a x^4 + b x^3 + c x^2 + d x+e .
\]
\end{theo}
\begin{proof}
Our rational divisor of degree $2$ is of the form $p+p'$. We have 
\begin{align*}
  \sL(p+p') &= k\oplus k x \\
  \sL(2 p+2 p') &= k\oplus k x\oplus k x^2\oplus k y \\
  \sL(4 p+4 p') &= \ldots 
\end{align*}
The rest is easy. 
\end{proof}





% !TEX root = sms.tex

\section{Basic analytic number theory}\label{sec:granville-ii}
\thanksauthor{Andrew Granville}





We are interested in asymptotics of $\sum_{n\leqslant x} a_n$ for various 
natural arithmetic sequences $\{a_n\}$. There are two main techniques: one is 
geometric, the other uses $L$-functions. 





\subsection{Geometric techniques}

\begin{enonce}[remark]{Example}
Consider the constant sequence $a_n=1$. We have 
$\lfloor x\rfloor = \sum_{1\leqslant n \leqslant x} 1$. More precisely, 
$\sum_{n\leqslant x} 1 = x+O(1)$. 
\end{enonce}

\begin{enonce}[remark]{Example}
We count lattice points inside a disk, i.e.~look at the asymptotics of 
$\#\{(x,y)\in \dZ^2:x^2+y^2\leqslant T\}$. This is approximately the area of 
$\{x^2+y^2\leqslant T\}$. The error comes from the boundary of the region 
$\{x^2+y^2\leqslant T\}$. It will be a bounded multiple of the radious 
$\sqrt T$. So 
\[
  \#\{(x,y)\in \dZ^2:x^2+y^2\leqslant T\} = \pi T + O(T^{1/2}) . 
\]
We might hope for an error term of the form $O(T^{1/2-\epsilon})$ for some 
$\epsilon>0$. 
\end{enonce}

Questions like this become quite subtle if we are looking at intersections 
$\Lambda\cap r\Omega$, where $\Lambda\subset \dR^n$ is a lattice and 
$\Omega\subset \dR^n$ is a bounded region. If $\partial\Omega$ is smooth, 
things work as expected. If, however $\Omega$ has a ``fractal-like'' boundary, 
one has to be very careful. 

\begin{enonce}[remark]{Example}
Let's count the number of lattice points inside an expanding triangle: 
\[
  \#\{(x,y)\in \dZ^2:x,y>0\text{ and }y+\alpha x\leqslant T\} .
\]
The area of the triangle 
$T\Delta = \{x,y>0\text{ and }y+\alpha x\leqslant T\}$ is 
$\frac{1}{2\alpha}T^2$. We get 
\[
  \#(T\Delta\cap \dZ^2) = \frac{1}{2\alpha} T^2 + O(T) .
\]
As with the circle, we could hope for an error term of the form 
$O(T^{1-\epsilon})$ for $\epsilon>0$. By considering $\alpha=-1$, we can see 
that this is not possible. What if $\alpha$ is not rational? If we 
consider 
\[
  \alpha = 1+N^{-1} + 2^{-N} + 2^{2^N} + \cdots
\]
then $\alpha$ is ``almost rational,'' which leads to a ``full error term'' 
$O(T)$ for $\# (T\Delta\cap \dZ^2)$. From this we see that diophantine 
approximation of transcendental numbers is relevant to these sorts of 
problems. 
\end{enonce}

\begin{enonce}[remark]{Example}
Let $d$ be the \emph{divisor function} defined by 
\[
  d(n) = \#\{(a,b)\in \dN:a b=n\} .
\]
We are interested in $\sum_{n\leqslant T} d(n)$. We can rewrite this as 
\begin{align*}
  \sum_{n\leqslant T} d(n) &= \sum_{n\leqslant T} \sum_{\substack{x,y\geqslant 1 \\ x y = n}} 1 = \sum_{\substack{x,y\geqslant 1 \\ x y\leqslant T}} 1 .
\end{align*}
So we're trying to count lattice points in $T\Omega$, where 
\[
  \Omega=\{(x,y)\in \dR^2:x,y>0\text{ and }x y\leqslant 1\} .
\]
But $\Omega$ has infinite area and a pathological boundary. Instead, let's 
count lattice points in 
\[
  \{(x,y)\in \dR^2:x,y>\frac 1 2\text{ and }x y\leqslant T\} .
\]
This is bounded, so we're in good shape. It's area is 
\begin{align*}
  \int_{1/2}^{2 T} \frac{T}{x}\, d x 
    &= T \log(4 T) \\
    &= T\log T+O(T) .
\end{align*}
\end{enonce}

Dirichlet has a beautiful trick for the asymptotics of the divisor function. We 
have 
\begin{align*}
  \sum_{\substack{a,b\geqslant 1 \\ a b\leqslant T}} 1 
    &= \sum_{T\geqslant a\geqslant 1} \sum_{1\leqslant b\leqslant T/a} 1 \\
    &= \sum_{1\leqslant a\leqslant T}\left(\frac T a+O(1)\right) \\
    &= T a\sum_{1\leqslant a\leqslant T} \frac 1 a + O(T) 
\end{align*}
We know that $\sum_{n\leqslant N} \frac 1 n = \log N+\gamma+O(N^{-1})$. This 
also gives us $T\log T+O(T)$. We would like a power-saving error term. 
Dirichlet's insight was that when summing pairs $(a,b)$ with $a b=n$, we can 
restrict to those with $a\leqslant b$. Write $m=\min\{a,b\}$ and 
$n=\max\{a,b\}$. We have 
\begin{align*}
  \sum_{\substack{a b\leqslant T \\ a,b\geqslant 1}} 1
    &= \sum_{1\leqslant m \leqslant \sqrt T} \sum_{m<n \leqslant \frac{T}{m}} 1+\sqrt T ,
\end{align*}
yielding a sum 
\[
  \sum_{n\leqslant T} d(n) = 2 T \sum_{m=1}^{\lfloor \sqrt T\rfloor} \frac 1 m - T+O(\sqrt T) = T\log T+(2\gamma-1)T + O(T^{1/2}) .
\]





\subsection{\texorpdfstring{$L$}{L}-functions}

\begin{enonce}[remark]{Example}
Consider the identity 
\[
  \int_0^1 e^{2\pi i n t}\, dt = \begin{cases} 0 & n=0 \\ 0 & n\ne 0\end{cases} 
\]
This is a characteristic function for the integer $n=0$. Suppose we wanted to 
attack Goldbach's conjecture, which says that each $2 N$ can be written as 
$p+q$ for primes $p+q$. We could look at 
\begin{align*}
  \sum_{p,q\text{ prime}} \begin{cases}1 & p+q-2 N = 0 \\ 0 & \ne 0\end{cases} 
    &= \sum_{p,q} \int_0^1 e^{2i\pi(p+q-2 N) t}\, dt \\
    &= \int_0^1 e^{4 i \pi N t}\left(\sum_p e^{2i \pi t}\right)^2\, dt .
\end{align*}
This is essentially the Hardy-Littlewood circle method. 
\end{enonce}

It would be nice if instead of just characteristic functions of points, we 
could get characteristic functions of more general regions via integrals. One 
has the \emph{Perron formula} 
\[
  \frac{1}{2i\pi} \int_{c-i\infty}^{c+i\infty} e^{s y}\, \frac{ds}{s} = \begin{cases} 1 & y>0 \\ \frac 1 2 & y=0 \\ 0 & y<0 \end{cases} 
\]
for $c>0$. This is essentially Cauchy's residue theorem. One integrates over 
increasingly large squares with $\{\Re z=c\}$ as their right side. 

Suppose $e^y=w$. Then we are integrating $w^s/s$, and the characteristic 
function is for $w>1$. Write 
\begin{align*}
  \sum_{n\leqslant x} a_n 
    &= \sum_{n\geqslant 1} a_n \begin{cases} 1 & x/n>1 \\ 0 & x/n<1 \end{cases} \\
    &= \sum_{n\geqslant 1} a_n \frac{1}{2i\pi} \int_{c-i\infty}^{c+i\infty} \left(\frac x n\right)^s\, \frac{ds}{s} \\
    &= \frac{1}{2i\pi} \int_{c-i\infty}^{c+i\infty} A(s) x^s\, \frac{ds}{s} 
\end{align*}
for $\Re c\gg 0$, where $A(s)=\sum_{n\geqslant 1} \frac{a_n}{n^s}$. 
Let's apply this approach to $\lfloor x\rfloor = \sum_{n\leqslant x} 1$. We get 
\[
  \lfloor x\rfloor = \frac{1}{2i\pi} \int_{2-i\infty}^{2+i\infty} \zeta(s) x^s\, \frac{ds}{s} .
\]
The function $\zeta$ is analytic except at $s=1$, where it has a pole of 
order $1$ with residue $1$. Thus 
\[
  \lfloor x\rfloor = x + \zeta(0) + \text{error} = x-\frac 1 2 + \text{error} .
\]
Our error term is the ``sawtooth function'' $x-\frac 1 2 - \lfloor x\rfloor$. 

Now we consider the more complicated sum $\sum_{n\leqslant x} d(n)$. Our 
corresponding Dirichlet series is 
\[
  D(s) = \sum_{n\geqslant 1} \frac{d(n)}{n^s} = \sum_{n\geqslant 1} n^{-s} \sum_{\substack{a b=n \\ a,b\geqslant 1}} 1 = \sum_{a,b\geqslant 1}\frac{1}{(a b)^s} = \zeta(s)^2 .
\]
Near $s=1$, we have $\zeta(s)=(s-1)^{-1} + \gamma + c(s-1) + \cdots$. So 
\begin{align*}
  \zeta(s)^2 \frac{x^s}{s} 
    &= \left(\frac{1}{s-1} + \gamma+ c_1(s-1)\right)^2 \cdot x \cdot \left(1+(s-1)\log x + \ldots\right)\cdots \\
    &= x\left(\frac{1}{(s-1)^2} + \frac{1}{s-1}(\log x+2\gamma-1) + \cdots\right) .
\end{align*}
This recovers Dirichlet's formula for $\sum_{n\leqslant x} d(n)$. 

Finally, let's review Riemann's original application of the zeta function. 
From the Euler product $\zeta(s) = \prod (1-p^{-s})^{-1}$ valid for 
$\Re s\geqslant 1$, we compute 
\[
  -\frac{\zeta'(s)}{\zeta(s)} = \sum_{\substack{p\text{ prime} \\ m\geqslant 1}} \frac{\log p}{p^{m s}} .
\]
It follows that 
\[
  \sum_{p^m\leqslant x} \log p = \frac{1}{2i\pi} \int_{2-i\infty}^{2+i\infty} -\frac{\zeta'(s)}{\zeta(s)} \frac{x^s}{s}\, ds .
\]
The poles of $\zeta'$ are easy to analyze. The other poles of the integrand 
come from zeros of $\zeta$. We get 
\[
  \sum_{p^m\leqslant x} \log p = x-\frac{\zeta'(0)}{\zeta(0)} - \sum_{\zeta(\rho)=0} \frac{x^\rho}{\rho} .
\]

It is trickier to count lattice points inside a circle using zeta functions. 
We have 
\[
  \sum_{a^2+b^2 \leqslant T} 1 = \sum_{n\leqslant T} R(n) ,
\]
where $R(n)=\#\{(a,b)\in \dZ^2:n=a^2+b^2\}=4 r(n)$. The Dirichlet series has 
an Euler product 
\[
  \left(1-\frac{1}{2^s}\right)^{-1} \prod_{p\equiv 1\pmod 4} \left(1-\frac{1}{p^s}\right)^{-2} \prod_{p\equiv 3\pmod 4} \left(1-\frac{1}{p^{2 s}}\right)^{-1} 
\]
If $\chi=\bigl(\frac{-4}{\cdot}\bigr)$, then 
\[
  L(s,\chi) = \prod_{p\equiv 1\pmod 4} \left(1-\frac{1}{p^s}\right)^{-1} \prod_{p\equiv 3\pmod 4} \left(1+\frac{1}{p^s}\right)^{-1} .
\]
It follows that our Dirichlet series is $\zeta(s) L(s,\chi)$. 





\subsection{Sieving}

Let's try to count square-free integers. We have 
\begin{align*}
  \sum_{\substack{n\leqslant x \\ n\text{ squarefree}}} 1 
    &= \lfloor x\rfloor - \sum_p \#\{n\leqslant x:p^2\mid n\} + \sum_{p,q} \#\{n\leqslant x:p^2 q^2\mid n\} \\
    &= \lfloor x\rfloor - \sum_p \left\lfloor\frac{x}{p^2}\right\rfloor + \sum_{p,q} \left\lfloor\frac{x}{p^2 q^2}\right\rfloor \\
    &= x+O(1) - \sum_{p\leqslant x} \left(\frac{x}{p^2} + O(1)\right) + \sum_{p,q} \left(\frac{x}{p^2 q^2}+O(1)\right) \\
    &= x\prod_p \left(1-\frac{1}{p^2}\right) + \text{error} \\
    &= \frac{6}{\pi^2} x + \text{error}
\end{align*}
A less risky approach (one that does not have as many error terms) is to write 
\begin{align*}
  \sum_{\substack{n\leqslant x \\ n\text{ squarefree}}} 1
    &= \sum_{\substack{n\leqslant x \\ p^2\mid n\text{ for all }p\leqslant y}} 1 + \text{error} ,
\end{align*}
where $y=\log x$ and 
\[
  |\text{error}| \leqslant \sum_{y<p<\sqrt x} \# \{n\leqslant x:p^2\mid n\} .
\]

A similar problem is 
\begin{align*}
  \#\{n\leqslant x:n^2+1\text{ is squarefree}\} = \sum_{\substack{n\leqslant x \\ p^2\nmid n^2+1\text{ for }p\leqslant y}} 1 + O\left(\sum_{y<p<x} \# \{n\leqslant x:p^2\mid n^2+1\}\right) .
\end{align*}
The summand inside the big-$O$ is bounded above by 
$2\left(\frac{x}{p^2} + 1\right)$. 

\begin{enonce}{Conjecture}
For all $\varepsilon>0$, there is a constant $\kappa_\varepsilon$ such that 
whenever $a+b=c$ with $(a,b)=1$, then 
\[
  \prod_{p\mid a b c} p > \kappa_\varepsilon \max\{|a|,|b|\}^{1-\varepsilon} = \kappa H(a,b)^{1-\varepsilon} .
\]
\end{enonce}
A remarkable article of Noam Elkies relates the $abc$-conjecture to 
Belyi maps. The $abc$ conjecture implies that if $F(x,y)\in \dZ[x,y]$ is a 
homogeneous polynomial, then 
\[
  \prod_{p\mid F(a,b)} p > \kappa H(a,b)^{\deg F-2-\varepsilon}
\]
Given $f$, set $F(x,y) = y^{d+1} f(x/y)$. Then a consequence of $abc$ is 
\[
  \prod_{p\mid f(n)} p > \kappa_\varepsilon |n|^{\deg f+1-\varepsilon} .
\]

See \cite{e91} for details. 

We could also consider $4 a^3+ 27 b^2$ divisible by $p^2$. 





% !TEX root = sms.tex

\section{Diophantine properties of curves}\label{sec:7}
\thanksauthor{Henri Darmon}





Let $X$ be a curve over a number field $k$. The main diophantine questions we 
are interested in are: 
\begin{itemize}
  \item What is $X(k)$?
  \item Is $X(k)$ finite?
  \item What is $\# X(k)$ for ``typical'' $X$?
\end{itemize}
We would like to phrase questions in a way that allow for us to talk about 
integral points on a curve -- e.g.~equations like the Pell equation 
$x^2-d y^2=1$. If $X$ is projective, then $X(\dZ)=X(\dQ)$, so there is no 
limitation in studying rational points. More generally, if $X/k$ is 
projective, then $X(\cO_k)=X(k)$. If $X$ is affine, we can choose an 
embedding $X\hookrightarrow \dA^n$ and put $X(\dZ)=X(\dQ)\cap \dA^n(\dZ)$. 
With this definition $X(\dZ)$ depepnds on the chosen equations for $X$, but 
hopefully the ``main features'' of $X(\dZ)$ do not depend on this embedding. 

So our question is: if $k$ is a number field, $S$ is a finite set of places of 
$S$ and $X/\cO_{k,S}$ is a curve, what is $X(\cO_{k,S})$? 

To $X$ we can attach some numerical invariants. The curve $X$ will be of the 
form $\widetilde X\smallsetminus \{x_1,\dots,x_s\}$ where $\widetilde X$ is 
proper. For $g$ the genus of $\widetilde X$, we define the 
\emph{Euler characteristic} of $X$ by 
\[
  \chi(X) = 2 - 2 g - s\in \dZ .
\]
A lot of the diophantine behavior of $X$ is governed by $\chi(X)$. The 
fundamental trichotomy comes from whether $\chi>0$, $\chi<0$, or 
$\chi=0$. 





\subsection{Positive Euler characteristic}

\begin{theo}
If $\chi(X)>0$, then $X(\cO_{k,S})$ is either empty or infinite. 
\end{theo}
\begin{proof}
If $X$ is affine, then $g=0$ and $s=1$, so $X=\dA^1$, whence 
$X(\cO_{k,S}) = \cO_{k,S}$. If $X$ is projective, then $g=s=0$. Then $X$ 
either has a rational point, in which case it is $\dP^1$, or $X$ is a conic 
with $X(k)=\varnothing$. 
\end{proof}

\begin{theo}[Hasse-Minkowski]
Let $X$ be a curve over $\dQ$ of genus zero. Then $X(\dQ)\ne\varnothing$ if and 
only if $X(\dQ_p)\ne\varnothing$ for all $p$ and $X(\dR)\ne \varnothing$. 
\end{theo}





\subsection{Negative Euler characteristic}

\begin{theo}[Siegel,Faltings]
If $\chi(X)<0$, then $\# X(\cO_{k,S})<\infty$. 
\end{theo}

The affine case was proven by Siegel in 1932. The prototypical examples are: 
\begin{center}
\begin{tabular}{cc|c}
$g$ & $s$ & $X$ \\ \hline
0 & 3 & $\dP^1\smallsetminus \{0,1,\infty\}$ \\
1 & 1 & $E\smallsetminus \{\infty\}$ 
\end{tabular}
\end{center}
The coordinate ring of $\cO_{\dP^1\smallsetminus \{0,1,\infty\}}$ is 
$\dZ[x,\frac 1 x,\frac{1}{1-x}]$, and 
\[
  (\dP^1\smallsetminus \{0,1,\infty\})(\cO_{k,S}) = \{v\in \cO_{k,S}^\times:v-1\in \cO_{k,S}^\times\} .
\]
This is an \emph{$S$-unit equation}, and Siegel proved that such equations have 
only finitely many solutions. 

If $g=1$, $s=1$, then the result amounts to showing that elliptic curves have 
only finitely many integral points. Since integral points are torsion, this 
follows from the Mordell-Weil Theorem. 

In the projective case, $g>1$, and the finiteness result is Faltings' Theorem, 
originally known as the Mordell Conjecture. 





\subsection{Zero Euler characteristic}

This is the most interesting case. 

\begin{theo}[Dirichlet,Mordell-Weil]
If $X(\cO_{k,S})$ is non-empty, then it is naturally an abelian group, and as 
such is finitely-generated. 
\end{theo}

In the affine case $g=0,s=2$, if $X(k)\ne\varnothing$, then (for the sake of 
illustration) $X = \dP^1\smallsetminus \{0,\infty\} = \dG_\multiplicative$, so 
$X(\cO_{k,S}) = \cO_{k,S}^\times$. The famous \emph{Dirichlet Unit Theorem} 
tells us this group is finitely generated. 

In the projective case $g=1,s=0$, $X$ is an elliiptic curve which we will 
denote by $E$. The Mordell-Weil Theorem says that $E(k)$ is finitely 
generated. 





\subsection{Ranks}

In the affine case, the rank of $\cO_{k,S}^\times$ is easily determined. 
Dirichlet's theorem says that 
\[
  \rank_\dZ(\cO_{k,S}^\times) = r+s-1+\# S ,
\]
where $r$ is the number of real places and $s$ is the number of complex 
places of $k$. 

In the projective case, the rank is much more subtle. If 
$X=\dP^1\smallsetminus \{p,p'\}$ for $p,p'$ conjugates in a quadratic 
extension $\dQ(\sqrt D)$, at least if $k=\dQ$. We are led to the equation 
$x^2-D y^2=1$. This has rank $0$ if $D<0$, and rank $1$ if $D>0$. 

For elliptic curves over $\dQ$, little is known. 

\begin{conjecture}
For $E$ ranging over elliptic curves defined over $\dQ$, 
is $\rank E=\rank_\dZ E(\dQ)$ bounded?
\end{conjecture}

\begin{conjecture}
As $E$ ranges over elliptic curves defined over $\dQ$, 
$\rank E$ is $0$ and $1$ with probability $\frac 1 2$ each. 
\end{conjecture}

Bhargava and Shankar have proved that there is a positive density set of 
elliptic curves having rank $0$ and $1$. 





\subsection{Proof of Mordell-Weil}

The proof has two main ingredients. The first is a height function 
$h:E(\dQ) \to \dR$ satisfying the property that for each $X$, the set 
$\{x\in E(\dQ):h(x)<X\}$ is finite. Moreover, 
$h(n\cdot x) = n^2 h(x)$ and $h(x+y)+h(x-y)=2 h(x) + 2 h(y)$. The second 
ingredient is the \emph{weak Mordell-Weil theorem}: 

\begin{theo}
For some $n>1$, the group $E(\dQ)/n$ is finite. 
\end{theo}

Proving Mordell-Weil from these two ingredients is a very old idea, going back 
to Fermat at least. Let $\{p_1,\dots,p_r\}$ be a set of representatives for 
$E(\dQ)/n$. Choose $X\gg h(p_j)$, and let 
$S=\{p_1,\dots,p_r\}\cup \{p:h(p)<X\}$. We claim that $S$ generates $E(\dQ)$. 
Let $p$ be a point not in $\langle S\rangle$ with minimal height with respect 
to this property. There exists some $j$ such that $p-p_j=n\cdot q$. One sees 
that $h(q)<h(p)$, so $q\in \langle S\rangle$. This implies 
$p\in \langle S\rangle$, a contradiction. 





\subsection{Proof of weak Mordell-Weil}

We do this for $n=2$. Assume $E[2]$ is defined over $\dQ$, i.e.~$E$ is of the 
form $y^2=(x-a)(x-b)(x-c)$. Given $P\in E(\dQ)$, choose some 
$\widetilde P\in E(\overline\dQ)$ such that $2\widetilde P=P$. Define a 
function $\delta(P):G_\dQ \to E[2]$ by 
$\delta(P)(\sigma) = \sigma(\widetilde P)-\widetilde P$. 

The function $\delta(P)$ is actually a continuous homomorphism 
$G_\dQ \to E[2]$. Moreover, $\delta(P_1)=\delta(P_2)$ if and only if 
$P_1-P_2\in 2 E(\dQ)$. So $\delta$ is an injection 
$E(\dQ)/2 \hookrightarrow \hom(G_\dQ,E[2])$. This doesn't solve our problem 
because $\hom(G_\dQ,E[2])$ is infinite. The necessary property of $\delta$ is 
the following. Let $L=\dQ(\sqrt\ell:\ell\mid 2(a-b)(b-c)(a-c))$. Then 
$\delta(P)$ factors through $\galois(L/\dQ)$. Indeed, if 
$P=(x,y)$, then $\widetilde P$ is defined over 
$\dQ(\sqrt{x-a},\sqrt{x-b},\sqrt{x-c})$. It is easy to check that if 
$P\in E[2]$, then $y=0$, and this implies $\widetilde P$ is defined over 
$L$. 

To conclude, $\delta$ is an injection 
$E(\dQ)/2\hookrightarrow \hom(\galois(L/\dQ),E[2])$, the latter being a finite 
set. Thus $E(\dQ)/2$ is finite. 

Let's give a more ``highbrow'' proof using Galois cohomology. Let $n>1$ be an 
integer. We have an exact sequence 
\[
  0 \to E[n] \to E(\overline\dQ) \xrightarrow n E(\overline\dQ) \to 0 .
\]
Take $G_\dQ$-invariants and we get an exact sequence 
\[
  0 \to E(\dQ)/n \xrightarrow\delta \h^1(G_\dQ,E[n]) \to \h^1(G_\dQ,E)[n] \to 0 .
\]
The middle set is still infinite. Repeat the process for each place  of $\dQ$:
\[\xymatrix{
  0 \ar[r] 
    & E(\dQ)/n \ar[r]^-\delta \ar[d] 
    & \h^1(G_\dQ,E[n]) \ar[r] \ar[d] 
    & \h^1(G_\dQ,E)[n] \ar[r] \ar[d] 
    & 0 \\
  0 \ar[r] 
    & E(\dQ_\ell)/n \ar[r] 
    & \h^1(G_{\dQ_\ell},E[n]) \ar[r] 
    & \h^1(G_{\dQ_\ell},E)[n] \ar[r] 
    & 0 
}\]
Define the \emph{$n$-Selmer group} and \emph{Tate-Shafarevich group} of $E$ by 
\begin{align*}
  \selmer_n(E) 
    &= \ker\left(\h^1(G_\dQ,E[n]) \to \bigoplus_v \h^1(G_{\dQ_v},E)\right) \\
  \sha(E) 
    &= \ker\left(\h^1(G_\dQ,E) \to \bigoplus_v \h^1(G_{\dQ_v},E)\right) .
\end{align*}
There is a canonical exact sequence 
\[
  0 \to E(\dQ)/n \to \selmer_n(E) \to \sha(E)[n] \to 0 .
\]
An elementary argument using the Hermite-Minkowski theorem shows that 
$\selmer_n(E)$ is finite. Since $E(\dQ)/n\hookrightarrow \selmer_n(E)$, we're 
done. 





\subsection{Geometric interpretation of \texorpdfstring{$\selmer_n(E)$}{SelnE}}

In general, we know that $\h^1(G_\dQ,\automorphism X)$ classifies 
$\dQ$-forms of $X$. We would like to find an object whose automorphism group is 
$E[n]$. Consider the isogeny $E\xrightarrow n E$. The automorphisms of this 
cover of $E$ are exactly elements of $E[n]$. 

\begin{defi}
An \emph{$n$-cover} of $E$ is a curve $C$ of genus $1$, equipped with a 
$\dQ$-rational map $\widetilde n:C\to E$ and a $\overline\dQ$-isomorphism 
$\varphi:C\iso E$ such that the following diagram commutes: 
\[\xymatrix{
  C \ar[r]^-\varphi \ar[d]^-{\widetilde n} 
    & E \ar[d]^-n \\
  E \ar@{=}[r] 
    & E
}\]
\end{defi}

Similarly, $\h^1(\dQ,E)$ can be identified with the set of isomorphism classes 
of curves of genus one such that $\jacobian C\simeq E$ over $\dQ$. The map 
$\h^1(\dQ,E[n]) \to \h^1(\dQ,E)$ comes from ``forgetting $\widetilde n$.'' 
It follows that $\selmer_n(E)$ can be identified with the set of isomorphism 
classes of $n$-covers $\widetilde n:C\to E$ such that 
$C(\dQ_\ell)\ne \varnothing$ for all $\ell$ and $C(\dR)\ne\varnothing$. 
Similarly $\sha(E)$ consists of isomorphism classes of genus-one curves $C$ 
such that $\jacobian C\simeq E$, and such that $C(\dQ_v)\ne\varnothing$ for all 
places $v$. 

\begin{theo}[Swinnerton-Dyer]
If $\widetilde 2:C\to E$ is an element of $\selmer_2(E)$, then $C$ has a 
$\dQ$-rational positive divisor of degree $2$. 
\end{theo}
\begin{proof}
Degree $2$ divisors of $C$ correspond to rational points on 
$\symmetric^2(C)$. Define a rational morphism 
$\varphi:\symmetric^2(C) \to E$ by $(P,Q)\mapsto P+Q = \widetilde 2(P)-(Q-P)$. 
(Recallhere that $E=\jacobian C$.) Then $X=\varphi^{-1}(0)$ is a curve. Over 
$\overline\dQ$, the map $\phi$ can be identified with the addition map 
$\symmetric^2(E) \to E$. So $X_{\overline\dQ}=\{(P,-P):P\in E(\dQ)\}$. In fact, 
$X_{\overline\dQ}=(E/-1)_{\overline\dQ}=\dP^1_{\overline\dQ}$. The same 
reasoning, replacing $\overline\dQ$ by $\dQ_\ell$ and using the fact that 
$C\simeq E$ over each $\dQ_v$, tels us that $X$ has a rational point in each 
$\dQ_v$. By Hasse-Minkowski, $X\simeq \dP^1$, whence the result. 
\end{proof}

\begin{coro}
If $\widetilde 2:C\to E$ is an element of $\selmer_2(E)$, then $C$ has an 
equation of the form $y^2=f(x)$, where $\deg f=4$. 
\end{coro}

This gives us the dictionary between elements of a $2$-Selmer group and 
binary quartic forms over $\dQ$. Bhargava and Shankar show that there is a 
positive proportion of $E$ with $\selmer_n E=0$, and also a positive proportion 
of $E$ with $\selmer_n E=dZ/n$. Both of these only hold for $n\in \{2,3,4,5\}$. 

\begin{theo}
1. If $\selmer_n E=0$, then $\rank E=0$. 

2. If $\selmer_n E=\dZ/n$, then $\rank E=1$. 
\end{theo}

Part 1 is trivial. Part 2 is incredibly deep. It uses in a crucial way the 
connection between elliptic curves and $L$-functions. This is a special case of 
the Birch and Swinnerton-Dyer conjecture. A big ingredient is the theory of 
complex multiplication discussed in \autoref{sec:3}.





% !TEX root = sms.tex

\section{More algebraic groups, representation theory and invariant theory}
\thanksauthor{Eyal Goren}





The notes for this lecture can be found online at 
\url{http://www.math.mcgill.ca/goren/AlgberaicGroups.SMS2014.pdf}. 





\subsection{Lie algebra}

Let $G$ be a reductive group over an algebraically closed field $k$ of 
characteristic zero. The \emph{Lie algebra} of $G$ is 
$\lie(G)=T_1 G$, the tangent space at the identity of $1$. We can identify 
$\lie G$ with the space of left-invariant derivations of $G$. This 
identification gives $\lie(G)$ the structure of a Lie algebra via 
$[X,Y]=X\circ Y-Y\circ X$. The operation $G\mapsto \lie(G)$ is a 
functor. In particular, the operation of $G$ on itself by inner 
automorphisms induces $\adjoint:G\to \generallinear(\frakg)$. 

The fundamental example is $G=\generallinear_N$, 
$\frakg=\mathfrak{gl}_n$, where $\adjoint(g)(X)=g X g^{-1}$. For 
$H\subset G$, $\fh=\lie H$ is found by ``1st-order approximations to the 
equations defining $H$.'' If $q$ is the Euclidean bilinear form on 
$k^n$, then $\orthogonal(q) = \{x:x \transpose x=1\}$. If we write 
$x=1+x '$, then 
\[
  (1+x') \transpose(1+x') = 1+x' + \transpose{x'} + x' \transpose{x'} = 1
\]
so infinitesimally, 
\[
  \mathfrak o = \mathfrak{so}(q) = \{X\in \mathfrak{gl}_n:X+\transpose X=0\} .
\]
Similarly, $\speciallinear_N=\{g\in \generallinear_N:\det g=1\}$. One easily 
computes $\mathfrak{sl}_n = \{X\in \mathfrak{gl}_n:\trace X=0\}$. 





\subsection{Root systems}

Let $T\subset G$ be a maximal torus and $\fh=\lie T$. Since $T$ is 
``universally semisimple,'' we have 
\[
  \frakg = \fh\oplus \bigoplus_{\alpha\in \Phi} \frakg_\alpha,
\]
where $\frakg_\alpha=\{X\in \frakg:\adjoint(t) X = \alpha(t) X\}$ and 
$\Phi=\Phi(G,T)=\{\alpha\in \character^\ast(T)\smallsetminus 0:\frakg_\alpha\ne 0\}$. 
The space $\character^\ast(T)_\dR$ is equipped with a canonical inner product 
coming from the Killing form on $\frakg$. The pair 
$(\character^\ast(T),\Phi)$ is a \emph{root system}, i.e.~$\Phi=-\Phi$, 
$s_\alpha(\Phi)=\Phi$ for all $\alpha\in \Phi$. Here 
$s_\alpha(v) = v-2\frac{\langle \alpha,v\rangle}{\langle \alpha,\alpha} \alpha$ 
is the reflection induced by $\alpha$. 

A \emph{Weyl chamber} $W$ is a connected component of 
$\character^\ast(T)_\dR\smallsetminus \bigcup_{\alpha\in \Phi} \alpha^\bot$. 
It determines an ordering of the roots: 
\[
  \alpha>0\text{ if }\langle \alpha,v\rangle>0\text{ for all }r\in W.
\]
A positive root is called \emph{simple} if it is not a non-trivial sum of 
other positive roots. Let $\Delta\subset \Phi$ be the set of simple roots. 
Any root can be written as $\sum_{\delta\in \Delta} a_\delta \delta$, where the 
$a_\delta$ are either all positive or all negative. 

The choice of a borel $B\supset T$ is equivalent to the choice of a Weyl chamber 
$W$. Given $B$, we put $\fb=\lie B=\sum_{\alpha\geqslant 0} \frakg_\alpha$. The 
subalgebra $\fu=\sum_{\alpha>0} \frakg_\alpha$ is the Lie algebra of the unipotent 
radical of $B$. We have $B=T\cdot U$ with $\lie U=\fu$. 

Given a subset $\Theta\subset \Delta$, set 
$S_\Theta=\left(\bigcap_{\alpha\in \Theta}\ker(\alpha)\right)^\circ$; this 
is a torus of rank $\rank G-\#\Theta$. Define 
$P_\Theta = Z(S_\Theta)\cdot U$; this is a \emph{standard parabolic subgroup} 
of $G$. 

\begin{theo}
There is a bijection between parabolic subgroups of $G$ (up to conjugation) 
and standard parabolics. There are exactly $2^{\# \Delta}$ such standard 
parabolics. 
\end{theo}

\begin{enonce}[remark]{Example}
If $G=\generallinear_N$ and $T$ is the diagonal torus, define  
$\lambda_i\in \character^\ast(T)$ by $\lambda_i(t_{j j}) = t_{i i}$. Then 
$\character^\ast(T)=\bigoplus \dZ\cdot \lambda_i$. The Lie algebra 
$\frakg=\lie G$ is just $\mathfrak{gl}_n=M_n$ with basis 
$\{E_{i j}\}_{i,j}$ of matrices with a single $1$ in the $(i,j)$ entry 
and zeros elsewhere. One can check that $t\in T$ acts as 
\[
  t E_{i j} t^{-1} = \frac{t_{ii}}{t_{jj}} E_{i j} = (\lambda_i-\lambda_j)(t) E_{i j} .
\]
So $\Phi=\{\lambda_i-\lambda_j:i\ne j\}$. The standard Borel of upper-triangular 
matrices induces the ordering $\lambda_i\geqslant \lambda_j$ if and only if 
$i\leqslant j$. Thus $\Phi^+=\{\lambda_i-\lambda_j:i<j\}$. The corresponding set 
of simple roots is 
$\Delta=\{\lambda_1-\lambda_2,\lambda_2-\lambda_3,\dots,\lambda_{n-1}-\lambda_n\}$. 
\end{enonce}

\begin{enonce}[remark]{Example}[$n=2$]
[do this]
\end{enonce}

\begin{enonce}[remark]{Example}[$n=3$]
[do this]
\end{enonce}





\subsection{Weights}

Let $G$ be a semisimple group, and fix $T\subset B$ a maximal torus contained 
in a Borel subgroup. Let $\rho:G\to \generallinear(V)$ be a representation of 
$G$. Then $V=\bigoplus_{\alpha\in \character^\ast(T)} V_\alpha$, where 
\[
  V_\alpha = \{v\in V:\rho(t)(v) = \alpha(t)\cdot v\text{ for all }t\in T\} ,
\]
is the \emph{weight space} of $\alpha$. The weights of $\rho$ are 
$\{\alpha:V_\alpha\ne 0\}$. 

The \emph{weight lattice} $\Lambda_\weight$ is the smallest subgroup of 
$\character^\ast(T)$ containing all weights of linear representations of $G$. 
The \emph{root lattice} of $G$, $\Lambda_\mathrm{r}$, is the $\dZ$-span of 
$\Phi$. It is known that $[\Lambda_\weight:\Lambda_\mathrm{r}]<\infty$. 

If $V$ is an irreducible representation of $G$, then there is a unique maximal 
weight $\alpha$ in the weights of $V$. This ``highest weight'' determines the 
representation. The set of possible $\alpha$ as highest weight vectors is 
$\Lambda_\weight\cap W$. For each such weight $\alpha$, let $U_\alpha$ be the 
unique irreducible representation of $G$ with highest weight $\alpha$. 

Given any representation $V$ of $G$, one can decompose $V$ into irreducible 
representations ``by hand.'' Find a maximal weight $\alpha$ of $V$. Then 
$V=U_\alpha\oplus V'$ for some $V'$. Continue the process. 





\subsection{Hilbert's invariants theorem}

As before, let $G$ be a reductive group and $\rho:G\to \generallinear(V)$ a 
linear representation. Let $\symmetric(V^\vee)$ be the ring of polynomial 
function on $V$. Indeed, if $\{e_1,\dots,e_n\}$ is a basis for $V$ and 
$\{x_1,\dots,x_n\}$ the dual basis, then 
$\symmetric(V^\vee) = k[x_1,\dots,x_n]$. Put $R=\symmetric(V^\vee)$. Then $G$ 
acts on $R$ by substitutions: $(g\cdot f)(r) = f(g^{-1} r)$. The fundamental 
problem of invariant theory is to give a presentation for the 
$k$-algebra $R^G$ consisting of $G$-invariant polynomial functions on $V$. 

\begin{theo}[Hilbert]
If $V$ is a representation of a reductive group $G$, then 
$\symmetric(V^\vee)^G$ is a finitely generated $k$-algebra. 
\end{theo}

\begin{enonce}[remark]{Example}
Consider $G=\speciallinear_2(k)$ acting on symmetric bilinear forms 
$q=\begin{pmatrix} a & b \\ b & d\end{pmatrix}$ by 
$g\cdot q = g q \transpose g$. Then $\det q=a d-b^2=\discriminant(q)$ is an 
invariant. Is it the only one? Are the invariants sufficient to classify 
(separate) the orbits? What is the nature of the map 
$\spectrum R \to \spectrum(R^G)$ corresponding to 
$V\to V/\!\!\! / G$? These questions, when $V$ is replaced by an arbitrary 
variety is the subject of \emph{Geometric invariant theory}. 
\end{enonce}

Write $\symmetric(V^\vee) = \bigoplus_{d\geqslant 0} \symmetric^d(V^\vee)$. 
The group $G$ acts on each $\symmetric^d(V^\vee)$, which is a finite-dimensional 
$k$-vector space. Decompose each of these into highest-weight representations. 
We get $R=\bigoplus_\alpha R_\alpha$, where $R_\alpha$ is the isotypical component 
of type $\alpha$. Any $f\in R$ can be written as a sum 
$f=\sum f_\alpha$ of isotypical components. Each $f_\alpha$ is a sum of 
homogeneous polynomials, all in $R_\alpha$. 

The \emph{Reynolds operator} $f\mapsto f^\natural$ sends $f$ to its ``trivial 
isotypical component.'' Then $(-)^\natural:R\to R^G$ is a projection. This map is 
$k$-linear and satisfies $(\varphi f)^\natural = \varphi f^\natural$ for 
$\varphi\in R^G$. Indeed, it is enough to show this for $f\in R[d]_\alpha$. 
Then 
$\varphi\cdot R[d]_\alpha \iso k\varphi\otimes_k R[d]_\alpha$ via 
$\varphi\otimes f\mapsto \varphi\cdot f$ is an isoorphism of 
$G$-modules, and the right-hand side is clearly of type $\alpha$. In fact, 
$(-)^\natural:R\to R^G$ is a homomorphism of $R^G$-algebras. 

Let $R_+^G=\bigoplus_{d>0} R[d]^G$. Consider the ideal 
$R\cdot R_+^G$ of $R$. By Hilbert's basis theorem there exist 
$f_1,\dots,f_N\in R_+^G$ such that 
$R\cdot R_+^G = \langle f_1,\dots,f_N\rangle$. Without loss of generality 
each $f_i$ is homogeneous of positive degree. We claim that 
$R^G=k[f_1,\dots,f_N]$. Let $\varphi\in R^G$. To show that 
$\varphi\in k[f_1,\dots,f_N]$ we may assume $\varphi$ is homogeneous. We 
argue by induction on $\deg\varphi$, the trivial case $\deg=0$ being 
obvious. Write $\varphi=\sum a_i f_i$, where without loss of generality 
$a_i\in R$ homogeneous with $\deg(a_i f_i) = \deg(\varphi)$. 
Then 
$\varphi=\varphi^\natural = \sum a_i^\natural f_i^\natural = \sum a_i^\natural f_i$. 
Without loss of generality the $a_i^\natural$ are homogeneous with 
$\deg(a_i^\natural f_i) = \det(\varphi)$. As 
$\deg(a_i^\natural)<\deg\varphi$, each $a_i^\natural\in k[f_1,\dots,f_N]$, 
whence the result. 

\begin{enonce}[remark]{Example}
As before, $\speciallinear_2$ acts on symmetric bilinear forms 
$q=\begin{pmatrix} a & b \\ b & d \end{pmatrix}$ by 
$g\cdot q= g q \transpose g$. The orbits are represented by 
$\begin{pmatrix} \ast \\ & 1 \end{pmatrix}$, 
$\begin{pmatrix} \\ & 1 \end{pmatrix}$, and $0$. The function 
$\det(q)$ is the only invariant. In other words, 
\[
  R^G=k[\det(q)]\simeq k[x] .
\]
So $k^3/ \! \! \! / \speciallinear_2\simeq \dA^1$, but the orbits are not in 
bijection with points of $\dA^1$. So there are ``not enough'' invariant 
functions to separate orbits. 
\end{enonce}

A similar phenomenon occurs for the action of $\speciallinear_N$ on 
symmetric bilinear forms in $N$ variables. There is only one invariant (the 
discriminant) but many orbits. 

\begin{theo}[Chevalley-Iwahori-Nagata]
The set of orbits always surjects onto $V^G$ (for any action of a reductive 
group on an affine algebraic variety). t is bijective if and only if each orbit 
is Zariski-closed. 
\end{theo}

In our example above, we have
\[
  \speciallinear(2)\cdot \begin{pmatrix} \\ & 1\end{pmatrix} = \left\{\begin{pmatrix} a & b \\ b & d \end{pmatrix} : a d-b^2\ne 0\right\} 
\]
which is not Zariski-closed. 

[Borel, Serre, Humphreys, Fulton-Harris, Mumford GIT, 
Dieudonn\'e-Carrell, Goodman-Wallach]





% !TEX root = sms.tex

\section{Cubic rings}\label{sec:wood-i}
\thanksauthor{Melanie Matchett-Wood}





Recall that the goal of this conference is ``counting arithmetic 
objects.'' The first sort of objects we might try to count are number fields. 
To do this, we would like a parameterization of all number fields of a given 
degree. For degree $2$, this is easy. All quadratic fields are of the form 
$\dQ(\sqrt D)$, where $D$ is a square-free integer. 
In general, a degree $n$ number field is of the form $\dQ(\theta)$. Let $f$ 
be the minimal polynomial of $\theta$; it will be of the form 
$f(x)=x^n+a_{n-1} x^{n-1} + \cdots + a_0\in \dZ[x]$. The problem is: determining 
whether two monic irreducible polynomials yield the same field is quite 
difficult. 

Rather than counting number fields, we will try to count their maximal orders 
instead. 





\subsection{Some definitions}

\begin{defi}
A \emph{rank-$n$ ring} is a (commutative, unital) ring $R$ such that 
$R\simeq \dZ^n$ as a $\dZ$-module. 
\end{defi}

For $n=2,3,4,5$, we call these rings quadratic, cubic, quartic, and quintic. 
Typical cubic rings are maximal orders $R=\cO_K$ in cubic number fields $K$. 
But non-maximal orders in cubic number fields are also cubic rings. For 
example, $\dZ[3\sqrt[3] 2]\subset \dQ(2^{1/3})$ is a perfectly good cubic 
ring, with $\dZ$-basis $\{1,3\sqrt[3] 2,9\sqrt[3] 4\}$. A still more 
pathological example is $\dZ[X]/X^3$, which has $\dZ$-basis $\{1,X,X^2\}$. If 
$R$ is any quadratic ring, then $\dZ\times R$ is another ``pathological'' cubic 
ring. 

In the end, we'll count number fields by counting their maximal orders, which 
we will count by including them into a larger class of rank-$n$ rings. 

\begin{defi}
Let $R$ be a rank-$n$ ring. The \emph{trace map} $\trace:R\to\dZ$ is 
defined by $\trace(r) = \trace(\cdot r:R\to R)$. 
\end{defi}

If $\alpha_1,\dots,\alpha_n$ is a $\dZ$-basis of $R$, then the 
\emph{discriminant} of $R$ is 
$\discriminant(R) = \det(\trace(\alpha_i \alpha_j)_{i,j})$. 





\subsection{Quadratic rings}

We know that quadratic rings (up to isomorphism) are in bijection with 
$\{D\in \dZ:d\equiv 0,1\pmod 4\}$. Given a quadratic ring $R$, the 
corresponding integer is $D=\discriminant(R)$. (It is an old theorem that 
$\discriminant(R)$ is a square modulo $4$.) Given such an integer $D$, 
the corresponding ring is 
\[
  \dZ[\tau]/\left(\tau^2 - D\tau + \frac{D^2-D}{4}\right)
\]
If, for example $D=0$, the corresponding ring is $\dZ[\tau]/\tau^2$. 

From our parameterization of quadratic rings, it is easy to count them! 





\subsection{Cubic rings}

Let $R$ be a cubic ring with $\dZ$-basis $\{1,W,T\}$. A key invariant is 
$W T = q+r W + s T$ for some $q,r,s\in \dZ$. Choose a new 
$\dZ$-basis $\{1,\omega,\theta\}$ with $\omega=W-s$ and $\theta=T-r$. In our 
new basis, $\omega\theta = n$ for some $n\in \dZ$. We call such a basis 
\emph{normalized}. Every basis of $R/\dZ$ has a unique normalized lift to 
$R$. 

We also write $\omega^2=m-b \omega + a\theta$ and 
$\theta^2 = \ell-d\omega+c\theta$. The fact that multiplication on $R$ is 
associative forces some conditions on $\{m,b,a,\ell,d,c\}$. The 
associativity relations are exactly: 
\begin{align*}
  n &= - a d \\
  \ell &= - b c \\
  m &= - a c .
\end{align*}
This induces a bijection between isomorphism classes of cubic rings with 
choice basis of $R/\dZ$ and quadruples $(a,b,c,d)\in \dZ^4$. Clearly 
$\generallinear_2(\dZ)$ acts on cubic rings with basis of $R/\dZ$; orbits of 
this action are isomorphism classes of cubic rings. The only thing here that 
is mysterious is the $\generallinear_2(\dZ)$-action on $\dZ^4$. 

If we write $f(x,y) = a x^3 + b x^2 y + c xy^2 + d y^3$ and 
$g\in \generallinear_2(\dZ)$, then 
\[
  (g f)(x,y) = \frac{1}{\det(g)} f\left(\begin{pmatrix} x & y \end{pmatrix} \cdot g\right) .
\]
In other words, if we identify $\dZ^4$ with the space of cubic forms in 
two variables, then the action of $\generallinear_2(\dZ)$ is the natural one 
(with a twist by $\det^{-1}$). 
This parameterization of cubic rings goes back to \cite{df64}. It was also 
used in \cite{dh69}, and saw its first modern formulation in 
\cite{ggs02}. 

\begin{enonce}[remark]{Example}
The cubic ring corresponding to $(a,b,c,d)=(0,0,0,0)$ is pretty pathological -- 
namely $\dZ[\omega,\theta]/(\omega,\theta)^2$. 
\end{enonce}

[\ldots wasn't able to take notes to the end\ldots]





% !TEX root = sms.tex

\section{Quartic and quintic rings}\label{sec:wood-ii}
\thanksauthor{Melanie Matchett-Wood}





The plan is to first review a bit of the theory of cubic rings. We'll spend 
most of the time on quartic rings, then give a brief treatment of quintic 
rings. 





\subsection{Cubic rings revisited}

We are interested in passing from cubic rings to forms geometrically. 
For simplicity, we assume $R=\cO_k$ is the maximal order in a cubic number 
field. Consider the affine scheme $\spectrum(\cO_k)$; we want to embed this 
into $\dP_\dZ^1$. In general, a map $X\to \dP_\dZ^1$ is determined by a line 
bundle $\sL$ on $X$ together with two global sections that generate $\sL$. 
For $\spectrum(\cO_k)$, this consists of an ideal $\fa\subset \cO_k$ together 
with two elements generating $\fa$. We choose the inverse different 
$\fD^{-1}\subset \cO_k$, and two elements of $\fD^{-1}$ having trace zero as 
our global sections. 
It turns out that the image of $\spectrum(\cO_k) \to \dP_\dZ^1$ is the zero-set 
of a binary cubic form. 

If $\cO_k$ is the maximal order in a number field $k$ with $[k:\dQ]=n$, the 
same construction embeds $\spectrum(\cO_k)$ into $\dP_\dZ^{n-2}$. 
It turns out that for any maximal order, a basis of 
$\ker(\trace:\fD^{-1} \to \dZ)$ will always generate $\fD^{-1}$ as a 
$\cO$-module. 





\subsection{Quartic rings}

Start as we did with cubic rings. Given a quartic ring $Q$, write down a 
$\dZ$-basis $\{1,\alpha_1,\alpha_2,\alpha_3\}$ for $Q$. Just as with cubic 
rings, we can encode the multiplication of $Q$ by  
\[
  \alpha_i \alpha_j = c_{i j}^0 + \sum_{k=1}^3 c_{i j}^k \alpha_k .
\]
We could ``shift'' some of the basis elements in order to make some of the 
$c_{i j}^k=0$. Associativity gives us some conditions on the 
$c_{i j}^k$. We're left with a ``moduli scheme'' for quartic rings with 
basis, but the polynomial relations between the $c_{i j}^k$ are complicated 
enough that a direct, explicit approach, does not get you very far. 

In the paper \cite{wy92}, Write and Yukie showed that quartic fields are 
parameterized by pairs of ternary and quadratic forms, moduli the natural 
action of $\generallinear_2(\dQ)\times \generallinear_3(\dQ)$. Unfortunately 
this approach isn't very useful either. More recently, in the paper 
\cite{b04}, Bhargava realized that the problem needed to be understood over 
$\dZ$, and that  cubic resolvent are essential. 

We start with cubic resolvent fields $K/\dQ$ with $[K:\dQ]=4$. Let 
$\widetilde K$ be the Galois closure of $K$; we assume 
$\galois(\widetilde K/\dQ)\simeq S_4$. The group $S_4$ has a canonical 
subgroup $D_4$ of index $3$, consisting of permutations which are 
symmetries of the square 
\[\xymatrix{
  1 \ar@{-}[r] \ar@{-}[d] 
    & 2 \ar@{-}[d] \\
  4 \ar@{-}[r] 
    & 3
}\]
Galois theory gives us a subfield $K_3\subset \widetilde K$ such that 
$[K_3:\dQ]=3$; this is the \emph{cubic resolvent} of $K$. For 
$k\in K$, let $k^{(1)},k^{(2)},k^{(3)},k^{(4)}$ be the Galois conjugates of 
$K$. Then $k^{(1)} k^{(3)} + k^{(2)} k^{(4)}$ is an element of $K_3$. Put 
$\phi_{4,3}(k) = k^{(1)} k^{(3)} + k^{(2)} k^{(4)}$; this is a 
discriminant-preserving map $K\to K_3$. 

Next we define the cubic resolvent of a ring. Let $Q$ be a quartic ring, which 
for simplicity we assume is an order in a $S_4$-quartic field $K$. 

\begin{defi}
A \emph{cubic resolvent ring} of $Q$ is a cubic ring $R$ in the resolvent 
field $K_3$ such that 
\begin{enumerate}
  \item $\discriminant(R)=\discriminant(Q)$
  \item for all $q\in Q$, $\phi_{4,3}(q)\in R$. 
\end{enumerate}
\end{defi}
We have a quadratic map $\phi_{4,3}:Q\to R$. It descends to a map 
$\phi_{4,3}:Q/\dZ \to R/\dZ$; here the quotients are taken as 
$\dZ$-modules, not as rings. The $\dZ$ in the quotient is the 
sub-$\dZ$-module generated by the multiplicative unit in the ring. 

\begin{exercise}
Show that $\phi_{4,3}$ descends to a map $Q/\dZ\to R/\dZ$. 
\end{exercise}

An element of $Q/\dZ$ can be written as $\ell\alpha_1+m \alpha_2 + n \alpha_3$ 
for $\ell,m,n\in \dZ$. It's image under $\phi_{4,3}$ will be a linear combination 
of $\omega$ and $\theta$, i.e. 
\[
  \phi_{4,3}(\ell \alpha_1 + m\alpha_2 + n\alpha_3) = A(\ell,m,n) \omega + B(\ell,m,n)\theta .
\]
The functions $A$ and $B$ are ternary quadratic forms with coefficients in 
$\dZ$. 

It is not clear whether every quartic ring even has a cubic resolvent. Even if 
it does, how many? Luckily, every quartic ring does have a cubic resolvent, but 
this resolvent is not necessarily unique. But maximal quartic rings (which 
include maximal quartic orders) have a unique cubic resolvent. 

We want to parameterize isomorphism classes of pairs $(Q,R)$, where $Q$ is a 
quartic ring and $R$ is a cubic resolvent of $Q$. The main result of 
\cite{b04} is that these are in bijection with 
$\generallinear_2(\dZ)\times \generallinear_3(\dZ)$-classes of pairs of ternary 
quadratic forms. This bijection preserves discriminants, and allows you to 
detect prime splitting, automorphism groups, \ldots as the parameterization of 
cubic rings. 

To be completely explicit, a \emph{ternary quadratic form} is of the form 
\[
  A(\ell,m,n) = a_{1 1} \ell^2 + a_{1 2} \ell m + a_{1 3} \ell n + \cdots + a_{3 3} n^2 .
\]
The $\generallinear_2$ and $\generallinear_3$ action can be written down 
explicitly. We identify the form $A$ with a $3\times 3$ matrix 
\[
  \begin{pmatrix} a_{11} & a_{12}/2 & a_{13}/2 \\ a_{12}/2 & a_{22} & a_{23}/2 \\ a_{13}/2 & a_{23}/2 & a_{33} \end{pmatrix} 
\]
and similarly for $B$. The quantity $4\det(A x+B y)$ is a binary cubic 
form with coefficients in $\dZ$. The corresponding cubic ring is the cubic 
resolvent ring in the pair $(Q,R)$. What we haven't done is describe how to 
construct the quartic ring $Q$ from $A$ and $B$. 





\subsection{Geometric perspective}

Write $A=a_{11} x^2+a_{1 2} x y + a_{1 3} x z + \cdots$ and similarly for $B$. 
These cut out a subscheme of $\dP^2$. Over $\dQ$, we would expect this subscheme 
to have four points (over $\overline\dQ$). The Galois conjugates of any such 
point $p$ are among the four intersection points, so $p$ is defined over a 
(at most) quartic extension of $\dQ$. 

Let's do this over $\spectrum \dZ$. A good reference is \cite{w11}. The two 
forms $A,B$ over $\dZ$ cut out a subscheme $V_{A,B}$ of $\dP_\dZ^2$. The 
functions on $V_{A,B}$ recover the quartic ring corresponding to $A,B$. 

If $\cO$ is the maximal order in a quartic field $K$, then using the inverse 
different we can embed $\spectrum(\cO)\hookrightarrow \dP_\dZ^2$. From 
\cite{ce96}, such subschemes of $\dP_\dZ^2$ are cut out by pairs of ternary 
quadratic forms. 





\subsection{Quintic rings}

In \cite{b08}, it is proved that there is a bijection between isomorphism 
classes of pairs $(R,S)$, where $R$ is a quintic ring and $S$ is a 
sextic resolvent of $R$, and $\generallinear_4(\dZ)\times \speciallinear_5(\dZ)$-orbits 
of quadruples of $5\times 5$ skew-symmetric matrices (alternatively, quinary 
alternating forms). 

We'll briefly describe the sextic resolvent of a quintic ring. At the level of 
Galois groups of fields, we're looking for an index $6$ subgroup of 
$S_5$. Make a pentagon out of $\{1,2,3,4,5\}$. There are $6$ ways to put them 
into two disjoint $5$-cycles. The permutations that fix this decomposition into 
$2$-cycles gives us such a subgroup. 

The resolvent associated to $(R,S)$ is a map $\Lambda^2 S^\vee \to R^\vee$. 
From a geometric perspective: over $\dQ$, in $\dP_{\overline\dQ}^3$, we have 
$A=A_1 x+A_2 y+A_3 z+A_4 w$, where $A_1$ is an alternating $5\times t$ matrix. 
The matrix $A$ has five skew-symmetric $4\times 4$ minors. The determinant of 
such a minor is a a square, so the Pfaffian $\sqrt{\det(\text{minor})}$ is a 
quadratic form in $4$ variables. So from $A$ we get forms $Q_1,\dots,Q_5$. 
These cut out a subvariety of $\dP_{\overline\dQ}^3$, which is the analogue 
of the ring $R$. 

We should expect the quadratic forms $Q_1,\dots,Q_5$ to have no common zeros! 
But when a $5$-tuple of quadratic forms come from Pfaffians of an alternating 
matrix, the subscheme they cut out is non-empty. 

We would have $\spectrum(\cO)\subset \dP_\dZ^3$. There is a geometric analogue 
of this in \cite{be77} which shows that such a subvariety of $\dP^3$ has to be 
cut out by $5$ quadratic forms which come from Pfaffians of an alternating 
matrix. 




% !TEX root = sms.tex

\section{How to count rings and fields I}
\thanksauthor{Manjul Bhargava}





In this lecture, we'll use the parameterization of cubic rings discussed in 
\autoref{sec:wood-i} to count cubic rings. Recall that there is a bijection 
\[
  \{\text{cubic rings}/\sim\} \iso \{\text{int.~bin.~cubic forms}\} /\generallinear_2(\dZ) 
\]
which preserves discriminant. So counting cubic rings by discriminant is 
equivalent to counting cubic forms by discriminant. The corresponding problem 
for binary quadratic forms was solved by Gauss and Lipschitz (and first done 
rigorously by Mertens and Siegel). 





\subsection{Lipschitz principle and generalizations}

Lipschitz's principle is that if $R$ is some region in the Euclidean plane with 
area $T$, then the number of lattice points in $R$ is approximately $T$. 
Better, 
\[
  \# \{\text{lattice points in }R\} = T + O(T^{1/2}) .
\]
The implied constant in $O(T^{1/2})$ will depend on $R$. Here we assume the 
region $R$ is ``homogeneously expanding'' via homothety. This principle, though 
completely elementary, was sufficient to count binary quadratic forms. 

Davenport realized that in order to attack binary cubic forms, one needs a 
version of the Lipschitz principle where the region $R$ is fixed (not 
necessarily expanding homogeneously). 

\begin{theo}[Davenport] 
Let $R$ be a bounded semi-algebraic region in $\dR^n$ defined by at most $k$ 
inequalities each of degree at most $\ell$. Then the number of lattice points 
in $R$ is $\volume(R)+O_{k,\ell}(\max\{\volume(\overline R),1\})$, where 
$\volume(\overline R)$ denotes the greatest $d$-dimensional volume of a 
projection of $R$ onto a $d$-dimensional coordinate hyperplane, 
$1\leqslant d\leqslant n-1$. 
\end{theo}
\begin{proof}
See the papers \cite{d51,d64}. 
\end{proof}

The following is Davenport's cubic version of Gauss' formula for the number of 
binary quadratic forms. 

\begin{theo}[Davenport]
Let $H(D)$ be the number of irreducible integer binary cubic forms, up to 
$\generallinear_2(\dZ)$-equivalence, having discriminant $D$. Then 
\begin{align*}
  \sum_{0<D<X} H(D) &= \frac{\pi^2}{72} X + O(X^{15/16}) \\
  \sum_{0<-D<X} H(D) &= \frac{\pi^2}{24} X + O(X^{15/16}) .
\end{align*}
\end{theo}

The proof uses Davenport's refined Lipschitz principle, along with analysis of 
the cusps using explicit inequalities. 

Attempts to mimic Davenport's methods to count quartic forms fail because the 
inequalities involved are far too complicated to be analyzed explicitly. We 
will reprove Davenport's theorem using Davenport's principle, but without 
having to write down explicit inequalities. 





\subsection{Proof of Davenport's theorem}

Let $V$ be the space of binary cubic forms, $G=\generallinear(2)$. So 
$V(\dR)$ is the real vector space of binary cubic forms, and $V(\dZ)$ is the 
lattice of integer binary cubic forms. Similarly for $G(\dR)$ and $G(\dZ)$. 
We have a representation $G\to \generallinear(V)$ defined over 
$\spectrum(\dZ)$. 

Let $E$ be a fundamental domain for the action of $G(\dZ)$ on $V(\dR)$. We can 
write 
\begin{align*}
  \sum_{0<D<X} H(D) 
    &= \# \{x\in E\cap V(\dZ)^\irreducible:0<|\discriminant(x)| < X\} \\
    &= \# \{x\in E\cap V^+(\dZ)^\irreducible:|\discriminant(x)|<X\} .
\end{align*}
We start by constructing any fundamental domains $E$. Fix $v\in V^+(\dR)$, and 
let $\cF$ be a fundamental domain for the left action of $G(\dZ)$ on $G(\dR)$. 
Then the multiset $\cF v=\{x=g v:g\in \cF\}$ is the union of six fundamental 
domains for the action of $G(\dZ)$ on $V(\dR)$. Indeed, 
\[
  \left(G(\dZ)\backslash G(\dR)\right) \cdot \left(G(\dR) \backslash V^+(\dR)\right) = G(\dZ)\backslash V^+(\dR) .
\]
A brief consideration of stabilizers yields the number of fundamental domains. 

A nice $\cF$ to take was defined by Gauss: 
\[
  \cF = \{g\in \generallinear_2(\dR):g\cdot i \in \Omega\},
\]
where $\generallinear_2(\dR)$ acts on the upper half-plane 
$\fH=\{z\in \dC:\Im z>0\}$ as usual, and 
\[
  \Omega=\left\{z\in \fH:-\frac 1 2 \leqslant \Re z\leqslant \frac 1 2\text{ and }|z|\geqslant 1\right\} .
\]
Davenport's fundamental domain for $G(\dZ)$ on $V(\dR)$ occurs when 
$v=x^2 y - x y^2$. Now we apply the Taski-Seidenberg theorem to conclude that 
$\cF v$ is semi-algebraic. This was generalized by Borel and Harish-Chandra in 
\cite{bh62} to arbitrary reductive groups over number fields. Note that 
$\cF = N' A' K \Lambda$, where 
\begin{align*}
  N' &= \left\{\begin{pmatrix} 1 \\ n & 1 \end{pmatrix} : |n|\leqslant \frac 1 2\right\} \\
  A' &= \left\{\begin{pmatrix} t \\ & t^{-1}\end{pmatrix} : t\geqslant \frac{\sqrt[4] 3}{\sqrt 2}\right\} \\
  K &= \left\{\begin{pmatrix}\cos \theta & \sin \theta \\ -\sin \theta & \cos \theta\end{pmatrix} : 0\leqslant \theta < 2\pi\right\} \\
  \Lambda &= \{\lambda\in \dR^\times:\lambda>0\} .
\end{align*}
Let $B$ be a compact set in $V^+(\dR)$ that is the closure of a non-empty open 
set, on which $\discriminant\geqslant 1$. We will allow $v$ to vary in $B$. 

Next, we need estimates on reducibility. 

\begin{lemm}
Let $R_X(v)=\cF v\cap \{|\discriminant |<X\}$. Then the number of reducible 
integral forms $a x^3 + \cdots + d y^3$ in $R_X(v)$ with $a\ne 0$ is 
$O(X^{3/4+\epsilon})$. 
\end{lemm}
\begin{proof}
In $R_X(v)$, we have $a=O(X^{1/4})$, $b=O(X^{1/4})$, $a b c=O(X^{3/4})$, 
$a b d=O(X^{3/4})$, \ldots. If $d=0$, at most $O(X^{3/4+\epsilon})$ such 
forms in $R_X(v)$. If $d\ne 0$, then $r x+s y\mid f(x,y)=a x^3+\cdots + d y^3$, 
which implies $r\mid a,s\mid d$. Fixing $a,b,d$ (for which there are 
$O(X^{3/4+\epsilon})$ choices) there are $O(X^\epsilon)$ choices for $r$ and 
$s$. The fact that $f(-s,r)=0$ determines $c$. The result follows. 
\end{proof}

Finally we average. Recall that $|\discriminant(v)|^{-1}\, dv$ is the unique 
(up to scalar) $G(\dR)$-invariant measure on $V(\dR)$. This lets us compute  
\begin{align*}
  N^+(X) 
    &= \sum_{0<D<X} H(D) \\
    &= \frac{\displaystyle\int_B \#\{x\in \cF v\cap V(\dZ)^\irreducible:|\discriminant(x)|<X\}\cdot |\discriminant(v)|^{-1}\, dv}{6\displaystyle\int_B|\discriminant(v)|^{-1}\, dv} \\
    &= \frac{1}{M} \int_\cF\#\{x\in g B\cap V(\dZ)^\irreducible:|\discriminant(x)|<X\}\, dg \\
    &= \frac 1 6 \volume(R_X(v)) + O(X^{5/6}) ,
\end{align*}
the last equality coming from a uniform application of Davenport's inequality. 
Note that we have proved a stronger version of Davenport's result, namely one 
with an error term of $O(X^{5/6})$ instead of $O(X^{15/16})$. In \cite{bst13}, 
a more careful version of this proof shows that there is a second term of the 
form $c X^{5/6}+O(X^{3/4+\epsilon})$. So the error term is in many ways as good 
as possible. 





% !TEX root = sms.tex

\section{Rings associated to binary \texorpdfstring{$n$}{n}-ic forms, composition of \texorpdfstring{$2\times n\times n$}{2*n*n} boxes and class groups}\label{sec:wood-iii}
\thanksauthor{Melanie Matchett Wood}





We'll focus especially on the case of binary quartic forms, and the 
parameterization of ideal classes. Good references are 
\cite{n89,s01,s03,s05,w11-rings}. 






\subsection{The construction}

Let $f=a_0 x^n + a_1 x^{n-1} y + \cdots + a_n y^n$ be a binary $n$-ic form with 
the $a_i\in \dZ$. We will construct a rank $n$ ring $R_f$. 

First we'll give an explicit construction. 
If $a_0\ne 0$, consider the ring $\dQ[\theta]/f(\theta,1)$. This contains 
elements 
\begin{align*}
  1 \\
  a_0 \theta \\
  a_0 \theta^2 + a_1 \theta \\
  \ldots \\
  a_0 \theta^{n-1} + a_1 \theta^{n-2} + \cdots + a_{n-2} \theta .
\end{align*}
Let $R_f$ be the $\dZ$-module generated by these. It turns out that $R_f$ is 
closed under multiplication, so it is a rank-$n$ ring. For example, 
\begin{align*}
  (a_0\theta)^2 
    &= a_0^2 \theta^2 \\
    &= a_0(a_0 \theta^2 + a_1 \theta) - a_1(a_0 \theta) ,
\end{align*}
so $(a_0 \theta)^2\in R_f$. The same phenomenon occurs for all of the additive 
generators of $R_f$. 

A more highbrow construction is as follows. Inside $\dP_\dZ^1$ we have a 
closed subscheme $V_f$ cut out by $f=0$. The ring $R_f$ is just 
$\h^0(V_f,\sO)$, the ring of regular functions on $V_f$ (at least if not all 
the $a_i=0$). When $a_0\ne 0$, these functions are determined by their 
restriction to $\dA_\dZ^1\cap V_f$. The regular functions on $\dA_\dZ^1$ are 
polynomials in $\frac x y$, i.e.~$\h^0(\dA_\dZ^1,\sO) = \dZ[\frac x y]$. We 
have for example 
\[
  a_0 \frac x y = -\left(a_1 + a_2 \frac y x + \cdots \right) .
\]
We claim that the $\dZ$-span of 
$1,a_0 \frac x y,a_0(\frac x y)^2+a_1 \frac x y,\ldots$ is the whole ring of 
regular functions on $V_f$. This recovers our explicit definition of $R_f$. 

\begin{prop}
Let $f$ be a binary $n$-ic form. Then 
\begin{itemize}
  \item If $R_f$ is a domain, its fraction field is $\dQ[\theta]/f(\theta,1)$. 
  \item $\discriminant(R_f) = \discriminant(f)$ .
  \item $R_f$ is a domain if and only if $f$ is irreducible in $\dQ[x,y]$ 
  \item $R_f$ is a maximal order if and only if certain conditions modulo $p^2$  
    for every $p$ are satisfied. 
  \item If $R_f$ is maximal, then a prime $p$ splits in $R_f$ as $f(x,y)$ 
    factors modulo $p$. 
\end{itemize}
\end{prop}

The special case $a_0=1$ yields monogenic rings $\dZ[\alpha]$. The bad news is 
that for $n>3$ we do not obtain all rank $n$ rings (or all rank $n$ maximal 
orders). Heuristically, this works as follows. Given a form $f$ we can produce 
$V_f\subset \dP_\dZ^1$. A map to $\dP^1$ is determined by a line bundle with 
two generating global sections. In the maximal order case, this is an ideal 
class of the ring with two generating elements. If we think of 
$R_f\subset \dQ[\theta]/f(\theta,1)$, then the ideal class is 
$I_f=\langle 1,\theta\rangle$, which is a fractional ideal class unless 
$a_0=1$. 

The general story goes as follows. There is a bijection between binary $n$-ic 
forms up to $\generallinear_2(\dZ)$-equivalence and rank-$n$ rings with 
an ``ideal class'' satisfying some conditions on the ideal class. We don't 
get all rank $n$ rings because not all rings have ideal classes satisfying the 
conditions. For $n=2$, the conditions on the ideal class are trivial, so we 
get all quadratic rings and all ``ideal classes.'' (We put ideal class in 
quotation marks because we haven't defined such things for rings that are not 
integral domains.) 





\subsection{Geometric story of Gauss composition}

Start with a binary quadratic form $f(x,y)=a x^2 + b x y + c y^2$ for 
$a,b,c\in \dZ$. To this we can associate a quadratic ring with an ideal class. 
Geometrically, $\{f=0\}$ cuts out a subscheme $V_f$ of $\dP_\dZ^1$. The ring 
$\h^0(V_f,\sO)$ of regular functions on $V_f$ is the associated ring, and the 
invertible sheaf coming from $V_f \hookrightarrow \dP^1$ is the ideal class. 

We work with pathological rings (having zero-divisors and nilpotents) because 
we want to be able to talk about the behavior of a form when it is reduced 
modulo $p$. Even if things are well-behaved over $\dZ$, their reduction modulo 
$p$ can be pathological. 

We define ideal classes in general rings. Let $C$ be an order in a quadratic 
field. If $\fa,\fb$ are ideals in the same class, then $\fa= k \fb$ for some 
$k\in C\otimes \dQ$. The map $k:\fa \to \fb$ is an isomorphism of $C$-modules. 
Moreover, the converse holds: ideals that are isomorphic as $C$-modules are in 
the same ideal class. So we can replace the notion of an ``ideal class'' with 
an ``isomorphism class of modules.'' But certainly not all $C$-modules come 
from ideal classes. For example, $C^2$ is not an ideal class. However, all 
modules isomorphic to $\dZ^2$ as $\dZ$-modules are ideal classes. 

The reference for what follows is \cite{w11-gauss}. 

\begin{theo}
There is a bijection between 
$\generallinear_2(\dZ)\times \generallinear_1(\dZ)$-classes of 
$a x^2 + b x y+c y^2$ and isomorphism classes of $(C,M)$, where $C$ is a 
quadratic ring and $M$ is a $C$-module such that $M$ is $\simeq \dZ^2$ as a 
$\dZ$-module and $\trace:C\to \dZ$ is the same using multiplication on $C$ or 
$M$. 
\end{theo}

Recall that for $n=2$ we get all ideal classes. For $n=3$, the conditions 
on our ``ideal class'' force $I_f$ to be the inverse different. So we get 
precisely isomorphism classes of cubic rings. For $n>3$, the conditions on 
the ideal class become non-trivial. In particular, we don't get all maximal 
orders. For example, when $n=4$, we get exactly the quartic rings with a 
monogenic cubic resolvent. A good reference is \cite{w12}. 





\subsection{Parameterization of ideal classes of \texorpdfstring{$R_f$}{Rf}}

References here are \cite{b04-i,b04-ii,w14}. Let 
$A\in \dZ^2\otimes \dZ^n\otimes \dZ^n$, the space of ``pairs $(A_1,A_2)$ of 
$n\times n$ matrices with coefficients in $\dZ$.'' Put 
$f=\det(A)=\det(A_1 x_1 + A_2 x_2)$; this is a binary $n$-ic form. The group 
$G\subset \generallinear_2(\dZ) \times\generallinear_n(\dZ)$ consisting of 
$(g,h)$ with $\det(g)\det(h)=1$ acts on $\dZ^2\otimes \dZ^n\otimes \dZ^n$ by 
left and right multiplication on $(A_1,A_2)$: 
\[
  (g,h)\cdot (A_1,A_2) = (g A_1 h,g A_2 h) .
\]

\begin{theo}
For primitive, irreducible binary $n$-ic forms $f$ with coefficients in 
$\dZ$, there is a bijection between ideal classes of $R_f$ and 
$G$-classes of $A\in \dZ^2\otimes \dZ^n\otimes \dZ^n$ with $\det(A)=f$. 
\end{theo}

The ideal classes appearing in this theorem are not necessarily invertible. 
If we unravel this when $f$ is monic, the theorem generalizes the classical 
result parameterizing ideal classes of monogenic orders by conjugacy classes 
of matrices. We will not cover the proof of this. We will describe the 
map from forms to ideal classes by restricting to the case when $R_f$ is a 
maximal order. 

Instead of viewing $\dZ^2\otimes \dZ^n\otimes \dZ^n$ as pairs of 
$n\times n$ matrices, we can view it as $n$-tuples $(a_1,\dots,a_n)$ of 
$2\times n$ matrices. Given such a tuple, 
$a_1 y_1 + a_2 y_2 + \cdots + a_n$ is a $2\times n$ matrix. Let 
$g_1,\dots,g_{\binom 2 n}$ be the determinants of maximal minors of the matrix 
$a_1 y_1 + \cdots + a_n y_n$. Each $g_i$ is a quadratic form in $n$ variables. 
Let $V_{\boldsymbol g}\subset \dP_\dZ^{n-1}$ be the subscheme cut out by the 
$g_i$. If the $g_i$ were generic, we would expect 
$V_{\boldsymbol g} = \varnothing$. In our case, the ring of functions on 
$V_{\boldsymbol g}$ is $R_f$. The map 
$V_{\boldsymbol g}\hookrightarrow \dP_\dZ^{n-1}$ comes from a line bundle with 
$n$ global sections. The line bundle gives us an ideal class in $R_f$, which 
can be any ideal class in $R_f$. 

When $n=2$, there are three ways of ``slicing up'' an element of 
$\dZ^2\otimes \dZ^2\otimes \dZ^3$. The same quadratic form $f$ produces a ring 
$R_f$ with ideal classes $M_A$, $N_A$, $I_f$. It turns out that 
$M_A N_A = I_f^{n-3}$, where $I_f$ is the ``standard'' ideal coming from $f$. 
Since $n=2$, $M_A N_A I_f = 1$. 

Let $M$ be an ideal in the ring $R_f$, and let $N$ be an ideal such that 
$M N=I_f^{n-3}$. Then with the multiplication map 
$\dZ^n\otimes \dZ^n = M\otimes_\dZ N \to I_f^{n-3} = \dZ^n$, forget all but the 
last two coordinates. This gives us two $n\times n$ matrices, which define 
$M$ and $N$ to begin with. For $n=2$, this can all be done quite explicitly. 





% !TEX root = sms.tex

\section{The zeta functions attached to prehomogeneous vector spaces}\label{sec:taniguchi}
\thanksauthor{Takashi Taniguchi}





\subsection{Introduction}

The first main example is the space of binary cubics. Let $G=\generallinear_2$, 
and let $V$ be the space of binary cubics. We consider the standard twisted 
action of $G$ on $V$: 
\[
  (g\cdot x)(u,v) = \frac{1}{\det g} x\left(\begin{pmatrix} u & v \end{pmatrix} \cdot g\right) .
\]
We know that $\discriminant(g\cdot x) = (\det g)^2 \discriminant(x)$. 

\begin{defi}[Shintani]
Define the function $\xi^\pm:\{\Im z>1\} \to \dC$ by 
\[
  \xi^\pm(s) 
    = \sum_{\substack{x\in G(\dZ)\backslash V(\dZ) \\ \pm \discriminant(x)>0}} \frac{\# \stabilizer(x)^{-1}}{\# \discriminant(x)^s} = \sum_{\substack{R\text{ cubic ring} \\ \pm \discriminant(R)>0}} \frac{\# \automorphism(R)^{-1}}{\discriminant(R)^s} .
\]
\end{defi}

\begin{theo}
1. The function $\xi^\pm$ has an analytic continuation to $\dC$. The % [analytic continuation]
function $(s-1)^2 (s-\frac 5 6)(s-\frac 7 6)\xi^\pm(s)$ is entire. 

2. Indeed, $\xi^\pm(s)$ is holomorphic except for simple poles % [principal part]
at $s=1,\frac 5 6$, with explicit residue formulas. 

3. There is a functional equation between $\xi^\pm(1-s)$ and % [functional equation]
$\widehat\xi^\pm(s)$. 
\end{theo}
\begin{proof}
1,3. These follow from the general theory of prehomogeneous vector spaces. 

2. This needs some careful analysis. 
\end{proof}

As an application [see Frank's talk] if we write 
\[
  \xi^\pm(s) = \sum_{n\geqslant 1} \frac{a_{\pm n}}{n^s},
\]
then 
\[
  \sum_{0<n<X} a_{\pm n} = r_1^\pm X + r_{5/6}^\pm \frac{X^{5/6}}{5/6} + O(X^{3/5+\epsilon}) .
\]
In \cite{s75}, Shintani separated the contributions of irreducible and reducible 
representations. 





\subsection{A proof of analytic continuation and functional equation for \texorpdfstring{$\zeta$}{zeta}}

Let $f\in \cS(\dR)$ be a smooth function that decays rapidly. The \emph{Fourier 
transform} of $f$ is 
\[
  \widehat f(y) = \int_\dR f(x) e^{2\pi i x y}\, dx .
\]
One proof of the analytic continuation of the Riemann $\zeta$ function uses the 
Poisson summation formula: 
\[
  \sum_{x\in \dZ} f(x) = \sum_{y\in \dZ} \widehat f(y) .
\]
A simple variation is that for $t\in \dR^\times$, we have 
\[
  \sum_{x\in \dZ} f(t x) = |t|^{-1} \sum_{y\in \dZ} \widehat f(t^{-1} y) .
\]
Simply let $f_t(x) =f (t x)$, and show using a change of variables that 
$\widehat{f_t}(y) = |t|^{-1} \widehat f(t^{-1} y)$. Applying Poisson summation 
to $f_t$ yields the formula. 

\begin{defi}[local zeta]
Define 
\[
  \Phi(f,s) = \int_\dR |x|^{s-1} f(x)\, dx ,
\]
where $f$ ranges over $\cS(\dR)$ and $s\in \dC$. 
\end{defi}

For $\Re s>0$, the map $\Phi(-,s)$ is a functional on $\cS(\dR)$. 

\begin{prop}
1. $\Phi(f,s)$ has a meromorphic continuation to $\dC$. Moreover, 
$\Gamma(s)^{-1} \Phi(f,s)$ is entire. 

2. $\Phi(\widehat f,s) = c(s) \Phi(f,1-s)$, where 
$c(s) = (2\pi)^{-s} \Gamma(s) (e^{i\pi s/2} + e^{-i\pi s/2})$. 
\end{prop}

Let's use this proposition to prove the functional equation for $\zeta$. 
We have 
\begin{align*}
  \zeta(s) \Phi(f,s) 
    &= \sum_{n\geqslant 1} \frac{1}{n^s} \int_\dR |x|s f(x) \frac{dx}{|x|} \\
    &= \sum_{n\geqslant 1} \int_\dR \left|\frac x n\right|^s f(x) \frac{dx}{|x|} \\
    &= \int_\dR |x|^s \sum_{n\geqslant 1} f(n x) \, \frac{dx}{|x|} \\
    &= \int_0^\infty |x|^s \sum_{n\in \dZ\smallsetminus 0} f(n x)\, \frac{dx}{x} .
\end{align*}
Recall that $d^\times t = \frac{dt}{t}$ is an invariant measure on 
$\dR^\times_+$. Define 
\begin{align*}
  Z(f,s) &= \int_0^\infty t^s \sum_{x\in \dZ\smallsetminus 0} f(t x)\, d^\times t \\
  Z_+(f,s) &= \int_1^\infty t^s \sum_{x\in \dZ\smallsetminus 0} f(t x) \, d^\times t .
\end{align*}

\begin{lemm}
The function $Z_+(f,s)$ is entire. 
\end{lemm}
\begin{proof}
Since $t\geqslant 1$, the convergence is better when $\Re s$ is smaller. 
\end{proof}

We compute: 
\begin{align*}
  Z(f,s) - Z_+(f,s) 
    &= \int_0^1 t^s \sum_{x\in \dZ\smallsetminus 0} f(t x)\, d^\times t \\
    &= \int_0^1 t^s \left(t^{-1} \sum_{y\in \dZ\smallsetminus 0} \widehat f(t^{-1} y) + t^{-1} \widehat f(0) - f(0)\right)\, d^\times t \\
    &= \int_0^1 t^{s-1} \sum_{y\in \dZ\smallsetminus 0} f(t^{-1} x)\, d^\times t + \widehat f(0) \int_0^1 t^{s-1} \, d^\times t - f(0) \int_0^1 t^s \, d^\times t \\
    &= Z_+(\widehat f,1-s) + \frac{\widehat f(0)}{s-1} + \frac{f(0)}{s} .
\end{align*}
We have shown that 
\begin{align*}
  Z(f,s) 
    &= Z_+(\widehat f,1-s) + \frac{\widehat f(0)}{s-1} + \frac{f(0)}{s} \\
    &= Z(\widehat f,1-s) .
\end{align*}
The functional equation follows: 
\begin{align*}
  \zeta(1-s) \Phi(f,1-s) 
    &= Z(f,1-s) \\
    &= Z(\widehat f,s) \\
    &= \zeta(s) \Phi(\widehat f,s) \\
    &= \zeta(s) c(s)\Phi(f,1-s) ,
\end{align*}
the last equality following from the previous proposition. Canceling the 
$\Phi(f,1-s)$ yields the functional equation we're looking for. 

We can also derive the residue of $\zeta$: 
\[
  \residue_{s=1} Z(f,s) 
    = \widehat f(0) 
    = \int_\dR f(x)\, dx .
\]
But $Z(f,s) = \zeta(s) \Phi(f,s)$, and 
\[
  \Phi(f,1) = \int_\dR |x|^{1-1} f(x)\, dx = \int_\dR f(x)\, dx .
\]
It follows that $\residue_{s=1} \zeta(s) = $. 





\subsection{Outline of proof properties of \texorpdfstring{$\xi^\pm$}{xi}}

Define 
\[
  \Phi_1(f,s) = \int_0^\infty x^{s-1} f(x)\, dx .
\]
Our ``main tool'' is integration by parts, using 
$x^{s-1} = \left(\frac{x^s}{s}\right)'$. We compute: 
\begin{align*}
  \frac{1}{\Gamma(s)} \Phi_1(f,s) 
    &= \frac{1}{s \Gamma(s)} \int_0^\infty (x^s)' f(x)\, dx \\
    &= \frac{1}{\Gamma(s+1)} \left(\left. x^s f(x)\right|_0^\infty - \int_0^\infty x^s f'(x)\, dx \right) \\
    &= \frac{-1}{\Gamma(s+1)} \Phi_1(f',s+1) .
\end{align*}
Repeat this $n$ times, obtaining 
\[
  \frac{1}{\Gamma(s)} \Phi_1(f,s) = \frac{(-1)^n}{\Gamma(s+n)} \Phi_1(f^{(n)},s+n) ,
\]
where the right-hand side is holomorphic for $\Re s>-n$. This proves part 1 of 
our main proposition. 

For part 2 of the main proposition, compute 
\begin{align*}
  \Phi(f_t,s) 
    &= \int_\dR |x|^s f(t x)\, dx \\
    &= |t|^{-s} \Phi(f,s) .
\end{align*}
It follows that  
\[
  \Phi(\widehat{f_t},s) = |t|^{-1} \Phi(\widehat{f_{t^{-1}}},s) = |t|^{s-1} \Phi(\widehat f,s) .
\]
The distributions $f\mapsto \Phi(\widehat f,s)$ and $f\mapsto \Phi(f,1-s)$ 
satisfy the same transformation properties. There is a general theorem of 
uniqueness of relatively invariant distributions on homogeneous spaces. It 
implies that the two distributions coincide up to a constant $c=c(s)$. 
The theory of prehomogeneous vector spaces provides a way to generalize this. 

There is a simpler proof of the functional equation that comes from a clever 
choice of $f$. Pick $f_0\in C_c^\infty(\dR\smallsetminus 0)$ and put 
$f=\frac{d}{dx} f_0 = f_0'$. By an elementary argument, we can prove 
\[
  \widehat f(y) = \widehat{f_0}(y) = y \widehat{f_0}(y) .
\]
This implies $f(0) = \widehat f(0) = 0$. From the Poisson summation formula, 
we get 
\[
  \sum_{x\in \dZ\smallsetminus 0} f(x) = \sum_{y\in \dZ\smallsetminus 0} \widehat f(y) .
\]
This implies $Z(f,s) = Z_+(f,s) + Z_+(\widehat f,1-s)$, which is entire. 
Similarly 
\[
  Z(f,s) = \zeta(s) \Phi(f,s) = \zeta(s) (s-1) \Phi(f_0,s-1) .
\]
For all $s$, there exists $f_0$ such that $\Phi(f_0,s-1)\ne 0$, hence 
$\zeta(s)(s-1)$ is entire. 





\subsection{Generalizing the result}

\begin{defi}[Sato] % M. Sato
Let $G$ be an algebraic group over a field $k$. A finite-dimensional 
representation $V$ of $G$ is a \emph{prehomogeneous vector space} if 
there exists $x\in V_{\bar k}$ such that the orbit 
$G_{\bar k}\cdot x\subset Z_{\bar k}$ is Zariski-open. 
\end{defi}

We say that a non-constant $P\in k[V]$ is a \emph{relative invariant 
polynomial} if there exists $\chi\in \character^\ast(G)$ such that 
$P(g\cdot x) = \chi(g) P(x)$ for all $g\in G$, $x\in V$. 

Let $(G,V)$ be a prehomogeneous vector space defined over $\dR$. Sato proved 
that if $P\in \dR[V]$ is a relative invariant, then 
\[
  \Phi^{(i)}(f,s) = \int_{V_\dR^{(i)}} |P(x)|^s f(x)\, dx
\]
has analytic continuation and satisfies a functional equation. In \cite{ss74}, 
Sato and Shintani proved that there is a zeta-function associated to $(G,V)$. 

The following are basic examples of prehomogeneous vector spaces: 

\begin{enonce}[remark]{Example}
Let $G=\generallinear_1$, $V=\dA^1$. Then the natural action of $G$ on $V$ 
has Zariski-open orbits. The function $P(x)=x$ leads to the standard Riemann 
zeta function $\zeta(s)$. 
\end{enonce}

\begin{enonce}[remark]{Example}
If $(G,V)$ is $\generallinear_2$ acting in binary cubics and 
$P(x)=\discriminant(x)$, then the associated zeta function is 
$\xi^\pm(s)$. 
\end{enonce}





\subsection{Shintani zeta function}

Recall that for $G=\generallinear_2$ and $V$ the space of binary cubics, 
the associated relative invariant polynomial is $P(x)=\discriminant(x)$. Let 
\[
  V' = \{x\in V:P(x)\ne 0\} ;
\]
this consists of $x\in V$ having no multiple roots in $\dP^1$. Let 
$S=\{x\in V:P(x)=0\} = V\smallsetminus V'$. Then $V_\dC'$ is a single 
$G_\dC$-orbit in $V_\dC$. Over the real numbers, $V_\dR'$ breaks up into two 
orbits $V_\dR^+\cup V_\dR^-$ corresponding to $P(x)>0$ and $P(x)<0$. 

Define a local zeta function by 
\[
  \Phi^\pm(f,s) = \int_{V_\dR^\pm} |P(x)|^{s-1} f(x)\, dx ;
\]
this converges when $\Re s>1$. For $x,y\in V_\dR$, define 
\[
  \langle x,y\rangle = x_1 y_4 - \frac 1 3 x_2 y_3 + \frac 1 3 x_3 t_2 - x_4 y_1 .
\]
This bilinear form is invariant in the sense that 
$\langle g x,g^\ast y\rangle = \langle x,y\rangle$, where 
$g^\ast = (\det g)^{-1} g$. We can use this form to identify $V_\dR$ with its 
dual, and define a Fourier transform 
\[
  \widehat f(y) = \int_{V_\dR} f(x) e^{2\pi i \langle x,y\rangle} \, dx .
\]

\begin{prop}
1. The function 
\[
  \frac{1}{\Gamma(s)^2 \Gamma(s-\frac 1 6) \Gamma(s+\frac 1 6)} \Phi^\pm(f,s) 
\]
is entire. 

2. There exists $M(s)$ such that the following functional equations hold:
\begin{align*}
  \Phi^+(\widehat f,s) &= M(s) \Phi^+(f,1-s) \\
  \Phi^-(\widehat f,s) &= M(s) \Phi^-(f,1-s) .
\end{align*}
\end{prop}
\begin{proof}
There exists a differential operator $Q(\frac{\partial}{\partial x})$ such that 
\begin{align*}
  Q\left(\frac{\partial}{\partial x}\right) e^{\langle x,y\rangle} &= P(y) e^{\langle x,y\rangle} \\
  Q\left(\frac{\partial}{\partial x}\right) P(x)^s &= b(s) P(x)^{s-1} ,
\end{align*}
where $b(s) = s^2 (s-\frac 1 6)(s+\frac 1 6)$. The rest is routine. 
\end{proof}





\subsection{Proof of analytic continuation and functional equation for \texorpdfstring{$\xi^\pm(s)$}{xi}}

We define a ``extended zeta function'' by 
\begin{align*}
  Z(f,s) 
    &= \int_{G(\dR)/G(\dZ)} |\det g|^{2 s} \sum_{\substack{x\in V(\dZ) \\ P(x)\ne 0}} \Phi(g x)\, dg \\
    &= \sum_{\substack{x\in G(\dZ)\backslash V(\dZ) \\ P(x)\ne 0}} \int_{G(\dR)} |P(g x)|^s f(g x)\, d g .
\end{align*}
Because of the invariance of the measure $d g$, the integrand on the second 
line depends only on the $G(\dR)$-orbit of $x$. But there are only two such 
orbits! 

In general, if $\varphi$ is a function on $V(\dR)$, we have a formula 
\[
  \int_{G(\dR)} \varphi(g y)\, dy = \frac{m_\pm}{2\pi} \int_{G(\dR)\cdot x} \varphi(y)\, \frac{dy}{|P(y)|} ,
\]
where $m_\pm\in \{2,6\}$ is the degree of the covering $G_\dR \to V_\dR^\pm$ 
defined by $g\mapsto g x$. Returning to our definition of $Z(f,s)$, we see that 
\begin{align*}
  \int_{G(\dR)} |P(g x)|^s f(g x)\, dg 
    &= \frac{m_\pm}{2\pi} \int_{G(\dR)\cdot x} |P(y)|^s f(y)\, \frac{dy}{|P(y)|} \\
    &= \frac{m_\pm}{2\pi} \Phi^\pm(f,s) .
\end{align*}
Thus we have 
\[
  Z(f,s) = \begin{pmatrix}\xi^+(s) & \xi^-(s)\end{pmatrix} \begin{pmatrix} \frac{3}{\pi} \Phi^+(f,s) \\ \frac{1}{\pi} \Phi^-(f,s) \end{pmatrix} . 
\]
Now choose $f_0\in C_c^\infty(V_\dR')$ and set 
$f=Q(\frac{\partial}{\partial x}) f_0$. We get 
$\widehat f(y) = P(y) \widehat{f_0}(y)$, which implies 
$f|_{S_\dR} = \widehat f|_{S_\dR} = 0$, where $S$ is the singular set. It 
follows that 
\[
  Z(f,s) = Z_+(f,s) + Z_+(\widehat f,1-s) = Z(\widehat f,1-s) ,
\]
an entire function. From an earlier proposition, we get the functional 
equation. By a similar argument, $\xi^\pm(s) b(s-1)$ is an entire function. 
Since $b(s-1) = (s-1)^2 (s-\frac 5 6)(s-\frac 7 6)$, this recovers our main 
proposition on $\xi^\pm$. 

The hard part is to analyze the integral 
\begin{align*}
  X(f,s) &- Z_+(f,s) - Z_+(\widehat f,1-s) \\
    &= \int_{\substack{G(\dR)/G(\dZ) \\ |\det|\leqslant 1}} |\det g|^{2 s} \left(|\det g|^{-2} \sum_{y\in V_\dZ^\ast\cap S} \widehat f(g^\ast y) - \sum_{x\in V_\dZ\cap S} f(g x)\right)\, dg .
\end{align*}
There are three types of singular points. Namely, 
\[
  S=\{0\} \cup \{\text{triple root}\}\cup \{\text{distinct double root and single root}\} .
\]
We can compute 
\begin{align*}
  X(f,s) &- Z_+(f,s) - Z_+(\widehat f,1-s) \\
    &= \left(\frac{\widehat f(0)}{2 s-2} - \frac{f(0)}{2 s}\right) \volume(G_\dR^1/G_\dZ) + \int \cdots \left(\sum_{y\ne 0} \cdots - \sum_{x\ne 0} \cdots\right)\, dg .
\end{align*}
In the cubic case, this computation is carried out in \cite{s72}, and the 
quartic case is done in \cite{y92}. The general case remains open. We can 
impose congruence conditions, obtaining:  
\[
  \sum_{x\in x_0+N V_\dZ} f(x) = \frac{1}{N^4}\sum_{y\in \frac 1 N V_\dZ^\ast} e^{2\pi i\langle x_0,y\rangle} \widehat f(y) .
\]





% !TEX root = sms.tex

\section{How to count rings and fields II}\label{sec:bhargava-ii}
\thanksauthor{Manjul Bhargava}





d


% !TEX root = sms.tex

\section{Heuristics for number field counts and applications to curves over finite fields}\label{sec:wood-iv}
\thanksauthor{Melanie Matchett Wood}





d


% !TEX root = sms.tex

\section{Moduli space of rings}\label{sec:poonen}
\thanksauthor{Bjorn Poonen}





d


% !TEX root = sms.tex

\section{Zeta function methods}
\thanksauthor{Frank Thorne}





\subsection{Motivation}

The question is: what are zeta functions good for? Let $N_3^\pm(X)$ be the 
number of cubic fields $K$ with $0<\pm\discriminant(K)<X$. Define 
\begin{align*}
  C^\pm &= \begin{cases} 1 & + \\ 3 & - \end{cases} \\
  K^\pm &= \begin{cases} 1 & + \\ \sqrt 3 & -\end{cases}
\end{align*}

\begin{theo}
The following holds: 
\[
  N_3^\pm(X)=C^\pm \frac{1}{12\zeta(3)}X + K^\pm \frac{4\zeta(1/3)}{5\Gamma(2/3)^3\zeta(5/3)} X^{5/6} + O(X^{2/3+\epsilon}) .
\]
\end{theo}

We would like to understand how zeta functions can be used to provide such 
good error terms. 





\subsection{Definitions}

As we have done before, put 
\begin{align*}
  V(\dZ) &= \{a u^3 + b u^2 v + c u v^2 + d v^3:a,b,c,d\in \dZ\} \\
  \widehat V(\dZ) &= \{\cdots : 3\mid b,c\} .
\end{align*}
The group $\generallinear_2(\dZ)$ acts on both of these via 
\[
  (\gamma\cdot f)(u,v) = \frac{1}{\det\gamma}f\left(\begin{pmatrix} u & v\end{pmatrix}\cdot \gamma\right) .
\]
\begin{theo}
There is a natural bijection 
\[
  \generallinear_2(\dZ)\backslash V(\dZ) \iso \{\text{cubic rings}\}/\sim .
\]
\end{theo}

\begin{defi}[Shintani]
Put 
\begin{align*}
  \xi^\pm(s) 
    &= \sum_{x\in \generallinear_2(\dZ)\backslash V^\pm(\dZ)} \frac{1}{\#\stabilizer(x)}|\discriminant(x)|^{-s} \\
    &= \sum_{\substack{R\text{ cubic ring} \\ 
  \pm \discriminant(R)>0}} \frac{1}{\#\automorphism(R)} |\discriminant(R)|^{-s} . 
\end{align*}
\end{defi}

\begin{theo}[Shintani]
The functions $\xi^\pm(s)$ have analytic continuation to $\dC$ except for 
poles at $s=1,\frac 5 6$, explicit residue formulas at these poles, and a 
functional equation 
\[
  \begin{pmatrix} \xi^+(1-s) \\ \xi^-(1-s) \end{pmatrix} = \Gamma\left(s-\frac 1 6\right)\Gamma(s)^2 \Gamma\left(s+\frac 1 6\right) \frac{3^{6 s-2}}{2\pi^{4 s}} \begin{pmatrix} \sin(2\pi s) & \sin(\pi s) \\ 3\sin(\pi s) & \sin(2\pi s)\end{pmatrix} \begin{pmatrix} \widehat \xi^+(s) \\ \widehat\xi^-(s) \end{pmatrix} 
\]
where $\widehat\xi^\pm$ are defined in terms of $\widehat V(\dZ)$ instead of 
$V(\dZ)$. 
\end{theo}
This was proved in \cite{s72}. 





\subsection{How analytic number theorists count}

We'll see explicitly how analytic properties of $\xi^\pm$ translate into 
asymptotic estimates for the number of cubic rings. 

\begin{enonce}{Principle}[Perron's formula]
Given any Dirichlet series $B(s)=\sum_{n\geqslant 1} b(n) n^{-s}$ which is 
absolutely convergent for $\Re s=2$, then 
\[
  \sum_{n\leqslant X} b(n) = \frac{1}{2\pi i}\int_{2-i\infty}^{2+i\infty} B(s) X^s \, \frac{\mathrm{d}s}{s} .
\]
\end{enonce}

For the $b(n)$ completely arbitrary, this is not very helpful. The idea is: for 
specific $b(n)$, shift the contour of this integral. For example, if we are 
trying to count integers less than $X$, apply Davenport's lemma to conclude 
that there are $X+O(1)$ integers between $0$ and $X$. We define 
\[
  \zeta(s) = \sum_{n\geqslant 1} n^{-s} .
\]
By Perron's formula, we get 
\[
  \sum_{1\leqslant n<X} 1 = \frac{1}{2\pi i} \int_{2-i\infty}^{2+i\infty} \zeta(s) X^s\frac{\mathrm{d}s}{s} .
\]
See if you can spot the mistake in the following computation: 
\begin{align*}\tag{$\ast$}\label{eq:diverge}
  \sum_{n\leqslant X} 1 
    &= \residue_{s=1} + \residue_{s=0}\left(\zeta(s)X^s\frac{\mathrm{d}s}{s}\right) + \frac{1}{2\pi i}\int_{-1-i\infty}^{1+i\infty} \zeta(s) X^s\frac{\mathrm{d}s}{s} \\
    &= X+\zeta(0) + (\text{error}) .
\end{align*}
Does the integral in \eqref{eq:diverge} converse? For this, we need some bounds 
on $\zeta$. 

\begin{prop}
If $\sigma<0$, we have $\zeta(-\sigma+i t) \ll (1+|t|)^{1/2+\sigma}$. 
\end{prop}

\begin{enonce}{Exercise}
Use the functional equation and Stirling's approximation to prove the 
Proposition. 
\end{enonce}

Inside the critical strip, controlling the behavior of $\zeta$ is a large 
problem. But outside $\{0<\Re s<1\}$, things are relatively straightforward. 

We can use the above Proposition to show that the integral appearing in 
\eqref{eq:diverge} diverges. 





\subsection{The Landau method}

A big principle in analytic number theory is that it is important to work with 
``smooth sums.'' 

\begin{prop}
Let $b:\dN\to \dC$ and $B(s)=\sum b(n) n^{-s}$. Then 
\begin{align*}
  \sum_{n<X} b(n)\left(1-\frac n X\right) &= \frac{1}{2\pi i} \int_{2-i\infty}^{2+i\infty} B(s) \frac{X^s}{s(s+1)}\, \mathrm{d}s \\
  \frac 1 2\sum_{n<X}\left(1-\frac n X\right)^2 &= \frac{1}{2\pi i}\int_{2-i\infty}^{2+i\infty} B(s) \frac{X^s}{s(s+1)(s+2)}\, \mathrm{d} s.
\end{align*}
\end{prop}

\begin{enonce}{Exercise}
Prove and generalize the Proposition. 
\end{enonce}

These are all special cases of Mellin inversion. Using the Proposition, we 
continue our counting of integers: 
\[
  \sum_{n<X} (X-n) = \frac{X^2}{2} - X + \frac{1}{2\pi i} \int_{-1-i\infty}^{1+i\infty} \zeta(s) \frac{X^{1+s}}{s(s+1)}\, \mathrm{d} x .
\]
The integral is $O(X^{1/2+\epsilon})$. Moreover, 
\[
  \sum_{n<X+Y} (X+Y-n) = \frac{X+Y}{2} - (X+Y) + O((X+Y)^{1/2+\epsilon}) .
\]
Subtract the first equation from the second, to conclude that 
\[
  \sum_{n<X} 1 = X+O(X^{1/4+\epsilon}) .
\]
A more careful analysis of the integrals gets better error terms. 





\subsection{Why does the zeta function have such good analytic properties?}

\begin{defi}[Shintani]
The \emph{global zeta function} is, for ``nice test functions'' 
$f:V_\dR\to \dC$, 
\[
  Z(f,s) = \int_{\generallinear_2(\dR)/\generallinear_2(\dR)} |\det g|^{2 s} \left(\sum_{x\in V(\dZ)\smallsetminus S} f(g x)\right)\, \mathrm{d} g ,
\]
where $S=\{x\in V(\dZ):\discriminant(x)=0\}$. 
\end{defi}

\begin{prop}
We have the following decomposition:
\[
  Z(f,s) = \frac{1}{4\pi} \xi^+(s) \int_{V^+(\dR)} |\discriminant(x)|^{s-1} f(x)\, \mathrm{d}x + \frac{\xi_-(s)}{2\pi} \int_{V^-(\dR)} |\discriminant(x)|^{s-1} f(x)\,\mathrm{d} x .
\]
\end{prop}

Recall from Bhargava's lectures that the number of irreducible 
$\generallinear_2(\dZ)$-orbits in $G(\dZ)$ with $|\discriminant|<X$ is 
\[
  C \frac{\displaystyle\int_\cF\#\{x\in g B\cap V(\dZ)^\mathrm{irr}:|\discriminant(x)|<X\}\,\mathrm{d}g}{\displaystyle\int_B |\discriminant(b)|^{-1}\, \mathrm{d} v} .
\]
The sieve gives the estimate 
\[
  N_\mathrm{max}^\pm(X) = \sum_{q\geqslant 1} \mu(q) N^\pm(q,x) 
\]
where $N^\pm_\mathrm{max}(X)$ is the number of maximal cubic rings $R$ with 
$0<\pm \discriminant(R)<X$ and 
$N^\pm(q,X)$ is the number of cubic rings ``nonmaximal at $q$.'' Define the 
$q$-nonmaximal zeta functions: 
\[
  \xi_q^\pm(s) = \sum_{x\in \generallinear_2(\dZ)\backslash V^\pm(\dZ)} \frac{1}{\#\stabilizer(x)} \Phi_q(x) |\discriminant(x)|^{-s} ,
\]
where $\Phi_q(x)$ is the characteristic function of the set of cubic forms 
nonmaximal at $q$. We get 
\[
  \widehat\xi_q^\pm(s) = q^{8 s-8} \sum_{x\in \generallinear_2(\dZ)\backslash \widehat V_\dZ} \frac{1}{\#\stabilizer(x)}\widehat\Phi_q(x) |\discriminant(x)|^{-s} ,
\]
where 
\[
  \widehat\Phi_q(x) = \sum_{y\in V(\dZ/q^2)} \Phi_q(y) \exp\left(\frac{2\pi i[x,y]}{q^2}\right) ,
\]
and 
\[
  [x,y] = x_r y_1 - \frac 1 3 x_3 y_2 + \frac 1 3 x_2 y_3 - x_1 y_4 .
\]





% !TEX root = sms.tex

\section{Counting Artin representations and modular forms of weight one}
\thanksauthor{Eknath Ghate}





We'll start by spending a little bit of time explaining what modular forms are. 





\subsection{Brief introduction to modular forms}

Let $f = \sum_{n\geqslant 1} a_n q^n$ be a primitive cusp form of weight $1$, 
level $N\geqslant 1$, and character 
$\varepsilon:(\dZ/N)^\times \to \dC^\times$. Then $f$ is a holomorphic function 
$f:\fH\to \dC$, where $\fH=\{z\in \dC:\Im z>0\}$ is the upper-half plane. The 
function $f$ transforms by 
\[
  f(\gamma z) = \varepsilon(d) (c z+d) f(z) ,
\]
for all matrices $\begin{pmatrix} a & b \\ c & d \end{pmatrix}\in \Gamma_0(N)$. 
Moreover, $f$ is ``holomorphic at cusps.'' 

Deligne and Serre proved that to asuch a modular form is attached a continuous 
Galois representation $\rho_f:G_\dQ \to \generallinear_2(\dC)$ such that 
$\trace \rho_f(\frobenius_\ell) = a_\ell$ and 
$\det\rho_f(\frobenius_\ell) = \varepsilon(\ell)$ for all primes $\ell\nmid N$. 
Since $G_\dQ$ is compact and totally disconnected, the image of $\rho_f$ is 
finite. Once we projectivize, the image of 
$\widetilde\rho_f:G_\dQ\to \projectivegenerallinear_2(\dC)$ is one of 
\begin{align*}
  D_m && \text{dihedral group of order }2m \\
  A_4 && \text{tetrahedral} \\
  S_4 && \text{octahedral} \\
  A_5 && \text{icosahedral}
\end{align*}
We call the representations with projective image $A_4,S_4,A_5$ exotic. They 
are quite rare, with the first one occurring when $N=800$. 

\begin{enonce}{Conjecture}
The number of exotic forms of prime level $N$ is $O(N^\epsilon)$. 
\end{enonce}

If the level $N$ is prime, then only octahedral or icosahedral images occur. 
In this talk we will restrict to counting octahedral forms of prime level. Here 
is recent progress on the conjecture:
\begin{center}
\begin{tabular}{c|c}
person & bound \\ \hline
Duke & $O(N^{7/8+\epsilon})$ \\
Wang & $O(N^{5/6+\epsilon})$ \\
M-V & $O(N^{4/5+\epsilon})$ \\
Ganguly & $O(N^{3/4+\epsilon})$ \\
Ellenberg & $O(N^{2/3+\epsilon})$
\end{tabular}
\end{center}

\begin{theo}[Bhargava-Ghate]
Let $N_\mathrm{oct}^\mathrm{prime}(X)$ be the number of octohedral forms of 
prime level $<X$. Then 
\[
  N_\mathrm{oct}^\mathrm{prime}(X) = O(X/\log X) .
\]
\end{theo}
This is proven in \cite{bg09}. So on average, the number of octahedral forms 
of prime level is bounded by a constant. 

\begin{proof}
The idea is to ``count forms by counting forms.'' That is, we use the fact 
that octahedral modular forms of weight one correspond to Galois 
representations, which in turn correspond go quartic number fields, which 
come from quartic forms. 

Step 1. It is enough to count (linear) Galois representations. The Artin 
conjecture says that there is a bijection between octahedral forms and 
isomorphism classes of $\rho:G_\dQ \to \generallinear_2(\dC)$ with 
$\rho(G_\dQ)\simeq S_4$ and $\det\rho(c)=-1$. The direction $\Rightarrow$ was 
proved in \cite{ds74}, while $\Leftarrow$ is the Langlands-Tunnell theorem 
proved in \cite{l80,t81}.

One uses Serre's conjecture (proved in full generality in 
\cite{kw09-i,kw09-ii}) to prove the Artin conjecture. Choose 
$\cO\subset \dC$, the ring of integers in a number field such that 
$\rho(G_\dQ)\subset \generallinear_2(\dC)$. We get a commutative diagram 
\[\xymatrix{
  G_\dQ \ar[r]^-\rho \ar[dr]_-{\bar\rho} 
    & \generallinear_2(\cO) \ar[d] \\ 
  & \generallinear_2(\overline{\dF_p}) .
}\]
Serre's conjecture predicts that if $\bar\rho$ is odd and irreducible, then 
$\bar\rho\sim \bar\rho_g$ for a modular form $g$, where 
$g\in S_1(\Gamma_0(N),4)$. Write $g=\sum b_n q^n$. Then 
$a_\ell\equiv b_\ell\pmod p$ for all $\ell\nmid N$. Since this works for 
infinitely many $p$, there exists $g$ such that $a_\ell=b_\ell$. Thus 
$\rho$ is modular. 

Step 2. It is enough to count projective Galois representations. There is 
a surjection from isomorphism of odd $\rho:G_\dQ\to \generallinear_2(\dC)$ 
with $\widetilde\rho(G_\dQ)\simeq S_4$, to isomorphism classes of 
$\widetilde\rho:G_\dQ\to \projectivegenerallinear_2(\dC)$ with 
$\widetilde\rho(G_\dQ)\simeq S_4$ and $\widetilde\rho(c)\ne 1$. Surjectivity 
follows from $\h^2(G_\dQ,\dC^\times)=0$. The map is not injective: if 
$\chi$ is any character, Tate proved that 
$\rho\otimes\chi\mapsto\widetilde\rho$. But we can control the map when $N$ is 
square-free. If $p\| N$, then 
\[
  \rho|_{I_p} \sim \begin{pmatrix} \varepsilon_p \\ & 1 \end{pmatrix} .
\]
Choose $\chi$ with $\chi^{12}=1$, and apply this with $\rho\otimes\chi$ 
instead of $\rho$. The new character is $(\varepsilon\chi^2)^{12}$. When 
$N=p$, there are only two such $\chi$ so that $\rho\otimes\chi$ map to the 
same $\widetilde\rho$. We get that $\varepsilon=\varepsilon_p$ is odd. Some 
technical manipulations yield that $\rho\mapsto \widetilde\rho$ is 
2-to-1 when the level $N$ is prime. 

Step 3. It is enough to count quartic fields with Galois closure having group 
$S_4$. A projective representation 
$\widetilde\rho:G_\dQ \to \projectivegenerallinear_2(\dC)$ with 
$\widetilde\rho(G_\dQ)\simeq S-4$ and $\widetilde\rho(c)\ne 1$ cuts out a field 
$K$ over $\dQ$ that is not totally real. 

There is a key technical problem here. We want to count modular forms by level, 
but we usually count number fields by discriminant. Let $N$ be the level of 
$f$. If $K$ is the field corresponding to $f$ and $D=\discriminant(K)$, then 
we might have $N\ne D_K$. However, a prime $p\mid N_f$ if and only if 
$p\mid D_K$. Note that if $p\geqslant 5$, then either $p\| N_f$ or 
$p^2\| N_f$. However, it is possible for $p^3\| D_K$. Consider the following 
ramification table in the octahedral case (for minimal forms): 
\begin{center}
\begin{tabular}{c|c|c|c|c|c}
$I_p$ & $G_p$ & ram.~in $K$ & $D_K$ & $N_f$ & $p\equiv$ \\ \hline
$(12)$     & $I_p$       & $1^2 11$ & $p$ & $p$ \\
$(12)$     & $(12),(34)$ & $1^2 2$  & $p$ & $p^2$ \\
$(12)(34)$ & $I_p$       & $1^2 1^{\underline 2}$ & $p^2$ & $p$\\
$(13)(24)$ & $(1234)$    & $2^2$    & $p^2$ & $p$ \\
$(12)(34)$ & $V_4$       & $2^2$    & $p^2$ & $p^2$ \\
$(12)(34)$ & $(12)(34)$  & $1^2 1$  & $p^2$ & $p^2$ \\
$(123)$    & $I_p$       & $1^3 1$  & $p^2$ & $p$   & $1\pmod 3$ \\
$(123)$    & $S_3$       & $1^3 1$  & $p^2$ & $p^2$ & $2\pmod 3$ \\
$(1234)$   & $I_p$       & $1^4 $   & $p^3$ & $p$   & $1\pmod 4$ \\
$(1234)$   & $D_4$       & $1^4$    & $p^3$ & $p^2$ & $3\pmod 4$ 
\end{tabular}
\end{center}
[\ldots some notation I don't understand\ldots]. 
Five times, the power of $p$ in $D_K$ is at most the power of $p$ in 
$N_f$. The other five possibilities, this fails. For $4/5$ of the time we can 
still control things. The other possibility is $p\equiv 1\pmod 3$. 

We use some facts about quartic fields. Consider field extensions 
$E-\supset K_6\supset K_3\supset \dQ$ corresponding to the inclusions 
[\ldots missed this part\ldots]. 

The extension $K_6/K_3$ has Galois group $S_4$ if and only if 
\begin{enumerate}
  \item $K_3/\dQ$ has Galois group $S_3$ 
  \item $\norm_\dQ(\discriminant(K_6/K_3))=n^2$ for $n$ square-free 
  \item $\discriminant(K_4) = \discriminant(k_3) n^2$ 
  \item $K_6/K_3$ ramifies if and only if $p=1^4$ or 
    $1^2 1^2$ or $2^2$. 
\end{enumerate} 
We use the following theorem of Serre to simplify the table: 
\begin{center}
\begin{tabular}{c|c|c|c|c}
Ram.~in $K_4$ & $|\discriminant(K_4)|$ & level & $|\discriminant(K_3)|$ & $n$ \\ \hline
$1^2 11$ & $p$ & $p$ & $p$ & $1$ \\
$1^4$ & $p^3$ & $p$ & $p$ & $p$ 
\end{tabular}
\end{center}

Step 4. Use Bhargava's counting results in the quartic case, as well as some 
sieve methods. Recall, from \cite{b04} that isomorphism classes of maximal
$S_4$-quartic orders are in bijection with 
$\generallinear_2(\dZ)\times \generallinear_3(\dZ)$-classes of pairs of 
irreducible ternary quadratic forms $(A,B)$. From \cite{b05}, we the number 
$N_4(X)$ of number fields of $S_4$-quartic fields of $|\discriminant|<X$ is 
$O(X)$. A technical modification shows that the number $N_4^\mathrm{prime}(X)$ 
of $S_4$-quartic fields of prime level $<X$ is $O(X/\log X)$. As a 
corollary, 
\[
  \sum_{\substack{0<|\discriminant(K_3)|<X \\ \text{prime}}} h_2^\ast(K_3) \leqslant C\frac{X}{\log X} .
\]
We can finally count the number of octahedral of prime level $<X$ modular 
forms.  
\end{proof}

\begin{theo}[Serre]
In prime level, the discriminant of $K_4$ is either $p^3$ or $-p$. 
\end{theo}





% !TEX root = sms.tex

\section{Binary quartic forms: bounded average rank of elliptic curves I}\label{sec:shankar-i}
\thanksauthor{Arul Shankar}





\subsection{Introduction}

Recall that every elliptic curve over $\dQ$ can be written as 
$y^2=x^3+A x+B$ for $A,B\in \dQ$. 

\begin{theo}[Mordell]
The abelian group $E(\dQ)$ is finitely generated. 
\end{theo}

So we can write $E(\dQ)=T\oplus \dZ^r$, where $T$ is a finite abelian group, 
and $r=\rank E$ is the \emph{rank} of $E$. We are going to study the average 
rank of elliptic curves. To do this, we need to order elliptic curves in some 
way. 

The elliptic curve $E_{A<B}:y^2=x^3+A x+B$ is isomorphic to 
$E_{n^4 A,n^6 B}:y^2=x^3+n^4 A x+n^6 B$ for all $n\in \dQ^\times$. So we can 
assume $A,B\in \dZ$. If we assume that $p^4\mid A\Rightarrow p^6\nmid B$, then 
the $A,B$ are unique. Thus there is a bijection between isomorphism classes of 
elliptic curves over $\dQ$ and 
\[
  \cE = \{E_{A,B}:(A,B)\in \dZ^2\text{ and }p^4\mid A\Rightarrow p^6\nmid B\} .
\]
We could also look at subfamilies of $\cE$ cut out by (possibly infinitely 
many) congruence conditions. We could also look at ``thin'' families consisting 
of all quadratic twists of some elliptic curve. 





\subsection{Ordering elliptic curves}

To talk about averages, we need to order elliptic curves in some way. The most 
obvious invariants to use are the discriminant and conductor. We have 
\[
  \discriminant(E_{A,B}) = \Delta(E_{A,B}) = 4 A^3 - 27 B^2 .
\]
The problem with ordering elliptic curves by discriminant is that we don't know 
how to count the number of elliptic curves with discriminant bounded by $X$. 
Essentially, the region $\{(x,y)\in \dR^2:4 x^2 - 27 y^3<X\}$ is noncompact, 
which makes the counting problem very hard. What is easier is to order elliptic 
curves by (naive) height: 
\[
  H(E_{A,B}) = \max\{|4 A^3|, 27 B^2\} .
\]

Let $f$ be a function on elliptic curves. The \emph{average} of $f$ is 
\[
  \average(f) = \lim_{X\to \infty} \frac{\sum_{H(E)<X} f(E)}{\sum_{H(E)<X} 1} .
\]
It's not hard to show that $\average(\# T)=1$. That is, on average an elliptic 
curve has no nontrivial torsion. This follows from Hilbert irreducibility. 

Our question is: what can we say about the average rank? 

\begin{conjecture}[Goldfeld, Katz-Sarnak]
$\average(\rank) = \frac 1 2$. Moreover, $50\%$ of elliptic curves have rank 
$0$ and $50\%$ have rank $1$. 
\end{conjecture}

The conjecture was originally made with elliptic curves ordered by conductor, 
but ordering by conductor, height and discriminant are all expected to yield 
the same average. 

Given an elliptic curve $E$, we can define an $L$-function which we denote 
$L_E(s)$. The \emph{completed $L$-function} $L_E^\ast(s)$ satisfies a 
functional equation 
\[
  L_E^\ast(s) = \omega(E) L_E^\ast(1-s), 
\]
where $\omega(E)$, the \emph{root number} of $E$, is $\pm 1$. The 
\emph{analytic rank} of $E$ is the order of vanishing of $L_E$ at 
$s=1/2$. The \emph{Birch and Swinnerton-Dyer conjecture} predicts that the 
analytic rank and algebraic rank of $E$ are the same. But the analytic 
rank of $E$ is also the analytic rank of $L_E^\ast$. So the BSD conjecture 
implies that $\rank(E)$ is even if and only if $\omega(E)=1$. It's expected 
that $\omega(E)$ is equidistributed, i.e.~half of elliptic curves have 
$\omega(E)=0$ and half have $\omega(E)=1$. Assuming BSD, it would follow 
that half of elliptic curves have even rank, and half have odd rank. We also 
expect the rank of an elliptic curve to be ``as small as it can get away 
with,'' which would force $100\%$ of elliptic curves with $\omega(E)=0$ to 
have rank zero, and $100\%$ of elliptic curves with $\omega(E)=1$ to have rank 
one. 

In \cite{ks99}, Katz and Sarnak studied the family of all $L$-functions of 
elliptic curves. Assuming GRH and BSD, the we have the following bounds on 
$\average(\rank)$:
\begin{center}
\begin{tabular}{c|c}
\cite{b92} & $\leqslant 2.14$ \\
\cite{h04} & $\leqslant 2$ \\
\cite{y06} & $\leqslant 1.79$
\end{tabular}
\end{center}

More recently, we have the following theorem proven in \cite{j02}. 

\begin{theo}[de Jong]
For the family of all elliptic curves over $\dF_q(t)$, the average rank is 
bounded above by $\frac 7 6+\epsilon(q)$, where $\epsilon(q)\to 0$ as 
$q\to \infty$. 
\end{theo}

The method is to bound $\average(\# \selmer_3)$. 





\subsection{Selmer groups}

The fundamental idea is that from the canonical short exact sequence 
\[
  0 \to E(\dQ)/p \to \selmer_p(E) \to \sha(E)[p] \to 0 ,
\]
we get a bound on $\rank E$ in terms of $\selmer_p(E)$. Note that 
$\#(E(\dQ)/p) \geqslant p^{\rank E}$. 

First, we want a parameterization of 2-Selmer elements of elliptic curves. 
More generally, if $\sigma\in \selmer_p(E)$. then we can think of $\sigma$ as a 
locally soluble $p$-covering of $E$. Such a covering is a twist of 
$[p]:E\to E$. It will be a genus-one curve $C$ isomorphic to $E$ over 
$\overline\dQ$, along with $C\to E$ such that the following diagram commutes:
\[\xymatrix{
  C \ar[d]^-\wr \ar[dr] \\
  E \ar[r]^-{[p]} 
    & E .
}\]
See \autoref{sec:7} for more details. The covering $C$ is \emph{soluble} if 
$C(\dQ)\ne \varnothing$, and it is \emph{locally soluble} if 
$C(\dQ_v)\ne\varnothing$ for all places $v$ of $\dQ$. Locally soluble 
$p$-coverings of $E$ are in natural bijection with $\selmer_p(E)$, and 
soluble $p$-coverings are in bijection with $E(\dQ)/p$. For the rest of this 
lecture, we will concentrate on $p=2$. 

It turns out that a locally soluble $2$-covering of $E$ yields a binary quartic 
form over $\dQ$. Conversely, a binary quartic form gives a $2$-cover. 

Let $V$ be the space of binary quartic forms. The group $\generallinear(2)$ 
acts on $V$ via 
\[
  (\gamma\cdot f)(x,y) = \frac{1}{(\det\gamma)^2}f\left(\begin{pmatrix} x & y \end{pmatrix} \cdot \gamma\right) .
\]
The center $Z(\generallinear_2)$ acts trivially, so the action descends to 
one of $\projectivegenerallinear(2)$ on $V$. The ring of invariants is 
freely generated by two elements $I,J$, which have degree $2$ and $3$ 
respectively in the coefficients of the form. 

\begin{theo}[Birch-Swinnerton-Dyer, Cremona-Fisher-Stoll]
There is a bijection between 2-Selmer elements and the quotient 
$\projectivegenerallinear_2(\dQ)\backslash V(\dQ)^\mathrm{ls}$, where 
$V(\dQ)^\mathrm{ls}$ is the subset of locally soluble forms, in which 
$(A,B)$ corresponds to $I=-3\cdot 2^6 A$ and $J=-272\cdot 2^6 B$, and 
$A(f)=-I(f)/3\cdot 2^4$ and $B(f) = -J(f)/27\cdot 2^6$. 
\end{theo}

We will write this as 
$\selmer_2(E_{A,B}) = \projectivegenerallinear_2(\dQ)\backslash V(\dQ)_{A,B}^\mathrm{ls}$. 
The identity element of $\selmer_2(E_{A,B})$ corresponds to the orbit of 
binary quadratic forms with a rational linear factor. 

We would like a parameterization of 2-Selmer elements that involves binary 
quartic forms with integral coefficients instead of just rational coefficients. 

\begin{lemm}[Birch, Swinnerton-Dyer]
If $f\in V(\dQ_p)$, then $A(f),B(f)\in \dZ_p$. If $f$ is $\dQ_p$-solvable, then 
$\projectivegenerallinear_2(\dQ_p)\cdot f\cap V(\dZ_p)\ne\varnothing$. 
\end{lemm}

\begin{theo}
If $f\in V(\dQ)$, then $A(f),B(f)\in \dZ$. If $f$ is locally soluble, then 
$\projectivegenerallinear_2(\dQ)\cdot f\cap V(\dZ)\ne\varnothing$. 
\end{theo}
\begin{proof}
For each prime, find $\gamma_p\in\projectivegenerallinear_2(\dQ_p)$ so that 
$\gamma_p\cdot f\in V(\dZ_p)$. Since $\projectivegenerallinear_2$ has class 
number $1$, there exists $\gamma\in \projectivegenerallinear_2(\dQ)$ so that 
$\gamma\cdot f\in V(\dZ_p)$ for all $p$, hence $\gamma\cdot f\in V(\dZ)$. 
\end{proof}

\begin{theo}[Birch, Swinnerton-Dyer and Cremona, Fisher, Stoll]
The set $\selmer_2(E_{A,B})$ is naturally in bijection with 
$\projectivegenerallinear_2(\dQ) \backslash V(\dZ)_{A,B}^\mathrm{ls}$. 
\end{theo}

Define the \emph{height} of a binary quartic form to be 
\[
  H(f) = \max\{4|A(f)|^3, 27 B(f)^2\} .
\]
So the goal is to count $\projectivegenerallinear_2(\dQ)$-equivalence classes 
of integral, locally soluble binary quartic forms with height bounded by $X$. 





\subsection{First step}

The goal is to count $\projectivegenerallinear_2(\dZ)$-orbits of 
$V(\dZ)_{H<X}^\mathrm{irr}$. The method is extremely similar to how Bhargava 
counted binary cubic forms. First we construct a fundamental domain $\cF_X$ for 
the action of $\projectivegenerallinear_2(\dZ)$ on $V(\dR)_{H<X}$. Next we 
estimate $\#\{\cF_X\cap V(\dZ)\}$ using averaging. 

We begin by finding a fundamental set for 
$\projectivegenerallinear_2(\dR)\backslash V(\dR)$. Over any field $k$ in 
which $6$ is invertible, 
$\projectivegenerallinear_2(k)\backslash V(k)_{A,B}^{k-\mathrm{sol}}$ is 
in bijection with $E_{A,B}(k)/2$. Given $A,B\in \dR$, the set 
$\projectivegenerallinear_2(\dR)\backslash V(\dR)_{A,B}^\mathrm{ls}$ is in 
bijection with $E_{A,B}(\dR)/2$. But the group of $\dR$-valued points of an 
elliptic curves is easy to understand. It is $\dZ/2\times S^1$ or $S^1$, 
depending on whether the discriminant is positive or negative. It follows that 
$E_{A,B}(\dR)/2$ has either $1$ or $2$ elements as $\Delta(E_{A,B})<0$ or 
$\Delta(E_{A,B})>0$. 

If $\Delta(E_{A,B})<0$, then 
$\projectivegenerallinear_2(\dR)\backslash V(\dR)_{A,B}$ is a singleton, and 
any $f$ in the set will have exactly $2$ real roots. If 
$\Delta(E_{A,B})>0$, there are two orbits, one consisting of forms with two 
real roots, and one consisting of forms with positive-definite binary quartic 
forms. Define 
\begin{align*}
  V(\dR)^{(0)} &= \{f\in V(\dR)\text{ with 4 real roots}\} \\
  V(\dR)^{(1)} &= \{f\in V(\dR)\text{ with 2 real roots}\} \\
  V(\dR)^{(2+)} &= \{f\in V(\dR)\text{ positive definite}\} .
\end{align*}
A fundamental set for $\projectivegenerallinear_2(\dR)\backslash V(\dR)^{(i)}$ 
for $i\in \{0,1,2+\}$ is $\{f\text{ having invariants }A,B\}$. We obtain 
\[
  \projectivegenerallinear_2(\dR)\backslash V(\dR)^{(i)}_{H<1} = R_1^{(i)} ,
\]
with, for example, 
\[
  R_1^{(0)} = \{x^3 y + A x y^3 + B y^4,(A,B)\in \dR^2,\Delta(E_{A,B})>0,H(E_{A,B})<1\} .
\]
For general height, we scale: 
\[
  \projectivegenerallinear_2(\dR)\backslash V(\dR)_{H<X}^{(i)} = X^{1/6} R_1^{(i)} = R_X^{(i)} .
\]

So $R_X^{(i)}$ is a fundamental domain for the action of 
$\projectivegenerallinear_2(\dR)$ on $V(\dR)_{H<X}^{(i)}$. We want a 
fundamental domain for the action of $\projectivegenerallinear_2(\dZ)$. Choose 
$\cF=\projectivegenerallinear_2(\dZ)\backslash \projectivegenerallinear_2(\dR)$; 
then $\cF\cdot R_X^{(i)}$ is an $n_i$-fold cover of 
$\projectivegenerallinear_2(\dZ)\backslash V(\dR)_X^{(i)}$. It turns out that 
$n_1=2$ and $n_0 = n_{2+} = 4$. 

It follows that 
\[
  \#\projectivegenerallinear_2(\dZ)\backslash V(\dZ)_{H<X}^{(i),\mathrm{irr}} = \frac{1}{n_i} \#(\cF\cdot R_X^{(i)}\cap V(\dZ)^\mathrm{irr}) .
\]
As with binary cubic forms, we average: 
\begin{align*}
  \#\projectivegenerallinear_2(\dZ)\backslash V(\dZ)_{H<X}^{(i),\mathrm{irr}} 
    &= \frac{1}{n_i \volume(G_0)} \int_{G_0} \#\left(\cF gR_X^{(i)} \cap V(\dZ)^\mathrm{irr}\right)\, \mathrm{d} g \\
    &= \frac{1}{n_i\volume(G_0)} \int_\cF \#\left(g G_0 R_X^{(i)}\cap V(\dZ)^\mathrm{irr}\right)\, \mathrm{d} g \\
    &= \frac{1}{n_i \volume(G_0)} \int_\cF \volume\left(g G_0 R_X^{(i)}\right)\, \mathrm{d} g + O(X^{3/4}) \\
    &= \frac{1}{n_i\volume(G_0)} \int_{G_0} \volume\left(\cF g R_X^{(1)}\right)\, \mathrm{d} g + O(X^{3/4}) \\
    &= \frac{1}{n_i} \volume\left(\cF R_X^{(i)}\right) + O(X^{3/4}) .
\end{align*}

We can summarize all of this in the following theorem. 

\begin{theo}[Bhargava, Shankar]
\[
  \#\left(\projectivegenerallinear_2(\dZ)\backslash V(\dZ)_{H<X}^{(i),\mathrm{irr}}\right) = \frac{1}{n_i} \volume\left(\cF R_X^{(i)}\right) + O(X^{3/4}) .
\]
\end{theo}

As an easy corollary, the average rank of elliptic curves is bounded. 
We can compute this as  
\[
  \frac{1}{n_i} \volume\left(\cF R_X^{(i)}\right) = \frac{1}{n_i} |J|\volume(\cF) \volume\left(R_X^{(i)}\right) + O(X^{5/6}) 
\]
So the number of elliptic curves with height $<X$ is some constant multiple of 
$X^{5/6}$. Thus $\average(\#\selmer_2)$ is bounded, whence $\average(\rank)$ is 
bounded. 

In \autoref{sec:shankar-ii}, we'll derive an explicit bound for 
$\rank(\average)$. 





% !TEX root = sms.tex

\section{Selmer groups and heuristics I}
\thanksauthor{Bjorn Poonen}





d


% !TEX root = sms.tex

\section{Rational points on curves}
\thanksauthor{Michael Stoll}





The goal is to discuss how one can show that a hyperelliptic curve has no 
rational points. 





\subsection{Hyperelliptic curves}

Let $k$ be a field of characteristic not $2$. 

\begin{defi}
A \emph{hyperelliptic curve} over $k$ is the smooth projective curve associated 
to an affine curve of the form $y^2=f(x)$ with $f\in k[x]$ squarefree (and 
$\deg f\geqslant 5$). 
\end{defi}

We will write $C:y^2=f(x)$; $C$ means the projective curve associated to 
the subvariety $V(y^2-f)$ of the affine plane. We can define the projective 
curve $C$ as follows: homogenize $f$ as $f(x)=F(x,1)$ with 
$F\in k[x,z]$ squarefree and homogeneous of even degree. Then 
$y^2=F(x,z)$ describes $C\subset \dP_{1,g+1,1}^2$ when $\deg F=2 g+2$. 

There are points at infinity. If 
$f(x)=f_{2 g+2} x^{2 g+2} + \cdots + f_1 x + f_0$, then 
$F(x,z) = f_{2 g+2} x^{2 g+2} + \cdots + f_0 z^{2 g+2}$. If $\deg f$ is odd, 
there is just one point $\infty=(1:0:0)$ at infinity. If $\deg f$ is even, 
then there are two points at infinity: $\infty_{\pm s} = (1:\pm s:0)$, where 
$f_{2 g+2}=s^2$. If $f_{2 g+2}$ is not a square in $k$, these points at 
infinity will not be defined over $k$. 

It is known that $C$ has genus $g$. The set of $k$-rational points of $C$ is 
\[
  C(k) = \{(\xi,\eta)\in k^2 : \eta^2 = f(\xi)\} \cup 
  \begin{cases} 
    \{\infty\} & \deg f\text{ odd} \\ 
    \{\infty_s,\infty_{-s}\} & \deg f\text{ even, $f_{2 g+2}$ a square} \\ 
    \varnothing & \text{otherwise} 
  \end{cases}
\]

The condition $\deg f\geqslant 5$ implies $g\geqslant 2$. Faltings' theorem 
tells us that $C(\dQ)$ is finite. Our motivating problem is: determine 
$C(\dQ)$ explicitly for given $C$. This is wide open at the present. 
Heuristically, we expect $100\%$ of hyperelliptic curves of genus $g$ to have 
no rational points. This needs some explanation. For a hyperelliptic curve 
$C:y^2=f$ defined over $\dQ$, we can assume $f\in \dZ[x]$. We can then order 
hyperelliptic curves by the height of $f$. 





\subsection{Local solubility}

Suppose we have some set $A$ which we want to prove is empty. One way to 
do this is to construct a map $A\to B$ and show that $B=\varnothing$. If, for 
example $A=C(\dQ)$, then for each place $v$ of $\dQ$ we have a natural 
injection $C(\dQ)\hookrightarrow C(\dQ_v)$. If $C(\dQ_v)=\varnothing$ for 
some $v$, then $C(\dQ)=\varnothing$. 

\begin{defi}
A curve $C$ is said to be \emph{everywhere locally soluble} if 
$C(\dR)\ne\varnothing$ and $C(\dQ_p)\ne\varnothing$ for all primes $p$. 
\end{defi}

Equivalently, $C$ is everywhere locally soluble if $C(\dA)\ne\varnothing$. 

\begin{theo}
If $C$ is not everywhere locally soluble, then $C(\dQ)=\varnothing$. 
\end{theo}

\begin{example}
Consider $C:y^2-x^6-x^2-17$. Then $C(\dR)=\varnothing$. 
\end{example}

\begin{example}
The curve $C:y^2=-x^6-3 x^5+4 x^4+2 x^3 +4 x^2 - 3 x-1$ has 
$C(\dQ_{11})=\varnothing$, because $C(\dF_{11})=\varnothing$. 
\end{example}

As a general principle, local questions are computable, whereas global 
questions are very hard. In our case, we have the following result. 

\begin{prop}
There is an algorithm that decides if a hyperelliptic curve over $\dQ$ is 
everywhere locally soluble or not. 
\end{prop}
\begin{proof}[Sketch of proof]
We need to take care of two problems: 
\begin{itemize}
  \item there are infinitely many places of $\dQ$  
  \item for each place $v$, the field $\dQ_v$ is uncountable 
\end{itemize}
It is easy to check whether $C(\dR)=\varnothing$. This is the case if and only 
if $f$ has no real roots, which happens exactly when $f$ is strictly positive 
or strictly negative. This is easy to decide. 
For fixed $p$, ``$C(\dQ_p)=\varnothing$'' reduces to a question modulo $p^n$ 
via Hensel's lemma. So for any completion $v$ of $\dQ$, it is a finite problem 
to check whether $C(\dQ_v)=\varnothing$. For $p$ sufficiently large, 
$p\nmid\discriminant(f)$, so we have $C(\dQ_p)\ne\varnothing$ via a theorem of 
Weil, namely $\#C(\dF_p)\geqslant p+1-2 g\sqrt p$. For 
$p\nmid\discriminant(f)$, the curve $C$ is smooth over $\dF_p$, so points in 
$C(\dF_p)$ lift to points in $C(\dQ_p)$. 
\end{proof}

A curve which is everywhere locally soluble does not necessarily have a 
solution in $\dQ$. For fixed $g\geqslant 2$, the everywhere locally soluble 
curves of genus $g$ have a density $\delta_g$. For example 
$\delta_2\approx 0.84$ and $\delta_g\to 1$ as $g\to \infty$. So we expect 
$100\%$ of hyperelliptic curves to have no rational points, but also for 
$100\%$ of hyperelliptic curves to be everywhere locally soluble. 





\subsection{Descent}

Consider $C:y^2=f(x)$ with $f(x)=f_1(x) f_2(x)$, with 
$\deg f_1,\deg f_2$ even and $f_1,f_2\in \dZ[x]$. Let $P=(\xi,\eta)\in C(\dQ)$. 
Then $f_1(\xi)\ne 0$, $f_2(\xi)\ne0$, or both. There exists a unique squarefree 
$d\in \dZ$, together with $\eta_1,\eta_2\in \dQ$, such that 
$f_1(\xi) = d\eta_1^2$, $f_2(\xi) = d\eta_2^2$. More geometrically, $P$ lifts 
to a rational point on $D_d:d y_1^2 = f_1(x),d y2^2=f_2(x)$ with 
$\pi_d:D_d\to C$ defined by $(x,y_1,y_2)\mapsto (x,d y_1 y_2)$. So we have 
reduced the problem of finding rational points on $C$ to that of finding 
rational points on the family $\{D_d:d\in \dZ\}$ of curves. 

If $p\mid d$, then $\overline\xi$ a common root of 
$\overline{f_1},\overline{f_2}\in \dF_p[x]$ implies 
$p\mid \resultant(f_1,f_2)\ne 0$. This is possible for only finitely many 
values of $d$. Let $T$ be the set of possible $d$. For each $d\in T$, we can 
use an algorithm to check if $D_d$ is everywhere locally soluble. If none of 
them are, then $C(\dQ)=\varnothing$. 

\begin{example}
Let $C:y^2=f_1 f_2$, where 
\begin{align*}
  f_1 &= -x^2-x-1 \\
  f_2 &= x^4 + x^3 + x^2 + x+2 .
\end{align*}
As an exercise, check that $C$ is everywhere locally soluble. We have 
$\resultant(f_1,f_2)=\pm 19$, so we can set $T=\{\pm 1,\pm 19\}$. If 
$d<0$, then $D_d(\dR)=\varnothing$. If $d\equiv 1\pmod 3$, then 
$D_d(\dF_3)=\varnothing$. It follows that $C(\dQ)=\varnothing$. 
\end{example}

This approach can be generalized to unramified coverings $\pi:D\to C$ that are 
Galois over $\overline\dQ$. 





\subsection{The (fake) 2-Selmer set}

Recall our strategy for showing that a set is empty. There is a more 
sophisticated version. Instead of looking at maps $A\to B$, look at 
commutative diagrams 
\[\xymatrix{
  A \ar[r] \ar[d] 
    & B \ar[d]^-\beta \\
  C \ar[r]^-\gamma 
    & D .
}\]
If $\image(\beta)\cap \image(\gamma)=\varnothing$, then $A=\varnothing$. We 
will construct such a diagram with $A=C(\dQ)$. 

Let $L=\dQ[x]/f$ and $L_v=L\otimes \dQ_v$. Write $T$ for the image of $x$ in 
$L$ and each $L_v$. Define 
\[
  H = \left\{\alpha\in L^\times/(L^\times)^2 \dQ^\times:N_{L/\dQ}(\alpha) = \leadingcoefficient(f)\cdot (\text{square})\text{ in }\dQ^\times\right\}
\]
where $\leadingcoefficient(f)$ is the leading coefficient of $f$. Similarly 
define $H_v$ for each place $v$. There are natural maps $\rho_v:H\to H_v$. 
Define $\delta:C(\dQ) \to H$ by 
\begin{align*}
  (\xi,\eta) &\mapsto (\xi-T) (L^\times)^2 \dQ^\times && \text{if }\eta\ne 0 \\
  (\xi,0) &\mapsto (\xi-T-f_1(T))(L^\times)^2 \dQ^\times && \text{if }f(x)=(x-\xi)f_1(x) \\
  \infty_{\pm s} &\mapsto (L^\times)^2 \dQ^\times .
\end{align*} 
Similarly define $\delta_v:C(\dQ_v)\to H_v$. We have a commutative diagram 
\[\xymatrix{
  C(\dQ) \ar[r]^-\delta \ar[d] 
    & H \ar[d]^-{(\rho_v)_v} \\
  C(\dA) \ar[r]^-{\prod \delta_v} 
    & \prod H_v .
}\]

\begin{defi}
The \emph{(fake) 2-Selmer set} of $C$ is 
\[
  \selmer_2^\mathrm{fake}(C) = \{\alpha\in H:\rho_v(\alpha)\in \image(\delta_v)\text{ for all }v\} .
\]
\end{defi}

If $\selmer_2^\mathrm{fake}(C)=\varnothing$, then $C(\dQ)=\varnothing$. 
There is a ``2-Selmer set'' $\selmer_2(C)$ with a map 
$\selmer_2(C) \to \selmer_2^\mathrm{fake}(C)$ that is surjective. It is 
bijective for some curves, but is usually 2-to-1. 

The set $\selmer_2^\mathrm{fake}(C)$ can be computed. Let $S$ be the set of 
places of $L$ dividing one of 
$\{2,\infty,\discriminant(f),\leadingcoefficient(f)\}$. Then 
\[
  \selmer_2^\mathrm{fake}(C)\subset H_S = \{\alpha (L^\times)^2\dQ^\times:v_\fp(\alpha)\text{ even for all }\fp\notin S\} .
\]
The set $H_S$ is finite by standard arguments. 

In \cite{bg13}, it is shown that the (upper) density of genus $g$ 
with $\selmer_2^\mathrm{fake}(C)\ne\varnothing$ is $o(2^{-g})$. 

\begin{example}[Bruin, Stoll]
Of the $\sim 200000$ isomorphism classes of genus $2$ curves of height 
$\leqslant 3$, all but $\sim 1500$ either 
\begin{itemize}
  \item have a rational point, 
  \item fail to be everywhere locally soluble, or 
  \item have empty 2-Selmer set .
\end{itemize}
\end{example}





% !TEX root = sms.tex

\section{Binary quartic forms: bounded average rank of elliptic curves II}\label{sec:shankar-ii}
\thanksauthor{Arul Shankar}





\subsection{Review}

Recall that if $E_{A,B}$ is the elliptic curve $y^2=x^3+A x+B$, then 
$\selmer_2(E_{A,B})$ is in bijection with the quotient 
$\projectivegenerallinear_2(\dQ)\backslash V(\dZ)_{(A,B)}^\mathrm{ls}$, where 
$V(\dZ)_{(A,B)}^\mathrm{ls}$ is the set of locally soluble integral binary 
quartic forms with invariants $A,B$. Even though the action of 
$\projectivegenerallinear_2(\dQ)$ does not preserve 
$V(\dZ)$, it still induces an equivalence relation, where $f\sim g$ whenever 
there is $\gamma\in \generallinear_2(\dZ)$ such that $\gamma \cdot f = g$. 
For example, the forms $p^4 x^4 + p^2 x y^3 + y^4$ and 
$x^4 + p^4 x y^3 p^4 y^4$ are $\projectivegenerallinear_2(\dQ)$-equivalent via 
the matrix $\begin{pmatrix} p^{-1} \\ & p \end{pmatrix}$. For asymptotics, we 
could restrict to irreducible quartic forms. We defined subsets 
$V(\dZ)_{(A,B)}^{(i)}$ of $V(\dZ)$; for definitions, see 
\autoref{sec:shankar-i}. We ended up with an estimate 
\[
  \#\left(\projectivegenerallinear_2(\dZ)\backslash V(\dZ)_{H<X}^{\mathrm{irr},(i)}\right) = \frac{1}{n_i} |J| \volume(\cF) \volume\left(R_X^{(i)}\right) + O(X^{3/4}) .
\]
In this lecture, we will prove the following theorem. 

\begin{theo}[Bhargava, Shankar]
$\average(\#\selmer_2) = 3$. 
\end{theo}

To do this, we will need to replace $V(\dZ)^\mathrm{irr}$ by 
$V(\dZ)^{\mathrm{irr},\mathrm{ls}}$, and replace 
$\projectivegenerallinear_2(\dZ)$-orbits by 
$\projectivegenerallinear_2(\dQ)$-equivalence classes. 





\subsection{Local solubility}

For each prime $p$, let $V(\dZ_p)^\mathrm{s}$ be the subset of $V(\dZ_p)$ 
consisting of soluble binary quartic forms. Let $V(\dZ)^{\mathrm{s}(p)}$ be the 
set of forms in $V(\dZ)$ whose image in $V(\dZ_p)$ is soluble. We start by 
computing 
\[
  \#\left(\projectivegenerallinear_2(\dZ)\backslash V(\dZ)_{H<X}^{\mathrm{irr},\mathrm{s}(p)}\right) = \frac{|J|}{n_i} \volume(\cF) \volume\left(R_X^{(i)}\right) \volume\left(V(\dZ_p)^\mathrm{s}\right) + O(X^{3/4}) .
\]
To do this for locally soluble forms, we will need to look at ``soluble at $p$ 
forms'' for all $p$. For that, we need a sieve. 

We want 
\[
  \#\left(\projectivegenerallinear_2(\dZ)\backslash V(\dZ)_{H<X}^{\mathrm{irr},p^2\mid \Delta}\right) = O(X^{5/6}/p^{1+\delta}) ,
\]
for any $\delta>0$. In fact, this is stronger than we need. All the proof 
requires is 
\[
  \sum_{p>M} \#\left(\projectivegenerallinear_2(\dZ)\backslash V(\dZ)_{H<X}^{\mathrm{irr},p^2\mid\Delta}\right) = O(X^{5/6}/f(M)) ,
\]
where $f(M)\to \infty$ as $M\to \infty$. 

Recall the (naive) height is $H(E_{A,B}) = \max\{4 |A|^3,27 B^2\}$. 
We want 
\[
  \#\{E_{A,B}:H(E_{A,B})<X\text{ and }p^2\mid \Delta(E_{A,B})\} = O\left(\frac{X^{5/6}}{p^{1+\delta}}\right),
\]
where $\delta>0$. When $p$ is large, map $E_{A,B}$ to the binary cubic form 
$x^3 + A x y^2 + B y^3$, which goes to 
$\generallinear_2(\dZ)\cdot (x^3+A x y^2 + B y^3)$. We have defined a map 
\[
  \varphi:U_1(\dZ) \to U(\dZ) \to \generallinear_2(\dZ)\backslash U(\dZ) ,
\] 
where $U_1(\dZ)$ is the space of elliptic curves and $U$ is the space of all 
binary cubic forms. This map is discriminant-preserving. 

\begin{theo}[Delone, Nagell, Siegel, Evertse, Akhtari]
The map $\varphi$ is at most 7-to-1 for elements with large enough 
discriminant. 
\end{theo}

This is very deep. 
As one application, a binary cubic form represents one at most seven times. 
It follows from the theorem that 
\[
  \#\left\{E_{A,B}:H(E_{A,B})<X\text{ and }p^2\mid \Delta(E_{A,B})\right\} = O\left(\frac{X}{p^2}\right) . 
\]
This isn't quite what we wanted because the estimate has $X$ instead of 
$X^{5/6}$. But for $p$ sufficiently large, it works. 

Suppose $p^2\mid \Delta(E_{A,B})$ for ``modulo $p$ reasons,'' i.e.~if $E_{A,B}$ 
has additive reduction. If $p>3$, then $A\equiv B\equiv 0\pmod p$. The bound in 
this case is 
\[
  O\left(\left(\frac{X^{1/3}}{p^2}+1\right)\left(\frac{X^{1/2}}{p}+1\right)\right) = O\left(\frac{X^{5/6}}{p^2} + \frac{X^{1/2}}{p}+1\right) .
\]
If $p\mid \Delta(E_{A,B})$ for ``modulo $p^2$ reasons,'' then fixing $A$ 
determines $B$ modulo $p^2$. In this case, the bound is 
\[
  O\left(X^{1/3} \cdot\left(\frac{X^{1/2}}{p}+1\right)\right) = O\left(X^{5/6}/p^2 + X^{1/3}\right) .
\]
Combining these estimates yields the uniform bound 
\[
  \#\{E:H(E)<X\text{ and }p^2\mid \Delta(E)\} = O\left(\frac{X^{5/6}}{p^{3/2}}\right) .
\]

Recall that there is a bijection between 
$\projectivegenerallinear_2(\dZ)\backslash V(\dZ)$ and the set of $(Q,C,x)$, 
where $Q$ is a quartic ring, $C$ is a cubic resolvent ring, and $x$ generates 
$C$ (?) The map $V(\dZ) \to \dZ^2\otimes \symmetric^2(\dZ^3)$ induces 
the map sending $(Q,C,x)$ to $(Q,C)$. On the side of forms, the map sends 
$a x^4 + b x^3 y + c x^2 y^2 + d x y^3 + e y^4$ to 
\[
  \begin{pmatrix} & & 1/2 \\ & -1 \\ 1/2 \end{pmatrix}, \begin{pmatrix} a & b/2 \\ b/2 & c & d/2 \\ & d/2 & e \end{pmatrix} .
\]
Our map yields the bound 
\[
  \#\left(\projectivegenerallinear_2(\dZ)\backslash V(\dZ)_{H<X}^{p^2\mid \Delta}\right) = O\left(\frac{X}{p^2}\right) .
\]
Combining everything, we get 
\begin{align*}
  \#\left(\projectivegenerallinear_2(\dZ)\backslash V(\dZ)_{H<X}^{p^2\mid \Delta,\mod p^2}\right) 
    &= O\left(\frac{X^{5/6}}{p^2} + X^{2/3}\right) \\
  \sum_{p<M} \#\left(\projectivegenerallinear_2(\dZ)\backslash V(\dZ)_{H<X}^{p^2\mid\Delta}\right) 
    &= O\left(\frac{X^{5/6}}{\log M}\right) .
\end{align*}





\subsection{Weights}

The problem is that a single $\projectivegenerallinear_2(\dQ)$-class in 
$V(\dZ)$ could break up into seven different 
$\projectivegenerallinear_2(\dZ)$-orbits. Given a form $f$, let 
\[
  B_f = \projectivegenerallinear_2(\dZ)\backslash \left(\projectivegenerallinear_2(\dQ)\cdot f\cap V(\dZ)\right) .
\]
For $f\in V(\dZ)$, we define 
\[
  W(f) = \begin{cases} 0 & \text{if $f$ is not locally soluble} \\ \left(\displaystyle\sum_{g\in B_f} \frac{\#\automorphism_\dQ(g)}{\automorphism_\dZ (g)}\right)^{-1} & \text{otherwise} \end{cases} 
\]
The weight of $f$ is a product of local weights. That is, define for each $p$ 
\[
  W_p(f) = \begin{cases} 0 & \text{if $f$ is not locally soluble} \\ \left(\displaystyle\sum_{g\in B_p(f)} \frac{\#\automorphism_{\dQ_p}(g)}{\automorphism_{\dZ_p} (g)}\right)^{-1} & \text{otherwise} \end{cases} 
\]
Then we have the following proposition (3.3 in my paper). 

\begin{prop}
For $f\in V(\dZ)$, we have $W(f) = \prod_p W_p(f)$. 
\end{prop}

We get the following formula: 
\[
  \#\left(\projectivegenerallinear_2(\dZ)\backslash V(\dZ)_{H<X}^{\mathrm{irr},W,(i)}\right) = \frac{|J|}{n_i} \volume(\cF) \volume\left(R_X^{(i)}\right) \prod_p \int_{V(\dZ_p)} W_p(f)\, \mathrm{d} f .
\]

\begin{prop}
\[
  \int_{V(\dZ_p)}W_p(f)\, \mathrm{d} f = |J|_p \volume(\projectivegenerallinear_2(\dZ_p)) \int_{\dZ_p^2} \frac{\#\left(E_{A,B}(\dQ_p)/2\right)}{\#E_{A,B}[2](\dQ_p)}\, \mathrm{d} (A,B) .
\]
\end{prop}

We also have 
\[
  |J|\volume\left(\projectivegenerallinear_2(\dZ)\backslash \projectivegenerallinear_2(\dR)\right) \int_{\{(A,B)\in\dR^2:H(E_{A,B})<X\}} \frac{\#(E_{A,B}(\dR)/2)}{\# E_{A,B}[2](\dR)}\, \mathrm{d}(A,B) .
\]

\begin{prop}[Brummer and Kramer]
\[
  \frac{\#(E(\dQ_v)/2)}{\# E[2](\dQ_v)} = \begin{cases} 1/2 & v=\infty \\ 2 & v=2 \\ 1 & \text{otherwise} \end{cases}
\]
\end{prop}

Thus the average of $\#\selmer_2)-1$ is the following limit: 
\[
  \lim_{X\to \infty} \frac{\displaystyle\volume(\projectivegenerallinear_2(\dZ)\backslash \projectivegenerallinear_2(\dR)) \prod_p \volume(\projectivegenerallinear_2(\dZ_p) + O(X^{3/2})) \int_{H(A,B)<X}\,\mathrm{d}(A,B)}{\displaystyle\int_{H<X} \mathrm{d}(A,B) + O(X^{1/2})} .
\]
The integrals cancel, and the product over all places in the numerator is the 
Tamagawa number $\tau(\projectivegenerallinear_2)=2$. It follows that 
\[
  \average(\#\selmer_2)=2+1=3 .
\]

When we generalize to $\selmer_n$ for $n\geqslant 3$, things get a bit more 
complicated, as in the following table: 
\begin{center}
\begin{tabular}{c|c|c|c|c}
$n$ & group & space & $\tau(G)$ & $\average(\#\selmer_n)$ \\ \hline
2 & $\projectivegenerallinear_2(\dZ)$ & $\symmetric^4(\dZ^2)$ & 2 & 3\\ 
3 & $\projectivegenerallinear_3(\dZ)$ & $\symmetric^3(\dZ^3)$ & 3 & 4\\ 
4 & qt.~of $\generallinear_2\times \generallinear_4$ & $\dZ^2\otimes\symmetric^2(\dZ^4)$ & 4 & 7\\
5 & qt.~of $\generallinear_5\times \generallinear_5$ & $\dZ^5\otimes \bigwedge^2 \dZ^5$ & 5 & 6
\end{tabular}
\end{center}





% !TEX root = sms.tex

\section{Coregular spaces and genus one curves}\label{sec:ho}
\thanksauthor{Wei Ho}





\subsection{Introduction and motivation}

Something we have done many times is take a representation $V$ of a group $G$ 
and study the orbits $V/G$. The stabilizers of points in $V$ are also 
important. We have tried to describe the orbits in terms of ``arithmetically 
interesting'' objects, e.g.~elliptic curves with extra data. If we impose 
local conditions on $V/G$, we get elliptic curves with Selmer elements. One 
final note: we want stabilizers in $G$ to match up with automorphism groups of 
the ``arithmetically interesting objects.'' This is a subtle but important 
point. 

In \autoref{sec:shankar-i} and \autoref{sec:shankar-ii}, the ``extra data'' 
attached to an elliptic curve $E$ was an $n$-covering of $E$. This consists 
of a $E$-torsor $C$ with a degree $n$ line bundle on $C$. 

\begin{example}
In the binary quartic case, the form $f(x,y) = a x^4 + \cdots$ corresponds to 
the curve $C:z^2 = a x^4 + b x^3 y + c x^2 y^2 + d x y^3 + e y^4$; this is a 
double cover of $\dP^1$ ramified at four points. The invariants of $f$ 
determine an elliptic curve. The map $C\to \dP^1$ determines the line bundle 
on $C$. 
\end{example}

\begin{example}
Recall that we can describe 3-Selmer elements with ternary cubics (homogeneous 
degree three polynomials in three variables). The representation involved 
is $\symmetric^3(3) = \symmetric^3\dA^3$. Any ternary cubic $f$ gives a genus 
one curve in $\dP^2$. It's Jacobian is canonically the elliptic curve with 
invariants those of $f$. The degree three line bundle on $C$ comes from the 
embeddigng $C\hookrightarrow \dP^2$. The rings of invariants of the action of 
$\speciallinear(3)$ on $\symmetric^3(3)$ is polynomial in two generators 
$S,T$, of degree 4 and 6 respectively. The Jacobian of $C_f$ is 
$E_{S(f),T(f)}$. 
\end{example}

To generalize these ideas, we need to find other representations that 
parameterize interesting data. Let $k$ be an algebraically closed field. If 
$(G,V)$ is a prehomogeneous vector space over $k$ and $U\subset V$ is open and 
$G$-stable, then $U(k)/G(k)$ might be ``zero-dimensional,'' e.g.~a single 
Zariski-open orbits. But there are lots of non-isomorphic elliptic curves, 
even over an algebraically closed field. so we would want 
$U(k)/G(k)$ to be ``bigger,'' e.g.~the affine line. 

More geometrically, we are trying to find prehomogeneous vector spaces 
$(G,V)$ such that the coarse moduli space $V/G$ is a moduli space we already 
are familiar with. In all our examples, the invariant rings are polynomial 
rings. We call such representations \emph{coregular}. 

The moduli space of elliptic curves is birational to $\dP(2,3)$. So we should 
look for coregular representations with ring of invariants free in two 
variables. Elliptic curves with one marked point are of the form 
$y^2 + d_3 y = x^3 + d_2 x^2 + d_4 x^2$. If we take our scaling, we get 
$\dP(2,3,4)$, or $\dA^3$ if we also keep track of the differential. 

To summarize: many (but not all!) families of elliptic curves with extra data 
have coarse moduli space (look over an algebraically closed field) birational 
to a weighted projective space. This means the invariant ring of any possible 
$(G,V)$ is a polynomial ring. 

[\ldots couldn't follow\ldots]





% !TEX root = sms.tex

\section{Arithmetic invariant theory and hyperelliptic curves I}\label{sec:gross-i}
\thanksauthor{Benedict Gross}





Let $k$ be a field, $G$ a reductive group over $k$, and 
$G\to \generallinear(V)$ a representation of $G$. The ring 
$\symmetric^\bullet(V^\vee)$ contains a subring 
$\symmetric^\bullet(V^\vee)^G$ of invariant polynomials. An important theorem 
is that $\symmetric^\bullet(V^\vee)^G$ is a finitely generated $k$-algebra. 
Write $V\gq G$ for the variety 
$\spectrum\left(\symmetric^\bullet(V^\vee)^G\right)$; this comes with a 
canonical ``projection'' $\pi:V\to V\gq G$. 





\subsection{First examples and results}

\begin{example}
Consider $G=\generallinear(W)$ and $V=\mathfrak{gl}(W)=\End W$, with the 
adjoint action of $G$ on $V$. It is known that 
$\mathfrak{gl}(W)\gq \generallinear(W)$ is an affine space with coordinates the 
``coefficients of the symmetric polynomial.'' 
\end{example}

More generally, if $G$ is a reductive group and $\frakg=\lie G$ under the 
adjoint representation, then $\frakg\gq G$ is affine. That is, Chevalley 
proved that the adjoint representation of a reductive group is coregular. 
Winberg generalized this even further. If $\theta:G\to G$ is an 
automorphism of order $m$ and $\frakg=\bigoplus_a \frakg(a)$, then 
the action of $G^\theta$ on each $\frakg(a)$ is coregular. 

Suppose $f\in (V\gq G)(k)$. Let $V_f$ be the fiber in $V$ of $\pi$ over $f$. 
Then $V_f(k)$ is a (possibly empty) union of $G(k)$-orbits. 

\begin{example}
When $G=\generallinear(n)$ and $V=\mathfrak{gl}(n)$, then $V_f$ is all linear 
$T$ with fixed characteristic polynomial $f$. The set $V_f(k)$ is always 
nonempty. Indeed, let $L=k[x]/f$ and $\theta:L\to L$ be ``multiplication 
by $x$.'' Then $\theta$ is a $k$-linear transformation with characteristic 
polynomial $f$. Choosing an isomorphism $L\simeq k^n$ gives an element in 
$V_f(k)$. Roughly, ``every polynomial is the characteristic polynomial of a 
map defined over the base field.'' 

If the discriminant $\Delta(f)\ne 0$, there 
is a single orbit of $G(k)$ on $V_f(k)$. For any $T\in V_f(k)$, 
the stabilizer $G_T$ is isomorphic to the Weil restriction 
$\Pi_{L/k}\dG_\multiplicative$; a maximal torus in $\generallinear(V)$ if $L$ 
is \'etale. 
If $f(x)=x^k$, then $V_f$ is known as the \emph{nilpotent cone}; the orbit 
we constructed is the \emph{regular nilpotent}. 
\end{example}

\begin{example}
Consider the action of $\speciallinear(W)$ on 
$\mathfrak{sl}(W)=\End(W)^{\trace=0}$. The ring of invariants is freely 
generated by all but the constant term of the characteristic polynomial, so 
$\mathfrak{sl}(n)\gq \speciallinear(n) \simeq \dA^{n-1}$. If $\Delta(f)\ne 0$, 
then orits in $V_f(k)$ are in bijection with $k^\times / \norm(L^\times)$. 

If for example $\dim W=2$ and $f(x)=x^2+1$, then the orbit space 
$V_f(k)/G_f(k)$ is in bijection with $\dQ^\times/\norm(\dQ(i)^\times)$; a 
(huge) abelian 2-group. 
\end{example}

For $\generallinear(n)$ and $\speciallinear(n)$, we obtained that 
$V_f(k)$ was a torsor over $\h^1(k,G_f)$. But this used 
$\h^1(k,\speciallinear_n) = \h^1(k,\generallinear_n) = 0$. 





\subsection{Principles of arithmetic invariant theory}

\begin{principle}
Assume $V_f(k)\ni v$, and that $G(k^s)$ acts transitively on 
$V_f(k^s)$. Then the set of $G(k)$-orbits on $V_f(k)$ is in bijection with the 
kernel of the map of pointed sets 
$\h^1(k,G_v) \to \h^1(k,G)$. 
\end{principle}
\begin{proof}
Say $v'\in V_f(k)\subset V_f(k^s)$, write $v'=g(v)$ for some $g\in G(k^s)$. 
Send the orbit of $v'$ to the class in $\h^1(k,G_v)$ of the cocycle 
$\sigma\mapsto c_\sigma = g^{-1} \circ g^\sigma$. Checking that this is a 
bijection is easy. 
\end{proof}

If $G$ is one of $\generallinear(n)$, $\speciallinear(n)$, $\symplectic(n)$, 
then $\h^1(k,G)=0$, so $V_f(k)/G(k) = \h^1(k,G_v)$. 

\begin{example}
Let $W$ be a split orthogonal space over $k$ of dimension $n=2 g+1$. So $W$ 
is a direct sum of $g$ hyperbolic planes and a single copy of $k$. Let 
$G=\specialorthogonal(W)$. For example, if 
$g=1$, then $G\iso \projectivegenerallinear(2)\iso \specialorthogonal_3$. 
Let $V=\mathfrak{so}(W)$; the space of trace-zero self-adjoint operators on 
$W$. Then 
$\symmetric^\bullet(W^\vee)^G=k[c_2,\dots,c_{2g+1}]$, freely generated on 
the coefficients of the characteristic polynomial. It's a bit more 
difficult to show that the fibers of 
$\mathfrak{so}(W) \to \mathfrak{so}(W)\gq \specialorthogonal(W)$ are all 
nonempty. Given $f$ in the quotient, define as before $L=k[x]/f$. This is a 
$k$-algebra of rank $2 g+1$. Let $\langle\lambda,\mu\rangle$ be the coefficient 
$x^{2 g}$ in $\lambda\mu$; also $\trace_{L/k}(\lambda\mu/f'(x))$. The 
operator ``multiplication by $x$'' on $L$ is self-adjoint with characteristic 
polynomial $f$. A bit of work shows that this gives an element of $V_f(k)$. 
\end{example}

In fact, for $V=\mathfrak{so}(n)$, $G=\specialorthogonal(n)$, the map 
$V(k) \to V\gq G$ has a standard section known as the Kostant section. 

Let's compute stabilizers. For $f$ with $\Delta(f)\ne 0$, it is easy to see 
that $G_v=\specialorthogonal(W)\cap L^\times$. This is 
$L^\times[2]^{\norm=1}$. As a group scheme, this is 
$\ker(\Pi_{L/k}\boldsymbol\mu_2 \xrightarrow{\norm{}}\boldsymbol\mu_2)$. 
An easy computation of Galois cohomology shows that 
$\h^1(k,G_v) = (L^\times/2)^{\norm=0}$. In this case, $\h^1(k, G_v)$ is also 
$\h^1(k,J[2])$, where $J$ is the Jacobian of $y^2=f(x)$. 

In general, our first principle is not very useful, because the map 
$\gamma:\h^1(k,G_v) \to \h^1(k,G)$ can be pretty complicated, and is not easy to 
pin down explicitly. 

\begin{principle}
For any $c\in \h^1(k,G)$, there is a twisted group $G^c$ over $k$ and twisted 
representation $V^c$ over $k$. The fiber of $\gamma$ over $c$ is the set of 
orbits of $G^c(k)$ in $V_f^c(k)$. 
\end{principle}

Recall our map 
$\h^1(k,G_v) = \h^1(k,J[2]) \xrightarrow\gamma \h^1(k,\specialorthogonal(W))$, 
where $J$ is the Jacobian of $y^2=f$. The 2-Selmer group $\selmer_2(J)$ is a 
subset of $\h^1(k,J[2])$, and in \cite{bg13}, Bhargava and I showed that 
$\selmer2(J) = \ker(\gamma)$. In general, when is $V_f$ empty for all 
$G^c$?

\begin{principle}
Assume $G(k^s)$ acts transitively on $V_f(k^s)$ and $G_v(k^s)$ is abelian. 
\begin{enumerate}
  \item If the class of $d_f$ is non-trivial in $\h^2(k,G_f)$, there is no 
    $k$-rational point in any fiber. 
  \item If $d_f=0$, then the fiber is nontrivial for some pure inner form $G^c$. 
\end{enumerate}
\end{principle}
\begin{proof}
Take $v\in V_f(k^s)$. Then $\sigma_v = \prescript{\sigma}{}{f}\circ f$ is also 
in $V_f(k^s)$. Define $\theta_\sigma:G_{c_v} \to G_v$ by 
$\alpha\mapsto g_\sigma \alpha g_\sigma^{-1}$. This map is an isomorphism that 
does not depend on $v$. Since 
$\theta_{\sigma\tau} = \theta_\sigma\circ\prescript{\sigma}{}{\theta_\tau}$, 
this descends $G_v$ to a group $G_f$ defined over $k$. 
\end{proof}

So if $G_v(k^s)$ is abelian and $f\in (V\gq G)(k)$, there is a ``stabilizer'' 
$G_f$ of $f$ even if $V_f(k)=\varnothing$. 

\begin{example}
Let $G=\speciallinear(W)$, where $\dim W=2 g+2$. Let 
$V=\symmetric^2(W^\vee)\oplus \symmetric^2(W^\vee)$. We can think of 
$V$ as the space of pairs $v=(A,B)$ of symmetric matrices. The ring of 
invariants is freely generated by the coefficients of the bilinear form 
$f(x,y) = (-1)^{g+1}\det(x A-y B)$. Assume $\Delta(f)\ne 0$. Then put 
$G_f=(\Pi_{L/k}\boldsymbol\mu_2)^{\norm=1}$, where $L/k$ is the extension 
constructed earlier. If we put $f(x,1) = f_0 \cdot g(x)$, then 
$L=k[x]/g$. What is the class $d_f\in \h^2(k,G_f)$? This cohomology group 
has a subgroup $k^\times / k^\times \norm(L^\times)$. The class $d_f$ is just 
the class of $f_0$ in $k^\times / k^\times \norm(L^\times)\subset \h^2(k,G_f)$. 
\end{example}

For example, if $k=\dR$, $g=0$, $f_0=-1$, $f=-x^2-y^2$, and $g=x^2+1$, then there 
are no orbits. 





% !TEX root = sms.tex

\section{Most hyperelliptic curves have no rational points}\label{sec:bhargava-iv}
\thanksauthor{Manjul Bhargava}





\subsection{Summary of results}

According to Don Zagier, the title should be ``most hyperelliptic curves are 
pointless.'' Recall that a \emph{hyperelliptic curve} is a smooth projective 
geometrically irreducible curve with a degree-two map to $\dP^1$. More 
concretely, any hyperelliptic curve over $\dQ$ can be expressed in the form 
\begin{equation*}\tag{$\ast$}\label{eq:hyper}
  C:z^2 = f_0 x^n + f_1 x^{n-1} y + \cdots + f_{n-1} x y^{n-1} + f_n y^n ,
\end{equation*}
where $n=2 g+2$ and $g$ is the genus of $C$, at least if 
$\discriminant(f)\ne 0$. By scaling, we may assume the $f_i\in \dZ$. Define a 
height on hyperelliptic curves by $H(C)=\max\{|f_i|\}$. We will order 
hyperelliptic curves by height. The results of this section are mostly from 
\cite{bg13}. 

\begin{theo}
Order all hyperelliptic curves \eqref{eq:hyper} over $\dQ$ of genus $g$ by 
height. Then as $g\to \infty$, a density approaching $100\%$ of hyperelliptic 
curves of genus $g$ have no rational points. 
\end{theo}

More precisely, the upper density of hyperelliptic curves of genus $g$ having a 
rational point is $o(2^{-g})$. Since most ($>75\%$) hyperelliptic curves of 
genus $g\geqslant 1$ have points over $\dQ_v$ for all places $v$ (everywhere 
locally soluble), we obtain the following. 

\begin{coro}
As $g\to \infty$, a density approaching $100\%$ of everywhere locally soluble 
hyperelliptic curves \eqref{eq:hyper} of genus $g$ fail the Hasse principle. 
\end{coro}

For $g=1$, the density is $>20\%$, and for $g=2$, the proportion of 
$>50\%$. For $g=10$, the density is $>99\%$. In the 1940's, Lind and Reichardt 
independently gave examples of equations of the form $z^2=f(x,y)$, where $f$ is 
a quartic, failing the Hasse principle. Later on, Selmer gave an example of an 
elliptic curve (minus the origin) failing the Hasse principle. A more 
elementary reformulation of these results is that binary forms rarely take 
square values. 





\subsection{Key construction}

The main idea is: use the representation 
$V(\dZ)=\dZ^2\otimes\symmetric^2(\dZ^n)$ of $G(\dZ)=\generallinear_n(\dZ)$. 
We can view elements of $V(\dZ)$ as pairs $(A,B)$ of $n\times n$ symmetric 
matrices with integer entries. The group $G$ acts by 
$\gamma\cdot (A,B) = (\gamma A\transpose \gamma, \gamma B \transpose \gamma)$. 
Given such a $v=(A,B)\in V(\dZ)$, define the \emph{invariant binary $n$-form} 
$f_v(x,y)=-1^{n/2}\det(A x-B y)$. The coefficients of $f_v(x,y)$ give 
invariants for the action of $G(\dZ)$ on $V(\dZ)$. In fact, these freely 
generate the ring of invariants over $\dC$. 

Given a binary $n$-ic form $f$ over $\dZ$, when does it arise as $f_v$ for some 
$v=(A,B)\in V(\dZ)$? Unfortunately not always. 

\begin{prop}
Let $f$ be a binary $n$-ic form over $\dZ$. If $z^2=f(x,y)$ has a rational 
point, then $f=f_v$ for some $v\in V(\dZ)$. 
\end{prop}
\begin{proof}
We use a classification of the orbit space $G(\dZ)\backslash V(\dZ)$. Given 
a rational point, we'll produce an ``algebraic object,'' which will give our 
pair $v=(A,B)$ of $n\times n$ symmetric matrices. Given 
a binary $n$-ic form $f$ over $\dZ$, assume $\discriminant(f)\ne 0$ and 
$f_0\ne 0$, where $f=f_0 x^n + \cdots + f_n y^n$. Let 
$K_f = \dQ[x]/f(x,1) = \dQ[\theta]$; this is an $n$-dimensional $\dQ$-algebra. 
Inside $K_f$, there is a lattice $R_f$ with basis 
$\{1,\zeta_1,\zeta_2,\dots,\zeta_{n-1}\}$, where 
\[
  \zeta_k = f_0 \theta^k + f_1 \theta^{k-1} + \cdots + f_{k-1} \theta .
\]
See \autoref{sec:wood-iii} for details of this construction. The $\zeta_k$ 
are integral over $\dZ$. In \cite{bm72}, Birch and Merriman proved that 
$\discriminant(R_f)=\discriminant(f)$. Much more recently, Nakagawa proved 
that $R_f$ is a ring. Define further lattices in $K_f$: 
\[
  I_f(k) = \langle 1,\theta,\theta^2,\dots,\theta^k,\zeta_{k+1},\dots,\zeta_{n-1}\rangle ,
\]
for any $0\leqslant k<n$. Then $I_f(k)$ is an $R_f$-module and 
$I_f(k)=I_f^k$. Note that $I_f(0)=R_f$; we define $I_f=I_f(1)$. Checking this 
is a good exercise. The $I_f(k)$ come equipped with bases. Given 
$(I,\alpha)$ as in the following theorem, take coefficients of $\zeta_{n-1}$ 
and $\zeta_{n-2}$ in $\frac 1 \alpha:I\times I \to I_f(n-3)$. These 
coefficients are the $(A,B)$ we wanted to construct. 
\end{proof}

\emph{Warning}: the converse to the Proposition is false. Recall that the 
ring $R_f$ is constructed in \autoref{sec:wood-iii} as the ring of global 
functions on the subscheme of $\dP_\dZ^1$ cut out by $f$. The module 
$I_f(k)$ is global sections of the pullback of $\sO(k)$. 

\begin{theo}[Wood]
The set $(G(\dZ)\backslash V(\dZ))_f$ is naturally in bijection with the set of 
equivalences classes of pairs $(I,f)$, where $I$ is a fractional ideal of $R_f$ 
and $\alpha\in K_f^\times$ such that $I^2\subset \alpha I_f(n-3)$ and 
$\norm(I)^2 = \norm(\alpha)\norm(I_f^{n-3})$. Here the equivalence relation is 
$(I,\alpha)\sim (\kappa I,\kappa^2\alpha)$ for $\kappa\in K_f^\times$. 
\end{theo}

\begin{theo}
Let $n$ be an odd integer. Then there always exists $v=(A,B)\in V(\dZ)$ such 
that $f_v=f$. 
\end{theo}
\begin{proof}
Take $\alpha=1$, $I=I_f^{(n-3)/2}$. 
\end{proof}

\begin{theo}
Let $n$ be an even integer. Assume $z^2=f(x,y)$ has a rational point. Then 
there exists $v\in V(\dZ)$ such that $f=f_v$. 
\end{theo}
\begin{proof}
We can further assume that $f(0,1)$ is square, i.e.~$f_n=c^2$ for some 
$c\in \dZ$. Take $\alpha=\theta$ and 
\[
  I=\langle c,\theta,\theta^2,\dots,\theta^{(n-2)/2},\zeta_{n/2},\dots,\zeta_{n-1}\rangle .
\]
Check that $I^2\subset \theta I_f^{n-3}$ and 
$\norm(I^2)=\norm(\theta)\norm(I_f^{n-3})$. By the theorem of Wood, this gives 
rise to a pair $v=(A,B)$. 
\end{proof}

If for example $n=6$, there is an explicit formula for $v$: 
\[
  (A,B) = \left(
  \begin{pmatrix} 
    -1 & 0 & 0 & 0 & 0 & 0 \\ 
    0 & 0 & 0 & 0 & 0 & 1 \\ 
    0 & 0 & 0 & 0 & 1 & 0 \\ 
    0 & 0 & 0 & f_0 & f_1 & f_2 \\
    0 & 0 & 1 & f_1 & f_2 & f_3 \\
    0 & 1 & 0 & f_2 & f_3 & f_4 
  \end{pmatrix},
  \begin{pmatrix}
    0 & 0 & 0 & 0 & 0 & c \\
    0 & 0 & 0 & 0 & 1 & 0 \\
    0 & 0 & 0 & 1 & 0 & 0 \\
    0 & 0 & 1 & f_1 & f_2 & 0 \\
    0 & 1 & 0 & f_2 & 0 & 0 \\
    c & 0 & 0 & 0 & 0 & -f_5
  \end{pmatrix}
  \right) .
\]

Now count the number of orbits of $G(\dZ)$ on $V(\dZ)$ having bounded height. 

\begin{theo}
The number of orbits of $G(\dZ)$ on $V(\dZ)$ having height $<X$ is 
\[
  C\cdot X^{n+1} = C(X^{1/n})^{n(n+1)} + O(X^{x+1-1/n}) .
\]
So the number of orbits per $f$ is bounded by a constant $C'$ on average. 
\end{theo}

This is not good enough for our purposes. We need the number of orbits per 
$f$ to be strictly less than $1$ on average. But we are only interested in 
some orbits. The number of orbits per $f$ locally looking like $f_v$ is 
$C''<C'$, where $C''=o(2^{-g})$. 

\begin{coro}
As $C$ ranges over hyperelliptic curves of genus $g$, 
$\average(\#\selmer_2^\mathrm{fake}(C)) = o(2^{-g})$. 
\end{coro}






% !TEX root = sms.tex

\section{Selmer groups and heuristics II}\label{sec:poonen-iii}
\thanksauthor{Bjorn Poonen}





d


% !TEX root = sms.tex

\section{Pencils of quadrics: the geometry}
\thanksauthor{Jerry Wang}





Pencils of quadrics have shown up many times (though not under that name) in 
this summer school. I will explain some of the geometry of quadrics. 





\subsection{Notation}

Let $k$ be a perfect field of characteristic not $2$. Let $\sL$ be a 
rational generic pencil of quadrics in $\dP^{2n+1}$. Such an $\sL$ will be 
of the form 
\[
  \{x Q-y Q_2:(x:y)\in \dP^1\}
\]
where $Q_1,Q_2\subset \dP^{2n+1}$ are quadrics. ``Rational'' means the $Q_i$ 
are defined over $k$. ``Generic'' means the binary form 
$f(x,y) = (-1)^{n+1}\det(x Q_1-y Q_2)$ has no repeated factors, 
i.e.~$\discriminant(f)\ne 0$. Alternatively, $C:z^2=f(x,y)$ should be a smooth 
hyperelliptic curve of genus $n$. The \emph{base locus} $B=Q_1\cap Q_2$ will 
be smooth of dimension $2 n-1$. 

Let $F$ be the variety of maximal linear subspaces of $B$. That is: 
\[
  F=\{X\simeq \dP^{n-1}:X\subset B\} .
\]
For example, when $n=1$, we have two quadrics $Q_1,Q_2\subset \dP^3$. We have 
$F=B=Q_1\cap Q_2$, a genus one curve. So over $\bar k$, $F$ is isomorphic to an 
elliptic curve. This is basically the construction Bhargava used to study 
Selmer elements of elliptic curves. 

If $n=2$, $B=Q_1\cap Q_2$ is a degree $4$ three-fold. The variety $F$ turns 
out to be an abelian surface (over $\bar k$). 

This apparent pattern holds. 

\begin{theo}[Reid, Donagi, Desale-Ramanan]
Over $\bar k$, the variety $F\simeq \jacobian C$; an abelian variety of 
dimension $n$. 
\end{theo}

To obtain arithmetic information about $C$, we need a result that works over 
an arbitrary (possibly non algebraically closed) base field. 

\begin{theo}
The variety $F$ is a $J=\jacobian(C)$-torsor. Moreover, there is an algebraic 
group structure on $G=J\sqcup F\sqcup J^1\sqcup F$ compatible with that of 
$J$, and for which $G/J\simeq \dZ/4$. 
\end{theo}


Here $J^1=\picard^1(C)$, the moduli space of degree-$1$ line bundles on $C$. 

For $n=1$, $Q_1,Q_2\subset \dP^3$, we had $F=Q_1\cap Q_2$. The curve $C$ is 
defined by $z^2=\det(x Q_1-y Q_2)$. The curve $\det(x Q_1-y Q_2)$ is cut out 
by a binary quartic form. There is a canonical isomorphism $H^1=C$, so we have 
a group structure on $G=E\sqcup F\sqcup F\sqcup F$. Multiplication by $2$ on 
$G$ gives a map $2:C\to E$; this is a $2$-cover of $E$. This is the cover used 
in the study of 2-Selmer groups of elliptic curves. Multiplication by $4$ gives 
a map $4:F\to E$; this is the $4$-cover used in the study of 4-Selmer groups of 
elliptic curves. 

For general $n\geqslant 2$, there are a couple cases. 

Case 1: $C(k)\ne\varnothing$. Choose $\infty\in C(k)$. Put 
$F[2]_\infty = \{X\in F:X+X=(\infty)\}$; this is a $J[2]$-torsor. So we get an 
element of $\h^1(k, J[2])$. There are two subcases 
corresponding to whether or not $\infty$ is a Weierstrass point. If $C$ has a 
Weierstrass point, we get all torsors of $J[2]$ in this way. When $\infty$ is a 
non-Weierstrass point, we don't get all of $\h^1(k,J[2])$, but we do get 
``enough'' points in $\h^1(k,J[2])$, namely the entire kernel of 
$\gamma:\h^1(k,J[2]) \to \h^1(k,?)$ as in \autoref{sec:gross-i}. 

Case 2: the map $2:F\to J^1$ is a $2$-cover of $J^1$. If $k$ is a global field 
and $C$ is everywhere locally soluble, we get all locally soluble 2-covers of 
$J^1$ using this method. 





% !TEX root = sms.tex

\section{Arithmetic invariant theory and hyperelliptic curves II}
\thanksauthor{Benedict Gross}





d


% !TEX root = sms.tex

\section{Chabauty methods and hyperelliptic curves}\label{sec:poonen-iv}
\thanksauthor{Bjorn Poonen}





d


% !TEX root = sms.tex

\section{Topological and algebro-geometric methods over function fields I}\label{sec:ellenberg-i}
\thanksauthor{Jordan Ellenberg}





I will give a ``sales pitch'' for thinking about these problems in the context 
of global function fields. The idea is that the main problems can be approached 
more geometrically. Some problems in arithmetic statistics are much easier in 
the function field context. 





\subsection{Motivating examples}

\begin{enonce}{Question}
How many integers are there between $N$ and $2 N$?
\end{enonce}

See \autoref{sec:granville-ii} for an interesting (and sophisticated) approach 
to this question. 

\begin{enonce}{Question}
How many squarefree integers are there between $N$ and $2 N$?
\end{enonce}

Call this number $\squarefree(N)$. To be squarefree is to be indivisible by 
$4,9,25,49,\ldots$, i.e.~not divisible by $p^2$ for any prime $p$. One might 
expect ``being indivisible by $p^2$'' to be independent for distinct $p$, so 
\begin{align*}
  \squarefree(N) 
    &\sim N\cdot\left(1-\frac 1 4\right)\left(1-\frac 1 9\right) \cdots \\
    &= N\cdot \prod_p \left(1-p^{-2}\right)^{-1} \\
    &= \zeta(2)^{-1} N .
\end{align*}
So $\lim_{N\to \infty}\frac{\squarefree{N}}{N} = \zeta(2)^{-1}$. This is a 
common phenomenon in arithmetic statistics -- some kind of behavior 
asymptotically occurs an $L$-value percent of the time. 

First, let's understand how this problem looks over more general global fields. 

\begin{defi}
A \emph{global field} is either 
\begin{itemize}
  \item A number field, i.e.~a finite extension of $\dQ$. 
  \item The function field of a curve over a finite field $\dF_q$. 
    (Equivalently, a field isomorphic to a finite extension of $\dF_q(t)$.)
\end{itemize}
\end{defi}

We will be quite loose in identifying a curve over $\dF_q$ and its function 
field, because there is an (anti-)equivalence of categories between smooth 
proper geometrically irreducible (aka ``nice'') curves over $\dF_q$ and 
field extensions of $\dF_q$ of transcendence degree $1$. 





\subsection{The analogy between number fields and function fields}

For a number field $K$, there is a unique embedding $\dQ\hookrightarrow K$. But 
there might be \emph{many} ways to embed $\dF_q(t)$ into a global field $K$. 
For example, $\dF_q(t^{17})\subset\dF_q(t)$ is a ``non-standard'' embedding of 
$\dF_q(t)$ into $\dF_q(t)$. So global function fields do not ``come with'' the 
structure of an extension of $\dF_q(t)$. This phenomenon lies behind the fact 
that Mordell-Weil ranks are unbounded over function fields. (See examples of 
Ulmer.) In this lecture we'll mainly talk about $\dF_q(t)$. 

What is the function-field analogue of counting squarefree integers in a box? 
One problem is that $\dQ$ has only one ``nice'' subring, whereas $\dF_q(t)$ has 
lot of ``nice'' subrings. We'll use the following analogy:
\begin{center}
\begin{tabular}{c|c}
number fields & function fields \\ \hline
$\dQ$ & $\dF_q(t)$ \\
$\dZ$ & $\dF_q[t]$ \\
$|\cdot|:\dZ\to \dR$ & $|f|_\infty = q^{\deg f}$ \\
$[N, 2 N]=\{n\in \dN:|n|\sim N\}$ & set of monic polys with $|f|= N=q^n$ \\
$\#(\dN\cap [N,2 N])\sim N$ & $\#($monic polys with $|f|=N)=N$ \\
of these, $\sim \zeta_\dZ(2)^{-1} N$ are squarefree & $\sim \zeta_{\dF_q[t]}(2)^{-1}$ are squarefree
\end{tabular}
\end{center}

That is, the limiting proportion of squarefree monic polynomials in 
$\dF_q[t]$ is 
\[
  \prod_p \left(1-|p|^{-2}\right) = 1-q^{-1} .
\]
as $p$ ranges over monic irreducible polynomials in $\dF_q[t]$. In fact, the 
number of squarefree monic polynomials of degree $n$ in $\dF_q[t]$ is exactly 
$q^n-q^{n-1}$ for all $n\geqslant 2$, and $q$ for $n=1$. So we have a 
power-saving result with \emph{much} better error term over function fields. So 
we shouldn't think of there being an analogy between any particular number field 
and any particular function field. Rather, there is an analogy between the 
\emph{class} of number fields and the \emph{class} of function fields. 





\subsection{Geometric picture}

What is geometric about what we've done? We introduce yet another function 
field, $\dC(t)$. We can once again think about the set of monic squarefree 
polynomials of degree $n$ in $\dC[t]$. This set is not just a set -- it is a 
\emph{space} (namely an algebraic variety). The space of monic squarefree 
polynomials of degree $n$ is called the (unordered) \emph{configuration space} 
of $\dC$, denoted $\configuration^n \dC$. It parameterizes $n$-tuples of 
distinct points in $\dC$, up to permutation. This isomorphism is given by 
$f\mapsto \{\text{roots of $f$}\}$. The inverse sends an $n$-tuple 
$(z_1,\dots,z_n)$ to the polynomial $f(t)=(t-z_1)\cdots(t-z_n)$. 

We are morally constrained to think of this configuration space not just as 
a complex manifold, but as a scheme over $\spectrum\dZ$. Namely, there is a 
scheme $\configuration^n \dA^1$ over $\spectrum\dZ$ such that 
\[
  (\configuration^n\dA^1)(K) = \{\text{monic squarefree polynomials of degree $n$ in $K[t]$}\}
\]
for any field $K$. In fact, this has a simple description. Namely, the moduli 
space of \emph{all} monic polynomials of degree $n$ is $\dA^n$. A polynomial $f$ 
is squarefree if and only if the discriminant $\Delta(f)$ is nonzero, where 
$\Delta$ is a polynomial in the coefficients of $f$. For example, 
\[
  \Delta(t^2+a_1 t+a_2) = a_1^2 - 4 a_2 .
\]
So $\configuration^n\dA^1$ is $\dA^n\smallsetminus V(\Delta)$. Note: we would 
get a different space if we parameterized ordered $n$-tuples numbers, where 
we care about ordering. We'll call that $\pureconfiguration^n$, the 
\emph{pure configuration space}. The group $S_n$ acts on $\pureconfiguration^n$ 
by permuting the $n$-tuples, and the quotient 
$\pureconfiguration^n/S_n$ is $\configuration_n$. Note that 
$\pureconfiguration^n\dA^1=\dA^n\smallsetminus \bigcup_{i\ne j} V(z_i-z_j)$, 
where $z_1,\dots,z_n$ are the the coordinates of $\dA^n$. 

The set of monic squarefree polynomials in $\dF_q[t]$ of degree $n$ is just 
$\configuration^n\dA^1(\dF_q)$. So our counting problem is: what is 
$\#\configuration^n\dA^1(\dF_q)$? We saw that the answer is $q^n-q^{n-1}$. 

What if we only cared about what happens as $q\to \infty$? For example, what is 
the probability that a degree-$n$ polynomial over $\dF_q[t]$ is squarefree? We 
had been fixing $q$ and letting $n\to \infty$. A simpler question is fixing $n$ 
and letting $q\to \infty$. As $q\to \infty$, 
\[
  \lim_{q\to\infty}\frac{\#\configuration^n\dA^1(\dF_q)}{q^n} 
    = \lim_{q\to \infty}\frac{\#(\dA^n\smallsetminus V(\Delta))(\dF_q)}{q^n} 
    = 1 .
\]





\subsection{M\"obius functions}

\begin{enonce}{Question}
What is the average of the M\"obius function?
\end{enonce}

Recall the \emph{M\"obius function} $\mu$ is the arithmetic function defined by 
\[
  \mu(n) = \begin{cases} 0 & n\text{ is squarefree} \\ 1 & n\text{ the product of an even number of distinct primes} \\ -1 & n\text{ the product of an odd number of distinct primes} \end{cases}
\]
Earlier, we computed that the expected value of $\mu^2$ is 
$\expected(\mu^2)=\zeta(2)^{-1}$. How does this look for function fields? We 
can define the M\"obius function of a polynomial in exactly the same way. But 
over function fields, we have the beautiful 

\begin{enonce}{Fact}
\[
  \mu(f) = (-1)^{\deg f} \left(\frac{\Delta(f)}{q}\right) 
\]
where $\left(\frac{\cdot}{q}\right)$ is the Legendre symbol. 
\end{enonce}

Note that $(-1)^n\mu(f)+1$ is the number of square roots of $\Delta(f)$ in 
$\dF_q$, where $n=\deg f$. Let's make a variety geometrizing this problem. 
Define $Y_n$ to be the space parameterizing pairs $(f,y)$, where $f$ is a monic 
squarefree degree $n$ polynomial, and $y$ is a square root of $\Delta(f)$. Then 
\[
  \#Y_n(\dF_q) = \sum_{f:\deg f=n} \left((-1)^n \mu(f) + 1\right)
\]
So $\#Y_n(\dF_q) - q^n = (-1)^n \sum_f \mu(f)$. We expect 
\[
  \# Y_n(\dF_q) = q^n + o(q^n) .
\]
Why? There is a map $Y_n\to \dA^n$ given by $(f,y)\mapsto f$. This is a 
double branched at the vanishing locus of $\Delta$. In general, we expect an 
$n$-dimensional variety to have approximately $q^n$ points over $\dF_q$. But 
this expectation only is valid when the variety is irreducible. So we think 
$\# Y_n(\dF_q)\sim q^n$ because we think $Y_n$ is irreducible. Indeed, the 
Weil conjectures guarantee that if $Y_n$ is geometrically irreducible, then 
\[
  \lim_{q\to \infty} \frac{\#Y_n(\dF_q)}{q^n} = 1 ,
\]
so the limit as $q\to \infty$ of the average of the M\"obius function is zero. 

But how do we \emph{actually know} that $Y_n$ is irreducible? What if 
$\Delta(a_1,\dots,a_n)$ were actually $G^2$ for some other polynomial $G$? 
Then $\mu(f)$ would be $(-1)^n$ for \emph{all} $f$ of degree $n$. I argue that 
the underlying idea here is a computation of monodromy. 





\subsection{Monodromy}

Recall the $S_n$-Galois cover 
$\pureconfiguration^n \twoheadrightarrow \configuration^n$. The normal subgroup 
$A_n\subset S_n$ corresponds to an intermediate (degree-2) Galois cover 
$U_n\to \configuration^n$. In fact, the following diagram is Cartesian with 
the left arrow being an \'etale cover with group $\dZ/2$:
\[\xymatrix{
  U_n \ar[r] \ar[d] 
    & Y_n \ar[d] \\
  \configuration^n{} \ar@{^{(}->}[r] 
    & \dA^n .
}\]
It is sufficient to show that $U_n$ is irreducible. A $\dZ/2$-cover of 
$\configuration^n$ is a map $\pi_1(\configuration^n)\to \dZ/2$, and the cover 
is irreducible if and only if this map is surjective. Whenever we have a 
``cover of moduli spaces'' $Y\to X$ of degree $n$, we have a map 
$\pi_1(X)\to S_n$. The image of this map is called the \emph{monodromy group} 
of the cover and $Y$ is irreducible if and only if the monodromy group is 
transitive. 

In \autoref{sec:ellenberg-ii}, we'll look at the idea that big monodromy 
implies ``averages are what you expect'' in the large $q$ regime. Sometimes, 
monodromy is not big, and its ``smallness'' can sometimes explain the failure 
of of heuristics. 





\subsection{Computational question}

This is related to the discussion of variation of Mordell-Weil ranks. As 
discussed above, there are $q^n-q^{n-1}$ squarefree onic polynomials of degree 
$n$ in $\dF_q[t]$. For each such $f(t)$, let 
\[
  C_f:y=f(t) 
\]
be the corresponding hyperelliptic curve. Its zeta function has the form 
\[
  \zeta(C_f,s) = \frac{P_f(q^{-s})}{(1-q^{-s})(1-q^{1-s})} ,
\]
where $P_f\in \dZ[X]$ has degree $2 g$, and all its roots have absolute value 
$q^{1/2}$. 

The question is: for how many $f$ does $P_f$ have $q^{1/2}$ as a root. Does 
this proportion look like $q^{\alpha n}$ for some $0<\alpha<1$?





% !TEX root = sms.tex

\section{Counting methods over global fields}\label{sec:wang-ii}
\thanksauthor{Jerry Wang}





In the previous lectures, we have seen how to parameterize objects of 
arithmetic interest by looking at orbits of group actions, and we have counted 
these orbits using analytic methods. In this lecture, we'll generalize the 
techniques to global fields. 





\subsection{Terminology}

A \emph{global field} K is one of the following:
\begin{itemize}
  \item number field (finite extension of $\dQ$)
  \item finite separable extension of $\dF_q(T)$
\end{itemize}
For global fields of finite characteristic, we are implicitly choosing a map 
from the corresponding curve to $\dP^1$. With that map, we can always define a 
ring of integers $\cO$, and a set $M_\infty$ of infinite places. Our underlying 
example will be the average size of $\selmer_2(E/K)$. 

Throughout, $K$ is a global field, not of characteristic $2$ or $3$. 





\subsection{Heights for global fields}

Let $K$ be a global field. An elliptic curve $E$ over $K$ can be written as 
$E:y^2=x^3+A x+B$ with $A,B\in K$. We think of $(A,B)\in \dP(4,6)$, where 
$\dP(4,6)=\dG_\multiplicative\backslash \dA^2\smallsetminus 0$ via the action 
$\alpha\cdot (A,B)=(\alpha^4A,\alpha^6 B)$. If $(A,B)\in \dA^2(K)$, define 
a fractional ideal 
\[
  I = \{\alpha\in K:\alpha\cdot (A,B)\in \dA^2(\cO)\} .
\]
Set 
\[
  H(A,B) = \norm(I)\prod_{v\in M_\infty} \max\left\{|A|_v^{1/4},|B|_v^{1/6}\right\} .
\]
A simple application of the product formula shows that this height is 
invariant under the action of $\dG_\multiplicative$. Unfortunately, the set 
$\dA^2(K)_{<X}=\{x\in \dA^2(K):H(x)<X\}$ might not be ``bounded.'' The solution 
is to construct a nice fundamental domain for 
$\dG_\multiplicative(K)\backslash S(K)$ (Here and elsewhere 
$S=\dA^2\smallsetminus 0$) so that 
$(\dG_\multiplicative(K)\backslash S(K))\cap S(K)_{<X}$ is bounded. 





\subsection{Orbit parameterization over \texorpdfstring{$K$}{K}}

There is a bijection between $\selmer_2(E)$ and the set of locally soluble 
orbits for the action of $G(K)$ on $V(K)$. Here $G=\projectivegenerallinear(2)$ 
and $V=\symmetric^4(2)$. If $K$ has characteristic not $2$ or $3$, everything 
works fine. 





\subsection{Locally soluble \texorpdfstring{$K$}{K}-orbits to integral orbits}

The key input over $\dQ$ is that if $v\in V(\dQ_p)^\mathrm{sol}$ with integral 
invariants, then there exists $g\in G(\dQ_p)$ such that $g v\in V(\dZ_p)$. 
Unlike our definition of heights, which used $h_\dQ=1$ and needed to be 
modified for general $K$, things here translate easily. 

\begin{lemm}
If $v\in V(K_\fp)^\mathrm{sol}$ with invariants in $\cO_\fp$, 
$\fp\notin M_\infty$, then there exists $g\in G(K_\fp)$ such that 
$g v\in V(\cO_\fp)$. 
\end{lemm}

Morally, replace $\dQ$ with $K$, $\dQ_p$ with $K_\fp$, and $\dZ_p$ with 
$\cO_\fp$. But we need to be careful: there is a fundamental difference in 
behavior between $\dZ$ and general $\cO_K$. 

Suppose $v\in V(K)^\mathrm{ls}$ with invariants in $\cO$. Then for all 
$\fp\notin M_\infty$, there exists $g_\fp\in G(K_\fp)$ such that 
$g_\fp v\in V(\cO_\fp)$. Put $g=(g_\fp)_{\fp\notin M_\infty}\in G(\dA_f)$, 
where $\dA_f$ is the ring of finite adeles. Inside $G(\dA_f)$ are two 
subgroups. One is $U=\prod_{\fp\notin M_\fp} G(\cO_\fp)$, the other is 
$G(K)$. If $K$ has trivial class group, the double quotient 
$U\backslash G(\dA_f)/G(K)$ will be trivial, but in general it is only finite. 
For number fields, this due to Borel \cite{b63}, and for function fields this is 
due to Conrad \cite{c12}. Put 
\[
  G(\dA_f) = \coprod_{\beta\in \class(G)} U \beta G(K) .
\]
There exists $g_\fp'\in G(\cO_\fp)$, $\beta\in \class(G)$, $h\in G(K)$ such 
that for all $\fp\notin M_\infty$ we have $G_\fp = g_\fp' \beta h$. Define 
\begin{align*}
  V_\beta &= V(K)\cap \beta^{-1}\Bigl(\prod_{\fp\notin M_\infty} V(\cO_\fp)\Bigr) \\
  G_\beta &= G(K)\cap \beta^{-1} U .
\end{align*}
Then the groups $V_\beta$ and $G_\beta$ are commensurable with 
$V(\cO)$ and $G(\cO)$. Since $\beta h v\in V(\cO_\fp)$ for all 
$\fp\notin M_\infty$, we have $h v\in V_\beta$. 

For any subgroup $G_0\subset G(K)$ and any $G_0$-invariant subset 
$V_0\subset V(K)$, $X>)$, let $N(V_0,G_0,X)$ be the number of irreducible 
$G_0$-orbits in $V_0$ of height $\leqslant X$, where each $G_0 v$ is weighted 
by 
\[
  \frac{1}{\#\stabilizer_{G_0}(v)} .
\]
Let $m:V(K)\to [0,1]$ be a $G_0$-invariant map defined by some congruence 
conditions (i.e.~$m=\prod_\fp m_\fp$). Let $N_m(V_0,G_0,X)$ be defined as 
$N(V_0,G_0,X)$, but weighted by 
\[
  \frac{m(v)}{\#\stabilizer_{G_0}(v)} .
\]

\begin{theo}
Define a weight $m'$ by 
\[
  m'(v)=\chi_{V(K)^\mathrm{ls}}(v) \frac{1}{\#\stabilizer_{G(K)}(v)}\left(\sum_{\beta\in \class(G)}\sum_{v_\beta\in G_\beta\backslash V_\beta\cap V(K)v} \frac{1}{\#\stabilizer_{G_\beta}(v_\beta)}\right)^{-1} .
\]
Then 
\[
  N(V(K)^\mathrm{ls},G(K),X) = \sum_{\beta\in \class(G)}N_{m'}(V_\beta,G_\beta,X) . 
\]
\end{theo}

It is nontrivial (but true) that $m' = \prod m'_\fp$. 





\subsection{Count integral orbits soluble at infinity}

Define $N_{m_\infty'}(V_\beta,G_\beta,X)$. Put 
$K_\infty = \prod_{v\in M_\infty} K_v$. Construct 
$G_\beta\backslash V(K_\infty)_{<X}$. Set 
\begin{align*}
  R(X) &= G(K_\infty) \backslash V(K_\infty)_{<X} \\
  \cF_X &= G_\beta\backslash G(K_\infty) 
\end{align*}
Then $\cF_\beta R(X)\twoheadrightarrow G_\beta\backslash V(K_\infty)_{<X}$. The 
fiber over $G_\beta v$ has size 
\[
  \frac{\#\stabilizer_{G(K_\infty)}(v)}{\#\stabilizer_{G_\beta}(v)} .
\]
We want 
\[
  N_{m_\infty}(V_\beta,G_\beta,X) = \int_{\cF_\beta R(X)} \frac{m_\infty(v)}{\#\stabilizer_{G(K_\infty)}(v)}\,\mathrm{d} v_{\infty,\beta} + \text{error} ,
\]
were we normalize our Haar measure by 
$v_{\infty,\beta}(V(K_\infty)/V_\beta)=1$. 
To continue, we need a version of Davenport's lemma for function fields. It is 
proved using Poisson summation. A more serious problem is the cusps. Without 
loss of generality assume $G_\beta - G(\cO)$ and $V_\beta=V(\cO)$. Reduction 
theory (worked out by Springer) tells us what $G(\cO)\backslash G(K_\infty)$ 
looks like for all local fields. There is still a ``$NAK$ decomposition.'' 

When $G=\projectivegenerallinear(2)$, we have 
\begin{align*}
  N &= \begin{pmatrix} 1 \\ \ast & 1 \end{pmatrix} \\
  A &= \left\{\begin{pmatrix} t^{-1} \\ & t\end{pmatrix}:t>\frac{\sqrt 3}{2}\right\} 
\end{align*}
Just as when $K=\dQ$, we cut off the cusps. Restrict the representation $V$ to 
the torus $A$: it decomposes as $V=\bigoplus_{\chi\in U_0} \chi$. For 
example, 
\[
  \symmetric^4(2) = \underbrace{\chi_{x^4}}_{t^{-4}} \oplus \underbrace{\chi_{x^3 y}}_{t^{-2}}\oplus \underbrace{\chi_{x^2 y^2}}_1 \oplus \underbrace{\chi_{x y^3}}_{t^2}\oplus \underbrace{\chi_{y^4}}_{t^4} .
\]
Describe reducibility using subsets of $U_0$. The cusp set  
$V(K_\infty)^\mathrm{cusp}\subset V(K_\infty)$ is defined by 
$|v(\chi)|<c_1$ for some $\chi\in U_0$, where $c_1$ is chosen so that if 
$v\in V(\cO)$, $|v(\chi)|<c_1$, then $v(\chi)=0$. 

There is a combinatorial condition on the characters of $A$ that implies 
\begin{enumerate}
  \item The number of irreducible elements of the cusp is small. 
  \item The volume of the cusp is small. 
\end{enumerate}
This condition only depends on the field $K$ through the torus $T$. For 
example, if the group is split over $\dQ$, the condition does not depend on the 
field at all. This has been worked out for all the representations we've seen 
so far. 

Finally, we need an estimation of reducibility. This is a purely local 
computation. 





\subsection{Solubility at finite primes}

We do this through the weight function $m_\fp'$. Again, this is purely 
local. 





\subsection{Uniformity estimate}

This is more-or-less local. The methods work over any global field. It has 
been worked out for $\selmer_n(E)$ with $n\in \{2,3,4,5\}$, and to count field 
extensions. 





\subsection{Compute local integrals}

This is (obviously) a local computation. 

The final result is: 
\[
  N(V(K)^\mathrm{ls},G(K),X) = \tau(G/K) \mu_\infty(X) \prod_{\fp\notin M_\infty} \mu_\fp .
\]

\begin{theo}
When all $E/K$ are ordered by height, then for $n\in \{2,3,4,5\}$, we have 
\[
  \average(\#\selmer_n E) = \sum_{d\mid n} d .
\]
\end{theo}





% !TEX root = sms.tex

\section{The Chabauty method and symmetric powers of curves}\label{sec:park}
\thanksauthor{Jennifer Park}





\subsection{Introduction}

Following Poonen, we say a curve is \emph{nice} if it is smooth, projective, 
and geometrically irreducible. 

\begin{enonce}{Question}
Let $X$ be a nice curve over $\dQ$ of genus $g\geqslant 2$. Find all degree-$d$ 
points on $X$. 
\end{enonce}

If $x\in X(\overline\dQ)$, we say $x$ has \emph{degree $d$} if 
$[\kappa(x):\dQ]\leqslant d$, where $\kappa(x)=\sO_{X,x}/\fm_x$ is the residue 
field at $x$. We could rephrase the problem as: find 
\[
  \bigcup_{[K:\dQ]\leqslant d} X(K) .
\]
The problem is: the compositum of all degree-$\leqslant d$ extensions of $\dQ$ 
is not a number field, so we can't apply Faltings' theorem to conclude this set 
is finite. 

Throughout, we assume $X$ has an effective divisor of degree $d$. This is not 
harmful, because if $X$ had no such divisor, it would have no points of degree 
$d$. We also assume $X$ has a rational point $0\in X(\dQ)$. 

\begin{enonce}{Question}
Let $X$ be a nice curve over $\dQ$ of genus $g\geqslant 2$. Find all 
$\dQ$-points on $\symmetric^d X = \overbrace{X\times \cdots X}^d / S^d$. 
\end{enonce}

A point on $\symmetric^d X$ will be a multiset $\{x_1,\dots,x_d\}$; this lies 
in $(\symmetric^d X)(\dQ)$ if and only if the $x_i$ are $\sigma$-conjugates, 
where $\sigma\in \galois(\overline\dQ/\dQ)$ has order $\leqslant d$. We can 
apply a generalized theorem of Faltings, proved in \cite{f94}, to study 
$(\symmetric^d X)(\dQ)$. 

\begin{theo}[Faltings]
Let $A$ be an abelian variety over $\dQ$, $Y\subset A$ a closed subvariety. Then 
there exists finitely many subvarieties $Y_i\subset Y$ such that each $Y_i$ is 
the translate of an abelian subvariety of $A$, and 
\[
  Y(\dQ) = \bigcup Y_i(\dQ) .
\]
\end{theo}

There is a map $j:\symmetric^d X\to J=\jacobian X$ defined by 
$\{x_1,\dots,x_d\}\mapsto [x_1+\cdots + x_d - d(0)]$. The fibers are $\dP^n$ 
for varying $n$. Applying Faltings' theorem to the image of the map $j$, we 
get $(j(\symmetric^d X))(\dQ) = \bigcup Y_i(\dQ)$, 
whence 
\begin{align*}
  (\symmetric^d X)(\dQ) 
    &= \bigcup_{n_i} \dP^{n_i}(\dQ) \cup \bigcup j^{-1}(Y_i(\dQ)) \\
    &= \bigcup_{n_i} \dP^{n_i}(\dQ) \cup \bigcup_{\dim Y_i\geqslant 1} j^{-1}(Y_i(\dQ)) \cup \bigcup_{\dim Y_i=0} j^{-1}(Y_i(\dQ)) .
\end{align*}
Since $\bigcup \dP^{n_1}(\dQ)$ is definitely infinite and 
$\bigcup_{\dim Y_i\geqslant 1} j^{-1}(Y_i(\dQ))$ could be infinite, we will 
count the quantity $\bigcup_{\dim Y_0=0} j^{-1}(Y_i(\dQ))$. The following 
theorem proved in \cite{hs91} is useful. 

\begin{theo}[Harris-Silverman]
Let $X$ be a nice curve over $\dC$. If $\symmetric^2 X$ contains an elliptic 
curve, then $X$ is either hyperelliptic or bielliptic. 
\end{theo}

Here, a \emph{bielliptic curve} is a double cover of an elliptic curve. If 
$X$ is the hyperelliptic $y^2=f$, we get $\dP^1\subset \symmetric^2 X$, and 
$\{(x,\sqrt{f(x)}),(x,-\sqrt{f(x)}):x\in \dQ\}\subset (\symmetric^2 X)(\dQ)$. 

\begin{defi}
The \emph{special set} $\cS(V)$ of a $\dQ$-variety $V$ is the Zariski closure 
of the union of the images of all nonconstant maps $f:G\to V$, where $G$ ranges 
over group varieties defined over $\overline\dQ$. 
\end{defi}

We will try to count $(\symmetric^d X)(\dQ)\smallsetminus \cS(\symmetric^d X)$. 
When $d=1$, we have the following theorem proved in \cite{c85}. 

\begin{theo}[Coleman]
Fix $g\geqslant 2$ and a prime $p$. There is an effectively computable bound 
$N(g,p)$ such that if $X$ is a nice curve over $\dQ$ of genus $g$ with good 
redution at $p$ and with $g>\rank J(\dQ)$, then $\# X(\dQ)\leqslant N(g,p)$. 
\end{theo}

If $p>2 g$, then $\# X(\dQ)\leqslant \# X(\dF_p)+(2 g-2)$. In that case, Stoll 
improved the bound to $\# X(\dF_p)+2r$. If $p>2$, Stoll improved it further 
to $\# X(\dF_p) + \lfloor\frac{2r}{p-2}\rfloor$. 

When $d\geqslant 2$, some things are known. In 1993, Klassen counted points on 
$\symmetric^d X$ away from a divisor of dimension $d-1$, where $X$ has gonality 
$>d$. Here, the \emph{gonality} of a curve $X$ is the minimal degree of a map 
$X\to \dP^1$. 

Baker-Bhargava-Wetherell explicitely found all points on 
$(\symmetric^2 X)(\dQ)$ for $X$ hyperelliptic. 

In 2009, Siksek removed the gonality hypothesis from Klassen's result. 

\begin{theo}[Park]
Let $d\geqslant 1$, $p$ be a prime, and $g\geqslant 2$. Then there exists an 
effectively computable bound $N(g,d,p)$ such that for every nice curve $X$ over 
$\dQ$ of genus $g$ with good reduction at $p$ with $\rank J\leqslant g-d$, 
satisfying (*), then 
\[
  \#\left(\symmetric^d(\dQ)\smallsetminus \cS(\symmetric^d X)\right) \leqslant N(g,d,p) .
\]
\end{theo}

If $\rank J\leqslant 1$, the hypothesis (*) is unnecessary. It is a rather 
technical hypothesis that we won't go into here. 





\subsection{Some applications}

\begin{prop}[Park]
We can take $N(3,3,2)=1539$ for any degree-$7$ odd hyperelliptic curve $X$ 
such that $\rank J\leqslant 1$, with good reduction at $2$. 
\end{prop}

This bound, while not fantastic, is far better than the one from Faltings' 
theorem. 

\begin{prop}[Park]
A positive proportion of hyperelliptic curves with $X$ genus $g\geqslant 3$ 
have no points of $\deg\leqslant 2$-points in $\symmetric^2 X$ outside of the 
special set of $\symmetric^2 X$. 
\end{prop}

For these curves, $(\symmetric^2 X)(\dQ)$ is parameterized by $\dP^1$. 





\subsection{Chabauty's method}

For a more in-depth introduction, see \autoref{sec:poonen-iv}. Let $X/\dQ$ be 
a nice curve with rank $r<g$ and good reduction at $p$. For 
$\omega\in \Gamma(J_{\dQ_p},\Omega^1)$, there is a unique group homomorphism 
$\eta_\omega:J(\dQ_p) \to \dQ_p$ sending $x$ to $\int_0^x\omega$ when this 
integral is defined. It is known that there exists $\omega$ such that 
$\eta_\omega(J(\dQ))=0$. 

On each residue disk $U$, there is a coordinate system in which 
\[
  \omega|_U = \sum w_i(t_1,\dots,t_g)\, \mathrm{d} t_i .
\]
Restricting to the curve $X|_U$, we get 
$\omega|_{X\cap U} = w(t)\, \mathrm{d}t$. Then 
\[
  \#\{x\in U:\eta_\omega(x)=0\} \geqslant \#(X(\dQ)\cap U) .
\]
Consider the restriction $\eta_\omega:(\symmetric^d X)(\dQ_p) \to \dQ_p$, 
given by 
\begin{align*}
  \{x_1,\dots,x_d\} 
    &\mapsto \int_0^{[x_1+\cdots + x_d-d(0)]} \omega \\
    &= \int_0^{[x_1-0]} \omega + \int_0^{[x_2 + \cdots + x_d-(d-1)(0)]}\omega \\
    &= \int_0^{[x_1-0]}\omega + \cdots + \int_0^{[x_d-0]}\omega \\
    &= I(t_1) + \cdots + I(t_d) .
\end{align*}
We can estimate the zeros of this map in terms of the power series $I$. We have 
to assume $r+d\leqslant g$. From this we get $d$ independent power series in 
$K_{x_i}\llbracket t_i\rrbracket$, where $[K_{x_i}:\dQ_p]\leqslant d$, 
vanishing on $(\symmetric^d X)(\dQ)$. 

There is a theory of generalized Newton polygons, partially done by Rabinoff 
and Bernstein. 





% !TEX root = sms.tex

\section{Topological and algebra-geometric methods over function fields II}\label{sec:ellenberg-ii}
\thanksauthor{Jordan Ellenberg}





In \autoref{sec:ellenberg-i}, we finished by looking at the equidistribution of 
the M\"obius function. In this lecture, we'll show how the underlying geometry 
gives some hints as to why this is the case. 





\subsection{M\"obius function and big monodromy}

Recall: we found that ``average of $\mu(f)=0$'' comes down to 
$\# Y_n(\dF_q) = q^n+o(q^n)$, where $Y_n$ parameterizes pairs $(f,y)$ with $y$ 
a square root of $\Delta(f)$. The scheme $Y_n$ is a branched double cover of 
$\dA^n=\symmetric^n\dA^1$, the space of monic integral polynomials of degree 
$n$.  For this to work, we need $Y_n$ to be irreducible. 

On the level of function fields (that is, generic points) the extension 
corresponding to $Y_n \to \dA^n$ is the quadratic extension 
\[
  K = k(a_1,\dots,a_n) \hookrightarrow K(\sqrt\Delta) .
\]
This quadratic extension is given by a map $G_K=\galois(\bar K/K)\to \dZ/2$. 
The extension is a field precisely when $G_K\to \dZ/2$ is nontrivial. That is, 
$G_K\to \dZ/2$ surjectivs if and only if $K(\sqrt\Delta)$ is a fied, if and 
only if $\Delta$ is \emph{not} a square in $K=k(a_1,\dots,a_n)$. The image of 
$G_K\to \dZ/2$ is the monodromy group, so this is a ``large monodromy'' result. 

\begin{enonce}{Conjecture}[Cohen-Lenstra]
Let $\ell$ be an odd prime and $E_{r,\ell,N}$ be the expected value of 
\[
  \surjection\left(\class\left(\dQ(\sqrt{-d})\right), (\dZ/\ell)^{\oplus r}\right) 
\]
for $d$ random in $[N,2 N]$. Then $\lim_{N\to\infty} E_{r,\ell,N} = 1$. 
\end{enonce}

For example, when $r=1$, we have $\surjection(A,\dZ/\ell) = \#A[\ell]-1$. We 
will try to give a function-field analogue of the Cohen-Lenstra heuristic. 





\subsection{Cohen-Lenstra over function fields}

The analogue of $\dQ(\sqrt{-d})$ for $d\in [N,2 N]$ is 
$\dF_q(t,\sqrt f)$ for $\deg f=n$. The analogue of $-d<0$ is to require the 
extension $\dF_q(t)(\sqrt f)$ to be ramified at infinity. This happens exactly 
when $n$ is odd. So we're looking at ``hyperelliptic curves of odd degree.'' 
The analogue of an ideal on a curve is a divisor, and the analogue of the class 
group is the Jacobian, namely 
\[
  \class\left(\dF_q(t)(\sqrt f)\right) \simeq \jacobian(C_f)(\dF_q) .
\]
where $C_f$ is the curve $y^2=f(t)$. For simplicity, write $J_f$ instead of 
$\jacobian(C_f)$. Note that $J_f$ is a $g$-dimensional abelian variety, and 
$J_f[\ell](\overline{\dF_q}) \simeq (\dZ/\ell)^{2 g}$. Here $g=(n-1)/2$. 

Define a new space $\configuration_1^n(\ell)$, which parameterizes pairs 
$(f,P)$, where $f$ is squarefree of degree $n$ and 
$P\in J_f[\ell]\smallsetminus 0$. So 
$\pi:\configuration_1^n(\ell) \to \configuration^n$ is an \'etale cover of degree 
$\ell^{2g}-1$. Define $E_{q,\ell,r,n}$ to be the expected value of 
$\#\surjection\left(J_f(\dF_q),(\dZ/\ell)^{\oplus r}\right)$. Cohen-Lenstra 
would suggest that $\lim_{n\to \infty} E_{q,\ell,r,n} = 1$. 

When $r=1$, this is 
\begin{align*}
  \expected_f\left(\# J_f(\dF_q)[\ell]-1\right) 
    &= \expected_f\left(\pi^{-1}(f)(\dF_q)\right) \\
    &= \frac{\#\configuration_1^n(\ell)(\dF_q))}{\#\configuration^n(\dF_q)} \\
    &= \frac{\#\configuration_1^n(\ell)(\dF_q)}{q^n-q^{n-1}} .
\end{align*}
So Cohen-Lenstra suggests that $\#\configuration_1^n(\ell)(\dF_q)\sim q^n$ as 
$n\to \infty$. Note that to get 
\[
  \lim_{q\to \infty} \frac{\#\configuration_1^n(\ell)(\dF_q)}{\#\configuration^n(\dF_q)} = 1,
\]
we would need $\configuration_1^n(\ell)$ to be irreducible. This is known to 
be true, and is proven via a monodromy computation. 

Once again, let $K=\dF_q(a_1,\dots,a_n)$, the function field of 
$\configuration^n$ over $\dF_q$. Let $L$ be the function field of 
$\configuration_1^n(\ell)$ over $\dF_q$. Then $L$ is $K$ adjoined an 
$\ell$-torsion point of $J_f$, where $f=t^n+a_1 t^{n-1} + \cdots + a_n$. 

A curve $C_f$ over $K$ gives rise to a Galois representation 
\[
  \rho:G_K=\galois(\bar K/K) \to \generalsymplectic_{2 g}(\dZ/\ell) = \generalsymplectic(J_f[\ell](K^s),e),
\]
where $e$ denotes the Weil pairing. Our question is: is $L$ a field? 
Equivalently, does the monodromy group $\rho(G_K)$ act transitively on 
$(\dZ/\ell)^{2 g}\smallsetminus 0$? It's a good exercise to show that 
$\generalsymplectic_n(\dF_q)$ acts transitively on $\dF_q^n\smallsetminus 0$ 
for any finite field $\dF_q$. 

\begin{theo}[J-K Yu, 1995]
For $K$ as above, $\rho(G_K) = \generalsymplectic_{2 g}(\dF_\ell)$. 
\end{theo}

This is a nice ``big monodromy theorem.'' That is, the monodromy group is as 
large as it can be. 





\subsection{What happens when \texorpdfstring{$r=2$}{r=2}}

We could define $\configuration_2^n(\ell)$, which parameterizes triples 
$(f,P,Q)$, where $f$ is above and $P,Q$ are linearly independent points on 
$J_f[\ell]$. Then 
\[
  \frac{\#\configuration_2^n(\ell)(\dF_q)}{\#\configuration^n(\dF_q)} = E_{q,2,\ell,n} ,
\]
because $\#\surjection(J_f(\dF_q),(\dZ/\ell)^2)$ is (by duality for finite 
abelian groups) the number of injections $(\dZ/\ell)^2 \to J_f(\dF_q)$. 
Unfortunately, the space $\configuration_2^n(\ell)$ is not irrereducible. The 
components of $\configuration_2^n(\ell)$ are naturally identified with the 
orbits of $\galois(\bar K/K\overline{\dF_q})$ on the set of injections 
$(\dZ/\ell)^2 \hookrightarrow (\dZ/\ell)^{2g}$. Even when $g=1$, this action is 
not transitive. Let $V_0=\dF_\ell^2$ and 
$\iota:V_0\hookrightarrow\dF_\ell^{2g}$. The Weil pairing $\omega$ pulls back 
to a pairing on $V_0$. This pairing is preserved by the monodromy group. The 
quantity $\omega(P,Q)$ is an invariant. 

So in fact, $\configuration_2^n(\ell)$ has $\ell$ components, parameterized by 
$\langle P,Q\rangle$ via the map 
$(f,P,Q)\mapsto \langle P,Q\rangle\in \mu_\ell$. How does 
$G_{\dF_q} = \galois(\overline{\dF_q}/\dF_q)$ act on the components? It (the 
Frobenius at $q$) just multiplies $\langle P,Q\rangle$ by $q$. So the 
number of $\dF_q$-rational components of $\configuration_2^n(\ell)$ is the 
number of elements $x\in \dZ/\ell$ such that $q x=x$. This depends strongly on 
$q$. It is 1 if $q\not\equiv 1\pmod\ell$, but $\ell$ when $q\equiv 1\pmod\ell$.
That is, when $\mu_\ell\cap \dF_q=1$ there is one component, and 
$\ell$ components if $\mu_\ell\subset \dF_q$. 

The following is work of Derek Garton. You can compute the ``modified 
Cohen-Lenstra distribution,'' whose moments match those given by 
\[
  E_{q,r,\ell,n} = \#\{\text{$\dF_q$-rational components of $\configuration_r^n(\ell)$}\} ,
\]
a geometrically motivated repair of Cohen-Lenstra in the presence of extra 
roots of unity. 





\subsection{Selmer groups}

The inspiration here is the beautiful paper \cite{j02}. What does the 3-Selmer 
group of a random $E/\dF_q(t)$ look like? If $E:y^2=f_t(x)$, let $\cE$ be the 
elliptic surface $y^2=f(t,x)$. It turns out that 
$\selmer_3(E/\dF_q(t))$ ``is'' $\h^2(\cE_{\overline{\dF_q}},\dZ/3)(\dF_q)$. You 
can play the same game, expressing the average number of nontrivial elements of 
$\selmer_p(E)$ as $\frac{\#Y_n(\dF_q)}{\#X_n(\dF_q)}$, where $Y_n\to X_n$ is a 
cover corresponding to the action of $G_K$ on 
$\h^2(\cE_{\overline{\dF_q}},\dZ/\ell)$. This cohomology group has a canonical 
symmetric pairing, so ``big monodromy'' would mean $\rho(G_K)$ is the entire 
orthogonal group. Given a big monodromy theorem, the number of components would 
be the number of orbits of $\orthogonal_d(\dZ/\ell)$ acting on 
$(\dZ/\ell)^d$. It is $\ell+1$. 

It seems like a proof that ``in the large $q$ limit,'' 
$\average(\#\selmer_\ell)=\ell+1$. 





\subsection{From \texorpdfstring{$q\to \infty$}{q to infty} to the fixed \texorpdfstring{$q$}{q} regime}

The reason we know that $\# X_n(\dF_q)=q^n+o(q^n)$ when $X_n$ irreducible is 
the Weil bounds. What's really going on is the Grothendieck-Lefschetz trace 
formula 
\[
  \# X(\dF_q) = \sum_{i=0}^{2\dim X} (-1)^i \trace\left(\frobenius_q,\h_c^i(X_{\overline\dF_q},\dQ_\ell)\right) .
\]
The Weil bounds tell you that as $q\to \infty$, the $i=0$ term contributes 
$100\%$ of this sum. But $\h^0$ is just the vector space generated by connected 
components. 

If we want to let $q$ stay fixed, we have to show that the $\h^i$ contribute 
nothing to start with. The easiest way for this to be true is for the higher 
$\h^i$ to vanish. This fits into the general idea that ``families of moduli 
spaces have vanishing higher cohomology,'' i.e.~cohomological stability 
results. 





% !TEX root = sms.tex

\section{Future perspectives}
\thanksauthor{Manjul Bhargava}





\subsection{Future directions / things to think about after the workshop}

We now have a general technique to count (nondegenerate, a.k.a.~irreducible) 
orbits having bounded invariants in a representation $G(\dZ)$ acting on 
$V(\dZ)$, for a representation $(G,V)$ defined over $\dZ$. Our first goal 
is thus to find interesting representations! Of particular interest would be 
representations where the orbits correspond to objects or arithmetic / 
geometric / topological interest. 

In the future, it would be nice to parameterize rings of rank $n>5$ 
(i.e.~parameterize $n$ points in $\dP^{n-2}$). For example, three points in 
$\dP^1$ are parameterized by binary cubic forms. Four points in 
$\dP^2$ are parameterized by pencils of conics, i.e.~pairs of ternary 
quadratic forms, which we know correspond to quartic rings. 
Five points in $\dP^3$ arise as quadruple of $5\times 5$ symmetric matrices via 
the Pfaffians of their linear combinations. This representation of 
$\speciallinear(4)\times \speciallinear(5)$ parameterizes quintic rings. 
In general, we would like to understand $n$ points in $\dP^{n-2}$ in terms of 
forms. Even for $n=6$, we do not know how to do this (in terms of 
prehomogeneous vector spaces or something similar). 

It would also be nice to parameterize $n$-Selmer elements of elliptic curves 
for $n>5$. Geometrically, we want to parameterize maps of genus one curves $C$ 
to $\dP^{n-1}$ of degree $n$. For example, maps $C\to \dP^1$ of degree $2$ 
ramify at four points, so they correspond to binary quartic forms. Maps 
$C\to \dP^2$ are plane cubics, parameterized by ternary cubic forms. Maps 
$C\to \dP^3$ are the intersection of two quadrics, hence parameterized by 
pairs $2\otimes \symmetric^2(4)$. Finally, maps $C\to \dP^4$ are the 
intersection of five quadrics, which come from $5\otimes \bigwedge^2 5$ 
(quintuples of $5\times 5$ skew-symmetric matrices). 

All these constructions were (in some form) known in the 19th century, though 
not over $\dZ$. If we added yet another variable, we would go from points, to 
genus one curves, to K3 surfaces. See \cite{bhk13} for work along these lines. 

We could look for analogous maps for objects involving surfaces or 
higher-dimensional varieties. Also, we could try to count objects parameterized 
by non-coregular surfaces, as in \cite{by13}. Aside from a few ``baby steps,'' 
essentially nothing is known. 

Parameterizing rings of rank $\leqslant 5$ allowed us to parameterize number 
fields. We could add more structure and try to parameterize fields with special 
Galois groups. For example, we don't know how to count $A_4$-quartic, 
$A_5$-quintic, or $D_5$-quintic fields with bounded discriminant. Some work 
along these lines is in \cite{bs13}, in which the action of 
$\specialorthogonal(x^2+x y+y^2)$ on a 2-dimensional space is used to 
parameterize $C_3$-cubic rings. The subspace of pairs of ternary quadratic 
forms 
\[
  \left(\begin{pmatrix} 0 & 0 & 0 \\ 0 & \ast & \ast \\ 0 & \ast & \ast \end{pmatrix}, \ast \right) 
\]
is preserved by a parabolic in $\speciallinear(2)\times \speciallinear(3)$. 
This group parameterizes $D_4$-quartic orders. 

We could try to parameterize $n$-Selmer elements in families of 
elliptic curves with extra structures (e.g.~marked rational points). We know 
how to parameterize 2 rational points, but have no idea how to do $k>2$ 
rational points. 

Similarly, we could try to parameterize $n$-Selmer group / set elements of 
higher genus curves, or $k$-tuples of elements of the this type with given 
invariants. (This is needed to get the $k$-th moments.) For example, 
$2\times 2\times 2$ cubes up to $\speciallinear_2(\dZ)$-action parameterize 
pairs of ideal classes in a cubic ring. Kevin Wilson has done some work along 
these lines. 

There are lists of coregular representations. One way of proving new theorems 
is to look through these lists and try to find interesting geometric 
interpretations of these representations. 

Families of modular forms could also be attacked via orbit-spaces of 
representations. 

Is there a systematic way of constructing ``good'' representations for a given 
arithmetic object? Coregular representations always seem to work. There is 
great work of Jack Thorne \cite{t13} in this direction, using Vinberg theory. 
Also there is upcoming work of Ho / Sam. 

Currently, we have no way of showing that a given group action on a 
unirational variety can't parameterize something interesting. Prehomogeneous 
vector spaces and coregular spaces have \emph{not} all been classified. The 
classification in \cite{sk77} is only of irreducible reductive reduced 
prehomogeneous vector spaces. Even their list is not finite -- it contains 
some infinite families. There are already known interesting 
(a.k.a.~parameterize something interesting) prehomogeneous 
vector spaces that are non-reductive or non-reduced, for examples. It would be 
very nice if there were a complete classification of prehomogeneous vector 
spaces. 

It's important to further develop counting techniques. Right now, we have 
two methods: geometry of number and zeta functions. So far, zeta-function 
methods have not worked on higher-dimensional representations. Hopefully, there 
is a way of unifying these (or just using both) to strengthen results. The 
adelic counting method described in \cite{p13} can at least explain some of the 
cancellations that we saw in \autoref{sec:shankar-ii}. 

We could try to connect with other problems in other areas. For example, 
Miller has maps from families of knots with given invariants into certain 
orbit spaces. Also, we could to connect counting techniques as in 
\autoref{sec:wang-ii} with topological techniques as in 
\autoref{sec:ellenberg-ii}. Chabauty methods as in \autoref{sec:poonen-iv} 
and \autoref{sec:park} could also be strengthened. Finally, we should try to 
develop better heuristics for all of the above, along with arithmetic 
justifications for the heuristics. Ellenberg and Venjakob have results in this 
direction for imaginary quadratic fields. Finally, Skinner and Urban's work 
\cite{su14} on $p$-adic $L$-functions vis-\`a-vis the Iwasawa main conjecture 
for $\generallinear(2)$ is also relevant. 





\subsection{Applications to the Birch and Swinnerton-Dyer conjecture}

This is partially an advertisement for the November workshop ``Counting 
arithmetic objects (ranks of elliptic curves)'' 
(\url{http://www.crm.umontreal.ca/act/theme/theme_2014_2_en/counting_e.php}) at 
the Centre de Recherches Math\'ematiques. 
Here are some examples of the sort of results that will be covered. 

\begin{theo}[Bhargava, Shankar]
A positive proportion of elliptic curves over $\dQ$ have rank $0$. 
\end{theo}
\begin{proof}
Use Dokchitser-Dokchitser to show existence of curves with odd (resp.~even) 
$p$-Selmer rank. From $\average(\#\selmer_5)=6$, we conclude that the average 
rank of an elliptic curve is $\leqslant 1.05$. This could be achieved if 
$95\%$ have rank $1$ and $5\%$ have rank $2$. This situation cannot happen if 
$p$-Selmer rank parities are well-distributed, as they are (often enough to 
rule out this scenario). This reduces the average rank bound to 
$\leqslant 0.885$. 
\end{proof}

\begin{theo}[Bhargava, Skinner]
A positive proportion of elliptic curves have rank $1$. 
\end{theo}
\begin{proof}
Dokchitser-Dokchitser implies a lot of elliptic curves have $5$-Selmer rank 
$5$. We want to show that these actually have rank $1$. Use work of Skinner 
\cite{s14} on Heegner points. 
\end{proof}

\begin{theo}[Bhargava, Skinner, Zhang]
At least $>66.48\%$ elliptic curves over $\dQ$ satisfy the rank part of the BSD 
conjecture. They also satisfy the $p$-part of BSD for all $p\ne 2$. 
\end{theo}

\begin{coro}
Most elliptic curves over $\dQ$ have finite Tate-Shafarevich group. 
\end{coro}
\begin{proof}
This follows from work of Kolyvagin. 
\end{proof}

Clearly, the standard assumption ``$p\ne 2$'' at the start of most papers 
needs to be removed. 





% !TEX root = sms.tex

\section{Exercises}





Those directing the problem sessions were (in alphabetical order): 
Jordan Ellenberg, Wei Ho, Jennifer Park, Arul Shankar, Frank Thorne, and 
Jerry Wang. 





\subsection{The parameterization of cubic, quartic, and quintic rings}


\subsubsection{Directly from Wood's cubic rings lecture}
\begin{enumerate}[\indent a)]
  \item Prove that the inverse maps $R\mapsto \discriminant(R)$ and 
    $D\mapsto \dZ[\tau]/\left(\tau^2-D\tau+\frac{D^2-D}{4}\right)$ induce a 
    bijection between the set of quadratic rings (up to isomorphism) and 
    \[
      \{D\in \dZ:D\equiv 0,1\pmod 4\} . 
    \]
  \item In the Delone-Faddeev equations 
    \begin{align*}
      \omega\theta &= n \\
      \omega^2 &= m-b\omega + a\theta \\
      \theta^2 &= \ell-d\omega + c\theta 
    \end{align*}
    prove that associativity is equivalent to the equations 
    \begin{align*}
      n &= -a d \\
      \ell &= -b d \\
      m &= - a c
    \end{align*}
  \item Wood mentioned that if you write $+b$ and $+d$ in place of $-b$ and 
    $-d$ above, the correspondence comes out slightly wrong. Try it and see 
    what happens. 
  \item Orders in cubic number fields correspond to irreducible cubic forms 
    $f(x,y)$ and the number field can be recovered as 
    $\dQ[\theta]/f(\theta,1)$. What happens if you substitute $f(1,\theta)$ for 
    $f(\theta,1)$. (What \emph{must} happen?)
  \item For a cubic form $f$, prove that the functions on its vanishing set 
    $V_f$ determine a cubic ring, which is the same ring obtained by the 
    Delone-Faddeev correspondence. (Describe any special conditions, 
    e.g.~$f\ne 0$, which are necessary in your proof.) 
\end{enumerate}


\subsubsection{Other exercises concerning cubic rings}

We give more exercises for cubic than for quartic or quintic rings. Note that 
most or all of these exercises are interesting for all three parameterizations 
being discussed. You are \emph{strongly encouraged} to extrapolate problems 
from one section to another! What is the same, and what is different? 
\begin{enumerate}[\indent a)]
  \item A good way to get started is to compute lots of examples of the 
    Delone-Faddeev correspondence. (If you don't do any of the other 
    exercises, you should probably do at least this, and the quartic and 
    quintic analogues!) What binary cubic form $f$ corresponds to the cubic 
    ring $\dZ^3$? To $\dZ[\sqrt[3] n]$? Conversely, what cubic rings 
    corresponds to the cubic form $u^3-u v^2+v^3$? To $u(u-v)(u+v)$? To $u^3$? 
    To $0$? Work out these, as well as other examples of your own invention, 
    and compute all of their discriminants. 
  \item Another good way to get started is to work out the details of the 
    Delone-Faddeev and Davenport-Heilbronn correspondences. The exposition 
    given in \cite[\S 2]{bst13} leaves many small details to be checked by the 
    reader. Pick your favorite lemma or proposition and work out the proof in 
    more detail than given in the paper. 
  \item The Delone-Faddeev correspondence is very interesting over $\dF_p$. 
    Assuming for simplicity that $p\ne 2,3$, determine all of the cubic rings 
    over $\dF_p$ as well as the $\generallinear_2(\dF_p)$-equivalence classes 
    of cubic forms over $\dF_p$. How many equivalence classes are there? On the 
    cubic forms side, how large is each $\generallinear_2(\dF_p)$-equivalence 
    class, and how big is each of the corresponding stabilizer groups? If you 
    reduce an integral binary cubic form modulo $p$, what is the relationship 
    between the cubic ring over $\dZ$ and the cubic ring over $\dF_p$?
  \item Work out what the Delone-Faddeev correspondence says over $\dC$: The 
    fact that $\generallinear_2(\dC)$ acts prehomogeneously on binary cubic 
    forms over $\dC$ can be restated by saying that all nonsingular binary 
    cubic forms form a single $\generallinear_2(\dC)$-orbit, and therefore (by 
    Delone-Faddeev) that there is exactly nondegenerate cubic ring over $\dC$ 
    up to isomorphism. Work out this case of the Delone-Faddeev correspondence, 
    describe this cubic ring, and prove its uniqueness directly. 
  \item Classify the set of those $\generallinear_2(\dZ_p)$-orbits on 
    $V(\dZ_p)$ whose discriminants are not divisible by $p$. What about those 
    whose discriminants are exactly divisible by $p$ or $p^2$? Which of those 
    extensions are maximal? 
\end{enumerate}





\subsection{Some cohomological computations for the representation \texorpdfstring{$V=\symmetric_2(n)\oplus \symmetric_2(n)$}{V=Sym2(n)+Sym2(n)} of \texorpdfstring{$G=\speciallinear_n$}{G=SLn} and \texorpdfstring{$H=\speciallinear_n/\boldsymbol\mu_2$}{H=SL2/mu2}}

Suppose $G$ is a reductive group with a representation $V$ over a field $k$. 
et $V\gq G=\spectrum k[V]^G$ denote the canonicl quotient. Let 
$f\in (V\gq G)(k)$ be a rational invariant and suppose $G(k^s)$ acts 
transitively on $V_f(k^s)$ with abelian stabilizers. In Gross' talk, we 
learned two obstructions for the existence of a rational element $v\in V(k)$ 
with invariant $f$. In this worksheet, we will make some computations regarding 
these obstructions when the representation $V$ is the space 
$\symmetric_2(n)\oplus\symmetric_2(n)$ of pairs of symmetric bilinear forms and 
when the reductive group is either $\speciallinear_n$ or 
$\speciallinear_n/\boldsymbol\mu_2$ with the action given by 
$g\cdot (A,B)=(\transpose g A g, \transpose g B g)$. 


\subsubsection{Warm up}

Consider the conjugation action of $G=\generallinear(W)$ on $V=\End(W)$. One 
can obtain invariants by taking the coefficients $c_1,\dots,c_n$ of the 
characteristic polynomial. 

\begin{exercise}
Show that the ring of polynomial invariants $k[V]^G=k[c_1,\cdots,c_n]$ via the 
following steps (or however you want to). 
\begin{enumerate}
  \item Show that for any $c_1,\dots,c_n$ in $k^s$, there exists some 
    $T\in V(k^s)$ with characteristic polynomial 
    \[
      \det(x\cdot 1-T) = x^{2n+1} + c_1 x^{2n} + \cdots + c_{2n+1} .
    \]
    This shows that there is no relation among the invariants. 
  \item Show that for any $c_1,\dots,c_n$ in $k^s$ such that 
    $f(x)=x^{2n+1} + c_1 x^{2n} + \cdots + c_{2n+1}$ has no repeated roots and 
    for any $T,T'\in V_f(k^s)$, there exists some $g\in G(k^s)$ such that 
    $g T g^{-1} = T'$. This shows that there are no other invariants. 
\end{enumerate}
\end{exercise}

Next we consider the conjugation action of the subgroup $H=\speciallinear(W)$ 
on $V=\End(W)$. Let $T\in V(k)$ be a regular semisimple operator, that is its 
characteristic polynomial $f(x)$ has no repeated factors. Let $L=k[x]/f(x)$ be 
the associated $k$-vector space of dimension $n$. 

\begin{exercise}
\begin{enumerate}
  \item Show that the stabilizer $H_T$ of $T$ is isomorphic to 
    $(\weilres_{L/k}\dG_\multiplicative)^{\norm=1}$, the kernel of the norm 
    map $\weilres_{L/k}\dG_\multiplicative\to\dG_\multiplicative$. 
  \item Show that $\h^1(k,H_T)  \simeq k^\times/\norm(L^\times)$ by taking 
    Galois cohomology of the short exact sequence 
    \[
      1 \to H_f \to \weilres_{L/k}\dG_\multiplicative \to \dG_\multiplicative \to 1 .
    \]
    Note Shapiro's lemma implies that $\h^1(k,\weilres_{L/K} M) = \h^1(L,M)$ 
    for any $\galois(L^s/L)$-module $M$. 
  \item Using the same idea, show that 
    $\h^1\left(k,(\weilres_{L/k}\boldsymbol\mu_2)^{\norm=1}\right) \simeq (L^\times/2)^{\norm=\square}$. 
\end{enumerate}
\end{exercise}


\subsubsection{Cohomological obstructions}

The reference for his section is \cite[\S 2]{bgw13}. As shown in 
\autoref{sec:gross-ii}, the abelian groups $G_v$ for $v\in V_f(k^s)$ descend to 
a commutative group scheme $G_f$ over $k$ unique up to unique isomorphism. In 
other words, there are canonical isomorphisms $\iota_v:G_f(k^s)\iso G_v(k^s)$ 
for any $v\in V_f(k^s)$ such that for any $h\in G(k^s)$, $b\in G_f(k^s)$, 
$\sigma\in \galois(k^s/k)$, 
\begin{align*}
  \iota_{h v}(b) &= h \iota_v(b) h^{-1} \\
  \prescript{\sigma}{}{\iota_v(b)} &= \iota_{\prescript{\sigma}{}{v}}(\prescript{\sigma}{}{b}) .
\end{align*}
For any $v\in V_f(k^s)$ and any $\sigma\in \galois(k^s/k)$, choose $g_\sigma$ 
such that $g_\sigma \prescript{\sigma}{}{v}=v$. Define 
\[
  d_{\sigma,\tau} = \iota_v^{-1}(g_\sigma g_\tau g_{\sigma\tau}^{-1}) \in G_f(k^s) .
\]

\begin{exercise}
\begin{enumerate}
  \item Show that $(d_{\sigma,\tau})$ is a 2-cocycle whose image $d_f$ in 
    $\h^2(k,G_f)$ does not depend on the choice of $g_\sigma$. That is, show 
    that for any $\sigma,\tau,\mu\in \galois(k^s/k)$, 
    \[
      \prescript{\sigma}{}{d_{\tau,\mu}} d_{\sigma,\tau\mu} = d_{\sigma\tau,\mu} d_{\sigma,\tau} .
    \]
    and that if each $g_\sigma$ is changed to $h_\sigma g_\sigma$ for some 
    $h_\sigma$ in $G_v$, then 
    \[
      d_{\sigma,\tau}' = d_{\sigma,\tau} \iota_v^{-1} (h_\sigma) (\prescript{\sigma}{}{\iota_v}^{-1}(h_\tau)) (\iota_v^{-1}(h_{\sigma\tau})^{-1} .
    \]
  \item Show that the 2-cochain $(d_{\sigma,\tau})$ does not depend on the 
    choice of $v\in V_f(k^s)$. 
\end{enumerate}
\end{exercise}

Given a class $c\in \h^1(k,G)$, one can obtain a pure inner form $G^c$ of $G$ 
and a representation $V^c$ of $G^c$ as follows. Suppose $c$ is given by the 
1-cocycle $(c_\sigma)$ with values in $G(k^s)$. Then $G^c(k^s) = G(k^s)$ with 
action 
\begin{equation}\label{eq:gross-2}
  \sigma(h) = c_\sigma \prescript{\sigma}{}{h} c_\sigma^{-1} .
\end{equation}
If we compose the cocycle $c$ with values in $G(k^s)$ with the homomorphism 
$\rho:G\to \generallinear(V)$, we obtain a cocycle $\rho(c)$ with values in 
$\generallinear(V)(k^s)$. By the generalization of Hilbert's Theorem 90, we 
have $\h^1(k,\generallinear(V))=1$. Hence there is an element $g$ in 
$\generallinear(V)(k^s)$, well-defined up to left multiplication by 
$\generallinear(V)(k)$, such that 
$\rho(c_\sigma)=g^{-1}\prescript{\sigma}{}{g}$ for all $\sigma$ in 
$\galois(k^s/k)$. We use the element $g$ to define a twisted representation of 
the group $G^c$ on the vector space $V$ over $k$. The homomorphism 
\[
  \rho_g:G^c(k^s) \to \generallinear(V)(k^s) 
\]
defined by $\rho_g(h)=g \rho(h) g^{-1}$ commutes with the respective Galois 
actions, so defines a representation over $k$. We emphasize that the Galois 
action on $G^c(k^s)$ is defined as in \eqref{eq:gross-2}, whereas the Galois 
action on $\generallinear(V)(k^s)$ is the usual action. We write $V^c$ for the 
representation $\rho_g$ of $G^c$. For any $f\in (V\gq G)(k^s)$, we write 
\[
  V_f^c(k) = (k)\cap g V_f(k^s) .
\]

\begin{exercise}
\begin{enumerate}
  \item Show that the isomorphism class of $G^c$ does not depend on the choice 
    of the representative $(c_\tau)$. 
  \item Show that the isomorphism class of the representation $V^c$ of $G^c$ 
    does not depend on the choice of the element $g$. 
  \item Observe that the group scheme $G_f$ over $k$ does not depend on the 
    twist $c\in \h^1(k,G)$. Show that the class $d_f\in \h^2(k,G_f)$ does not 
    depend on the twist $c\in \h^1(k,G)$. 
\end{enumerate}
\end{exercise}

\begin{theo*}
Let $G$ be a reductive group with representation $V$. Suppose there exists 
$v\in V(k)$ with invariant $f\in (V\gq G)(k)$ and stabilizer $G_v$ such that 
$G(k^s)$ acts transitively on $V_f(k^s)$. Then there is a bijection between the 
set of $G^c(k)$-orbits on $V_f^c(k)$ and the fiber $\gamma^{-1}(c)$ of the map 
\[
  \gamma:\h^1(k,G_v) \to \h^1(k,G) 
\]
above the class $c\in \h^1(k,G)$. In particular, the image of $\h^1(k,G_v)$ in 
$\h^1(k,G)$ determines the set of pure inner forms of $G$ for which the 
$k$-rational invariant $f$ lifts to a $k$-rational orbit of $G^c$ on $V^c$. 
\end{theo*}

\begin{exercise}
Prove the theorem via the following steps (or however you want to). 
\begin{enumerate}
  \item Fix some $c\in \h^1(k,G)$. Show that $V_f^c(k)$ is nonempty if and only 
    if $c$ is in the image of $\gamma$. 
  \item Suppose $V_f^c(k)$ is nonempty and take $w\in V_f^c(k)$. Then there is a 
    bijection between $G^c(k)\backslash V_f^c(k)$ and $\ker(\gamma_c)$, where 
    $\gamma_c$ is the natural map of sets $\h^1(G_w^c)\to \h^1(k,G^c)$. Show 
    that there is a bijection between $\gamma^{-1}(c)$ and $\ker(\gamma_c)$. 
\end{enumerate}
\end{exercise}

\begin{theo*}
Suppose that $f$ is a rational invariant, and that $G(k^s)$ acts transitively 
on $V_f(k^s)$ with abelian stabilizers. Then $d_f=0$ in $\h^2(k,G_f)$ if and 
only if there exists a pure inner form $G^c$ of $G$ such that $V_f^c(k)$ is 
nonempty. That is, the condition $d_f=0$ is necessary and sufficient for the 
existence of rational orbits for some pure inner twist of $G$. In particular, 
when $\h^1(k,G)=1$, the condition $d_f=0$ in $\h^2(k,G_f)$ is necessary and 
sufficient for the existence of rational orbits of $G(k)$ on $V_f(k)$. 
\end{theo*}

\begin{exercise}
$\Leftarrow$ is trivial. Prove $\Rightarrow$ via the following steps (or 
however you want to). 
\begin{enumerate}
  \item Show that there exists a 1-cochain $(e_\sigma)$ with values in 
    $G_v(k^s)$ such that $c=(e_\sigma g_\sigma)$ is a 1-cocycle. 
  \item Show that $V_f^c(k)$ is nonempty. 
\end{enumerate}
\end{exercise}


\subsubsection{The representation $V=\symmetric_2(n)\oplus \symmetric_2(n)$ of $G=\speciallinear_n$ and $H=\speciallinear_n/\boldsymbol\mu_2$}

The ring $k[V]^G$ of polynomial invariants is freely generated by the 
coefficients of the invariant binary form $f(x,y)=(-1)^{n(n-1)/2}\det(A x-B y)$ 
for $(A,B)\in V$. Fix some binary $n$-ic form 
$f(x,y) = f_0 x^n + \cdots + f_n y^n$ with coefficients in $k$ and suppose that 
$f_0$ and $\Delta(f)$ are nonzero. Let $L$ denote the \'etale extension 
$k[x]/f(x,1)$. Then $G(k^s)$ acts transitively on $V_f(k^s)$ with abelian 
stabilizers. Moreover, we have 
\begin{align*}
  G_f &\simeq (\weilres_{L/k}\boldsymbol\mu_2)^{\norm=1} , \\
  H_f &\simeq (\weilres_{L/k}\boldsymbol\mu_2)^{\norm=1}/\boldsymbol\mu_2 .
\end{align*}
The groups $G_f$ and $H_f$ fit inside short exact sequences 
\begin{align*}
  1 \to (\weilres_{L/k}\boldsymbol\mu_2)^{\norm=1} &\to \weilres_{L/k}\boldsymbol\mu_2 \xrightarrow{\norm} \boldsymbol\mu_2 \to 1 \\
  1 \to \boldsymbol\mu_2 \to (\weilres_{L/k}\boldsymbol\mu_2)^{\norm=1} &\to (\weilres_{L/k}\boldsymbol\mu_2)^{\norm=1}/\boldsymbol\mu_2 \to 1 .
\end{align*}
Taking Galois cohomology gives long exact sequences 
\begin{align*}
  L^\times / 2 &\xrightarrow{\norm} k^\times / 2 \xrightarrow{\delta_0} \h^2(k,G_f) , \\
  \h^1(k,H_f) &\xrightarrow{\delta} \h^2(k,\boldsymbol\mu_2) \xrightarrow\alpha h^2(k,G_f) .
\end{align*}
Let $d_f^G$ (resp.~$d_f^H$) denote the corresponding classes in $\h^2(k,G_f)$ 
(resp.~$\h^2(k,H_F)$) that obstruct the existence of a rational lift for some 
pure inner form of $G$ (resp.~$H$). For the following exercise, see 
\cite[\S 4.5]{bgw13}. 

\begin{exercise}
\begin{enumerate}
  \item Show that $d_f^G=\delta_0(f_0)$. Since $\h^1(k,G)=1$, we see that 
    $V_f(k)$ is nonempty if and only if $f_0\in (k^\times)^2 \norm(L^2)$. 
  \item Show that $d_f^H$ is the image of $d_f^G$ under the natural map 
    $\h^2(k,G_f) \to \h^2(k,H_f)$. 
  \item Suppose now that $d_f^H=0$. We would like to know for which pure inner 
    forms of $H$ do there exist rational orbits with invariant $f$. 
     \begin{enumerate}
       \item Show that $d_f^G\in \h^2(k,G_f)$ is the image of some 
         $d\in \h^2(k,\boldsymbol\mu_2)$ under $\alpha$ where $d$ lies in the 
         image of the map $\delta_2:\h^1(k,H) \hookrightarrow \h^2(k,\boldsymbol\mu_2)$ 
         obtained from the short exact sequence $1\to \boldsymbol\mu_2 \to G\to H\to 1$. 
       \item Let $c$ be an element of $\h^1(k,H)$ such that $\delta_2(c)=d$. 
         Show that $v_f^c(k)$ is nonempty. 
       \item Show that the pure inner forms of $H$ for which rational orbits 
         exist with invariant $f$ correspond to classes $c\in \h^1(k,H)$ such 
         that $\alpha(\delta_2(c))=d_f^G$. 
     \end{enumerate}
\end{enumerate}
\end{exercise}










\bibliographystyle{alpha}
\bibliography{sms}

\end{document}
