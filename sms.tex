\documentclass{article}

\usepackage{amsmath,amssymb,amsthm,enumerate,extarrows,mathrsfs,mathtools,sms,stmaryrd}
\usepackage[colorlinks=true]{hyperref}
\usepackage[all]{xy}
\newtheorem{conjecture}[subsubsection]{Conjecture}
\newtheorem{coro}[subsubsection]{Corollary}
\newtheorem{fact}[subsubsection]{Fact}
\newtheorem{heuristic}[subsubsection]{Heuristic}
\newtheorem{lemm}[subsubsection]{Lemma}
\newtheorem{prediction}[subsubsection]{Prediction}
\newtheorem{principle}[subsubsection]{Principle}
\newtheorem{prop}[subsubsection]{Proposition}
\newtheorem{question}[subsubsection]{Question}
\newtheorem*{theo*}{Theorem}
\newtheorem{theo}[subsubsection]{Theorem}
\theoremstyle{definition}
\newtheorem{defi}[subsubsection]{Definition}
\newtheorem{example}[subsubsection]{Example}
\newtheorem*{exercise}{Exercise}


\SetSymbolFont{stmry}{bold}{U}{stmry}{m}{n}

\title{Counting arithmetic objects}
\date{June 23 -- July 4, 2014}
\author{Notes taken by Daniel Miller}

\begin{document}
\setcounter{tocdepth}{1}
\maketitle
\tableofcontents





% !TEX root = sms.tex

\section{Introduction and perspective}\label{sec:bhargava-i}
\thanksauthor{Manjul Bhargava}





\subsection{Motivation}

The main question we are interested in is: given a class $\cC$ of objects ``of 
arithmetic interest,'' how many objects are there in $\cC$, up to isomorphism, 
having bounded invariants? 

\begin{example}
The following are the main examples we're interested in: 
\begin{center}
\begin{tabular}{c|c}
$\cC$ & invariant \\ \hline
number fields of given degree & discriminant \\
class group elements of number fields of given degree & '' \\
rational points on curves & height \\
elliptic curves weighted by rank & '' \\
$n$-Selmer elements of Jacobians of curves & '' 
\end{tabular}
\end{center}
All of these will be defined precisely later on. 
\end{example}

Given such a class of objects of arithmetic interest, how they are distributed 
(asymptotically) with respect to their basic invariants? Beyond the cases of 
degree $2$ number fields and genus $0$ curves, little was known at the 
beginning of the 20th century. 





\subsection{Strategy}

Direct methods of counting arithmetic objects generally fail except in the 
``easy'' cases of degree $2$ number fields and genus $0$ curves. The modern 
approach uses representation theory. We try to find a map 
\[
  \cC/\simeq \hookrightarrow G(\dZ)\backslash V(\dZ) ,
\]
where $G$ is an algebraic group and $V$ is a representation of $G$, both 
defined over $\dZ$. More precisely, one finds such a map that sends the 
invariants of objects in $\cC$ to the ring of fundamental polynomial 
invariants of the action of $G$ on $V$. Good choices of such maps often come 
from algebraic geometry, but we have to work out the theory over $\dZ$. 

\begin{example}[Gauss]
In his \emph{Disquisitiones}, Gauss constructs a map 
\[
  \left\{\begin{array}{c}\text{ideal classes of (orders } \\ \text{in) quadratic fields with} \\ \text{non-square discriminant}\end{array}\right\}\bigr/\simeq 
  \iso 
  \speciallinear_2(\dZ)\bigl\backslash \left\{\begin{array}{c}\text{integer binary quadratic} \\ \text{forms }a x^2+b x y+c y^2\end{array} \right\} 
\]
The map sends the discriminant of such an ideal class to $b^2-4 a c$ (the 
discriminant of the quadratic form, which is the unique polynomial invariant 
of quadratic forms). If $I=\langle \alpha,\beta\rangle$, then 
the corresponding quadratic form is $\norm(\alpha x+\beta x)/\norm(I)$. 
\end{example}

\begin{example}[Levi, Delone-Faddeev] % 20s, 40s [Granville talk]
Recall that a \emph{cubic ring} is a (commutative, unital) ring structure 
on $\dZ^3$. They constructed a map 
\[
  \{\text{cubic rings}\}/\simeq \iso \generallinear_2(\dZ)\bigl\backslash\left\{\begin{array}{c}\text{integer binary cubic forms} \\ ax^3+b x^2 y+c x y^2+d y^3\end{array}\right\} ,
\]
which ``preserves discriminant.'' The discriminant of a binary cubic form is 
$b c^2-4 a c^3 - 4 b^3 d - 27 a^2 d^2 + 18 a b c d$. If 
$R=\langle 1,\alpha,\beta\rangle$ is a cubic ring, the associated binary cubic 
form is 
$\sqrt{\discriminant(\alpha x+\beta y)/\discriminant(R)} = [R:\dZ[\alpha x+\beta y]]$. 
See \cite{df64} for Delone-Faddeev's original approach. Bhargava's thesis 
\cite{b01} and Wood's thesis \cite{w09} develop things further. See 
\autoref{sec:granville-i} for more details. 
\end{example}

\begin{example} % [Wood II]
There are similar maps 
\[
  \{\text{quartic (resp.~quintic) fields}\}/\sim \to \{\text{representations\ldots}\} .
\]
These are also discriminant-preserving (in all these cases, the discriminant on 
the right is the unique polynomial invariant). See \autoref{sec:wood-ii} for 
more details. 
\end{example}

\begin{example}[Birch, Swinnerton-Dyer] % Shankar
There is a map 
\[
  \left\{\begin{array}{c}\sigma\in E(\dQ)/2:E\text{ of the } \\ \text{form }y^2=x^3+A x+B\end{array}\right\} \hookrightarrow\generallinear_2(\dQ)\backslash \left\{\begin{array}{c}\text{integer binary} \\ \text{quartic forms}\end{array}\right\} .
\]
The program \texttt{mwrank} created by Cremona uses this. In fact, this map 
factors through the $2$-Selmer group $\selmer_2(E)$. Write $E_{A,B}$ for the 
elliptic curve $y^2=x^3+A x+B$. This map sends $E_{A,B}$ to the fundamental 
invariants $I,J$ (of degree $2$, $3$ respectively) of binary quartics. See 
\autoref{sec:shankar-i} and \autoref{sec:shankar-ii} for more. 
\end{example}

\begin{example}[Cremona-Fisher-Stoll]
For $n\in \{3,4,5\}$, there is a map 
\[
  \{\sigma \in E(\dQ)/n:E\text{ of the form }E_{A,B}\} \hookrightarrow G(\dZ)\backslash V(\dZ) ,
\]
where $A,B$ are send to the fundamental invariants $I,J$ on the right-hand 
side. 
\end{example}

\begin{example} % [Gross I, Poonen IV]
There is a non-injective, but still useful map 
\[
  \left\{\begin{array}{c}\text{rational points on odd hyperelliptic} \\ \text{curves }y^2=x^{2 g+1} + a_1 x^{2 g} + \cdots + a_{2g+1} \end{array} \right\} \to \specialorthogonal_{2 g+1}(\dZ) \backslash \symmetric^2(\dZ^{2 g+1}) .
\]
This sends the $a_i$ to the invariants on the right-hand side. See 
\autoref{sec:gross-i} and \autoref{sec:poonen-iv} for more. 
\end{example}

\begin{example} % [Gross II, Ho] -- also see Jack Thorne, [Wood III]
There is a (non-injective) map 
\[
  \left\{\begin{array}{c}\text{rational points on even hyperelliptic} \\ \text{curves } z^2=a_0 x^{2 g+2} + \cdots + a_{2 g+2} y^{2 g+2} \end{array} \right\} \to \text{product of SL's} \backslash\cdots ,
\]
sending the $a_i$ to invariants. See \autoref{sec:gross-ii}, 
\autoref{sec:ho}, \autoref{sec:thorne}, and \autoref{sec:wood-iii} for more. 
\end{example}

Once we've found our map from arithmetic objects to orbits, the question 
becomes: how many orbits of $G(\dZ)$ on $V(\dZ)$ are there having bounded 
invariants? Gauss worked this out for binary quadratic forms. Let $h(D)$ be 
the number of $\speciallinear_2(\dZ)$-orbits of integer binary quadratic forms of 
discriminant $D$. 

\begin{theo}[Gauss, Lipschitz, Mertens]
\[
  \sum_{0<-D<X} h(D) \sim \frac{\pi}{18} X^{3/2} .
\]
\end{theo}
\begin{proof}
Gauss shows that every integer binary quadratic form $a x^2+b x y+c y^2$ 
with $D=b^2-4 a c<0$ has a unique $\speciallinear_2(\dZ)$-equivalent form 
satisfying $|b|<a\leqslant c$ or $0<b=a\leqslant c$. 

We apply geometry of numbers to the counting problem 
\[
  \sum_{0<-D<X} h(D) \sim \# \{(a,b,c):0<4 a c-b^2<X\text{ and }|b|<a\leqslant c\} .
\]
Gauss's conjecture tells us that the number of points on the right is 
asymptotically the volume of the region. Proving this is tricky! In fact, the 
conjecture is false in general. If we consider the region 
$R=\{(a,b,c):0\leqslant 4 a c-b^2<X\text{ and }|b|\leqslant a \leqslant c\}$, 
then $R\cap \dZ^3$ contains infinitely points. 
\end{proof}

One can attack the lattice-point counting problem explicitly by writing the 
count as a triple sum $\sum_{a,b,c}$, approximating the sum by a triple 
integral, and keeping track of error terms. This is how Lipschitz and Mertens 
proved the result. Working through this is a good exercise. Davenport developed 
some general principles for bounded regions. He used the principle to reprove 
Gauss' count of binary quadratics, and extended the argument to a count of 
binary cubic forms. This requires knowing explicit inequalities for the region. 

A third approach uses zeta functions (or more generally $L$-functions). 
% [Granville, Taniguchi, Thorne]
Siegel first applied this to binary quaratic forms. Goldfeld-Hoffstein, 
Shintani, and Datskovsky extended these methods. See 
\autoref{sec:granville-ii}, \autoref{sec:taniguchi}, and 
\autoref{sec:thorne} for details. 

% [Shankar]
There is a hybrid method: average over a compact continuum of fundamental 
domains. It doesn't need explicit inequalities, but still uses elementary 
geometry of numbers. The method does adapt to situations where there are more 
than one invariant. For examples, it works on all the above examples. In 
particular, it gives a count of quartic and quintic fields, boundedness of 
average rank of elliptic curves, and produces lots of hyperelliptic curves with 
few rational points. See \autoref{sec:shankar-i} and \autoref{sec:shankar-ii} 
for details. 

What if we replace $\dQ$ with another base field, like a number field or 
function field? Over a function field, one can use algebro-geometric and 
topological methods. Boundedness of average rank was proved by de Jong. 
Ellenberg has proved many other results of this type. Also, the ``hybrid 
method'' works over an arbitrary global field, as is worked out in 
\autoref{sec:wang-ii}. 





% !TEX root = sms.tex

\section{Basics of binary quadratic forms and Gauss composition}
\thanksauthor{Andrew Granville}





d


% !TEX root = sms.tex

\section{Algebraic groups, representation theory, and invariant theory}
\thanksauthor{Eyal Goren}





This lecture will consist mostly of a review of the basic terminology, as 
well as a little bit of Galois cohomology. An ``official version'' of the 
notes can be found online at 
\url{http://www.math.mcgill.ca/goren/AlgberaicGroups.SMS2014.pdf}. 





\subsection{Algebraic groups}

For us, a \emph{linear algebraic group} is a Zariski-closed subgroup of 
$\generallinear_N(\bar k)$ for some integer $N\geqslant 1$, where $k$ is a 
fixed field of characteristic zero, and $\bar k$ is the algebraic closure of 
$k$. A good reference for linear algebraic groups is the eponymous book 
\cite{b91}. 

\begin{enonce}[remark]{Example}
The main example of a linear algebraic group is $G=\generallinear_N$. It 
contains several standard subgroups:
\begin{align*}
  B &= \begin{pmatrix} \ast & \cdots & \ast \\ & \ddots & \vdots \\ & & \ast \end{pmatrix} && \text{``standard Borel''} \\
  U &= \begin{pmatrix} 1 & \cdots & \ast \\ & \ddots & \vdots \\ & & 1\end{pmatrix} && \text{``unipotent radical of $B$''} \\
  T &= \begin{pmatrix} \ast \\ & \ddots \\ & & \ast \end{pmatrix} && \text{``maximal torus''}
\end{align*}
\end{enonce}

\begin{enonce}[remark]{Example}
Let $q$ be a symmetric bilinear form corresponding to the matrix 
$(q_{i j})_{1\leqslant i,j\leqslant N}$. Let 
\[
  \specialorthogonal(q) = \{g\in \generallinear_N: g q \transpose g = q\text{ and }\det g=1\} .
\]
If $q$ is the form 
$q(x_1,\dots,x_{2n+1}) = \frac 1 2 (x_1 x_{2n+1} + x_2 x_{2n} + \cdots + x_{n+1}^2$, 
then a maximal torus consists ``anti-triangular'' matrices with 
$(t_1,\dots,t_n,1,t_n^{-1},\dots,t_1^{-1})$. 
\end{enonce}

An affine group $G$ is determined by its coordinate ring $\bar k[G]$. The 
group operations $m:G\times G\to G$, $i:G\to G$, $e:1\to G$ correspond via 
the Yoneda lemma to $m^\ast:\bar k[G]\to \bar k[G]\otimes \bar k[G]$, 
$i^\ast:\bar k[G] \to \bar k[G]$, $e^\ast:\bar k[G] \to \bar k$. These give 
$\bar k[G]$ the structure of a Hopf algebra. 

A \emph{homomorphism} $f:G\to H$ of algebraic is a morphism of varieties that 
respects the group structures. It corresponds to a ring 
$f^\ast:\bar k[H] \to \bar k[G]$ that respects the comultiplication.

A \emph{character} of $G$ is a homomorphism 
$f:G\to \generallinear_1 = \dG_\multiplicative$. This corresponds to a 
Hopf-algebra homomorphism $\phi:\bar k[t^{\pm 1}] \to \bar k[G]$. If we let $f$ 
be the image of $t$ under this map, then the fact that $\phi$ respects 
comultiplication tells us that $m^\ast(f) = f\otimes f$. We call such 
elements \emph{grouplike}. Let $\character^\ast(G)$ be the group of characters 
of $G$; we have seen that $\character^\ast(G)$ is in bijection with the set 
of grouplike elements of $\bar k[G]$. 

If $g\in G$, put $\inner g:G\to G$ for the action of $g$ by inner 
automorphisms, i.e.\ $\inner g(x) = g x g^{-1}$. So we have a homomorphism 
$\inner:G\to \automorphism G$. 





\subsection{Non-abelian cohomology}

Let $G$ be a topological group acting continuously on a discrete group $M$. 
Define 
\begin{align*}
  \h^0(G,M) &= M^G = \{m\in M:g m=m\text{ for all }g\in G\} \\
  \h^1(G,M) &= \{\zeta:G\to M\text{ such that }\zeta(a b) = \zeta(a)\cdot a \zeta(b)\} / \sim 
\end{align*}
where $\zeta\sim \xi$ if there exists $m\in M$ such that 
$\zeta(a) = m^{-1} \xi(a) \cdot a m$ for all $a\in G$. The set $\h^0(G,M)$ is 
naturally a group, but $\h^1(G,M)$ is only a pointed set. If 
$0 \to A \to B \to C \to 0$ is an exact sequence of $G$-groups, we get a long 
exact sequence 
\[
  0 \to A^G \to B^G \to C^G \xrightarrow\delta \h^1(G,A) \to \h^1(G,B) \to \h^1(G,C) \to \h^2(G,A) \to \cdots 
\]
with the last map only existing if $A$ is central in $B$. A good source for all 
of this is \cite{s79}. 

We define $\delta$ directly. Given $c\in C^G$, lift $c$ to $b\in B$, and let 
$\zeta=\delta(c)$ be $\zeta(g)=b^{-1} \cdot g b$. One can check that the class 
of $\zeta$ in $\h^1(G,A)$ is well-defined. 





\subsection{Forms}

Suppose $G$ is defined over $k$. A \emph{$k$-form} of $G$ is an algebraic group 
$H$ over $k$ together with an isomorphism $f:G_{\bar k} \iso H_{\bar k}$. Let 
$\Gamma=\galois(\bar k/k)$. For all $\sigma\in G$, we have 
$\sigma f:G_{\bar k}\iso H_{\bar k}$. Then 
$f^{-1}\circ \sigma f\in \automorphism_{\bar k}(G)$. An easy exercise in the 
definitions shows that this is a cocycle. In fact, we have 

\begin{theo}
There is a natural isomorphism of pointed sets 
\[
  \{\text{$k$-forms of $G$}\}/\sim \iso \h^1(\Gamma,\automorphism_{\bar k} G) .
\]
\end{theo}

If $H$ corresponds to $\zeta:G\to M$, then $H(k)=G(\bar k)^\Gamma$, where 
$\Gamma$ now acts by $\tau \cdot g = \zeta(\tau)(\tau(g))$. 

\begin{enonce}[remark]{Example}[compact forms]
Let $\Gamma=\galois(\dC/\dR)=\langle c\rangle$. Let 
$\theta(g) = \transpose{\bar g}^{-1}$ be the \emph{Cartan involution}. The 
cocycle $\zeta$ given by $\zeta(c)=\theta$ corresponds to the real form 
$\unitary_N$ of $\generallinear_N(\dC)$. It is defined by 
$\unitary_N(\dR) = \{g\in \generallinear_N(\dC):\theta g=g\}$. 
\end{enonce}

\begin{theo}
Any (connected) reductive algebraic group $G$ over $\dR$ has a unique compact 
form. 
\end{theo}

All of the groups $\generallinear_n$, $\speciallinear_n$, $\specialorthogonal_n$, 
$\symplectic_{2 n}$,\ldots are reductive. 

\begin{enonce}[remark]{Example}
If $G=\dG_\multiplicative$, the compact form is 
$T(\dR)=\{z\in \dC^\times:z \bar z=1\}$. We have 
\[
  T\simeq \specialorthogonal(2) = \left\{\begin{pmatrix} a & b \\ -b & a \end{pmatrix} : a^2+b^2=1\right\} ,
\]
via $\begin{pmatrix} a & b\\ -b & a \end{pmatrix} \mapsto a+b i$. 
\end{enonce}

If $M$ is a $\Gamma$-module, we often put $\h^i(k,M) = \h^i(\Gamma,M)$. 
Also, if $G$ is an algebraic group defined over $k$, we put 
$\h^i(k,G) = \h^i(\Gamma,G(\bar k))$. 

\begin{enonce}[remark]{Example}
Start with the exact sequence 
$1 \to \dG_\multiplicative \to \generallinear_N \to \projectivegenerallinear_N \to 1$. 
The long exact sequence in cohomology is 
\[
  1 \to k^\times \to \generallinear_N(k) \to \projectivegenerallinear_N(\bar k)^\Gamma \to \h^1(k,\dG_\multiplicative) \to \h^1(k,\generallinear_N) \to \h^2(k,\dG_\multiplicative) .
\]
The famous \emph{Hilbert Theorem 90} tells us that 
$\h^1(k,\generallinear_N) = 1$ for all $N\geqslant 1$, so we get 
\[
  \projectivegenerallinear_N(\bar k)^\Gamma = \projectivegenerallinear_N(k) = \generallinear_N(k)/\dG_\multiplicative(k) .
\]
Moreover, $\h^1(k,\projectivegenerallinear_N)\hookrightarrow \h^2(k,\dG_\multiplicative)$. We call 
$\brauer(k)=\h^2(k,\dG_\multiplicative)$ the \emph{Brauer group} of $k$. 
Since 
$\projectivegenerallinear_N(\bar k) = \automorphism_{\bar k\text{-}\mathsf{Alg}}(M_N(\bar k))$, 
we see that $\h^1(k,\projectivegenerallinear_N)$ classifies $k$-forms of 
$M_N(\bar k)$, i.e.~central simple algebras over $K$ of rank $N^2$. 
\end{enonce}





\subsection{Jordan decomposition}

Any $g\in \generallinear_N(\bar k)$ has a unique decomposition 
$g=g_\simple g_\unipotent$, where $g_\simple$ is simple (i.e.~diagonalizable), 
$g_\unipotent$ is unipotent, and 
$g_\simple g_\unipotent = g_\unipotent g_\simple$. One has 
$(g_\unipotent - 1)^N=0$. For example, in the two-dimensional case, a matrix 
$\begin{pmatrix} t_1 & u \\ & t_2\end{pmatrix}$ is already diagonalizable if 
$t_1\ne t_2$, or $u=0$. If neither of those occur, we write it as 
\[
  \begin{pmatrix} t & u \\ & t \end{pmatrix} = \begin{pmatrix} t \\ & t\end{pmatrix} \begin{pmatrix} 1 & u/t \\ & 1 \end{pmatrix} .
\]
The Jordan decomposition enjoys very strong rigidity properties. Namely, 
if $g\in G\subset \generallinear_N$, then also $g_\simple\in G$ and 
$g_\unipotent \in G$. If $f:G\to H$ is a homomorpism of algebraic groups, then 
we have $f(g_\simple) = f(g)_\simple$ and $f(g_\unipotent) = f(g)_\unipotent$. 





\subsection{Tori}

A \emph{torus} $T$ is a form of $\dG_\multiplicative^N$ for some $N$. Tori of 
rank $N$ over $k$ are classified by 
$\h^1(\Gamma,\automorphism_{\bar k}(\dG_\multiplicative^N)) = \hom(\Gamma,\generallinear_N(\dZ))/\text{conj}$. 

If $T$ is a orus, then its group of characters 
$\character^\ast(T_{\bar k}) \simeq \character^\ast(\dG_\multiplicative^N) = \hom(\dG_\multiplicative^N,\dG_\multiplicative) = \dZ^N$ 
has a continuous action of $\Gamma$, via 
$\sigma \chi(g) = \sigma(\chi(\sigma^{-1}(g)))$. Tori are completely classified 
by this action. 

\begin{theo}
The functor $\character^\ast$ induces an anti-equivalence of categories 
\[
  \{\text{tori over }k\} \iso \{\text{finite free $\dZ$-modules with continuous $\Gamma$-action}\} .
\]
\end{theo}

A linear action of a torus $T$, namely $f:T\to \generallinear_N$, is 
simultaneously diagonalizable (every element of $T$ is semi-simple). This 
follows from rigidity properties of the Jordan Decomposition. 

All maximal tori in a linear algebraic group $G$ are conjugate. The common 
dimension of these tori is called the \emph{rank} of $G$, and written 
$\rank_{\bar k}(G)$. 

\begin{enonce}[remark]{Example}
The standard torus of diagonal matrices $T\subset\generallinear_N$ is 
maximal, so $\rank(\generallinear_N) = N$. 
\end{enonce}





\subsection{Solvable groups}

An algebraic group $G$ is called \emph{solvable} if it is solvable ``in the 
usual sense.'' In other words, there exists a filtration 
$1=G_0\subset G_1\subset \cdots \subset G_l = G$ such that each 
$G_i$ a normal algebraic subgroup of $G_{i+1}$, and each 
$G_{i+1}/G_i$ is abelian. The standard Borel $B\subset \generallinear_N$ is 
solvable. In fact, $B$ is a maximal solvable subgroup. 

\begin{theo}[Kolchin-Lie]
If $G\subset \generallinear_N$ is solvable, then $G$ can be conjugated 
into the standard Borel $B\subset \generallinear_N$. 
\end{theo}
\begin{proof}
One uses the fact that if $G$ acts on a projective space, then it has a fixed 
point. The standard representation $G\to \generallinear_N$ gives an action of 
$G$ on $\dP^{N-1}$, and a fixed point for $G$ in $\dP^{N-1}$ gives a line fixed 
by the action of $G$. 
\end{proof}

\begin{theo}[Borel]
If a solvable group $G$ acts on a proper variety, then $G$ has a fixed point. 
\end{theo}

For a group $G$, let $\rad(G)$ be the maximal connected normal solvable 
subgroup of $G$, and let $\rad_\unipotent(G)$ be the maximal connected normal 
unipotent subgroup of $G$. We say that $G$ is \emph{semisimple} if 
$\rad(G)=1$, and \emph{reductive} if $\rad_\unipotent(G)=1$. Clearly 
semisimple groups are reductive. The groups 
$\speciallinear_n$, $\specialorthogonal_n$, $\symplectic_n$ are semisimple 
and $\generallinear_n$, $\generalsymplectic_{2 n}$, $\generalspin_n$ are 
reductive. 

For any $G$, the quotient $G/\rad_\unipotent(G)$ is reductive, and 
$G/\rad(G)$ is semisimple. If $G$ is reductive, then $G/Z(G)$ is 
semisimple. A \emph{Levi subgroup} of a group $G$ is a subgroup $H$ such 
that $G=H\ltimes \rad_\unipotent (G)$. Such an $H$ will be a maximal 
reductive subgroup of $G$. 

A maximal connected solvable subgroup of $G$ is called a \emph{Borel subgroup}. 
The group $B$ of upper-triangular matrices is a borel subgroup of 
$\generallinear(n)$. Every torus is contained in a Borel subgroup, and if $G$ 
is a reductive group, then all Borel subgroups of $G$ are conjugate. 

A group $G$ over $\dC$ is reductive if and only if every representation 
$\rho:G\to \generallinear_N$ is semi-simple (a direct sum of irreducible 
representations). Alternatively, the ring $\dC[\rho(G)]$ should be semi-simple. 

\begin{enonce}[remark]{Example}
The group $\simeq U_1 = \begin{pmatrix} 1 & \ast \\ & 1 \end{pmatrix}$ 
is not reductive. 
\end{enonce}





\subsection{Parabolic subgroups}

A subgroup $P$ of a connected algebraic group $G$ is called \emph{parabolic} if 
the quotient $G/P$ is projective. The basic theorem is that $P$ is parabolic if 
and only if $P$ contains a Borel subgroup $B$. So in $\generallinear_N$, a 
subgroup is parabolic if it contains a conjugate of the subgroup of 
upper-triangular matrices. 

\begin{enonce}[remark]{Example}
Let $k$ be an algebraically closed field. Recall that a \emph{flag} in 
$k^n$ is a collection of subspaces 
$F=(0\subsetneq F_1\subsetneq \cdots \subsetneq F_a=k^n)$. The \emph{type} of 
$F$ is $\boldsymbol d=(\dim F_i)_i$. The space of type $\boldsymbol d$ is a 
projective variety $\flag_{\boldsymbol d}$ on which $\generallinear_n$ acts 
transitively. Let $P$ be a stabilizer of a flag. Then 
$G/P\simeq \flag_{\boldsymbol d}$ and $P$ is parabolic. For example, if 
$F_i$ is the span of $\{e_1,\dots,e_{d_i}\}$, then 
\[
  P = \begin{pmatrix} 1_{h_1} \end{pmatrix} [finish]
\]
\end{enonce}

[finish]





% !TEX root = sms.tex

\section{Basic algebraic number theory}
\thanksauthor{Eknath Ghate}





d


% !TEX root = sms.tex

\section{Geometric properties of curves}
\thanksauthor{Henri Darmon}





d


% !TEX root = sms.tex

\section{Basic analytic number theory}
\thanksauthor{Andrew Granville}





d


% !TEX root = sms.tex

\section{Diophantine properties of curves}\label{sec:7}
\thanksauthor{Henri Darmon}





Let $X$ be a curve over a number field $k$. The main diophantine questions we 
are interested in are: 
\begin{itemize}
  \item What is $X(k)$?
  \item Is $X(k)$ finite?
  \item What is $\# X(k)$ for ``typical'' $X$?
\end{itemize}
We would like to phrase questions in a way that allow for us to talk about 
integral points on a curve -- e.g.~equations like the Pell equation 
$x^2-d y^2=1$. If $X$ is projective, then $X(\dZ)=X(\dQ)$, so there is no 
limitation in studying rational points. More generally, if $X/k$ is 
projective, then $X(\cO_k)=X(k)$. If $X$ is affine, we can choose an 
embedding $X\hookrightarrow \dA^n$ and put $X(\dZ)=X(\dQ)\cap \dA^n(\dZ)$. 
With this definition $X(\dZ)$ depepnds on the chosen equations for $X$, but 
hopefully the ``main features'' of $X(\dZ)$ do not depend on this embedding. 

So our question is: if $k$ is a number field, $S$ is a finite set of places of 
$S$ and $X/\cO_{k,S}$ is a curve, what is $X(\cO_{k,S})$? 

To $X$ we can attach some numerical invariants. The curve $X$ will be of the 
form $\widetilde X\smallsetminus \{x_1,\dots,x_s\}$ where $\widetilde X$ is 
proper. For $g$ the genus of $\widetilde X$, we define the 
\emph{Euler characteristic} of $X$ by 
\[
  \chi(X) = 2 - 2 g - s\in \dZ .
\]
A lot of the diophantine behavior of $X$ is governed by $\chi(X)$. The 
fundamental trichotomy comes from whether $\chi>0$, $\chi<0$, or 
$\chi=0$. 





\subsection{Positive Euler characteristic}

\begin{theo}
If $\chi(X)>0$, then $X(\cO_{k,S})$ is either empty or infinite. 
\end{theo}
\begin{proof}
If $X$ is affine, then $g=0$ and $s=1$, so $X=\dA^1$, whence 
$X(\cO_{k,S}) = \cO_{k,S}$. If $X$ is projective, then $g=s=0$. Then $X$ 
either has a rational point, in which case it is $\dP^1$, or $X$ is a conic 
with $X(k)=\varnothing$. 
\end{proof}

\begin{theo}[Hasse-Minkowski]
Let $X$ be a curve over $\dQ$ of genus zero. Then $X(\dQ)\ne\varnothing$ if and 
only if $X(\dQ_p)\ne\varnothing$ for all $p$ and $X(\dR)\ne \varnothing$. 
\end{theo}





\subsection{Negative Euler characteristic}

\begin{theo}[Siegel,Faltings]
If $\chi(X)<0$, then $\# X(\cO_{k,S})<\infty$. 
\end{theo}

The affine case was proven by Siegel in 1932. The prototypical examples are: 
\begin{center}
\begin{tabular}{cc|c}
$g$ & $s$ & $X$ \\ \hline
0 & 3 & $\dP^1\smallsetminus \{0,1,\infty\}$ \\
1 & 1 & $E\smallsetminus \{\infty\}$ 
\end{tabular}
\end{center}
The coordinate ring of $\cO_{\dP^1\smallsetminus \{0,1,\infty\}}$ is 
$\dZ[x,\frac 1 x,\frac{1}{1-x}]$, and 
\[
  (\dP^1\smallsetminus \{0,1,\infty\})(\cO_{k,S}) = \{v\in \cO_{k,S}^\times:v-1\in \cO_{k,S}^\times\} .
\]
This is an \emph{$S$-unit equation}, and Siegel proved that such equations have 
only finitely many solutions. 

If $g=1$, $s=1$, then the result amounts to showing that elliptic curves have 
only finitely many integral points. Since integral points are torsion, this 
follows from the Mordell-Weil Theorem. 

In the projective case, $g>1$, and the finiteness result is Faltings' Theorem, 
originally known as the Mordell Conjecture. 





\subsection{Zero Euler characteristic}

This is the most interesting case. 

\begin{theo}[Dirichlet,Mordell-Weil]
If $X(\cO_{k,S})$ is non-empty, then it is naturally an abelian group, and as 
such is finitely-generated. 
\end{theo}

In the affine case $g=0,s=2$, if $X(k)\ne\varnothing$, then (for the sake of 
illustration) $X = \dP^1\smallsetminus \{0,\infty\} = \dG_\multiplicative$, so 
$X(\cO_{k,S}) = \cO_{k,S}^\times$. The famous \emph{Dirichlet Unit Theorem} 
tells us this group is finitely generated. 

In the projective case $g=1,s=0$, $X$ is an elliiptic curve which we will 
denote by $E$. The Mordell-Weil Theorem says that $E(k)$ is finitely 
generated. 





\subsection{Ranks}

In the affine case, the rank of $\cO_{k,S}^\times$ is easily determined. 
Dirichlet's theorem says that 
\[
  \rank_\dZ(\cO_{k,S}^\times) = r+s-1+\# S ,
\]
where $r$ is the number of real places and $s$ is the number of complex 
places of $k$. 

In the projective case, the rank is much more subtle. If 
$X=\dP^1\smallsetminus \{p,p'\}$ for $p,p'$ conjugates in a quadratic 
extension $\dQ(\sqrt D)$, at least if $k=\dQ$. We are led to the equation 
$x^2-D y^2=1$. This has rank $0$ if $D<0$, and rank $1$ if $D>0$. 

For elliptic curves over $\dQ$, little is known. 

\begin{enonce}{Conjecture}
For $E$ ranging over elliptic curves defined over $\dQ$, 
is $\rank E=\rank_\dZ E(\dQ)$ bounded?
\end{enonce}

\begin{enonce}{Conjecture}
As $E$ ranges over elliptic curves defined over $\dQ$, 
$\rank E$ is $0$ and $1$ with probability $\frac 1 2$ each. 
\end{enonce}

Bhargava and Shankar have proved that there is a positive density set of 
elliptic curves having rank $0$ and $1$. 





\subsection{Proof of Mordell-Weil}

The proof has two main ingredients. The first is a height function 
$h:E(\dQ) \to \dR$ satisfying the property that for each $X$, the set 
$\{x\in E(\dQ):h(x)<X\}$ is finite. Moreover, 
$h(n\cdot x) = n^2 h(x)$ and $h(x+y)+h(x-y)=2 h(x) + 2 h(y)$. The second 
ingredient is the \emph{weak Mordell-Weil theorem}: 

\begin{theo}
For some $n>1$, the group $E(\dQ)/n$ is finite. 
\end{theo}

Proving Mordell-Weil from these two ingredients is a very old idea, going back 
to Fermat at least. Let $\{p_1,\dots,p_r\}$ be a set of representatives for 
$E(\dQ)/n$. Choose $X\gg h(p_j)$, and let 
$S=\{p_1,\dots,p_r\}\cup \{p:h(p)<X\}$. We claim that $S$ generates $E(\dQ)$. 
Let $p$ be a point not in $\langle S\rangle$ with minimal height with respect 
to this property. There exists some $j$ such that $p-p_j=n\cdot q$. One sees 
that $h(q)<h(p)$, so $q\in \langle S\rangle$. This implies 
$p\in \langle S\rangle$, a contradiction. 





\subsection{Proof of weak Mordell-Weil}

We do this for $n=2$. Assume $E[2]$ is defined over $\dQ$, i.e.~$E$ is of the 
form $y^2=(x-a)(x-b)(x-c)$. Given $P\in E(\dQ)$, choose some 
$\widetilde P\in E(\overline\dQ)$ such that $2\widetilde P=P$. Define a 
function $\delta(P):G_\dQ \to E[2]$ by 
$\delta(P)(\sigma) = \sigma(\widetilde P)-\widetilde P$. 

The function $\delta(P)$ is actually a continuous homomorphism 
$G_\dQ \to E[2]$. Moreover, $\delta(P_1)=\delta(P_2)$ if and only if 
$P_1-P_2\in 2 E(\dQ)$. So $\delta$ is an injection 
$E(\dQ)/2 \hookrightarrow \hom(G_\dQ,E[2])$. This doesn't solve our problem 
because $\hom(G_\dQ,E[2])$ is infinite. The necessary property of $\delta$ is 
the following. Let $L=\dQ(\sqrt\ell:\ell\mid 2(a-b)(b-c)(a-c))$. Then 
$\delta(P)$ factors through $\galois(L/\dQ)$. Indeed, if 
$P=(x,y)$, then $\widetilde P$ is defined over 
$\dQ(\sqrt{x-a},\sqrt{x-b},\sqrt{x-c})$. It is easy to check that if 
$P\in E[2]$, then $y=0$, and this implies $\widetilde P$ is defined over 
$L$. 

To conclude, $\delta$ is an injection 
$E(\dQ)/2\hookrightarrow \hom(\galois(L/\dQ),E[2])$, the latter being a finite 
set. Thus $E(\dQ)/2$ is finite. 

Let's give a more ``highbrow'' proof using Galois cohomology. Let $n>1$ be an 
integer. We have an exact sequence 
\[
  0 \to E[n] \to E(\overline\dQ) \xrightarrow n E(\overline\dQ) \to 0 .
\]
Take $G_\dQ$-invariants and we get an exact sequence 
\[
  0 \to E(\dQ)/n \xrightarrow\delta \h^1(G_\dQ,E[n]) \to \h^1(G_\dQ,E)[n] \to 0 .
\]
The middle set is still infinite. Repeat the process for each place  of $\dQ$:
\[\xymatrix{
  0 \ar[r] 
    & E(\dQ)/n \ar[r]^-\delta \ar[d] 
    & \h^1(G_\dQ,E[n]) \ar[r] \ar[d] 
    & \h^1(G_\dQ,E)[n] \ar[r] \ar[d] 
    & 0 \\
  0 \ar[r] 
    & E(\dQ_\ell)/n \ar[r] 
    & \h^1(G_{\dQ_\ell},E[n]) \ar[r] 
    & \h^1(G_{\dQ_\ell},E)[n] \ar[r] 
    & 0 
}\]
Define the \emph{$n$-Selmer group} and \emph{Tate-Shafarevich group} of $E$ by 
\begin{align*}
  \selmer_n(E) 
    &= \ker\left(\h^1(G_\dQ,E[n]) \to \bigoplus_v \h^1(G_{\dQ_v},E)\right) \\
  \sha(E) 
    &= \ker\left(\h^1(G_\dQ,E) \to \bigoplus_v \h^1(G_{\dQ_v},E)\right) .
\end{align*}
There is a canonical exact sequence 
\[
  0 \to E(\dQ)/n \to \selmer_n(E) \to \sha(E)[n] \to 0 .
\]
An elementary argument using the Hermite-Minkowski theorem shows that 
$\selmer_n(E)$ is finite. Since $E(\dQ)/n\hookrightarrow \selmer_n(E)$, we're 
done. 





\subsection{Geometric interpretation of \texorpdfstring{$\selmer_n(E)$}{SelnE}}

In general, we know that $\h^1(G_\dQ,\automorphism X)$ classifies 
$\dQ$-forms of $X$. We would like to find an object whose automorphism group is 
$E[n]$. Consider the isogeny $E\xrightarrow n E$. The automorphisms of this 
cover of $E$ are exactly elements of $E[n]$. 

\begin{defi}
An \emph{$n$-cover} of $E$ is a curve $C$ of genus $1$, equipped with a 
$\dQ$-rational map $\widetilde n:C\to E$ and a $\overline\dQ$-isomorphism 
$\varphi:C\iso E$ such that the following diagram commutes: 
\[\xymatrix{
  C \ar[r]^-\varphi \ar[d]^-{\widetilde n} 
    & E \ar[d]^-n \\
  E \ar@{=}[r] 
    & E
}\]
\end{defi}

Similarly, $\h^1(\dQ,E)$ can be identified with the set of isomorphism classes 
of curves of genus one such that $\jacobian C\simeq E$ over $\dQ$. The map 
$\h^1(\dQ,E[n]) \to \h^1(\dQ,E)$ comes from ``forgetting $\widetilde n$.'' 
It follows that $\selmer_n(E)$ can be identified with the set of isomorphism 
classes of $n$-covers $\widetilde n:C\to E$ such that 
$C(\dQ_\ell)\ne \varnothing$ for all $\ell$ and $C(\dR)\ne\varnothing$. 
Similarly $\sha(E)$ consists of isomorphism classes of genus-one curves $C$ 
such that $\jacobian C\simeq E$, and such that $C(\dQ_v)\ne\varnothing$ for all 
places $v$. 

\begin{theo}[Swinnerton-Dyer]
If $\widetilde 2:C\to E$ is an element of $\selmer_2(E)$, then $C$ has a 
$\dQ$-rational positive divisor of degree $2$. 
\end{theo}
\begin{proof}
Degree $2$ divisors of $C$ correspond to rational points on 
$\symmetric^2(C)$. Define a rational morphism 
$\varphi:\symmetric^2(C) \to E$ by $(P,Q)\mapsto P+Q = \widetilde 2(P)-(Q-P)$. 
(Recallhere that $E=\jacobian C$.) Then $X=\varphi^{-1}(0)$ is a curve. Over 
$\overline\dQ$, the map $\phi$ can be identified with the addition map 
$\symmetric^2(E) \to E$. So $X_{\overline\dQ}=\{(P,-P):P\in E(\dQ)\}$. In fact, 
$X_{\overline\dQ}=(E/-1)_{\overline\dQ}=\dP^1_{\overline\dQ}$. The same 
reasoning, replacing $\overline\dQ$ by $\dQ_\ell$ and using the fact that 
$C\simeq E$ over each $\dQ_v$, tels us that $X$ has a rational point in each 
$\dQ_v$. By Hasse-Minkowski, $X\simeq \dP^1$, whence the result. 
\end{proof}

\begin{coro}
If $\widetilde 2:C\to E$ is an element of $\selmer_2(E)$, then $C$ has an 
equation of the form $y^2=f(x)$, where $\deg f=4$. 
\end{coro}

This gives us the dictionary between elements of a $2$-Selmer group and 
binary quartic forms over $\dQ$. Bhargava and Shankar show that there is a 
positive proportion of $E$ with $\selmer_n E=0$, and also a positive proportion 
of $E$ with $\selmer_n E=dZ/n$. Both of these only hold for $n\in \{2,3,4,5\}$. 

\begin{theo}
1. If $\selmer_n E=0$, then $\rank E=0$. 

2. If $\selmer_n E=\dZ/n$, then $\rank E=1$. 
\end{theo}

Part 1 is trivial. Part 2 is incredibly deep. It uses in a crucial way the 
connection between elliptic curves and $L$-functions. This is a special case of 
the Birch and Swinnerton-Dyer conjecture. A big ingredient is the theory of 
complex multiplication discussed in \autoref{sec:3}.





% !TEX root = sms.tex

\section{More algebraic groups, representation theory and invariant theory}
\thanksauthor{Eyal Goren}





d


% !TEX root = sms.tex

\section{Cubic rings}\label{sec:wood-i}
\thanksauthor{Melanie Matchett-Wood}





Recall that the goal of this conference is ``counting arithmetic 
objects.'' The first sort of objects we might try to count are number fields. 
To do this, we would like a parameterization of all number fields of a given 
degree. For degree $2$, this is easy. All quadratic fields are of the form 
$\dQ(\sqrt D)$, where $D$ is a square-free integer. 
In general, a degree $n$ number field is of the form $\dQ(\theta)$. Let $f$ 
be the minimal polynomial of $\theta$; it will be of the form 
$f(x)=x^n+a_{n-1} x^{n-1} + \cdots + a_0\in \dZ[x]$. The problem is: determining 
whether two monic irreducible polynomials yield the same field is quite 
difficult. 

Rather than counting number fields, we will try to count their maximal orders 
instead. 





\subsection{Some definitions}

\begin{defi}
A \emph{rank-$n$ ring} is a (commutative, unital) ring $R$ such that 
$R\simeq \dZ^n$ as a $\dZ$-module. 
\end{defi}

For $n=2,3,4,5$, we call these rings quadratic, cubic, quartic, and quintic. 
Typical cubic rings are maximal orders $R=\cO_K$ in cubic number fields $K$. 
But non-maximal orders in cubic number fields are also cubic rings. For 
example, $\dZ[3\sqrt[3] 2]\subset \dQ(2^{1/3})$ is a perfectly good cubic 
ring, with $\dZ$-basis $\{1,3\sqrt[3] 2,9\sqrt[3] 4\}$. A still more 
pathological example is $\dZ[X]/X^3$, which has $\dZ$-basis $\{1,X,X^2\}$. If 
$R$ is any quadratic ring, then $\dZ\times R$ is another ``pathological'' cubic 
ring. 

In the end, we'll count number fields by counting their maximal orders, which 
we will count by including them into a larger class of rank-$n$ rings. 

\begin{defi}
Let $R$ be a rank-$n$ ring. The \emph{trace map} $\trace:R\to\dZ$ is 
defined by $\trace(r) = \trace(\cdot r:R\to R)$. 
\end{defi}

If $\alpha_1,\dots,\alpha_n$ is a $\dZ$-basis of $R$, then the 
\emph{discriminant} of $R$ is 
$\discriminant(R) = \det(\trace(\alpha_i \alpha_j)_{i,j})$. 





\subsection{Quadratic rings}

We know that quadratic rings (up to isomorphism) are in bijection with 
$\{D\in \dZ:d\equiv 0,1\pmod 4\}$. Given a quadratic ring $R$, the 
corresponding integer is $D=\discriminant(R)$. (It is an old theorem that 
$\discriminant(R)$ is a square modulo $4$.) Given such an integer $D$, 
the corresponding ring is 
\[
  \dZ[\tau]/\left(\tau^2 - D\tau + \frac{D^2-D}{4}\right)
\]
If, for example $D=0$, the corresponding ring is $\dZ[\tau]/\tau^2$. 

From our parameterization of quadratic rings, it is easy to count them! 





\subsection{Cubic rings}

Let $R$ be a cubic ring with $\dZ$-basis $\{1,W,T\}$. A key invariant is 
$W T = q+r W + s T$ for some $q,r,s\in \dZ$. Choose a new 
$\dZ$-basis $\{1,\omega,\theta\}$ with $\omega=W-s$ and $\theta=T-r$. In our 
new basis, $\omega\theta = n$ for some $n\in \dZ$. We call such a basis 
\emph{normalized}. Every basis of $R/\dZ$ has a unique normalized lift to 
$R$. 

We also write $\omega^2=m-b \omega + a\theta$ and 
$\theta^2 = \ell-d\omega+c\theta$. The fact that multiplication on $R$ is 
associative forces some conditions on $\{m,b,a,\ell,d,c\}$. The 
associativity relations are exactly: 
\begin{align*}
  n &= - a d \\
  \ell &= - b c \\
  m &= - a c .
\end{align*}
This induces a bijection between isomorphism classes of cubic rings with 
choice basis of $R/\dZ$ and quadruples $(a,b,c,d)\in \dZ^4$. Clearly 
$\generallinear_2(\dZ)$ acts on cubic rings with basis of $R/\dZ$; orbits of 
this action are isomorphism classes of cubic rings. The only thing here that 
is mysterious is the $\generallinear_2(\dZ)$-action on $\dZ^4$. 

If we write $f(x,y) = a x^3 + b x^2 y + c xy^2 + d y^3$ and 
$g\in \generallinear_2(\dZ)$, then 
\[
  (g f)(x,y) = \frac{1}{\det(g)} f\left(\begin{pmatrix} x & y \end{pmatrix} \cdot g\right) .
\]
In other words, if we identify $\dZ^4$ with the space of cubic forms in 
two variables, then the action of $\generallinear_2(\dZ)$ is the natural one 
(with a twist by $\det^{-1}$). 
This parameterization of cubic rings goes back to \cite{df64}. It was also 
used in \cite{dh69}, and saw its first modern formulation in 
\cite{ggs02}. 

\begin{enonce}[remark]{Example}
The cubic ring corresponding to $(a,b,c,d)=(0,0,0,0)$ is pretty pathological -- 
namely $\dZ[\omega,\theta]/(\omega,\theta)^2$. 
\end{enonce}

[\ldots wasn't able to take notes to the end\ldots]





% !TEX root = sms.tex

\section{Quartic and quintic rings}\label{sec:wood-ii}
\thanksauthor{Melanie Matchett-Wood}





The plan is to first review a bit of the theory of cubic rings. We'll spend 
most of the time on quartic rings, then give a brief treatment of quintic 
rings. 





\subsection{Cubic rings revisited}

We are interested in passing from cubic rings to forms geometrically. 
For simplicity, we assume $R=\cO_k$ is the maximal order in a cubic number 
field. Consider the affine scheme $\spectrum(\cO_k)$; we want to embed this 
into $\dP_\dZ^1$. In general, a map $X\to \dP_\dZ^1$ is determined by a line 
bundle $\sL$ on $X$ together with two global sections that generate $\sL$. 
For $\spectrum(\cO_k)$, this consists of an ideal $\fa\subset \cO_k$ together 
with two elements generating $\fa$. We choose the inverse different 
$\fD^{-1}\subset \cO_k$, and two elements of $\fD^{-1}$ having trace zero as 
our global sections. 
It turns out that the image of $\spectrum(\cO_k) \to \dP_\dZ^1$ is the zero-set 
of a binary cubic form. 

If $\cO_k$ is the maximal order in a number field $k$ with $[k:\dQ]=n$, the 
same construction embeds $\spectrum(\cO_k)$ into $\dP_\dZ^{n-2}$. 
It turns out that for any maximal order, a basis of 
$\ker(\trace:\fD^{-1} \to \dZ)$ will always generate $\fD^{-1}$ as a 
$\cO$-module. 





\subsection{Quartic rings}

Start as we did with cubic rings. Given a quartic ring $Q$, write down a 
$\dZ$-basis $\{1,\alpha_1,\alpha_2,\alpha_3\}$ for $Q$. Just as with cubic 
rings, we can encode the multiplication of $Q$ by  
\[
  \alpha_i \alpha_j = c_{i j}^0 + \sum_{k=1}^3 c_{i j}^k \alpha_k .
\]
We could ``shift'' some of the basis elements in order to make some of the 
$c_{i j}^k=0$. Associativity gives us some conditions on the 
$c_{i j}^k$. We're left with a ``moduli scheme'' for quartic rings with 
basis, but the polynomial relations between the $c_{i j}^k$ are complicated 
enough that a direct, explicit approach, does not get you very far. 

In the paper \cite{wy92}, Write and Yukie showed that quartic fields are 
parameterized by pairs of ternary and quadratic forms, moduli the natural 
action of $\generallinear_2(\dQ)\times \generallinear_3(\dQ)$. Unfortunately 
this approach isn't very useful either. More recently, in the paper 
\cite{b04}, Bhargava realized that the problem needed to be understood over 
$\dZ$, and that  cubic resolvent are essential. 

We start with cubic resolvent fields $K/\dQ$ with $[K:\dQ]=4$. Let 
$\widetilde K$ be the Galois closure of $K$; we assume 
$\galois(\widetilde K/\dQ)\simeq S_4$. The group $S_4$ has a canonical 
subgroup $D_4$ of index $3$, consisting of permutations which are 
symmetries of the square 
\[\xymatrix{
  1 \ar@{-}[r] \ar@{-}[d] 
    & 2 \ar@{-}[d] \\
  4 \ar@{-}[r] 
    & 3
}\]
Galois theory gives us a subfield $K_3\subset \widetilde K$ such that 
$[K_3:\dQ]=3$; this is the \emph{cubic resolvent} of $K$. For 
$k\in K$, let $k^{(1)},k^{(2)},k^{(3)},k^{(4)}$ be the Galois conjugates of 
$K$. Then $k^{(1)} k^{(3)} + k^{(2)} k^{(4)}$ is an element of $K_3$. Put 
$\phi_{4,3}(k) = k^{(1)} k^{(3)} + k^{(2)} k^{(4)}$; this is a 
discriminant-preserving map $K\to K_3$. 

Next we define the cubic resolvent of a ring. Let $Q$ be a quartic ring, which 
for simplicity we assume is an order in a $S_4$-quartic field $K$. 

\begin{defi}
A \emph{cubic resolvent ring} of $Q$ is a cubic ring $R$ in the resolvent 
field $K_3$ such that 
\begin{enumerate}
  \item $\discriminant(R)=\discriminant(Q)$
  \item for all $q\in Q$, $\phi_{4,3}(q)\in R$. 
\end{enumerate}
\end{defi}
We have a quadratic map $\phi_{4,3}:Q\to R$. It descends to a map 
$\phi_{4,3}:Q/\dZ \to R/\dZ$; here the quotients are taken as 
$\dZ$-modules, not as rings. The $\dZ$ in the quotient is the 
sub-$\dZ$-module generated by the multiplicative unit in the ring. 

\begin{exercise}
Show that $\phi_{4,3}$ descends to a map $Q/\dZ\to R/\dZ$. 
\end{exercise}

An element of $Q/\dZ$ can be written as $\ell\alpha_1+m \alpha_2 + n \alpha_3$ 
for $\ell,m,n\in \dZ$. It's image under $\phi_{4,3}$ will be a linear combination 
of $\omega$ and $\theta$, i.e. 
\[
  \phi_{4,3}(\ell \alpha_1 + m\alpha_2 + n\alpha_3) = A(\ell,m,n) \omega + B(\ell,m,n)\theta .
\]
The functions $A$ and $B$ are ternary quadratic forms with coefficients in 
$\dZ$. 

It is not clear whether every quartic ring even has a cubic resolvent. Even if 
it does, how many? Luckily, every quartic ring does have a cubic resolvent, but 
this resolvent is not necessarily unique. But maximal quartic rings (which 
include maximal quartic orders) have a unique cubic resolvent. 

We want to parameterize isomorphism classes of pairs $(Q,R)$, where $Q$ is a 
quartic ring and $R$ is a cubic resolvent of $Q$. The main result of 
\cite{b04} is that these are in bijection with 
$\generallinear_2(\dZ)\times \generallinear_3(\dZ)$-classes of pairs of ternary 
quadratic forms. This bijection preserves discriminants, and allows you to 
detect prime splitting, automorphism groups, \ldots as the parameterization of 
cubic rings. 

To be completely explicit, a \emph{ternary quadratic form} is of the form 
\[
  A(\ell,m,n) = a_{1 1} \ell^2 + a_{1 2} \ell m + a_{1 3} \ell n + \cdots + a_{3 3} n^2 .
\]
The $\generallinear_2$ and $\generallinear_3$ action can be written down 
explicitly. We identify the form $A$ with a $3\times 3$ matrix 
\[
  \begin{pmatrix} a_{11} & a_{12}/2 & a_{13}/2 \\ a_{12}/2 & a_{22} & a_{23}/2 \\ a_{13}/2 & a_{23}/2 & a_{33} \end{pmatrix} 
\]
and similarly for $B$. The quantity $4\det(A x+B y)$ is a binary cubic 
form with coefficients in $\dZ$. The corresponding cubic ring is the cubic 
resolvent ring in the pair $(Q,R)$. What we haven't done is describe how to 
construct the quartic ring $Q$ from $A$ and $B$. 





\subsection{Geometric perspective}

Write $A=a_{11} x^2+a_{1 2} x y + a_{1 3} x z + \cdots$ and similarly for $B$. 
These cut out a subscheme of $\dP^2$. Over $\dQ$, we would expect this subscheme 
to have four points (over $\overline\dQ$). The Galois conjugates of any such 
point $p$ are among the four intersection points, so $p$ is defined over a 
(at most) quartic extension of $\dQ$. 

Let's do this over $\spectrum \dZ$. A good reference is \cite{w11}. The two 
forms $A,B$ over $\dZ$ cut out a subscheme $V_{A,B}$ of $\dP_\dZ^2$. The 
functions on $V_{A,B}$ recover the quartic ring corresponding to $A,B$. 

If $\cO$ is the maximal order in a quartic field $K$, then using the inverse 
different we can embed $\spectrum(\cO)\hookrightarrow \dP_\dZ^2$. From 
\cite{ce96}, such subschemes of $\dP_\dZ^2$ are cut out by pairs of ternary 
quadratic forms. 





\subsection{Quintic rings}

In \cite{b08}, it is proved that there is a bijection between isomorphism 
classes of pairs $(R,S)$, where $R$ is a quintic ring and $S$ is a 
sextic resolvent of $R$, and $\generallinear_4(\dZ)\times \speciallinear_5(\dZ)$-orbits 
of quadruples of $5\times 5$ skew-symmetric matrices (alternatively, quinary 
alternating forms). 

We'll briefly describe the sextic resolvent of a quintic ring. At the level of 
Galois groups of fields, we're looking for an index $6$ subgroup of 
$S_5$. Make a pentagon out of $\{1,2,3,4,5\}$. There are $6$ ways to put them 
into two disjoint $5$-cycles. The permutations that fix this decomposition into 
$2$-cycles gives us such a subgroup. 

The resolvent associated to $(R,S)$ is a map $\Lambda^2 S^\vee \to R^\vee$. 
From a geometric perspective: over $\dQ$, in $\dP_{\overline\dQ}^3$, we have 
$A=A_1 x+A_2 y+A_3 z+A_4 w$, where $A_1$ is an alternating $5\times t$ matrix. 
The matrix $A$ has five skew-symmetric $4\times 4$ minors. The determinant of 
such a minor is a a square, so the Pfaffian $\sqrt{\det(\text{minor})}$ is a 
quadratic form in $4$ variables. So from $A$ we get forms $Q_1,\dots,Q_5$. 
These cut out a subvariety of $\dP_{\overline\dQ}^3$, which is the analogue 
of the ring $R$. 

We should expect the quadratic forms $Q_1,\dots,Q_5$ to have no common zeros! 
But when a $5$-tuple of quadratic forms come from Pfaffians of an alternating 
matrix, the subscheme they cut out is non-empty. 

We would have $\spectrum(\cO)\subset \dP_\dZ^3$. There is a geometric analogue 
of this in \cite{be77} which shows that such a subvariety of $\dP^3$ has to be 
cut out by $5$ quadratic forms which come from Pfaffians of an alternating 
matrix. 




% !TEX root = sms.tex

\section{How to count rings and fields I}
\thanksauthor{Manjul Bhargava}





d


% !TEX root = sms.tex

\section{Rings associated to binary $n$-ic forms, composition of $2\times n\times n$ boxes and class groups}
\thanksauthor{Melanie Matchett Wood}





d


% !TEX root = sms.tex

\section{The zeta functions attached to prehomogeneous vector spaces}
\thanksauthor{Takashi Taniguchi}





d


% !TEX root = sms.tex

\section{How to count rings and fields II}\label{sec:bhargava-iii}
\thanksauthor{Manjul Bhargava}





In \autoref{sec:bhargava-ii}, we proved that 
$N^+(X) = \frac 1 6 \volume(\cF v\cap \{|\discriminant|<X\}) + O(X^{5/6})$. 
Recall that $N^+(X)$ is the number of irreducible positive binary cubic forms 
with $0<\discriminant<X$ up to $\generallinear_2(\dZ)$-equivalence. 






\subsection{The general theorem}

Recall that we defined $R_X(v) = \cF v\cap \{|\discriminant|<X\}$. 

\begin{prop}
Let $f\in C^0(V^+(\dR))$, and $v_0\in V^+(\dR)$. Then 
\[
  \int_\cF f(g v_0)\, \mathrm{d}g = \frac{1}{2\pi} \int_{\cF v_0} f(x)|\discriminant(x)|^{-1} \, \mathrm{d}x . 
\]
\end{prop}
\begin{proof}
Because of uniqueness of invariant measures, this comes down to a simple 
Jacobian calculation. 
\end{proof}

Now let $f=|\discriminant|$ when $|\discriminant|<X$, and $0$ elsewhere. Then 
\begin{align*}
  \frac 1 6 \volume(R_X(v)) 
    &= \frac{2\pi}{6} \int_1^{X^{1/4}} \lambda^4\, \mathrm{d}^\times \lambda \int_{\generallinear_2(\dZ)\backslash \generallinear_2^\pm(\dR)} \mathrm{d} g \\
    &= \frac{2\pi}{6} \frac{X}{4} \cdot \frac{\zeta(2)}{\pi} \\
    &= \frac{\pi^2}{72} X.
\end{align*}
This explains the coefficient of $X$ in our formula for $N^+(X)$. If we wanted 
to compute $N^-(X)$, then everything would be the same except that the 
$\frac 1 6$ would be replaced by $\frac 1 3$, so the coefficient of $X$ would 
be $\frac{\pi^2}{24}$. 

More generally, we've proved the following theorem: 

\begin{theo}
Let $S\subset V(\dZ)$ be a $G(\dZ)$-invariant subset defined by congruence 
conditions modulo finitely many prime powers. If we let $N^+(S;X)$ be the 
number of positive-discriminant irreducible integer binary cubic forms in $S$ 
with $|\discriminant|<X$, up to $\generallinear_2(\dZ)$-equivalence, then 
\[
  N^+(S;X) = \frac{\pi^2}{72} \prod_p\mu_p(S) \cdot X + O(X^{5/6}) .
\]
where $\mu_p(S)$ is the $p$-adic density of $S$ in $V(\dZ)$. 
\end{theo}

The implied constant in $O(X^{5/6})$ will depend on the set $S$. See 
\cite{bst13} for details. What if $S$ is defined by infinitely many congruence 
conditions? For example,if we want to count cubic fields via their maximal 
orders, maximality is determined by congruence conditions modulo $p^2$ for 
every $p$. So we need to know how the constant in $O(X^{5/6})$ changes as we 
vary $S$. 

Let $W_p$ be the set of integer binary cubic forms $f$ such that $R_f$ is 
\emph{not} maximal at $p$, i.e.~$R_f\otimes\dZ_p$ is not a maximal order in 
$R_f\otimes \dQ_p$. 

\begin{prop}
With notation as above, 
\[
  N(W_p,X) = O(X/p^2) ,
\]
with the implied constant is independent of $p$. 
\end{prop}
\begin{proof}
If $f\in W_p$ is a multiple of $p$, then $f/p$ has discriminant 
$\discriminant(f)/p^4$. The number of such $f$ is $O(X/p^4)$. If $f\in W_p$ is 
not a multiple of $p$, we use the following lemma to replace $f$ with an 
equivalent form satisfying $p\mid c$ and $p^2\mid d$. Let 
$f'=(a p,b,c/p,d/p)$; this is $\generallinear_2(\dQ)$-equivalent to $f$ via the 
matrix $\begin{pmatrix} 1 \\ & p^{-1} \end{pmatrix}$. Moreover, 
$\discriminant(f') = \discriminant(f)/p^2$. This gives at most $O(X/p^2)$ 
such $f$, because $[f]\mapsto [f']$ is at most 3-to-1 when 
$f'\not\equiv 0\pmod p$. 
\end{proof}

\begin{lemm}
Let $f\in W_p$ not be a multiple of $p$. Then there exists a binary cubic 
form $a x^3 + \cdots + d y^3$ that is $\generallinear_2(\dZ)$-equivalent to $f$ 
such that $p\mid c$ and $p^2\mid d$. 
\end{lemm}
\begin{proof}
Let $\cO$ be an order strictly containing $R_f$. Then there exist bases 
$1,\omega,\theta$ and $1,\omega',\theta'$ of $R_f$ and $\cO$ respectively, such 
that $\omega = p^i\omega'$ and $\theta = p^j \theta'$ and $i,j\geqslant 0$ are 
distinct. If we write down the multiplication table for $R_f$, we get 
$p\mid b$ and $p^2\mid a$, or $p\mid c$ and $p^2\mid d$ depending on which of 
$i$ and $j$ are bigger. 
\end{proof}

\begin{theo}[Davenport-Heilbronn]
The number of cubic fields of positive discriminant $<X$ is 
$\frac{1}{12\zeta(3)} X + o(X)$. The number of cubic fields with negative 
discriminant $<X$ is $\frac{1}{4\zeta(3)} X + o(X)$. 
\end{theo}
\begin{proof}
Let $U\subset V(\dZ)$ (resp.~$U_p\subset V(\dZ_p)$) be the set of binary cubic 
forms $f$ such that $R_f$ is maximal (resp.~maximal at $p$). Then 
$U=\bigcap_p U_p$. We know that 
\[
  N^+\left(\bigcap_{p<Y} U_p;X\right) = \frac{\pi^2}{72} \cdot \prod_{p<Y} \mu_p(U_p) \cdot X + o(X) .
\]
An elementary lemma gives $\mu_p(U_p) = \frac{(p^2-1)(p^3-1)}{p^5}$, so we 
have 
\begin{align*}
  \limsup_{X\to \infty} \frac{N^+(U;X)}{X} 
    &\leqslant \lim_{X\to \infty} \frac 1 X N^+\left(\bigcap_{p<Y} U_p; X\right)\\
    &= \frac{\pi^2}{72} \prod_{p<Y} \frac{(p^2-1)(p^3-1)}{p^5} \\
    &= \frac{1}{12\zeta(3)} .
\end{align*}
The latter bit coming from $Y\to \infty$. Now we want to show the $\liminf$ is 
sufficiently large. Since 
$U \subset \bigcap_{p<Y} U_p\subset U\cup \bigcup_{p\geqslant Y} W_p$, 
\[
  \liminf_{X\to \infty} \frac{N^+(U;X)}{X} \geqslant \lim_{X\to \infty} \frac 1 X N^+\left(\bigcap_{p<Y} U_p;X\right) - \sum_{p\geqslant Y} O(1/p^2) 
\]
Let $Y\to \infty$ and the sum tends to zero. Since the limit approaches 
$1/12\zeta(3)$, we get the lower bound. 
\end{proof}

What if we wanted to sieve to square-free discriminants, rather than just 
maximal orders? If a cubic ring has squarefree discriminant, it is maximal. But 
the converse does not hold. If any prime $p$ totally ramifies in a cubic 
extension $K/\dQ$, then $p^2\mid\discriminant(K)$. This is the only way 
$\discriminant(K)$ can be divisible by a square. So counting square-free 
discriminant forms is equivalent to counting cubic fields in which no prime 
totally ramifies. On the side of forms, if $R_f$ is maximal, the only way 
$\discriminant(f)$ is not squarefree is if $f$ is a constant times a cube 
of some linear form modulo $p$ for some $p$. If $R_f$ is not maximal, then 
$\discriminant(R_f)$ is automatically divisible by a square. This is a 
condition modulo $p^2$, whereas $\discriminant(R_f)$ being non-squarefree is 
a modulo $p$ condition. 





\subsection{The geometric sieve}

This is also known as the \emph{closed point sieve}, or the \emph{Ekedahl 
sieve}. 

\begin{theo}
Let $B$ be a bounded region in $\dR^n$ with finite volume. Let $Y$ be a 
subscheme of $\dA_\dZ^n$ of codimension $k$. Let $r,M>0$. Then 
\[
  \#\{a\in r B\cap V(\dZ):a\in Y(\dF_p)\text{ for some }p>M\} = O\left(\frac{r^n}{M^{k-1} \log M} + r^{n-k+1}\right) .
\]
\end{theo}
\begin{proof}[Sketch of proof]
We treat the case $k=2$. Suppose $Y$ is defined by integer polynomials 
$f(x_1,\dots,x_n)$ and $g(x_1,\dots,x_n)$. We can assume that $f$ does not 
involve $x_n$ (for example, by using elimination theory). Then the number of 
lattice points $(a_1,\dots,a_n)\in r B$ where $f=0$ or $g=0$ is 
$O(r^{n-1})$. Suppose $f$ and $g$ are nonzero on $a$. Let's count all bad pairs 
$(a,p)$. For $p\leqslant r$, this easy: the number of such $(a,p)$ is 
$O(r^n/p2)$. Fix $a_1,\dots,a_{n-1}$. If $p>r$, then $f(a_1,\dots,a_n)$ has at 
most $O(1)$ of prime factors $p>r$. For each such prime $p$, there are 
$O(1)$ values of $a_n$ such that $g(a_1,\dots,a_n)\equiv 0\pmod p$. So 
there are $O(r^{n-1})$ such $a=(a_1,\dots,a_n)$. 
\end{proof}





% !TEX root = sms.tex

\section{Heuristics for number field counts and applications to curves over finite fields}\label{sec:wood-iv}
\thanksauthor{Melanie Matchett Wood}





We'll discuss three things: local ($p$-adic) densities, applications to curves 
over finite fields, and heuristics for counting number fields. 
The motivating question is: how many number fields are there? 





\subsection{Local densities}

What proportion of degree $n$ number fields (ordered by $|\discriminant|$) have 
$7$ split completely? Of course, a similar question can be asked for any prime 
$p$. Note that we are fixing the prime and letting number fields vary, as 
opposed to the other way around. If we fix a number field and let primes 
vary, the splitting is controlled by the \v Cebotarev Density Theorem. 

To be more precise, we are interested in 
\[
  \lim_{X\to \infty} \frac{\#\{[K:\dQ]=n,|\discriminant K|<X:7\text{ splits completely}\}/\sim}{\#\{[K:\dQ]=n:|\discriminant K|<X\}/\sim} .
\]
It is not clear that this limit exists, and it is not known if $n>5$. It 
matters that we order by discriminant -- if we ordered by some other invariant, 
the limit would change. 

Let $K/\dQ$ be a degree $n$ number field. Write $K_7=K\otimes \dQ_7$; if 
$K=\dQ[\theta]/f$ this is $\dQ_7[\theta]/f$. If 
$(7)=\fp_1^{e_1}\cdots \fp_r^{e_r}$ in $\cO_K$ and each $\fp_i$ has inertia 
degree $f_i$, then $K_7=K_{\fp_1}\times \cdots \times K_{\fp_r}$, where each 
$K_{\fp_i}/\dQ_7$ is a field extension of degree $e_i f_i$. In the ring of 
integers of $K_{\fp_i}$, $(7)=\fp_i^{e_i}$. So $K_7$ carries all the splitting 
data $(e_i,f_i)$ as well as more information about how $7$ ramifies. 

\begin{enonce}[remark]{Example}
The field $\dQ_2$ has a unique unramified extension of degree $2$ (in which 
$e=1$, $f=2$). It has six different ramified extensions of degree $2$ (in 
which $e=2$, $f=1$). So there are six different ways a quadratic extension 
$K/\dQ$ can ramify at $2$. 
\end{enonce}

\begin{defi}
An \emph{\'etale $\dQ_p$-algebra} is a finite direct product of finite field 
extensions of $\dQ_p$. The \emph{degree} of an \'etale $\dQ_p$-algebra is its 
dimension as a $\dQ_p$-vector space. If $L$ is an \'etale $\dQ_p$-algebra, 
define $\cO_L$ as usual. The \emph{discriminant} of $L$ is 
$\discriminant_{\dZ_p}(\cO_L) = \langle \det(\trace(\alpha_i \alpha_j))\rangle$.
Put $|\discriminant_{\dZ_p}(\cO_L)| = \#(\dZ_p/\discriminant(\cO_L))$.  
\end{defi}

Let's look at all degree $n$ \'etale $\dQ_p$-algebras. There are only finitely 
many of these (for fixed $n$ and $p$). In fact, if $p>n$, they are 
well-understood (and have an easy classification). We can ask: how often does 
each \'etale $\dQ_p$-algebra occur as $K_p=K\otimes\dQ_p$ for a random degree 
$n$ number field $K$? The most naive guess would be a uniform distribution. 

\begin{enonce}[remark]{Example}
Consider $p=5,n=2$. There are four \'etale $\dQ_5$-algebras of degree $2$. 
These are $\dQ_5\times \dQ_5$, the unique unramified extension of degree $2$, 
and two ramified extensions. 
\end{enonce}

\begin{enonce}[remark]{Example}
Consider $p=5,n=3$. Here there are six \'etale algebras of this type: 
$\dQ_5\times \dQ_5\times \dQ_5$, $\dQ_5\times($any quadratic extension$)$, and one 
ramified and one unramified extension of degree $3$.
\end{enonce}

Our naive ``equidistribution'' guess is wrong. First, ramified algebras are 
rare, because lots of ramification corresponds to a large discriminant. Also, 
objects occur ``in nature'' inversely proportionally to cardinality of their 
automorphism groups. For example, if our objects are isomorphism classes of 
cubic fields and ``nature'' is $\overline\dQ$, then a Galois cubic field
occurs once in $\overline\dQ$, whereas non-Galois cubic fields occur three 
times, in keeping with the respective orders of automorphism groups. This 
principle is the basis for the Cohen-Lenstra Heuristics. 

We will construct a measure on the set of \'etale $\dQ_p$-algebras of degree 
$n$. Namely, 
\[
  \mu_p(\{L\}) = \frac{1}{\# \automorphism(L)|\discriminant L|} .
\]
This is \emph{not} a probability measure. Let $\widetilde \mu_p$ be the 
normalized version of $\mu_p$ so that $\widetilde\mu_p$ is a probability 
measure on the set of \'etale $\dQ_p$-algebras of degree $n$. 

\begin{enonce}{Heuristic}
A $\dQ_p$-algebra $L$ of degree $n$ occurs as $K_p$ for a random degree $n$ 
number field $K$ with probability $\widetilde \mu_p$. 
\end{enonce}

References for this are \cite{b07,m02,m04}. It is known to hold when 
$n=2,3,5$. For $n=2,3$, this follows from results of Davenport-Heilbront. 
For $n=5$, this was done by Bhargava via his count of quintic extensions. For 
$n=4$, the heuristic fails. About $16\%$ of quartic fields have Galois closure 
with group $D_4$, and about $83\%$ have Galois closure with group $S_4$. We can 
recover the heuristic for $n=4$ by restricting to $S_4$-quartic extensions. 





\subsection{Points on curves over finite fields}

There is a well-known analogy between number fields (e.g.~$\dQ\supset\dZ$) and 
function fields over finite fields (e.g.~$\dF_q(X)\supset \dF_q[X]$). For 
example, we should think of a quadratic field $\dQ(\sqrt D)$ as being analogous 
to an extension $\dF_q(X)(\sqrt{X^3+1})/\dF_q(X)$. We can talk about things 
like splitting of primes\ldots in both cases. 

The heuristic above corresponds to a conjecture for the proportion of curves 
over $\dF_q$ with a degree $n$ map to $\dP^1$ that have $k$ points for each 
$k$. 

\begin{theo}
For $n=3$, the conjecture is true. That is, when ordered by genus, the average 
number of points on a trigonal curve is $q+2 - \frac{1}{q^2+q+1}$. 
\end{theo}

See \cite{w12-trig} for a proof of this. 





\subsection{Zeta functions}

Let's return to the question: how many degree $n$ number fields are there? As 
is so common number theory, we define a zeta function: 
\[
  Z(s) = \prod_p \left(\sum_{\substack{K\text{ \'etale $\dQ_p$-algebra} \\ [K:\dQ_p]=n}} \frac{1}{\# \automorphism(K) |\discriminant K|^s}\right) = \sum_{n\geqslant 1} a_n n^{-s} .
\]
The $a_n$ don't literally count anything. But heuristically, 
the number of degree $n$ number fields with $|\discriminant K|<X$ should 
asymptotically be $\sum_{1\leqslant n\leqslant X} a_n$. There are conjectures 
due to Malle and Bhargava on hour many degree $n$ number fields there are with 
fixed Galois group. These conjectures are (at last naively) false. However, when 
interpreted more loosely (up to $O(X^\epsilon)$) they are known, e.g.~for nilpotent 
groups, \ldots. 





% !TEX root = sms.tex

\section{Moduli space of rings}\label{sec:poonen}
\thanksauthor{Bjorn Poonen}





All rings in this lecture are commutative with unit. Fix an integer 
$n\geqslant 0$. The main question is: is there a scheme $\cA_n$ such that 
$\cA_n(\dZ)$ is in bijection with the set of isomorphism classes of 
rings $A$ such that $A\simeq \dZ^n$ as $\dZ$-modules. Of course there is such 
a scheme! The set of rank-$n$ rings is countable, and there are lots of 
schemes with $\aleph_0$ points defined over $\dZ$. 

A good source for what follows is the paper \cite{p08}. 





\subsection{Fine moduli space}

A better question would be to ask for the existence of a scheme $\cA_n$ such 
that for all rings $k$, the set $\cA_n(k)$ is naturally in bijection with the 
set of isomorphism classes of $k$-algebras $A$ such that $A\simeq k^n$ as a 
$k$-module. What do we mean by ``natural''? For every ring homomorphism 
$k\to L$, we require the following diagram to commute:
\[\xymatrix{
  \cA_n(k) \ar[r] \ar[d] 
    & \{\text{rank-$n$ $k$-algebras}\}/\sim \ar[d]^-{-\otimes_k L} \\
  \cA_n(L) \ar[r] 
    & \{\text{rank-$n$ $L$-algebras}\} /\sim
}\]
If such a $\cA_n$ exists, it follows from Yoneda's lemma that $\cA_n$ is 
unique up to unique isomorphism. For, the functor 
\[
  k\mapsto \{\text{isomorphism cases of rank-$n$ $k$-algebras}\} 
\]
would be representable, and the standard argument shows that the representing 
object is unique. 

Unfortunately, for $n\geqslant 2$ no such scheme exists. We'll show why in the 
case $n=2$. The map $\cA_2(\dR) \to \cA_2(\dC)$ must be injective. That is, the 
map ``extension of scalars'' from rank-$2$ $\dR$-algebras to rank-$2$ 
$\dC$-algebras should be injective on isomorphism classes. The rings 
$\dR[x]/(x^2-1)\simeq \dR\times \dR$ and $\dR[x]/(x^2+1)\simeq \dC$ are 
certainly not isomorphic over $\dR$, but they are both isomorphic to 
$\dC\times \dC$ when tensored with $\dC$. 

Essentially, what is going on here is that twists of an $\dR$-algebra 
$A$ are in bijection with $\h^1(\dR,\automorphism A_\dC)$. Since $A_\dC$ can 
have nontrivial automorphisms (for example when $A=\dR[x]/(x^2+1)$, there is 
no way that $\cA_2(\dR)\to \cA_2(\dC)$ can be injective. This fits into the 
slogan that ``objects with nontrivial automorphisms have no coarse moduli 
scheme.'' The canonical example is that 
elliptic curves have isomorphisms, preventing the existence of a fine moduli 
scheme for all elliptic curves. Just as with elliptic curves, the way to remedy 
the situation is to add structure. 





\subsection{Moduli space of based algebras}

Unlike elliptic curves, it is very easy to prove that the moduli space of 
``based algebras'' exists. 

\begin{theo}
There exists a scheme $\cB_n$ representing the functor 
\[
  k\mapsto \{(A,\boldsymbol e)\}/\sim ,
\]
where $A$ ranges over $k$-algebras (abstractly) isomorphic to $k^n$ and 
$\boldsymbol=(e_1,\dots,e_n)$ is a $k$-basis for $A$. 
\end{theo}
\begin{proof}
A $k$-algebra $A$ with basis $\boldsymbol e=(e_1,\dots,e_n)$ is just a 
$k$-module $\bigoplus_i k e_i$ with multiplication table 
\[
  e_i e_j = \sum_l c_{i j}^l e_l .
\]
together with $1=\sum d_i e_i$. The $n^3+n$ elements 
$\{c_{i j}^l,d_i\}$ determine $A$, but commutativity, associativity, and unit 
force certain conditions on the $c$'s and $d$'s. These impose polynomial 
conditions. So 
\[
  \cB_n = \spectrum\left(\dZ[c_{i j}^l,d_m:1\leqslant i,j,l,m\leqslant n] / \text{relations}\right) .
\]
That is, $\cB_n$ is the subscheme of $\dA^{n^3+n}$ cut out by the polynomial 
relations encoding associativity, commutativity, and existence of unit. 
\end{proof}

The rest of this lecture will be concerned with understanding the geometry of 
$\cB_n$. 





\subsection{\texorpdfstring{$\generallinear(n)$}{GL(n)}-basis}

Reinterpret our scheme $\cB$ as representing the functor that assigns to a ring 
$k$ the set of isomorphism classes of pairs $(A,\phi)$, where $A$ is a 
$k$-algebra and $\phi:A\iso k^n$ is an isomorphism of $k$-modules. The group 
$\generallinear_n(k)$ on $\cB_n(k)$ by 
\[
  g\cdot (A,\phi) = (A,g\circ \phi) .
\]
This action is nicely functorial, so it gives an action of the group scheme 
$\generallinear(n)$ on the scheme $\cB_n$. 

\begin{prop}
For $(A,\phi)$ and $(A',\phi')$ in $\cB_n(k)$, we have 
\[
  \operatorname{Isom}_{k\text{-}\mathsf{Alg}}(A,A') = \{g\in \generallinear_n(k)\text{ mapping }(A,\phi)\text{ to }(A',\phi')\} .
\]
\end{prop}
\begin{proof}
An isomorphism $\alpha:A\iso A'$ corresponds to $g\in \generallinear_n(k)$ if  
$g\circ\phi = \phi'\circ \alpha$. 
\end{proof}

\begin{coro}
For each $k$, there is a natural isomorphism 
\[
  \{\text{algebras of rank $n$ over $k$}\}/\sim = \generallinear_n(k)\backslash \cB_n(k) . 
\]
\end{coro}

\begin{coro}
For any $(A,\phi)\in \cB_n(k)$, there is a natural isomorphism 
$\automorphism_{k\textnormal{-}\mathsf{Alg}}(A)\simeq \stabilizer_{\generallinear_n(k)}(A,\phi)$. 
\end{coro}

So the problem of classifying rank-$n$ algebras comes down to understanding the 
scheme $\cB_n$ together with its $\generallinear(n)$-action. 





\subsection{Example: \texorpdfstring{$\cB_3$}{B3} over \texorpdfstring{$\dC$}{C}}

Let's classify rank $3$ algebras $A$ over $\dC$. If $A$ is a finite-dimensional 
$\dC$-algebra, it will be artinian. As such, it will be a finite product of 
local artinian rings. Any finite-dimensional local artinian $\dC$-algebra $A$ 
has (by definition) a unique maximal ideal $\fm$, which is nilpotent. If 
$x_1,\dots,x_d\in A$ map to a basis of $\fm/\fm^2$, then we will have a 
surjection $\dC[x_1,\dots,x_d]/(x_1,\dots,x_d)^r\twoheadrightarrow A$ for some 
$r\gg 0$. 
\begin{center}
\begin{tabular}{l|l|l|l|l}
dim.~of factors & algebra  & $\automorphism$ & $\dim(\automorphism)$ & $\dim(\generallinear_3\text{-orbit})$ \\ \hline
1,1,1 & $\dC\times \dC\times \dC$ & $S_3$ & 0 & 9\\
2,1 & $\dC[x]/x^2\times \dC$ & $x\mapsto a x$ & 1 & 8\\
3 & $\dC[x]/x^3$ & $x\mapsto a x+b x^2$ & 2 & 7\\
& $\dC[x,y]/(x,y)^2$ & $\generallinear(2)$ & 4 & 5\\
\end{tabular}
\end{center}
So $\cB_3$ is $9$-dimensional, and contains an open dense orbit isomorphic to 
$\generallinear(3)/S_3$. Each orbit is in the closure of a higher-dimensional 
orbit. We can see this by watching families of algebras degenerate. Consider 
$\dC[x]/(x(x-t)(x-1))$. When $t\notin \{0,1\}$, this algebra is \'etale over 
$\dC$, namely $\dC\times\dC\times\dC$. When $t=0$, this algebra lies in the 
eight-dimensional orbit. Similar arguments yield the rest. 





\subsection{\texorpdfstring{$\cB_n(\dC)$}{BnC} for all \texorpdfstring{$n$}{n}}

Consider the following table: 
\begin{center}
\begin{tabular}{c|c}
$n$ & $\dim \cB_n$ \\ \hline
1 & 1 \\
2 & 4 \\ 
3 & 9 \\
\vdots \\
11 & $\geqslant 129$ \\
\vdots \\
$n$ & $\sim \frac{2}{27} n^3$
\end{tabular}
\end{center}
What's going on here? For small $n$, there is a large open orbit isomorphic to 
$\generallinear(n)/S_n$. Once $n$ gets sufficiently large, there are 
high-dimensional components outside the \'etale locus. 

The scheme $\cB_n$ is smooth if and only if $n\leqslant 3$. The set 
$\generallinear_n(\dC)\backslash \cB_n(\dC)$ is finite if and only if 
$n\leqslant 6$, as was proved in \cite{c54,d66,s56}. % [Charles 1954, Dyment 1966, Suprunenko 1956]
The scheme $\cB_n$ is irreducible if and only if $n\leqslant 7$; this is 
shown in \cite{cdvv09}. %[Cartwright, Ernan, Velasso, Viray, 2009]
The fact that $\dim(\cB_n)\sim \frac{2}{27} n^3$ is due to \cite{n87,p08}. 

These schemes can be used to construct lots of finite rings. Start with 
\[
  \dZ_2[x_1,\dots,x_d]/(2,x_2,\dots,x_d)^3
\]
and then quotient out by a vector space. It is conjectured that asymptotically, 
all finite rings are of this form, and have ``characteristic $8$.'' 





% !TEX root = sms.tex

\section{Zeta function methods}\label{sec:thorne}
\thanksauthor{Frank Thorne}





\subsection{Motivation}

The question is: what are zeta functions good for? Let $N_3^\pm(X)$ be the 
number of cubic fields $K$ with $0<\pm\discriminant(K)<X$. Define 
\begin{align*}
  C^\pm &= \begin{cases} 1 & + \\ 3 & - \end{cases} \\
  K^\pm &= \begin{cases} 1 & + \\ \sqrt 3 & -\end{cases}
\end{align*}

\begin{theo}
The following holds: 
\[
  N_3^\pm(X)=C^\pm \frac{1}{12\zeta(3)}X + K^\pm \frac{4\zeta(1/3)}{5\Gamma(2/3)^3\zeta(5/3)} X^{5/6} + O(X^{2/3+\epsilon}) .
\]
\end{theo}

We would like to understand how zeta functions can be used to provide such 
good error terms. 





\subsection{Definitions}

As we have done before, put 
\begin{align*}
  V(\dZ) &= \{a u^3 + b u^2 v + c u v^2 + d v^3:a,b,c,d\in \dZ\} \\
  \widehat V(\dZ) &= \{\cdots : 3\mid b,c\} .
\end{align*}
The group $\generallinear_2(\dZ)$ acts on both of these via 
\[
  (\gamma\cdot f)(u,v) = \frac{1}{\det\gamma}f\left(\begin{pmatrix} u & v\end{pmatrix}\cdot \gamma\right) .
\]
\begin{theo}
There is a natural bijection 
\[
  \generallinear_2(\dZ)\backslash V(\dZ) \iso \{\text{cubic rings}\}/\sim .
\]
\end{theo}

\begin{defi}[Shintani]
Put 
\begin{align*}
  \xi^\pm(s) 
    &= \sum_{x\in \generallinear_2(\dZ)\backslash V^\pm(\dZ)} \frac{1}{\#\stabilizer(x)}|\discriminant(x)|^{-s} \\
    &= \sum_{\substack{R\text{ cubic ring} \\ 
  \pm \discriminant(R)>0}} \frac{1}{\#\automorphism(R)} |\discriminant(R)|^{-s} . 
\end{align*}
\end{defi}

\begin{theo}[Shintani]
The functions $\xi^\pm(s)$ have analytic continuation to $\dC$ except for 
poles at $s=1,\frac 5 6$, explicit residue formulas at these poles, and a 
functional equation 
\[
  \begin{pmatrix} \xi^+(1-s) \\ \xi^-(1-s) \end{pmatrix} = \Gamma\left(s-\frac 1 6\right)\Gamma(s)^2 \Gamma\left(s+\frac 1 6\right) \frac{3^{6 s-2}}{2\pi^{4 s}} \begin{pmatrix} \sin(2\pi s) & \sin(\pi s) \\ 3\sin(\pi s) & \sin(2\pi s)\end{pmatrix} \begin{pmatrix} \widehat \xi^+(s) \\ \widehat\xi^-(s) \end{pmatrix} 
\]
where $\widehat\xi^\pm$ are defined in terms of $\widehat V(\dZ)$ instead of 
$V(\dZ)$. 
\end{theo}
This was proved in \cite{s72}. 





\subsection{How analytic number theorists count}

We'll see explicitly how analytic properties of $\xi^\pm$ translate into 
asymptotic estimates for the number of cubic rings. 

\begin{enonce}{Principle}[Perron's formula]
Given any Dirichlet series $B(s)=\sum_{n\geqslant 1} b(n) n^{-s}$ which is 
absolutely convergent for $\Re s=2$, then 
\[
  \sum_{n\leqslant X} b(n) = \frac{1}{2\pi i}\int_{2-i\infty}^{2+i\infty} B(s) X^s \, \frac{\mathrm{d}s}{s} .
\]
\end{enonce}

For the $b(n)$ completely arbitrary, this is not very helpful. The idea is: for 
specific $b(n)$, shift the contour of this integral. For example, if we are 
trying to count integers less than $X$, apply Davenport's lemma to conclude 
that there are $X+O(1)$ integers between $0$ and $X$. We define 
\[
  \zeta(s) = \sum_{n\geqslant 1} n^{-s} .
\]
By Perron's formula, we get 
\[
  \sum_{1\leqslant n<X} 1 = \frac{1}{2\pi i} \int_{2-i\infty}^{2+i\infty} \zeta(s) X^s\frac{\mathrm{d}s}{s} .
\]
See if you can spot the mistake in the following computation: 
\begin{align*}\tag{$\ast$}\label{eq:diverge}
  \sum_{n\leqslant X} 1 
    &= \residue_{s=1} + \residue_{s=0}\left(\zeta(s)X^s\frac{\mathrm{d}s}{s}\right) + \frac{1}{2\pi i}\int_{-1-i\infty}^{1+i\infty} \zeta(s) X^s\frac{\mathrm{d}s}{s} \\
    &= X+\zeta(0) + (\text{error}) .
\end{align*}
Does the integral in \eqref{eq:diverge} converse? For this, we need some bounds 
on $\zeta$. 

\begin{prop}
If $\sigma<0$, we have $\zeta(-\sigma+i t) \ll (1+|t|)^{1/2+\sigma}$. 
\end{prop}

\begin{enonce}{Exercise}
Use the functional equation and Stirling's approximation to prove the 
Proposition. 
\end{enonce}

Inside the critical strip, controlling the behavior of $\zeta$ is a large 
problem. But outside $\{0<\Re s<1\}$, things are relatively straightforward. 

We can use the above Proposition to show that the integral appearing in 
\eqref{eq:diverge} diverges. 





\subsection{The Landau method}

A big principle in analytic number theory is that it is important to work with 
``smooth sums.'' 

\begin{prop}
Let $b:\dN\to \dC$ and $B(s)=\sum b(n) n^{-s}$. Then 
\begin{align*}
  \sum_{n<X} b(n)\left(1-\frac n X\right) &= \frac{1}{2\pi i} \int_{2-i\infty}^{2+i\infty} B(s) \frac{X^s}{s(s+1)}\, \mathrm{d}s \\
  \frac 1 2\sum_{n<X}\left(1-\frac n X\right)^2 &= \frac{1}{2\pi i}\int_{2-i\infty}^{2+i\infty} B(s) \frac{X^s}{s(s+1)(s+2)}\, \mathrm{d} s.
\end{align*}
\end{prop}

\begin{enonce}{Exercise}
Prove and generalize the Proposition. 
\end{enonce}

These are all special cases of Mellin inversion. Using the Proposition, we 
continue our counting of integers: 
\[
  \sum_{n<X} (X-n) = \frac{X^2}{2} - X + \frac{1}{2\pi i} \int_{-1-i\infty}^{1+i\infty} \zeta(s) \frac{X^{1+s}}{s(s+1)}\, \mathrm{d} x .
\]
The integral is $O(X^{1/2+\epsilon})$. Moreover, 
\[
  \sum_{n<X+Y} (X+Y-n) = \frac{X+Y}{2} - (X+Y) + O((X+Y)^{1/2+\epsilon}) .
\]
Subtract the first equation from the second, to conclude that 
\[
  \sum_{n<X} 1 = X+O(X^{1/4+\epsilon}) .
\]
A more careful analysis of the integrals gets better error terms. 





\subsection{Why does the zeta function have such good analytic properties?}

\begin{defi}[Shintani]
The \emph{global zeta function} is, for ``nice test functions'' 
$f:V_\dR\to \dC$, 
\[
  Z(f,s) = \int_{\generallinear_2(\dR)/\generallinear_2(\dR)} |\det g|^{2 s} \left(\sum_{x\in V(\dZ)\smallsetminus S} f(g x)\right)\, \mathrm{d} g ,
\]
where $S=\{x\in V(\dZ):\discriminant(x)=0\}$. 
\end{defi}

\begin{prop}
We have the following decomposition:
\[
  Z(f,s) = \frac{1}{4\pi} \xi^+(s) \int_{V^+(\dR)} |\discriminant(x)|^{s-1} f(x)\, \mathrm{d}x + \frac{\xi_-(s)}{2\pi} \int_{V^-(\dR)} |\discriminant(x)|^{s-1} f(x)\,\mathrm{d} x .
\]
\end{prop}

Recall from Bhargava's lectures that the number of irreducible 
$\generallinear_2(\dZ)$-orbits in $G(\dZ)$ with $|\discriminant|<X$ is 
\[
  C \frac{\displaystyle\int_\cF\#\{x\in g B\cap V(\dZ)^\mathrm{irr}:|\discriminant(x)|<X\}\,\mathrm{d}g}{\displaystyle\int_B |\discriminant(b)|^{-1}\, \mathrm{d} v} .
\]
The sieve gives the estimate 
\[
  N_\mathrm{max}^\pm(X) = \sum_{q\geqslant 1} \mu(q) N^\pm(q,x) 
\]
where $N^\pm_\mathrm{max}(X)$ is the number of maximal cubic rings $R$ with 
$0<\pm \discriminant(R)<X$ and 
$N^\pm(q,X)$ is the number of cubic rings ``nonmaximal at $q$.'' Define the 
$q$-nonmaximal zeta functions: 
\[
  \xi_q^\pm(s) = \sum_{x\in \generallinear_2(\dZ)\backslash V^\pm(\dZ)} \frac{1}{\#\stabilizer(x)} \Phi_q(x) |\discriminant(x)|^{-s} ,
\]
where $\Phi_q(x)$ is the characteristic function of the set of cubic forms 
nonmaximal at $q$. We get 
\[
  \widehat\xi_q^\pm(s) = q^{8 s-8} \sum_{x\in \generallinear_2(\dZ)\backslash \widehat V_\dZ} \frac{1}{\#\stabilizer(x)}\widehat\Phi_q(x) |\discriminant(x)|^{-s} ,
\]
where 
\[
  \widehat\Phi_q(x) = \sum_{y\in V(\dZ/q^2)} \Phi_q(y) \exp\left(\frac{2\pi i[x,y]}{q^2}\right) ,
\]
and 
\[
  [x,y] = x_r y_1 - \frac 1 3 x_3 y_2 + \frac 1 3 x_2 y_3 - x_1 y_4 .
\]





% !TEX root = sms.tex

\section{Counting Artin representations and modular forms of weight one}
\thanksauthor{Eknath Ghate}





We'll start by spending a little bit of time explaining what modular forms are. 





\subsection{Brief introduction to modular forms}

Let $f = \sum_{n\geqslant 1} a_n q^n$ be a primitive cusp form of weight $1$, 
level $N\geqslant 1$, and character 
$\varepsilon:(\dZ/N)^\times \to \dC^\times$. Then $f$ is a holomorphic function 
$f:\fH\to \dC$, where $\fH=\{z\in \dC:\Im z>0\}$ is the upper-half plane. The 
function $f$ transforms by 
\[
  f(\gamma z) = \varepsilon(d) (c z+d) f(z) ,
\]
for all matrices $\begin{pmatrix} a & b \\ c & d \end{pmatrix}\in \Gamma_0(N)$. 
Moreover, $f$ is ``holomorphic at cusps.'' 

Deligne and Serre proved that to asuch a modular form is attached a continuous 
Galois representation $\rho_f:G_\dQ \to \generallinear_2(\dC)$ such that 
$\trace \rho_f(\frobenius_\ell) = a_\ell$ and 
$\det\rho_f(\frobenius_\ell) = \varepsilon(\ell)$ for all primes $\ell\nmid N$. 
Since $G_\dQ$ is compact and totally disconnected, the image of $\rho_f$ is 
finite. Once we projectivize, the image of 
$\widetilde\rho_f:G_\dQ\to \projectivegenerallinear_2(\dC)$ is one of 
\begin{align*}
  D_m && \text{dihedral group of order }2m \\
  A_4 && \text{tetrahedral} \\
  S_4 && \text{octahedral} \\
  A_5 && \text{icosahedral}
\end{align*}
We call the representations with projective image $A_4,S_4,A_5$ exotic. They 
are quite rare, with the first one occurring when $N=800$. 

\begin{enonce}{Conjecture}
The number of exotic forms of prime level $N$ is $O(N^\epsilon)$. 
\end{enonce}

If the level $N$ is prime, then only octahedral or icosahedral images occur. 
In this talk we will restrict to counting octahedral forms of prime level. Here 
is recent progress on the conjecture:
\begin{center}
\begin{tabular}{c|c}
person & bound \\ \hline
Duke & $O(N^{7/8+\epsilon})$ \\
Wang & $O(N^{5/6+\epsilon})$ \\
M-V & $O(N^{4/5+\epsilon})$ \\
Ganguly & $O(N^{3/4+\epsilon})$ \\
Ellenberg & $O(N^{2/3+\epsilon})$
\end{tabular}
\end{center}

\begin{theo}[Bhargava-Ghate]
Let $N_\mathrm{oct}^\mathrm{prime}(X)$ be the number of octohedral forms of 
prime level $<X$. Then 
\[
  N_\mathrm{oct}^\mathrm{prime}(X) = O(X/\log X) .
\]
\end{theo}
This is proven in \cite{bg09}. So on average, the number of octahedral forms 
of prime level is bounded by a constant. 

\begin{proof}
The idea is to ``count forms by counting forms.'' That is, we use the fact 
that octahedral modular forms of weight one correspond to Galois 
representations, which in turn correspond go quartic number fields, which 
come from quartic forms. 

Step 1. It is enough to count (linear) Galois representations. The Artin 
conjecture says that there is a bijection between octahedral forms and 
isomorphism classes of $\rho:G_\dQ \to \generallinear_2(\dC)$ with 
$\rho(G_\dQ)\simeq S_4$ and $\det\rho(c)=-1$. The direction $\Rightarrow$ was 
proved in \cite{ds74}, while $\Leftarrow$ is the Langlands-Tunnell theorem 
proved in \cite{l80,t81}.

One uses Serre's conjecture (proved in full generality in 
\cite{kw09-i,kw09-ii}) to prove the Artin conjecture. Choose 
$\cO\subset \dC$, the ring of integers in a number field such that 
$\rho(G_\dQ)\subset \generallinear_2(\dC)$. We get a commutative diagram 
\[\xymatrix{
  G_\dQ \ar[r]^-\rho \ar[dr]_-{\bar\rho} 
    & \generallinear_2(\cO) \ar[d] \\ 
  & \generallinear_2(\overline{\dF_p}) .
}\]
Serre's conjecture predicts that if $\bar\rho$ is odd and irreducible, then 
$\bar\rho\sim \bar\rho_g$ for a modular form $g$, where 
$g\in S_1(\Gamma_0(N),4)$. Write $g=\sum b_n q^n$. Then 
$a_\ell\equiv b_\ell\pmod p$ for all $\ell\nmid N$. Since this works for 
infinitely many $p$, there exists $g$ such that $a_\ell=b_\ell$. Thus 
$\rho$ is modular. 

Step 2. It is enough to count projective Galois representations. There is 
a surjection from isomorphism of odd $\rho:G_\dQ\to \generallinear_2(\dC)$ 
with $\widetilde\rho(G_\dQ)\simeq S_4$, to isomorphism classes of 
$\widetilde\rho:G_\dQ\to \projectivegenerallinear_2(\dC)$ with 
$\widetilde\rho(G_\dQ)\simeq S_4$ and $\widetilde\rho(c)\ne 1$. Surjectivity 
follows from $\h^2(G_\dQ,\dC^\times)=0$. The map is not injective: if 
$\chi$ is any character, Tate proved that 
$\rho\otimes\chi\mapsto\widetilde\rho$. But we can control the map when $N$ is 
square-free. If $p\| N$, then 
\[
  \rho|_{I_p} \sim \begin{pmatrix} \varepsilon_p \\ & 1 \end{pmatrix} .
\]
Choose $\chi$ with $\chi^{12}=1$, and apply this with $\rho\otimes\chi$ 
instead of $\rho$. The new character is $(\varepsilon\chi^2)^{12}$. When 
$N=p$, there are only two such $\chi$ so that $\rho\otimes\chi$ map to the 
same $\widetilde\rho$. We get that $\varepsilon=\varepsilon_p$ is odd. Some 
technical manipulations yield that $\rho\mapsto \widetilde\rho$ is 
2-to-1 when the level $N$ is prime. 

Step 3. It is enough to count quartic fields with Galois closure having group 
$S_4$. A projective representation 
$\widetilde\rho:G_\dQ \to \projectivegenerallinear_2(\dC)$ with 
$\widetilde\rho(G_\dQ)\simeq S-4$ and $\widetilde\rho(c)\ne 1$ cuts out a field 
$K$ over $\dQ$ that is not totally real. 

There is a key technical problem here. We want to count modular forms by level, 
but we usually count number fields by discriminant. Let $N$ be the level of 
$f$. If $K$ is the field corresponding to $f$ and $D=\discriminant(K)$, then 
we might have $N\ne D_K$. However, a prime $p\mid N_f$ if and only if 
$p\mid D_K$. Note that if $p\geqslant 5$, then either $p\| N_f$ or 
$p^2\| N_f$. However, it is possible for $p^3\| D_K$. Consider the following 
ramification table in the octahedral case (for minimal forms): 
\begin{center}
\begin{tabular}{c|c|c|c|c|c}
$I_p$ & $G_p$ & ram.~in $K$ & $D_K$ & $N_f$ & $p\equiv$ \\ \hline
$(12)$     & $I_p$       & $1^2 11$ & $p$ & $p$ \\
$(12)$     & $(12),(34)$ & $1^2 2$  & $p$ & $p^2$ \\
$(12)(34)$ & $I_p$       & $1^2 1^{\underline 2}$ & $p^2$ & $p$\\
$(13)(24)$ & $(1234)$    & $2^2$    & $p^2$ & $p$ \\
$(12)(34)$ & $V_4$       & $2^2$    & $p^2$ & $p^2$ \\
$(12)(34)$ & $(12)(34)$  & $1^2 1$  & $p^2$ & $p^2$ \\
$(123)$    & $I_p$       & $1^3 1$  & $p^2$ & $p$   & $1\pmod 3$ \\
$(123)$    & $S_3$       & $1^3 1$  & $p^2$ & $p^2$ & $2\pmod 3$ \\
$(1234)$   & $I_p$       & $1^4 $   & $p^3$ & $p$   & $1\pmod 4$ \\
$(1234)$   & $D_4$       & $1^4$    & $p^3$ & $p^2$ & $3\pmod 4$ 
\end{tabular}
\end{center}
[\ldots some notation I don't understand\ldots]. 
Five times, the power of $p$ in $D_K$ is at most the power of $p$ in 
$N_f$. The other five possibilities, this fails. For $4/5$ of the time we can 
still control things. The other possibility is $p\equiv 1\pmod 3$. 

We use some facts about quartic fields. Consider field extensions 
$E-\supset K_6\supset K_3\supset \dQ$ corresponding to the inclusions 
[\ldots missed this part\ldots]. 

The extension $K_6/K_3$ has Galois group $S_4$ if and only if 
\begin{enumerate}
  \item $K_3/\dQ$ has Galois group $S_3$ 
  \item $\norm_\dQ(\discriminant(K_6/K_3))=n^2$ for $n$ square-free 
  \item $\discriminant(K_4) = \discriminant(k_3) n^2$ 
  \item $K_6/K_3$ ramifies if and only if $p=1^4$ or 
    $1^2 1^2$ or $2^2$. 
\end{enumerate} 
We use the following theorem of Serre to simplify the table: 
\begin{center}
\begin{tabular}{c|c|c|c|c}
Ram.~in $K_4$ & $|\discriminant(K_4)|$ & level & $|\discriminant(K_3)|$ & $n$ \\ \hline
$1^2 11$ & $p$ & $p$ & $p$ & $1$ \\
$1^4$ & $p^3$ & $p$ & $p$ & $p$ 
\end{tabular}
\end{center}

Step 4. Use Bhargava's counting results in the quartic case, as well as some 
sieve methods. Recall, from \cite{b04} that isomorphism classes of maximal
$S_4$-quartic orders are in bijection with 
$\generallinear_2(\dZ)\times \generallinear_3(\dZ)$-classes of pairs of 
irreducible ternary quadratic forms $(A,B)$. From \cite{b05}, we the number 
$N_4(X)$ of number fields of $S_4$-quartic fields of $|\discriminant|<X$ is 
$O(X)$. A technical modification shows that the number $N_4^\mathrm{prime}(X)$ 
of $S_4$-quartic fields of prime level $<X$ is $O(X/\log X)$. As a 
corollary, 
\[
  \sum_{\substack{0<|\discriminant(K_3)|<X \\ \text{prime}}} h_2^\ast(K_3) \leqslant C\frac{X}{\log X} .
\]
We can finally count the number of octahedral of prime level $<X$ modular 
forms.  
\end{proof}

\begin{theo}[Serre]
In prime level, the discriminant of $K_4$ is either $p^3$ or $-p$. 
\end{theo}





% !TEX root = sms.tex

\section{Binary quartic forms: bounded average rank of elliptic curves}
\thanksauthor{Arul Shankar}





\subsection{Introduction}

Recall that every elliptic curve over $\dQ$ can be written as 
$y^2=x^3+A x+B$ for $A,B\in \dQ$. 

\begin{theo}[Mordell]
The abelian group $E(\dQ)$ is finitely generated. 
\end{theo}

So we can write $E(\dQ)=T\oplus \dZ^r$, where $T$ is a finite abelian group, 
and $r=\rank E$ is the \emph{rank} of $E$. We are going to study the average 
rank of elliptic curves. To do this, we need to order elliptic curves in some 
way. 

The elliptic curve $E_{A<B}:y^2=x^3+A x+B$ is isomorphic to 
$E_{n^4 A,n^6 B}:y^2=x^3+n^4 A x+n^6 B$ for all $n\in \dQ^\times$. So we can 
assume $A,B\in \dZ$. If we assume that $p^4\mid A\Rightarrow p^6\nmid B$, then 
the $A,B$ are unique. Thus there is a bijection between isomorphism classes of 
elliptic curves over $\dQ$ and 
\[
  \cE = \{E_{A,B}:(A,B)\in \dZ^2\text{ and }p^4\mid A\Rightarrow p^6\nmid B\} .
\]
We could also look at subfamilies of $\cE$ cut out by (possibly infinitely 
many) congruence conditions. We could also look at ``thin'' families consisting 
of all quadratic twists of some elliptic curve. 





\subsection{Ordering elliptic curves}

To talk about averages, we need to order elliptic curves in some way. The most 
obvious invariants to use are the discriminant and conductor. We have 
\[
  \discriminant(E_{A,B}) = \Delta(E_{A,B}) = 4 A^3 - 27 B^2 .
\]
The problem with ordering elliptic curves by discriminant is that we don't know 
how to count the number of elliptic curves with discriminant bounded by $X$. 
Essentially, the region $\{(x,y)\in \dR^2:4 x^2 - 27 y^3<X\}$ is noncompact, 
which makes the counting problem very hard. What is easier is to order elliptic 
curves by (naive) height: 
\[
  H(E_{A,B}) = \max\{|4 A^3|, 27 B^2\} .
\]

Let $f$ be a function on elliptic curves. The \emph{average} of $f$ is 
\[
  \average(f) = \lim_{X\to \infty} \frac{\sum_{H(E)<X} f(E)}{\sum_{H(E)<X} 1} .
\]
It's not hard to show that $\average(\# T)=1$. That is, on average an elliptic 
curve has no nontrivial torsion. This follows from Hilbert irreducibility. 

Our question is: what can we say about the average rank? 

\begin{enonce}{Conjecture}[Goldfeld, Katz-Sarnak]
$\average(\rank) = \frac 1 2$. Moreover, $50\%$ of elliptic curves have rank 
$0$ and $50\%$ have rank $1$. 
\end{enonce}

The conjecture was originally made with elliptic curves ordered by conductor, 
but ordering by conductor, height and discriminant are all expected to yield 
the same average. 

Given an elliptic curve $E$, we can define an $L$-function which we denote 
$L_E(s)$. The \emph{completed $L$-function} $L_E^\ast(s)$ satisfies a 
functional equation 
\[
  L_E^\ast(s) = \omega(E) L_E^\ast(1-s), 
\]
where $\omega(E)$, the \emph{root number} of $E$, is $\pm 1$. The 
\emph{analytic rank} of $E$ is the order of vanishing of $L_E$ at 
$s=1/2$. The \emph{Birch and Swinnerton-Dyer conjecture} predicts that the 
analytic rank and algebraic rank of $E$ are the same. But the analytic 
rank of $E$ is also the analytic rank of $L_E^\ast$. So the BSD conjecture 
implies that $\rank(E)$ is even if and only if $\omega(E)=1$. It's expected 
that $\omega(E)$ is equidistributed, i.e.~half of elliptic curves have 
$\omega(E)=0$ and half have $\omega(E)=1$. Assuming BSD, it would follow 
that half of elliptic curves have even rank, and half have odd rank. We also 
expect the rank of an elliptic curve to be ``as small as it can get away 
with,'' which would force $100\%$ of elliptic curves with $\omega(E)=0$ to 
have rank zero, and $100\%$ of elliptic curves with $\omega(E)=1$ to have rank 
one. 

In \cite{ks99}, Katz and Sarnak studied the family of all $L$-functions of 
elliptic curves. Assuming GRH and BSD, the we have the following bounds on 
$\average(\rank)$:
\begin{center}
\begin{tabular}{c|c}
\cite{b92} & $\leqslant 2.14$ \\
\cite{h04} & $\leqslant 2$ \\
\cite{y06} & $\leqslant 1.79$
\end{tabular}
\end{center}

More recently, we have the following theorem proven in \cite{j02}. 

\begin{theo}[de Jong]
For the family of all elliptic curves over $\dF_q(t)$, the average rank is 
bounded above by $\frac 7 6+\epsilon(q)$, where $\epsilon(q)\to 0$ as 
$q\to \infty$. 
\end{theo}

The method is to bound $\average(\# \selmer_3)$. 





\subsection{Selmer groups}

The fundamental idea is that from the canonical short exact sequence 
\[
  0 \to E(\dQ)/p \to \selmer_p(E) \to \sha(E)[p] \to 0 ,
\]
we get a bound on $\rank E$ in terms of $\selmer_p(E)$. Note that 
$\#(E(\dQ)/p) \geqslant p^{\rank E}$. 

First, we want a parameterization of 2-Selmer elements of elliptic curves. 
More generally, if $\sigma\in \selmer_p(E)$. then we can think of $\sigma$ as a 
locally soluble $p$-covering of $E$. Such a covering is a twist of 
$[p]:E\to E$. It will be a genus-one curve $C$ isomorphic to $E$ over 
$\overline\dQ$, along with $C\to E$ such that the following diagram commutes:
\[\xymatrix{
  C \ar[d]^-\wr \ar[dr] \\
  E \ar[r]^-{[p]} 
    & E .
}\]
See \autoref{sec:7} for more details. The covering $C$ is \emph{soluble} if 
$C(\dQ)\ne \varnothing$, and it is \emph{locally soluble} if 
$C(\dQ_v)\ne\varnothing$ for all places $v$ of $\dQ$. Locally soluble 
$p$-coverings of $E$ are in natural bijection with $\selmer_p(E)$, and 
soluble $p$-coverings are in bijection with $E(\dQ)/p$. For the rest of this 
lecture, we will concentrate on $p=2$. 

It turns out that a locally soluble $2$-covering of $E$ yields a binary quartic 
form over $\dQ$. Conversely, a binary quartic form gives a $2$-cover. 

Let $V$ be the space of binary quartic forms. The group $\generallinear(2)$ 
acts on $V$ via 
\[
  (\gamma\cdot f)(x,y) = \frac{1}{(\det\gamma)^2}f\left(\begin{pmatrix} x & y \end{pmatrix} \cdot \gamma\right) .
\]
The center $Z(\generallinear_2)$ acts trivially, so the action descends to 
one of $\projectivegenerallinear(2)$ on $V$. The ring of invariants is 
freely generated by two elements $I,J$, which have degree $2$ and $3$ 
respectively in the coefficients of the form. 

\begin{theo}[Birch-Swinnerton-Dyer, Cremona-Fisher-Stoll]
There is a bijection between 2-Selmer elements and the quotient 
$\projectivegenerallinear_2(\dQ)\backslash V(\dQ)^\mathrm{ls}$, where 
$V(\dQ)^\mathrm{ls}$ is the subset of locally soluble forms, in which 
$(A,B)$ corresponds to $I=-3\cdot 2^6 A$ and $J=-272\cdot 2^6 B$, and 
$A(f)=-I(f)/3\cdot 2^4$ and $B(f) = -J(f)/27\cdot 2^6$. 
\end{theo}

We will write this as 
$\selmer_2(E_{A,B}) = \projectivegenerallinear_2(\dQ)\backslash V(\dQ)_{A,B}^\mathrm{ls}$. 
The identity element of $\selmer_2(E_{A,B})$ corresponds to the orbit of 
binary quadratic forms with a rational linear factor. 

We would like a parameterization of 2-Selmer elements that involves binary 
quartic forms with integral coefficients instead of just rational coefficients. 

\begin{lemm}[Birch, Swinnerton-Dyer]
If $f\in V(\dQ_p)$, then $A(f),B(f)\in \dZ_p$. If $f$ is $\dQ_p$-solvable, then 
$\projectivegenerallinear_2(\dQ_p)\cdot f\cap V(\dZ_p)\ne\varnothing$. 
\end{lemm}

\begin{theo}
If $f\in V(\dQ)$, then $A(f),B(f)\in \dZ$. If $f$ is locally soluble, then 
$\projectivegenerallinear_2(\dQ)\cdot f\cap V(\dZ)\ne\varnothing$. 
\end{theo}
\begin{proof}
For each prime, find $\gamma_p\in\projectivegenerallinear_2(\dQ_p)$ so that 
$\gamma_p\cdot f\in V(\dZ_p)$. Since $\projectivegenerallinear_2$ has class 
number $1$, there exists $\gamma\in \projectivegenerallinear_2(\dQ)$ so that 
$\gamma\cdot f\in V(\dZ_p)$ for all $p$, hence $\gamma\cdot f\in V(\dZ)$. 
\end{proof}

\begin{theo}[Birch, Swinnerton-Dyer and Cremona, Fisher, Stoll]
The set $\selmer_2(E_{A,B})$ is naturally in bijection with 
$\projectivegenerallinear_2(\dQ) \backslash V(\dZ)_{A,B}^\mathrm{ls}$. 
\end{theo}

Define the \emph{height} of a binary quartic form to be 
\[
  H(f) = \max\{4|A(f)|^3, 27 B(f)^2\} .
\]
So the goal is to count $\projectivegenerallinear_2(\dQ)$-equivalence classes 
of integral, locally soluble binary quartic forms with height bounded by $X$. 





\subsection{First step}

The goal is to count $\projectivegenerallinear_2(\dZ)$-orbits of 
$V(\dZ)_{H<X}^\mathrm{irr}$. The method is extremely similar to how Bhargava 
counted binary cubic forms. First we construct a fundamental domain $\cF_X$ for 
the action of $\projectivegenerallinear_2(\dZ)$ on $V(\dR)_{H<X}$. Next we 
estimate $\#\{\cF_X\cap V(\dZ)\}$ using averaging. 

We begin by finding a fundamental set for 
$\projectivegenerallinear_2(\dR)\backslash V(\dR)$. Over any field $k$ in 
which $6$ is invertible, 
$\projectivegenerallinear_2(k)\backslash V(k)_{A,B}^{k-\mathrm{sol}}$ is 
in bijection with $E_{A,B}(k)/2$. Given $A,B\in \dR$, the set 
$\projectivegenerallinear_2(\dR)\backslash V(\dR)_{A,B}^\mathrm{ls}$ is in 
bijection with $E_{A,B}(\dR)/2$. But the group of $\dR$-valued points of an 
elliptic curves is easy to understand. It is $\dZ/2\times S^1$ or $S^1$, 
depending on whether the discriminant is positive or negative. It follows that 
$E_{A,B}(\dR)/2$ has either $1$ or $2$ elements as $\Delta(E_{A,B})<0$ or 
$\Delta(E_{A,B})>0$. 

If $\Delta(E_{A,B})<0$, then 
$\projectivegenerallinear_2(\dR)\backslash V(\dR)_{A,B}$ is a singleton, and 
any $f$ in the set will have exactly $2$ real roots. If 
$\Delta(E_{A,B})>0$, there are two orbits, one consisting of forms with two 
real roots, and one consisting of forms with positive-definite binary quartic 
forms. Define 
\begin{align*}
  V(\dR)^{(0)} &= \{f\in V(\dR)\text{ with 4 real roots}\} \\
  V(\dR)^{(1)} &= \{f\in V(\dR)\text{ with 2 real roots}\} \\
  V(\dR)^{(2+)} &= \{f\in V(\dR)\text{ positive definite}\} .
\end{align*}
A fundamental set for $\projectivegenerallinear_2(\dR)\backslash V(\dR)^{(i)}$ 
for $i\in \{0,1,2+\}$ is $\{f\text{ having invariants }A,B\}$. We obtain 
\[
  \projectivegenerallinear_2(\dR)\backslash V(\dR)^{(i)}_{H<1} = R_1^{(i)} ,
\]
with, for example, 
\[
  R_1^{(0)} = \{x^3 y + A x y^3 + B y^4,(A,B)\in \dR^2,\Delta(E_{A,B})>0,H(E_{A,B})<1\} .
\]
For general height, we scale: 
\[
  \projectivegenerallinear_2(\dR)\backslash V(\dR)_{H<X}^{(i)} = X^{1/6} R_1^{(i)} = R_X^{(i)} .
\]

So $R_X^{(i)}$ is a fundamental domain for the action of 
$\projectivegenerallinear_2(\dR)$ on $V(\dR)_{H<X}^{(i)}$. We want a 
fundamental domain for the action of $\projectivegenerallinear_2(\dZ)$. Choose 
$\cF=\projectivegenerallinear_2(\dZ)\backslash \projectivegenerallinear_2(\dR)$; 
then $\cF\cdot R_X^{(i)}$ is an $n_i$-fold cover of 
$\projectivegenerallinear_2(\dZ)\backslash V(\dR)_X^{(i)}$. It turns out that 
$n_1=2$ and $n_0 = n_{2+} = 4$. 

It follows that 
\[
  \#\projectivegenerallinear_2(\dZ)\backslash V(\dZ)_{H<X}^{(i),\mathrm{irr}} = \frac{1}{n_i} \#(\cF\cdot R_X^{(i)}\cap V(\dZ)^\mathrm{irr}) .
\]
As with binary cubic forms, we average: 
\begin{align*}
  \#\projectivegenerallinear_2(\dZ)\backslash V(\dZ)_{H<X}^{(i),\mathrm{irr}} 
    &= \frac{1}{n_i \volume(G_0)} \int_{G_0} \#\left(\cF gR_X^{(i)} \cap V(\dZ)^\mathrm{irr}\right)\, \mathrm{d} g \\
    &= \frac{1}{n_i\volume(G_0)} \int_\cF \#\left(g G_0 R_X^{(i)}\cap V(\dZ)^\mathrm{irr}\right)\, \mathrm{d} g \\
    &= \frac{1}{n_i \volume(G_0)} \int_\cF \volume\left(g G_0 R_X^{(i)}\right)\, \mathrm{d} g + O(X^{3/4}) \\
    &= \frac{1}{n_i\volume(G_0)} \int_{G_0} \volume\left(\cF g R_X^{(1)}\right)\, \mathrm{d} g + O(X^{3/4}) \\
    &= \frac{1}{n_i} \volume\left(\cF R_X^{(i)}\right) + O(X^{3/4}) .
\end{align*}

We can summarize all of this in the following theorem. 

\begin{theo}[Bhargava, Shankar]
\[
  \#\left(\projectivegenerallinear_2(\dZ)\backslash V(\dZ)_{H<X}^{(i),\mathrm{irr}}\right) = \frac{1}{n_i} \volume\left(\cF R_X^{(i)}\right) + O(X^{3/4}) .
\]
\end{theo}

As an easy corollary, the average rank of elliptic curves is bounded. 
We can compute this as  
\[
  \frac{1}{n_i} \volume\left(\cF R_X^{(i)}\right) = \frac{1}{n_i} |J|\volume(\cF) \volume\left(R_X^{(i)}\right) + O(X^{5/6}) 
\]
So the number of elliptic curves with height $<X$ is some constant multiple of 
$X^{5/6}$. Thus $\average(\#\selmer_2)$ is bounded, whence $\average(\rank)$ is 
bounded. 

In \autoref{sec:shankar-ii}, we'll derive an explicit bound for 
$\rank(\average)$. 





% !TEX root = sms.tex

\section{Selmer groups and heuristics I}
\thanksauthor{Bjorn Poonen}





d


% !TEX root = sms.tex

\section{Rational points on curves}
\thanksauthor{Michael Stoll}





d


% !TEX root = sms.tex

\section{Binary quartic forms: bounded average rank of elliptic curves II}\label{sec:shankar-ii}
\thanksauthor{Arul Shankar}





\subsection{Review}

Recall that if $E_{A,B}$ is the elliptic curve $y^2=x^3+A x+B$, then 
$\selmer_2(E_{A,B})$ is in bijection with the quotient 
$\projectivegenerallinear_2(\dQ)\backslash V(\dZ)_{(A,B)}^\mathrm{ls}$, where 
$V(\dZ)_{(A,B)}^\mathrm{ls}$ is the set of locally soluble integral binary 
quartic forms with invariants $A,B$. Even though the action of 
$\projectivegenerallinear_2(\dQ)$ does not preserve 
$V(\dZ)$, it still induces an equivalence relation, where $f\sim g$ whenever 
there is $\gamma\in \generallinear_2(\dZ)$ such that $\gamma \cdot f = g$. 
For example, the forms $p^4 x^4 + p^2 x y^3 + y^4$ and 
$x^4 + p^4 x y^3 p^4 y^4$ are $\projectivegenerallinear_2(\dQ)$-equivalent via 
the matrix $\begin{pmatrix} p^{-1} \\ & p \end{pmatrix}$. For asymptotics, we 
could restrict to irreducible quartic forms. We defined subsets 
$V(\dZ)_{(A,B)}^{(i)}$ of $V(\dZ)$; for definitions, see 
\autoref{sec:shankar-i}. We ended up with an estimate 
\[
  \#\left(\projectivegenerallinear_2(\dZ)\backslash V(\dZ)_{H<X}^{\mathrm{irr},(i)}\right) = \frac{1}{n_i} |J| \volume(\cF) \volume\left(R_X^{(i)}\right) + O(X^{3/4}) .
\]
In this lecture, we will prove the following theorem. 

\begin{theo}[Bhargava, Shankar]
$\average(\#\selmer_2) = 3$. 
\end{theo}

To do this, we will need to replace $V(\dZ)^\mathrm{irr}$ by 
$V(\dZ)^{\mathrm{irr},\mathrm{ls}}$, and replace 
$\projectivegenerallinear_2(\dZ)$-orbits by 
$\projectivegenerallinear_2(\dQ)$-equivalence classes. 





\subsection{Local solubility}

For each prime $p$, let $V(\dZ_p)^\mathrm{s}$ be the subset of $V(\dZ_p)$ 
consisting of soluble binary quartic forms. Let $V(\dZ)^{\mathrm{s}(p)}$ be the 
set of forms in $V(\dZ)$ whose image in $V(\dZ_p)$ is soluble. We start by 
computing 
\[
  \#\left(\projectivegenerallinear_2(\dZ)\backslash V(\dZ)_{H<X}^{\mathrm{irr},\mathrm{s}(p)}\right) = \frac{|J|}{n_i} \volume(\cF) \volume\left(R_X^{(i)}\right) \volume\left(V(\dZ_p)^\mathrm{s}\right) + O(X^{3/4}) .
\]
To do this for locally soluble forms, we will need to look at ``soluble at $p$ 
forms'' for all $p$. For that, we need a sieve. 

We want 
\[
  \#\left(\projectivegenerallinear_2(\dZ)\backslash V(\dZ)_{H<X}^{\mathrm{irr},p^2\mid \Delta}\right) = O(X^{5/6}/p^{1+\delta}) ,
\]
for any $\delta>0$. In fact, this is stronger than we need. All the proof 
requires is 
\[
  \sum_{p>M} \#\left(\projectivegenerallinear_2(\dZ)\backslash V(\dZ)_{H<X}^{\mathrm{irr},p^2\mid\Delta}\right) = O(X^{5/6}/f(M)) ,
\]
where $f(M)\to \infty$ as $M\to \infty$. 

Recall the (naive) height is $H(E_{A,B}) = \max\{4 |A|^3,27 B^2\}$. 
We want 
\[
  \#\{E_{A,B}:H(E_{A,B})<X\text{ and }p^2\mid \Delta(E_{A,B})\} = O\left(\frac{X^{5/6}}{p^{1+\delta}}\right),
\]
where $\delta>0$. When $p$ is large, map $E_{A,B}$ to the binary cubic form 
$x^3 + A x y^2 + B y^3$, which goes to 
$\generallinear_2(\dZ)\cdot (x^3+A x y^2 + B y^3)$. We have defined a map 
\[
  \varphi:U_1(\dZ) \to U(\dZ) \to \generallinear_2(\dZ)\backslash U(\dZ) ,
\] 
where $U_1(\dZ)$ is the space of elliptic curves and $U$ is the space of all 
binary cubic forms. This map is discriminant-preserving. 

\begin{theo}[Delone, Nagell, Siegel, Evertse, Akhtari]
The map $\varphi$ is at most 7-to-1 for elements with large enough 
discriminant. 
\end{theo}

This is very deep. 
As one application, a binary cubic form represents one at most seven times. 
It follows from the theorem that 
\[
  \#\left\{E_{A,B}:H(E_{A,B})<X\text{ and }p^2\mid \Delta(E_{A,B})\right\} = O\left(\frac{X}{p^2}\right) . 
\]
This isn't quite what we wanted because the estimate has $X$ instead of 
$X^{5/6}$. But for $p$ sufficiently large, it works. 

Suppose $p^2\mid \Delta(E_{A,B})$ for ``modulo $p$ reasons,'' i.e.~if $E_{A,B}$ 
has additive reduction. If $p>3$, then $A\equiv B\equiv 0\pmod p$. The bound in 
this case is 
\[
  O\left(\left(\frac{X^{1/3}}{p^2}+1\right)\left(\frac{X^{1/2}}{p}+1\right)\right) = O\left(\frac{X^{5/6}}{p^2} + \frac{X^{1/2}}{p}+1\right) .
\]
If $p\mid \Delta(E_{A,B})$ for ``modulo $p^2$ reasons,'' then fixing $A$ 
determines $B$ modulo $p^2$. In this case, the bound is 
\[
  O\left(X^{1/3} \cdot\left(\frac{X^{1/2}}{p}+1\right)\right) = O\left(X^{5/6}/p^2 + X^{1/3}\right) .
\]
Combining these estimates yields the uniform bound 
\[
  \#\{E:H(E)<X\text{ and }p^2\mid \Delta(E)\} = O\left(\frac{X^{5/6}}{p^{3/2}}\right) .
\]

Recall that there is a bijection between 
$\projectivegenerallinear_2(\dZ)\backslash V(\dZ)$ and the set of $(Q,C,x)$, 
where $Q$ is a quartic ring, $C$ is a cubic resolvent ring, and $x$ generates 
$C$ (?) The map $V(\dZ) \to \dZ^2\otimes \symmetric^2(\dZ^3)$ induces 
the map sending $(Q,C,x)$ to $(Q,C)$. On the side of forms, the map sends 
$a x^4 + b x^3 y + c x^2 y^2 + d x y^3 + e y^4$ to 
\[
  \begin{pmatrix} & & 1/2 \\ & -1 \\ 1/2 \end{pmatrix}, \begin{pmatrix} a & b/2 \\ b/2 & c & d/2 \\ & d/2 & e \end{pmatrix} .
\]
Our map yields the bound 
\[
  \#\left(\projectivegenerallinear_2(\dZ)\backslash V(\dZ)_{H<X}^{p^2\mid \Delta}\right) = O\left(\frac{X}{p^2}\right) .
\]
Combining everything, we get 
\begin{align*}
  \#\left(\projectivegenerallinear_2(\dZ)\backslash V(\dZ)_{H<X}^{p^2\mid \Delta,\mod p^2}\right) 
    &= O\left(\frac{X^{5/6}}{p^2} + X^{2/3}\right) \\
  \sum_{p<M} \#\left(\projectivegenerallinear_2(\dZ)\backslash V(\dZ)_{H<X}^{p^2\mid\Delta}\right) 
    &= O\left(\frac{X^{5/6}}{\log M}\right) .
\end{align*}





\subsection{Weights}

The problem is that a single $\projectivegenerallinear_2(\dQ)$-class in 
$V(\dZ)$ could break up into seven different 
$\projectivegenerallinear_2(\dZ)$-orbits. Given a form $f$, let 
\[
  B_f = \projectivegenerallinear_2(\dZ)\backslash \left(\projectivegenerallinear_2(\dQ)\cdot f\cap V(\dZ)\right) .
\]
For $f\in V(\dZ)$, we define 
\[
  W(f) = \begin{cases} 0 & \text{if $f$ is not locally soluble} \\ \left(\displaystyle\sum_{g\in B_f} \frac{\#\automorphism_\dQ(g)}{\automorphism_\dZ (g)}\right)^{-1} & \text{otherwise} \end{cases} 
\]
The weight of $f$ is a product of local weights. That is, define for each $p$ 
\[
  W_p(f) = \begin{cases} 0 & \text{if $f$ is not locally soluble} \\ \left(\displaystyle\sum_{g\in B_p(f)} \frac{\#\automorphism_{\dQ_p}(g)}{\automorphism_{\dZ_p} (g)}\right)^{-1} & \text{otherwise} \end{cases} 
\]
Then we have the following proposition (3.3 in my paper). 

\begin{prop}
For $f\in V(\dZ)$, we have $W(f) = \prod_p W_p(f)$. 
\end{prop}

We get the following formula: 
\[
  \#\left(\projectivegenerallinear_2(\dZ)\backslash V(\dZ)_{H<X}^{\mathrm{irr},W,(i)}\right) = \frac{|J|}{n_i} \volume(\cF) \volume\left(R_X^{(i)}\right) \prod_p \int_{V(\dZ_p)} W_p(f)\, \mathrm{d} f .
\]

\begin{prop}
\[
  \int_{V(\dZ_p)}W_p(f)\, \mathrm{d} f = |J|_p \volume(\projectivegenerallinear_2(\dZ_p)) \int_{\dZ_p^2} \frac{\#\left(E_{A,B}(\dQ_p)/2\right)}{\#E_{A,B}[2](\dQ_p)}\, \mathrm{d} (A,B) .
\]
\end{prop}

We also have 
\[
  |J|\volume\left(\projectivegenerallinear_2(\dZ)\backslash \projectivegenerallinear_2(\dR)\right) \int_{\{(A,B)\in\dR^2:H(E_{A,B})<X\}} \frac{\#(E_{A,B}(\dR)/2)}{\# E_{A,B}[2](\dR)}\, \mathrm{d}(A,B) .
\]

\begin{prop}[Brummer and Kramer]
\[
  \frac{\#(E(\dQ_v)/2)}{\# E[2](\dQ_v)} = \begin{cases} 1/2 & v=\infty \\ 2 & v=2 \\ 1 & \text{otherwise} \end{cases}
\]
\end{prop}

Thus the average of $\#\selmer_2)-1$ is the following limit: 
\[
  \lim_{X\to \infty} \frac{\displaystyle\volume(\projectivegenerallinear_2(\dZ)\backslash \projectivegenerallinear_2(\dR)) \prod_p \volume(\projectivegenerallinear_2(\dZ_p) + O(X^{3/2})) \int_{H(A,B)<X}\,\mathrm{d}(A,B)}{\displaystyle\int_{H<X} \mathrm{d}(A,B) + O(X^{1/2})} .
\]
The integrals cancel, and the product over all places in the numerator is the 
Tamagawa number $\tau(\projectivegenerallinear_2)=2$. It follows that 
\[
  \average(\#\selmer_2)=2+1=3 .
\]

When we generalize to $\selmer_n$ for $n\geqslant 3$, things get a bit more 
complicated, as in the following table: 
\begin{center}
\begin{tabular}{c|c|c|c|c}
$n$ & group & space & $\tau(G)$ & $\average(\#\selmer_n)$ \\ \hline
2 & $\projectivegenerallinear_2(\dZ)$ & $\symmetric^4(\dZ^2)$ & 2 & 3\\ 
3 & $\projectivegenerallinear_3(\dZ)$ & $\symmetric^3(\dZ^3)$ & 3 & 4\\ 
4 & qt.~of $\generallinear_2\times \generallinear_4$ & $\dZ^2\otimes\symmetric^2(\dZ^4)$ & 4 & 7\\
5 & qt.~of $\generallinear_5\times \generallinear_5$ & $\dZ^5\otimes \bigwedge^2 \dZ^5$ & 5 & 6
\end{tabular}
\end{center}





% !TEX root = sms.tex

\section{Coregular spaces and genus one curves}
\thanksauthor{Wei Ho}





\subsection{Introduction and motivation}

Something we have done many times is take a representation $V$ of a group $G$ 
and study the orbits $V/G$. The stabilizers of points in $V$ are also 
important. We have tried to describe the orbits in terms of ``arithmetically 
interesting'' objects, e.g.~elliptic curves with extra data. If we impose 
local conditions on $V/G$, we get elliptic curves with Selmer elements. One 
final note: we want stabilizers in $G$ to match up with automorphism groups of 
the ``arithmetically interesting objects.'' This is a subtle but important 
point. 

In \autoref{sec:shankar-i} and \autoref{sec:shankar-ii}, the ``extra data'' 
attached to an elliptic curve $E$ was an $n$-covering of $E$. This consists 
of a $E$-torsor $C$ with a degree $n$ line bundle on $C$. 

\begin{enonce}[remark]{Example}
In the binary quartic case, the form $f(x,y) = a x^4 + \cdots$ corresponds to 
the curve $C:z^2 = a x^4 + b x^3 y + c x^2 y^2 + d x y^3 + e y^4$; this is a 
double cover of $\dP^1$ ramified at four points. The invariants of $f$ 
determine an elliptic curve. The map $C\to \dP^1$ determines the line bundle 
on $C$. 
\end{enonce}

\begin{enonce}[remark]{Example}
Recall that we can describe 3-Selmer elements with ternary cubics (homogeneous 
degree three polynomials in three variables). The representation involved 
is $\symmetric^3(3) = \symmetric^3\dA^3$. Any ternary cubic $f$ gives a genus 
one curve in $\dP^2$. It's Jacobian is canonically the elliptic curve with 
invariants those of $f$. The degree three line bundle on $C$ comes from the 
embeddigng $C\hookrightarrow \dP^2$. The rings of invariants of the action of 
$\speciallinear(3)$ on $\symmetric^3(3)$ is polynomial in two generators 
$S,T$, of degree 4 and 6 respectively. The Jacobian of $C_f$ is 
$E_{S(f),T(f)}$. 
\end{enonce}

To generalize these ideas, we need to find other representations that 
parameterize interesting data. Let $k$ be an algebraically closed field. If 
$(G,V)$ is a prehomogeneous vector space over $k$ and $U\subset V$ is open and 
$G$-stable, then $U(k)/G(k)$ might be ``zero-dimensional,'' e.g.~a single 
Zariski-open orbits. But there are lots of non-isomorphic elliptic curves, 
even over an algebraically closed field. so we would want 
$U(k)/G(k)$ to be ``bigger,'' e.g.~the affine line. 

More geometrically, we are trying to find prehomogeneous vector spaces 
$(G,V)$ such that the coarse moduli space $V/G$ is a moduli space we already 
are familiar with. In all our examples, the invariant rings are polynomial 
rings. We call such representations \emph{coregular}. 

The moduli space of elliptic curves is birational to $\dP(2,3)$. So we should 
look for coregular representations with ring of invariants free in two 
variables. Elliptic curves with one marked point are of the form 
$y^2 + d_3 y = x^3 + d_2 x^2 + d_4 x^2$. If we take our scaling, we get 
$\dP(2,3,4)$, or $\dA^3$ if we also keep track of the differential. 

To summarize: many (but not all!) families of elliptic curves with extra data 
have coarse moduli space (look over an algebraically closed field) birational 
to a weighted projective space. This means the invariant ring of any possible 
$(G,V)$ is a polynomial ring. 

[\ldots couldn't follow\ldots]





% !TEX root = sms.tex

\section{Arithmetic invariant theory and hyperelliptic curves I}\label{sec:gross-i}
\thanksauthor{Benedict Gross}





Let $k$ be a field, $G$ a reductive group over $k$, and 
$G\to \generallinear(V)$ a representation of $G$. The ring 
$\symmetric^\bullet(V^\vee)$ contains a subring 
$\symmetric^\bullet(V^\vee)^G$ of invariant polynomials. An important theorem 
is that $\symmetric^\bullet(V^\vee)^G$ is a finitely generated $k$-algebra. 
Write $V\gq G$ for the variety 
$\spectrum\left(\symmetric^\bullet(V^\vee)^G\right)$; this comes with a 
canonical ``projection'' $\pi:V\to V\gq G$. 





\subsection{First examples and results}

\begin{enonce}[remark]{Example}
Consider $G=\generallinear(W)$ and $V=\mathfrak{gl}(W)=\End W$, with the 
adjoint action of $G$ on $V$. It is known that 
$\mathfrak{gl}(W)\gq \generallinear(W)$ is an affine space with coordinates the 
``coefficients of the symmetric polynomial.'' 
\end{enonce}

More generally, if $G$ is a reductive group and $\frakg=\lie G$ under the 
adjoint representation, then $\frakg\gq G$ is affine. That is, Chevalley 
proved that the adjoint representation of a reductive group is coregular. 
Winberg generalized this even further. If $\theta:G\to G$ is an 
automorphism of order $m$ and $\frakg=\bigoplus_a \frakg(a)$, then 
the action of $G^\theta$ on each $\frakg(a)$ is coregular. 

Suppose $f\in (V\gq G)(k)$. Let $V_f$ be the fiber in $V$ of $\pi$ over $f$. 
Then $V_f(k)$ is a (possibly empty) union of $G(k)$-orbits. 

\begin{enonce}[remark]{Example}
When $G=\generallinear(n)$ and $V=\mathfrak{gl}(n)$, then $V_f$ is all linear 
$T$ with fixed characteristic polynomial $f$. The set $V_f(k)$ is always 
nonempty. Indeed, let $L=k[x]/f$ and $\theta:L\to L$ be ``multiplication 
by $x$.'' Then $\theta$ is a $k$-linear transformation with characteristic 
polynomial $f$. Choosing an isomorphism $L\simeq k^n$ gives an element in 
$V_f(k)$. Roughly, ``every polynomial is the characteristic polynomial of a 
map defined over the base field.'' 

If the discriminant $\Delta(f)\ne 0$, there 
is a single orbit of $G(k)$ on $V_f(k)$. For any $T\in V_f(k)$, 
the stabilizer $G_T$ is isomorphic to the Weil restriction 
$\Pi_{L/k}\dG_\multiplicative$; a maximal torus in $\generallinear(V)$ if $L$ 
is \'etale. 
If $f(x)=x^k$, then $V_f$ is known as the \emph{nilpotent cone}; the orbit 
we constructed is the \emph{regular nilpotent}. 
\end{enonce}

\begin{enonce}[remark]{Example}
Consider the action of $\speciallinear(W)$ on 
$\mathfrak{sl}(W)=\End(W)^{\trace=0}$. The ring of invariants is freely 
generated by all but the constant term of the characteristic polynomial, so 
$\mathfrak{sl}(n)\gq \speciallinear(n) \simeq \dA^{n-1}$. If $\Delta(f)\ne 0$, 
then orits in $V_f(k)$ are in bijection with $k^\times / \norm(L^\times)$. 

If for example $\dim W=2$ and $f(x)=x^2+1$, then the orbit space 
$V_f(k)/G_f(k)$ is in bijection with $\dQ^\times/\norm(\dQ(i)^\times)$; a 
(huge) abelian 2-group. 
\end{enonce}

For $\generallinear(n)$ and $\speciallinear(n)$, we obtained that 
$V_f(k)$ was a torsor over $\h^1(k,G_f)$. But this used 
$\h^1(k,\speciallinear_n) = \h^1(k,\generallinear_n) = 0$. 





\subsection{Principles of arithmetic invariant theory}

\begin{enonce}{Principle}
Assume $V_f(k)\ni v$, and that $G(k^s)$ acts transitively on 
$V_f(k^s)$. Then the set of $G(k)$-orbits on $V_f(k)$ is in bijection with the 
kernel of the map of pointed sets 
$\h^1(k,G_v) \to \h^1(k,G)$. 
\end{enonce}
\begin{proof}
Say $v'\in V_f(k)\subset V_f(k^s)$, write $v'=g(v)$ for some $g\in G(k^s)$. 
Send the orbit of $v'$ to the class in $\h^1(k,G_v)$ of the cocycle 
$\sigma\mapsto c_\sigma = g^{-1} \circ g^\sigma$. Checking that this is a 
bijection is easy. 
\end{proof}

If $G$ is one of $\generallinear(n)$, $\speciallinear(n)$, $\symplectic(n)$, 
then $\h^1(k,G)=0$, so $V_f(k)/G(k) = \h^1(k,G_v)$. 

\begin{enonce}[remark]{Example}
Let $W$ be a split orthogonal space over $k$ of dimension $n=2 g+1$. So $W$ 
is a direct sum of $g$ hyperbolic planes and a single copy of $k$. Let 
$G=\specialorthogonal(W)$. For example, if 
$g=1$, then $G\iso \projectivegenerallinear(2)\iso \specialorthogonal_3$. 
Let $V=\mathfrak{so}(W)$; the space of trace-zero self-adjoint operators on 
$W$. Then 
$\symmetric^\bullet(W^\vee)^G=k[c_2,\dots,c_{2g+1}]$, freely generated on 
the coefficients of the characteristic polynomial. It's a bit more 
difficult to show that the fibers of 
$\mathfrak{so}(W) \to \mathfrak{so}(W)\gq \specialorthogonal(W)$ are all 
nonempty. Given $f$ in the quotient, define as before $L=k[x]/f$. This is a 
$k$-algebra of rank $2 g+1$. Let $\langle\lambda,\mu\rangle$ be the coefficient 
$x^{2 g}$ in $\lambda\mu$; also $\trace_{L/k}(\lambda\mu/f'(x))$. The 
operator ``multiplication by $x$'' on $L$ is self-adjoint with characteristic 
polynomial $f$. A bit of work shows that this gives an element of $V_f(k)$. 
\end{enonce}

In fact, for $V=\mathfrak{so}(n)$, $G=\specialorthogonal(n)$, the map 
$V(k) \to V\gq G$ has a standard section known as the Kostant section. 

Let's compute stabilizers. For $f$ with $\Delta(f)\ne 0$, it is easy to see 
that $G_v=\specialorthogonal(W)\cap L^\times$. This is 
$L^\times[2]^{\norm=1}$. As a group scheme, this is 
$\ker(\Pi_{L/k}\boldsymbol\mu_2 \xrightarrow{\norm{}}\boldsymbol\mu_2)$. 
An easy computation of Galois cohomology shows that 
$\h^1(k,G_v) = (L^\times/2)^{\norm=0}$. In this case, $\h^1(k, G_v)$ is also 
$\h^1(k,J[2])$, where $J$ is the Jacobian of $y^2=f(x)$. 

In general, our first principle is not very useful, because the map 
$\gamma:\h^1(k,G_v) \to \h^1(k,G)$ can be pretty complicated, and is not easy to 
pin down explicitly. 

\begin{enonce}{Principle}
For any $c\in \h^1(k,G)$, there is a twisted group $G^c$ over $k$ and twisted 
representation $V^c$ over $k$. The fiber of $\gamma$ over $c$ is the set of 
orbits of $G^c(k)$ in $V_f^c(k)$. 
\end{enonce}

Recall our map 
$\h^1(k,G_v) = \h^1(k,J[2]) \xrightarrow\gamma \h^1(k,\specialorthogonal(W))$, 
where $J$ is the Jacobian of $y^2=f$. The 2-Selmer group $\selmer_2(J)$ is a 
subset of $\h^1(k,J[2])$, and in \cite{bg13}, Bhargava and I showed that 
$\selmer2(J) = \ker(\gamma)$. In general, when is $V_f$ empty for all 
$G^c$?

\begin{enonce}{Principle}
Assume $G(k^s)$ acts transitively on $V_f(k^s)$ and $G_v(k^s)$ is abelian. 
\begin{enumerate}
  \item If the class of $d_f$ is non-trivial in $\h^2(k,G_f)$, there is no 
    $k$-rational point in any fiber. 
  \item If $d_f=0$, then the fiber is nontrivial for some pure inner form $G^c$. 
\end{enumerate}
\end{enonce}
\begin{proof}
Take $v\in V_f(k^s)$. Then $\sigma_v = \prescript{\sigma}{}{f}\circ f$ is also 
in $V_f(k^s)$. Define $\theta_\sigma:G_{c_v} \to G_v$ by 
$\alpha\mapsto g_\sigma \alpha g_\sigma^{-1}$. This map is an isomorphism that 
does not depend on $v$. Since 
$\theta_{\sigma\tau} = \theta_\sigma\circ\prescript{\sigma}{}{\theta_\tau}$, 
this descends $G_v$ to a group $G_f$ defined over $k$. 
\end{proof}

So if $G_v(k^s)$ is abelian and $f\in (V\gq G)(k)$, there is a ``stabilizer'' 
$G_f$ of $f$ even if $V_f(k)=\varnothing$. 

\begin{enonce}[remark]{Example}
Let $G=\speciallinear(W)$, where $\dim W=2 g+2$. Let 
$V=\symmetric^2(W^\vee)\oplus \symmetric^2(W^\vee)$. We can think of 
$V$ as the space of pairs $v=(A,B)$ of symmetric matrices. The ring of 
invariants is freely generated by the coefficients of the bilinear form 
$f(x,y) = (-1)^{g+1}\det(x A-y B)$. Assume $\Delta(f)\ne 0$. Then put 
$G_f=(\Pi_{L/k}\boldsymbol\mu_2)^{\norm=1}$, where $L/k$ is the extension 
constructed earlier. If we put $f(x,1) = f_0 \cdot g(x)$, then 
$L=k[x]/g$. What is the class $d_f\in \h^2(k,G_f)$? This cohomology group 
has a subgroup $k^\times / k^\times \norm(L^\times)$. The class $d_f$ is just 
the class of $f_0$ in $k^\times / k^\times \norm(L^\times)\subset \h^2(k,G_f)$. 
\end{enonce}

For example, if $k=\dR$, $g=0$, $f_0=-1$, $f=-x^2-y^2$, and $g=x^2+1$, then there 
are no orbits. 





% !TEX root = sms.tex

\section{Applications to the Birch and Swinnerton-Dyer conjecture}\label{sec:bhargava-iv}
\thanksauthor{Manjul Bhargava}





d


% !TEX root = sms.tex

\section{Selmer groups and heuristics II}\label{sec:poonen-iii}
\thanksauthor{Bjorn Poonen}





d


% !TEX root = sms.tex

\section{Pencils of quadrics: the geometry}
\thanksauthor{Jerry Wang}





Pencils of quadrics have shown up many times (though not under that name) in 
this summer school. I will explain some of the geometry of quadrics. 





\subsection{Notation}

Let $k$ be a perfect field of characteristic not $2$. Let $\sL$ be a 
rational generic pencil of quadrics in $\dP^{2n+1}$. Such an $\sL$ will be 
of the form 
\[
  \{x Q-y Q_2:(x:y)\in \dP^1\}
\]
where $Q_1,Q_2\subset \dP^{2n+1}$ are quadrics. ``Rational'' means the $Q_i$ 
are defined over $k$. ``Generic'' means the binary form 
$f(x,y) = (-1)^{n+1}\det(x Q_1-y Q_2)$ has no repeated factors, 
i.e.~$\discriminant(f)\ne 0$. Alternatively, $C:z^2=f(x,y)$ should be a smooth 
hyperelliptic curve of genus $n$. The \emph{base locus} $B=Q_1\cap Q_2$ will 
be smooth of dimension $2 n-1$. 

Let $F$ be the variety of maximal linear subspaces of $B$. That is: 
\[
  F=\{X\simeq \dP^{n-1}:X\subset B\} .
\]
For example, when $n=1$, we have two quadrics $Q_1,Q_2\subset \dP^3$. We have 
$F=B=Q_1\cap Q_2$, a genus one curve. So over $\bar k$, $F$ is isomorphic to an 
elliptic curve. This is basically the construction Bhargava used to study 
Selmer elements of elliptic curves. 

If $n=2$, $B=Q_1\cap Q_2$ is a degree $4$ three-fold. The variety $F$ turns 
out to be an abelian surface (over $\bar k$). 

This apparent pattern holds. 

\begin{theo}[Reid, Donagi, Desale-Ramanan]
Over $\bar k$, the variety $F\simeq \jacobian C$; an abelian variety of 
dimension $n$. 
\end{theo}

To obtain arithmetic information about $C$, we need a result that works over 
an arbitrary (possibly non algebraically closed) base field. 

\begin{theo}
The variety $F$ is a $J=\jacobian(C)$-torsor. Moreover, there is an algebraic 
group structure on $G=J\sqcup F\sqcup J^1\sqcup F$ compatible with that of 
$J$, and for which $G/J\simeq \dZ/4$. 
\end{theo}


Here $J^1=\picard^1(C)$, the moduli space of degree-$1$ line bundles on $C$. 

For $n=1$, $Q_1,Q_2\subset \dP^3$, we had $F=Q_1\cap Q_2$. The curve $C$ is 
defined by $z^2=\det(x Q_1-y Q_2)$. The curve $\det(x Q_1-y Q_2)$ is cut out 
by a binary quartic form. There is a canonical isomorphism $H^1=C$, so we have 
a group structure on $G=E\sqcup F\sqcup F\sqcup F$. Multiplication by $2$ on 
$G$ gives a map $2:C\to E$; this is a $2$-cover of $E$. This is the cover used 
in the study of 2-Selmer groups of elliptic curves. Multiplication by $4$ gives 
a map $4:F\to E$; this is the $4$-cover used in the study of 4-Selmer groups of 
elliptic curves. 

For general $n\geqslant 2$, there are a couple cases. 

Case 1: $C(k)\ne\varnothing$. Choose $\infty\in C(k)$. Put 
$F[2]_\infty = \{X\in F:X+X=(\infty)\}$; this is a $J[2]$-torsor. So we get an 
element of $\h^1(k, J[2])$. There are two subcases 
corresponding to whether or not $\infty$ is a Weierstrass point. If $C$ has a 
Weierstrass point, we get all torsors of $J[2]$ in this way. When $\infty$ is a 
non-Weierstrass point, we don't get all of $\h^1(k,J[2])$, but we do get 
``enough'' points in $\h^1(k,J[2])$, namely the entire kernel of 
$\gamma:\h^1(k,J[2]) \to \h^1(k,?)$ as in \autoref{sec:gross-i}. 

Case 2: the map $2:F\to J^1$ is a $2$-cover of $J^1$. If $k$ is a global field 
and $C$ is everywhere locally soluble, we get all locally soluble 2-covers of 
$J^1$ using this method. 





% !TEX root = sms.tex

\section{Arithmetic invariant theory and hyperelliptic curves II}\label{sec:gross-ii}
\thanksauthor{Benedict Gross}





\subsection{Redefining hyperelliptic curves}

We'll start with a slightly more general definition of a hyperelliptic curve. 
If $C$ is a curve of genus $g\geqslant 1$ over a field $k$, then 
$\h^0(C,\Omega^1)$ is a $g$-dimensional $k$-vector space. It is known that this 
has no base points. So we get a map 
$\pi:C\to \dP(\h^0(\Omega^1))\simeq \dP^{g-1}$; this is called the 
\emph{canonical map}. If $g\geqslant 2$, you can use the Riemann-Roch theorem 
to prove that this map is either an embedding with image a smooth curve of 
degree $2 g-2$, or it's 2-to-1 onto a rational normal curve $X$ of degree 
$g-1$. Say $C$ is \emph{hyperelliptic} if we're in the second case. 

If $X(k)\ne\varnothing$, then we get a map $C\to \dP^1$ and recover the 
standard definition of a hyperelliptic curve. This always happens if $g$ is 
even. But if $g$ is odd, we might not have a rational point. The 
image $X$ of $\pi:C\to \dP^{g-1}$ will be of the form $\{Q(x,y,z)=0\}$. 

[\ldots stuff I didn't catch\ldots]

We define $C\subset \dP(1,1,1,\frac{g+1}{2})$. 

We don't really need this, because we will be studying $C$ with local points 
everywhere. Via $C\to X$, the curve $X$ has local points everywhere. By the 
Hasse principle, $X(k)\ne\varnothing$, so $X\simeq \dP^1$. Thus $C\to X$ 
realizes $C$ as a curve of the form $z^2=F(x,y)$ inside 
$\dP(1,1,g+1)$. For the rest of this lecture, $C$ will be a hyperelliptic 
curve defined by an equation of this form. 





\subsection{Main result}

The following is a strenthening of the theorem Bhargava proved in 
\autoref{sec:bhargava-iv}. 

\begin{prop}
A positive proportion of such $C$ have no rational points over any extension 
of odd degree over $\dQ$. 
\end{prop}

For heuristic reasons, we suspect the proportion is $3/4$. 

Let $J$ be the Jacobian of $C$; this is an abelian variety over $k$ of 
dimension $g$. For each $n$, there is a variety $J^n=\picard^n(C)$ which 
classifies line bundles of degree $n$ over $C$. Each $J^n$ is a $J$-torsor. For 
hyperelliptic curves, there is a canonical element $h=\pi^\ast\sO(1)\in J^2(k)$ 
coming from the degree-two map $C\to \dP^1$. (Just pull back any point in 
$\dP^1$.) Thus $J^r\simeq J^{r+2n}$ for all $n$. The degree-one part $J^1$ is 
especially important because the Abel map $C\to J^1$ given by 
$x\mapsto [x]$ is an embedding defined over $k$. 

\begin{prop}
The following are equivalent:
\begin{itemize}
  \item $C$ has no rational points over any extension of odd degree 
  \item $J^1(\dQ)=\varnothing$
\end{itemize}
\end{prop}
\begin{proof}
Indeed, if $x\in C(L)$ and $[L:\dQ]=2n+1$, then $\sum [x^\sigma]$ will be a 
divisor of degree $2n+1$, hence an element of $J^1(\dQ)$. For the converse, we 
only have to prove that $J^1$ is \emph{not} isomorphic to $J$ over $\dQ$, 
because any element of $J^1(\dQ)$ gives an isomorphism $J\iso J^1$. 
\end{proof}

We're implicitly using the fact that the curve has local points everywhere. 
There is a ``Brauer obstruction class'' measuring whether a rational divisor 
class comes from a rational divisor. When our curve is everywhere locally 
soluble, the Brauer class vanishes locally, so in this case it vanishes. 





\subsection{Fundamental groups}

There is a beautiful idea, going back to Serre, Grothendieck, \ldots that 
studies varieties via their unramified covers. We will distinguish $J$ from 
$J^1$ by studying their \emph{arithmetic fundamental groups}. In particular, 
we will study their unramified 2-covers. 

Recall that a \emph{2-covering} of $J$ is a $J$-torsor $F$ with an \'etale 
covering $\pi:F\to J$ such that $\pi(f+a) = \pi(f)+2 $. So $\pi^{-1}(0)$ is a 
$J[2]$-torsor. It follows that $\pi:F\to J$ has degree $2^{2 g}$. There is an 
obvious notion of equivalence of 2-covers: via commutative diagrams 
\[\xymatrix{
  F' \ar[r]^-\pi \ar[d]^-\wr 
    & J \ar@{=}[d] \\
  F \ar[r]^-\pi 
    & J .
}\]
For a 2-covering $F$, put $F[2]=\pi^{-1}(0)$; the 2-covering $F$ is completely 
determined by the class of $F[2]$ in $\h^1(\dQ,J[2])$. Solvable 2-coverings 
correspond to the image of $J(\dQ)/2\hookrightarrow \h^1(\dQ,J[2])$. Put 
$\picard(C)=\coprod_{n\in \dZ} J^n$; the group 
$\picard(C)/\dZ h=J\sqcup J^1$. So multiplication by 2 induces a 2-cover 
$J^1\to J$. For this cover, $\pi^{-1}(0)=\{f\in J^1:2 f=h\}$, which we call 
$W[2]$. For $100\%$ of hyperelliptic curves, $W[2]$ is nontrivial. But 
$W[2]$ is locally soluble, so it gives a class in $\selmer_2(J)$. Since 
$W[2]$ is nontrivial, we see that $100\%$ of the time, 
$\selmer_2(J)\ne 0$. 

We can even define 2-covers of $J^1$. They are maps $\pi:F\to J^1$ such that 
$\pi(f+a)=\pi(f)+2 a$. Composing with $J^1\to J$, we get a 4-cover 
$F\to J$. So $\pi^{-1}(0_J)$ is a $J[4]$-torsor. Let $\selmer_2(J^1)$ be the 
set of locally soluble 2-covers of $J^1$. We will distinguish between 
$J$ and $J^1$ by showing that on average, 
$\#\selmer_2(J^1) < \# \selmer_2(J)$. Thus there will be many curves with 
$J\not\simeq J^1$. Note that 
\[
  \selmer_2(J^1) = \{\alpha\in \selmer_4(J):2\alpha = W[2]\} .
\]
So if $W[2]$ is not divisible by $2$, $J^1\not\simeq J$. We have a commutative 
diagram 
\[\xymatrix{
  0 \ar[r] 
    & J(\dQ)/4 \ar[r] \ar[d] 
    & \selmer_4 J \ar[r] \ar[d] 
    & \sha(J)[4] \ar[r] \ar[d] 
    & 0 \\
  0 \ar[r] 
    & J(\dQ)/2 \ar[r] 
    & \selmer_2(J) \ar[r] 
    & \sha(J)[2] \ar[r] 
    & 0 .
}\]
We will try to count the Selmer set $\selmer_2(J^1)$. Pencils 
$x A-y B$ of quadrics in $\dP^{2g+1}(\dQ)$ give 2-coverings 
$\pi:F\to J^1$. This we saw in \autoref{sec:wang-i}. 

We study the action of $\speciallinear(2 g+2)$ on 
$\symmetric^2(2 g+2)\oplus \symmetric^2(2g+2)$. Using Bhargava's methods, we 
show that $\average(\#\selmer_2 J^1) \leqslant 2=\tau(\speciallinear_{2g+2})$. 
This is (just barely) enough. We know that 
$\average(\#\selmer_2 J) \geqslant 2$. So we need to show that a positive 
proportion of the time, $\average(\#\selmer_2 J)>2$. The Dokchitser brothers 
have shown that the parity of $\selmer_2 J$ is equidistributed. So there have 
to be curves with $J\not\simeq J^3$. A better result would be 
$\average(\#\selmer_2 J)=3$, but even this is not good enough. 

\emph{Note}: one usually creates abelian covers of $C$ through the fundamental 
group of its Jacobian. But when $C$ has no rational points, $C$ embeds into 
$J^1$, not $J$, so we would have to look at covers of $J^1$. But these don't 
always exist! 

\begin{prop}
Let $f(x,y)$ be a binary form of degree $2g+2$ with $\Delta\ne 0$ and 
$f_0\ne 0$, defined over $\dQ$. The following are equivalent:
\begin{enumerate}
  \item There exists an orbit $(A,B)$ with $\discriminant(x A-y B)=f(x,y)$. 
  \item $f_0\in (\dQ^\times)^2 \norm(L^\times)$. 
  \item $J^1_\fm$ is divisible by 2 in $\h^1(\dQ,J_\fm)$, where 
    $\fm=[p_\infty]+[p_\infty']$ and $C_\fm:z^2=y^2 f(x,y)$ has genus $g+1$. 
  \item The maximal abelian 2-cover of $C$ ramified only at 
    $\{p_\infty,p_\infty'\}$ descends to a cover $D\to C$ over $\dQ$. 
\end{enumerate}
\end{prop}





% !TEX root = sms.tex

\section{Chabauty methods and hyperelliptic curves}\label{sec:poonen-iv}
\thanksauthor{Bjorn Poonen}





\subsection{Introduction}

For an elementary introduction to Chabauty's method, see \cite{mp12}. 

\begin{theo}[Faltings, 1983]
If $C$ is a curve of genus $\geqslant 2$ over $\dQ$, then $C(\dQ)$ is finite. 
\end{theo}

The proof of this is ``bad'' in the sense that it is highly ineffective. It 
does produce an upper bound on the number of rational points, but neither 
Faltings' or Vojta's later proof give horrible upper bounds for $\# C(\dQ)$, 
and give no upper bound on the height of those points. 

Much earlier Chabauty created a more effective method for attacking this 
problem for certain families of curves. 

\begin{theo}[Poonen, Stoll, 2013]
Let $C:y^2=f(x)$, where $\deg f=2 g+1$. 
\begin{enumerate}
  \item For each $g\geqslant 3$, the fraction of such $C$ satisfying 
    $C(\dQ)=\{\infty\}$ is positive. 
  \item This fraction tends to $1$ as $g\to \infty$. More precisely, it is 
    $\geqslant 1-(12 g+20)g^{-g}$. 
  \item Chabauty's method at the prime $2$ effectively determines $C(\dQ)$ for 
    such a fraction of curves. 
\end{enumerate}
\end{theo}
\begin{proof}
See \cite{ps13}. 
\end{proof}

Conjecturally, $100\%$ of such $C$ have $C(\dQ)=\{\infty\}$. This theorem is 
essentially the first case in which an effective version of Faltings' theorem 
was proven for a large class of curves. 





\subsection{Genus one}

Let $E$ be an elliptic curve over $\dC$. Then $E(\dC)\simeq \dC/\Lambda$ for 
some discrete subgroup $\Lambda\subset \dZ^2$ in $\dC$. The differential 
$d z$ is a well-defined holomorphic one-form on $\dC/\Lambda$, corresponding to 
an algebraic differential $\omega\in \Omega^1(E)$. If we had started with 
$\omega$, we can define $E(\dC)\to \dC/\Lambda$ by 
\[
  x\mapsto \int_0^x \omega .
\]
Changing the path $0\to x$ changes the integral by an element in the discrete 
lattice $\Lambda=\{\int_\gamma\omega:\gamma\in \pi_1 E(\dC)\}$. 





\subsection{Genus \texorpdfstring{$g$}{g}}

Let $C$ be a curve of genus $g$ over $\dC$. Then 
\[
  \{\textnormal{holomorphic 1-forms on $C$}\} = \Gamma(C,\Omega^1) .
\]
One definition of the genus of $C$ is that $h^0(\Omega^1) = g$, 
i.e.~$\Gamma(C,\Omega^1)$ is $g$-dimensional. Let $\omega_1,\dots,\omega_g$ be a 
basis. Fix $0\in C(\dC)$. Define a holomorphic map 
$i:C(\dC\to \dC^g/\Lambda$ by 
\[
  x\mapsto \left(\int_0^x \omega_1,\dots,\int_0^x \omega_g\right) .
\]
This is called the Abel-Jacobi map. If $z_i$ are the coordinates on 
$\dC^g/\Lambda$, we have $i^\ast \mathrm{d}z_j = \omega_j$. 

As mentioned in \autoref{sec:gross-ii}, Weil discovered an algebraic analogue 
of this. Let $C$ be a curve of genus $g$ over any field $k$, with chosen 
$0\in C(k)$. Then there exists a $g$-dimensional abelian variety $J$, the 
\emph{Jacobian} of $C$ such that 
\begin{itemize}
  \item For any field $L\supset k$, 
    \[
      J(L) \simeq \picard^0(C_L) = \divisors^0(C_L) / \textnormal{linear equivalence} .
    \]
  \item There is a morphism $i:C\to J$, $x\mapsto [x]-[0]$, which is an 
    embedding if $g\geqslant 1$. 
  \item The pullback map $i^\ast:\Gamma(J,\Omega^1) \to \Gamma(C,\Omega^1)$ is 
    an isomorphism. 
  \item If $k=\dC$, the analytic Abel-Jacobi map factors as 
    \[\xymatrix{
      C(\dC) \ar[r]^-i 
        & J(\dC) \ar[r]^-\sim 
        & \dC^g/\Lambda .
    }\]
\end{itemize}

One of the difficulties of dealing with curves of higher genus is that $C(k)$ 
is not naturally a group. But at least $C\hookrightarrow J$, and $J(k)$ is a 
group. 





\subsection{(Bad) real-analytic approach}

Let $C$ be a curve of genus $g\geqslant 2$ with $0\in C(\dQ)$. Let 
$i:C\hookrightarrow J$. First let's try a (bad) real-analytic approach 
to proving finiteness of $C(\dQ)$. We have a commutative diagram 
\[\xymatrix{
  C(\dQ) \ar[r] \ar[d] 
    & J(\dQ) \ar[d] \\
  C(\dR) \ar[r] 
    & J(\dR) .
}\]
The group $J(\dQ)$ is finitely generated by Mordell-Weil, let $r=\rank J(\dQ)$. 
Since $J$ is projective, $J(\dR)$ is a compact real Lie group, so 
$J(\dR) \simeq \dR^g/\dZ^g\oplus \text{finite}$. Note that 
$C(\dQ)=J(\dQ)\cap C(\dR)\subset J(\dR)$. But typically the group generated by 
some $x\in C(\dR)\subset J(\dR)$ is dense inside $J(\dR)$. Alternatively, the 
closure of $J(\dQ)$ in the classical topology is often open in $J(\dR)$. 





\subsection{\texorpdfstring{$p$}{p}-adic approach}

Chabauty suggested replacing $\dR$ with $\dQ_p$. Again we have a commutative 
diagram 
\[\xymatrix{
  C(\dQ) \ar[r] \ar[d] 
    & J(\dQ) \ar[d] \\
  C(\dQ_p) \ar[r] 
    & J(\dQ_p) .
}\]
We have $C(\dQ)\subset C(\dQ_p)\cap \overline{J(\dQ)}$, where here the closure 
is taken in the $p$-adic analytic topology. 

\begin{theo}[Chabauty 1941]
If $r<g$, then $C(\dQ_p)\cap \overline{J(\dQ)}$ is finite. In particular, 
$C(\dQ)$ is finite. 
\end{theo}

Often, the points in $C(\dQ_p)\cap \overline{J(\dQ)}$ can be approximated 
$p$-adically. 





\subsection{Structure of \texorpdfstring{$J(\dQ_p)$}{J(Qp)}}

For simplicity, assume $C$ has good reduction at $p$. So $C$ extends to a 
smooth proper curve over $\spectrum(\dZ_p)$. Weil's construction of the 
Jacobian works in great generality, so $J=\jacobian C$ also has good 
reduction, i.e.~extends to an abelian scheme over $\spectrum(\dZ_p)$. 

We can understand $J(\dQ_p)$ by looking at a reduction map. First note that by 
the ``valuative criterion for properness,'' $J(\dQ_p)=J(\dQ_p)$, so we have a 
homomorphism $J(\dZ_p) \to J(\dF_p)$, where $J(\dF_p)$ is a finite abelian 
group. Let $U$ be the kernel of this map. By smoothness, the map 
$J(\dZ_p) \to J(\dF_p)$ is surjective, so $J(\dZ_p)$ is a disjoint union of 
$J(\dF_p)$ copies of $U$. For suitable local coordinates $t_1,\dots,t_g$ at 
$0$, we get a $p$-adic analytic isomorphism 
$\boldsymbol t=(t_1,\dots,t_g):U\iso (p \dZ_p)^g$. We conclude that 
$J(\dZ_p)\iso \coprod_{J(\dF_p)} (p \dZ_p)^g$. 

We would like a $p$-adic analogue of the Abel-Jacobi map. For 
$\omega\in \Gamma(J,\Omega^1)$, there is a canonical homomorphism 
$\eta_\omega:J(\dQ_p) \to \dQ_p$ which we write 
\[
  x\mapsto \int_0^x \omega .
\]
If $\omega = \sum w_j(\boldsymbol t) \mathrm{d}t_j$ with 
$w_j\in \dZ_p\llbracket t_1,\dots,t_g\rrbracket$, just ``integrate formally'' 
and evaluate on $(p\dZ_p)^g$. This definition works for $x\in U$. If we require 
$\eta_\omega$ to be a homomorphism, there is a unique extension of 
$\eta_\omega$ from $U$ to $J(\dQ_p)$. If $x\notin U$, there exists some 
$n\geqslant 1$ such that $n x\in U$; then define 
\[
  \int_0^x \omega = \frac 1 n \int_0^{n\cdot x} \omega .
\]

Putting everything together, we get a $p$-adic analytic homomorphism 
$\log:J(\dQ_p) \to \dQ_p^g$ defined by 
\[
  \log x = \left(\int_0^x \omega_1,\dots,\int_0^x \omega_g\right) .
\]
This is a local diffeomorphism. 





\subsection{Consequences}

We know that $J(\dQ_p)\simeq \dZ_p^g\oplus \text{finite}$. If $r<g$, 
$\log J(\dQ)\subset \dQ_p^g$ will be contained in some hyperplane. Therefore 
there is some $0\ne \omega\in \Gamma(J_{\dQ_p},\Omega^1)$ such that 
$\eta_\omega|_{J(\dQ)}=0$. This gives an explicit way of finding a 
``smaller box'' inside $J(\dQ_p)$ in which $C(\dQ)$ fits. 

Just to recap, $C(\dQ_p)\cap \overline{J(\dQ)}$ is a subset of 
$C9\dQ_p)\cap \{\eta=0\}$, the set of zeros of $\int_0^x \omega$ on 
$C(\dQ_p)$. We can write $\omega=w(t)\, \mathrm{d}t$ with 
$w\in \dZ_p\llbracket t\rrbracket$. Then 
\[
  \eta = \int \omega \in \dQ_p\llbracket t\rrbracket .
\]
We have some control on the Newton polygon of $\eta$. If we write 
$\eta = \sum a_i t^i$, plot the points $(i,v_p(i))$. The lower-convex hull 
of these points is the Newton polygon, and from this we can understand the 
valuations of the zeros of $\eta$. In particular, the number of zeros can 
be controlled in terms of $g$. 





\subsection{Main result}

From \cite{bg13}, we know that $\average(\#\selmer_2 J)=3$. This wouldn't be 
sufficient, except that $\selmer_2 J$ carries ``one $2$-adic digit'' of 
information about $J(\dQ)$. We have a commutative diagram 
\[\xymatrix{
  C(\dQ) \ar@{^{(}->}[d] \ar@{^{(}->}[rr] 
    & & C(\dQ_p) \ar[d] \\
  J(\dQ) \ar@{^{(}->}[r] \ar@{->>}[d] 
    & \overline{J(\dQ)} \ar@{^{(}->}[r] \ar@{->>}[d] 
    & J(\dQ_p) \ar@{->>}[r]^-{\log} \ar@{->>}[d] 
    & \dZ_p^g \ar@{->>}[d] \ar@{.>}[dr]^-\rho \\
  J(\dQ)/p \ar@{->>}[r] \ar[dr] 
    & \overline{J(\dQ)}/p \ar[r] 
    & J(\dQ_p)/p \ar@{->>}[r] 
    & \dF_p^g \ar@{.>}[r] 
    & \dP^{g-1}(\dF_p) \\
  & \selmer_p J \ar[ur] 
}\]
We know that $J(\dQ)/p$ lives inside $\selmer_p J$ in $J(\dQ_p)/p$. We will 
concentrate on the case $p=2$. We can compare the image of $\selmer_p J$ and 
$\rho\log C(\dQ_p)$ inside $\dP^{g-1}(\dF_p)$. Bhargava and Gross proved an 
equidistribution result for Selmer elements. 





% !TEX root = sms.tex

\section{Topological and algebro-geometric methods over function fields I}\label{sec:ellenberg-i}
\thanksauthor{Jordan Ellenberg}





I will give a ``sales pitch'' for thinking about these problems in the context 
of global function fields. The idea is that the main problems can be approached 
more geometrically. Some problems in arithmetic statistics are much easier in 
the function field context. 





\subsection{Motivating examples}

\begin{enonce}{Question}
How many integers are there between $N$ and $2 N$?
\end{enonce}

See \autoref{sec:granville-ii} for an interesting (and sophisticated) approach 
to this question. 

\begin{enonce}{Question}
How many squarefree integers are there between $N$ and $2 N$?
\end{enonce}

Call this number $\squarefree(N)$. To be squarefree is to be indivisible by 
$4,9,25,49,\ldots$, i.e.~not divisible by $p^2$ for any prime $p$. One might 
expect ``being indivisible by $p^2$'' to be independent for distinct $p$, so 
\begin{align*}
  \squarefree(N) 
    &\sim N\cdot\left(1-\frac 1 4\right)\left(1-\frac 1 9\right) \cdots \\
    &= N\cdot \prod_p \left(1-p^{-2}\right)^{-1} \\
    &= \zeta(2)^{-1} N .
\end{align*}
So $\lim_{N\to \infty}\frac{\squarefree{N}}{N} = \zeta(2)^{-1}$. This is a 
common phenomenon in arithmetic statistics -- some kind of behavior 
asymptotically occurs an $L$-value percent of the time. 

First, let's understand how this problem looks over more general global fields. 

\begin{defi}
A \emph{global field} is either 
\begin{itemize}
  \item A number field, i.e.~a finite extension of $\dQ$. 
  \item The function field of a curve over a finite field $\dF_q$. 
    (Equivalently, a field isomorphic to a finite extension of $\dF_q(t)$.)
\end{itemize}
\end{defi}

We will be quite loose in identifying a curve over $\dF_q$ and its function 
field, because there is an (anti-)equivalence of categories between smooth 
proper geometrically irreducible (aka ``nice'') curves over $\dF_q$ and 
field extensions of $\dF_q$ of transcendence degree $1$. 





\subsection{The analogy between number fields and function fields}

For a number field $K$, there is a unique embedding $\dQ\hookrightarrow K$. But 
there might be \emph{many} ways to embed $\dF_q(t)$ into a global field $K$. 
For example, $\dF_q(t^{17})\subset\dF_q(t)$ is a ``non-standard'' embedding of 
$\dF_q(t)$ into $\dF_q(t)$. So global function fields do not ``come with'' the 
structure of an extension of $\dF_q(t)$. This phenomenon lies behind the fact 
that Mordell-Weil ranks are unbounded over function fields. (See examples of 
Ulmer.) In this lecture we'll mainly talk about $\dF_q(t)$. 

What is the function-field analogue of counting squarefree integers in a box? 
One problem is that $\dQ$ has only one ``nice'' subring, whereas $\dF_q(t)$ has 
lot of ``nice'' subrings. We'll use the following analogy:
\begin{center}
\begin{tabular}{c|c}
number fields & function fields \\ \hline
$\dQ$ & $\dF_q(t)$ \\
$\dZ$ & $\dF_q[t]$ \\
$|\cdot|:\dZ\to \dR$ & $|f|_\infty = q^{\deg f}$ \\
$[N, 2 N]=\{n\in \dN:|n|\sim N\}$ & set of monic polys with $|f|= N=q^n$ \\
$\#(\dN\cap [N,2 N])\sim N$ & $\#($monic polys with $|f|=N)=N$ \\
of these, $\sim \zeta_\dZ(2)^{-1} N$ are squarefree & $\sim \zeta_{\dF_q[t]}(2)^{-1}$ are squarefree
\end{tabular}
\end{center}

That is, the limiting proportion of squarefree monic polynomials in 
$\dF_q[t]$ is 
\[
  \prod_p \left(1-|p|^{-2}\right) = 1-q^{-1} .
\]
as $p$ ranges over monic irreducible polynomials in $\dF_q[t]$. In fact, the 
number of squarefree monic polynomials of degree $n$ in $\dF_q[t]$ is exactly 
$q^n-q^{n-1}$ for all $n\geqslant 2$, and $q$ for $n=1$. So we have a 
power-saving result with \emph{much} better error term over function fields. So 
we shouldn't think of there being an analogy between any particular number field 
and any particular function field. Rather, there is an analogy between the 
\emph{class} of number fields and the \emph{class} of function fields. 





\subsection{Geometric picture}

What is geometric about what we've done? We introduce yet another function 
field, $\dC(t)$. We can once again think about the set of monic squarefree 
polynomials of degree $n$ in $\dC[t]$. This set is not just a set -- it is a 
\emph{space} (namely an algebraic variety). The space of monic squarefree 
polynomials of degree $n$ is called the (unordered) \emph{configuration space} 
of $\dC$, denoted $\configuration^n \dC$. It parameterizes $n$-tuples of 
distinct points in $\dC$, up to permutation. This isomorphism is given by 
$f\mapsto \{\text{roots of $f$}\}$. The inverse sends an $n$-tuple 
$(z_1,\dots,z_n)$ to the polynomial $f(t)=(t-z_1)\cdots(t-z_n)$. 

We are morally constrained to think of this configuration space not just as 
a complex manifold, but as a scheme over $\spectrum\dZ$. Namely, there is a 
scheme $\configuration^n \dA^1$ over $\spectrum\dZ$ such that 
\[
  (\configuration^n\dA^1)(K) = \{\text{monic squarefree polynomials of degree $n$ in $K[t]$}\}
\]
for any field $K$. In fact, this has a simple description. Namely, the moduli 
space of \emph{all} monic polynomials of degree $n$ is $\dA^n$. A polynomial $f$ 
is squarefree if and only if the discriminant $\Delta(f)$ is nonzero, where 
$\Delta$ is a polynomial in the coefficients of $f$. For example, 
\[
  \Delta(t^2+a_1 t+a_2) = a_1^2 - 4 a_2 .
\]
So $\configuration^n\dA^1$ is $\dA^n\smallsetminus V(\Delta)$. Note: we would 
get a different space if we parameterized ordered $n$-tuples numbers, where 
we care about ordering. We'll call that $\pureconfiguration^n$, the 
\emph{pure configuration space}. The group $S_n$ acts on $\pureconfiguration^n$ 
by permuting the $n$-tuples, and the quotient 
$\pureconfiguration^n/S_n$ is $\configuration_n$. Note that 
$\pureconfiguration^n\dA^1=\dA^n\smallsetminus \bigcup_{i\ne j} V(z_i-z_j)$, 
where $z_1,\dots,z_n$ are the the coordinates of $\dA^n$. 

The set of monic squarefree polynomials in $\dF_q[t]$ of degree $n$ is just 
$\configuration^n\dA^1(\dF_q)$. So our counting problem is: what is 
$\#\configuration^n\dA^1(\dF_q)$? We saw that the answer is $q^n-q^{n-1}$. 

What if we only cared about what happens as $q\to \infty$? For example, what is 
the probability that a degree-$n$ polynomial over $\dF_q[t]$ is squarefree? We 
had been fixing $q$ and letting $n\to \infty$. A simpler question is fixing $n$ 
and letting $q\to \infty$. As $q\to \infty$, 
\[
  \lim_{q\to\infty}\frac{\#\configuration^n\dA^1(\dF_q)}{q^n} 
    = \lim_{q\to \infty}\frac{\#(\dA^n\smallsetminus V(\Delta))(\dF_q)}{q^n} 
    = 1 .
\]





\subsection{M\"obius functions}

\begin{enonce}{Question}
What is the average of the M\"obius function?
\end{enonce}

Recall the \emph{M\"obius function} $\mu$ is the arithmetic function defined by 
\[
  \mu(n) = \begin{cases} 0 & n\text{ is squarefree} \\ 1 & n\text{ the product of an even number of distinct primes} \\ -1 & n\text{ the product of an odd number of distinct primes} \end{cases}
\]
Earlier, we computed that the expected value of $\mu^2$ is 
$\expected(\mu^2)=\zeta(2)^{-1}$. How does this look for function fields? We 
can define the M\"obius function of a polynomial in exactly the same way. But 
over function fields, we have the beautiful 

\begin{enonce}{Fact}
\[
  \mu(f) = (-1)^{\deg f} \left(\frac{\Delta(f)}{q}\right) 
\]
where $\left(\frac{\cdot}{q}\right)$ is the Legendre symbol. 
\end{enonce}

Note that $(-1)^n\mu(f)+1$ is the number of square roots of $\Delta(f)$ in 
$\dF_q$, where $n=\deg f$. Let's make a variety geometrizing this problem. 
Define $Y_n$ to be the space parameterizing pairs $(f,y)$, where $f$ is a monic 
squarefree degree $n$ polynomial, and $y$ is a square root of $\Delta(f)$. Then 
\[
  \#Y_n(\dF_q) = \sum_{f:\deg f=n} \left((-1)^n \mu(f) + 1\right)
\]
So $\#Y_n(\dF_q) - q^n = (-1)^n \sum_f \mu(f)$. We expect 
\[
  \# Y_n(\dF_q) = q^n + o(q^n) .
\]
Why? There is a map $Y_n\to \dA^n$ given by $(f,y)\mapsto f$. This is a 
double branched at the vanishing locus of $\Delta$. In general, we expect an 
$n$-dimensional variety to have approximately $q^n$ points over $\dF_q$. But 
this expectation only is valid when the variety is irreducible. So we think 
$\# Y_n(\dF_q)\sim q^n$ because we think $Y_n$ is irreducible. Indeed, the 
Weil conjectures guarantee that if $Y_n$ is geometrically irreducible, then 
\[
  \lim_{q\to \infty} \frac{\#Y_n(\dF_q)}{q^n} = 1 ,
\]
so the limit as $q\to \infty$ of the average of the M\"obius function is zero. 

But how do we \emph{actually know} that $Y_n$ is irreducible? What if 
$\Delta(a_1,\dots,a_n)$ were actually $G^2$ for some other polynomial $G$? 
Then $\mu(f)$ would be $(-1)^n$ for \emph{all} $f$ of degree $n$. I argue that 
the underlying idea here is a computation of monodromy. 





\subsection{Monodromy}

Recall the $S_n$-Galois cover 
$\pureconfiguration^n \twoheadrightarrow \configuration^n$. The normal subgroup 
$A_n\subset S_n$ corresponds to an intermediate (degree-2) Galois cover 
$U_n\to \configuration^n$. In fact, the following diagram is Cartesian with 
the left arrow being an \'etale cover with group $\dZ/2$:
\[\xymatrix{
  U_n \ar[r] \ar[d] 
    & Y_n \ar[d] \\
  \configuration^n{} \ar@{^{(}->}[r] 
    & \dA^n .
}\]
It is sufficient to show that $U_n$ is irreducible. A $\dZ/2$-cover of 
$\configuration^n$ is a map $\pi_1(\configuration^n)\to \dZ/2$, and the cover 
is irreducible if and only if this map is surjective. Whenever we have a 
``cover of moduli spaces'' $Y\to X$ of degree $n$, we have a map 
$\pi_1(X)\to S_n$. The image of this map is called the \emph{monodromy group} 
of the cover and $Y$ is irreducible if and only if the monodromy group is 
transitive. 

In \autoref{sec:ellenberg-ii}, we'll look at the idea that big monodromy 
implies ``averages are what you expect'' in the large $q$ regime. Sometimes, 
monodromy is not big, and its ``smallness'' can sometimes explain the failure 
of of heuristics. 





\subsection{Computational question}

This is related to the discussion of variation of Mordell-Weil ranks. As 
discussed above, there are $q^n-q^{n-1}$ squarefree onic polynomials of degree 
$n$ in $\dF_q[t]$. For each such $f(t)$, let 
\[
  C_f:y=f(t) 
\]
be the corresponding hyperelliptic curve. Its zeta function has the form 
\[
  \zeta(C_f,s) = \frac{P_f(q^{-s})}{(1-q^{-s})(1-q^{1-s})} ,
\]
where $P_f\in \dZ[X]$ has degree $2 g$, and all its roots have absolute value 
$q^{1/2}$. 

The question is: for how many $f$ does $P_f$ have $q^{1/2}$ as a root. Does 
this proportion look like $q^{\alpha n}$ for some $0<\alpha<1$?





% !TEX root = sms.tex

\section{Counting methods over global fields}\label{sec:wang-ii}
\thanksauthor{Jerry Wang}





d


% !TEX root = sms.tex

\section{The Chabauty method and symmetric powers of curves}\label{sec:park}
\thanksauthor{Jennifer Park}





\subsection{Introduction}

Following Poonen, we say a curve is \emph{nice} if it is smooth, projective, 
and geometrically irreducible. 

\begin{question}
Let $X$ be a nice curve over $\dQ$ of genus $g\geqslant 2$. Find all degree-$d$ 
points on $X$. 
\end{question}

If $x\in X(\overline\dQ)$, we say $x$ has \emph{degree $d$} if 
$[\kappa(x):\dQ]\leqslant d$, where $\kappa(x)=\sO_{X,x}/\fm_x$ is the residue 
field at $x$. We could rephrase the problem as: find 
\[
  \bigcup_{[K:\dQ]\leqslant d} X(K) .
\]
The problem is: the compositum of all degree-$\leqslant d$ extensions of $\dQ$ 
is not a number field, so we can't apply Faltings' theorem to conclude this set 
is finite. 

Throughout, we assume $X$ has an effective divisor of degree $d$. This is not 
harmful, because if $X$ had no such divisor, it would have no points of degree 
$d$. We also assume $X$ has a rational point $0\in X(\dQ)$. 

\begin{question}
Let $X$ be a nice curve over $\dQ$ of genus $g\geqslant 2$. Find all 
$\dQ$-points on $\symmetric^d X = \overbrace{X\times \cdots X}^d / S^d$. 
\end{question}

A point on $\symmetric^d X$ will be a multiset $\{x_1,\dots,x_d\}$; this lies 
in $(\symmetric^d X)(\dQ)$ if and only if the $x_i$ are $\sigma$-conjugates, 
where $\sigma\in \galois(\overline\dQ/\dQ)$ has order $\leqslant d$. We can 
apply a generalized theorem of Faltings, proved in \cite{f94}, to study 
$(\symmetric^d X)(\dQ)$. 

\begin{theo}[Faltings]
Let $A$ be an abelian variety over $\dQ$, $Y\subset A$ a closed subvariety. Then 
there exists finitely many subvarieties $Y_i\subset Y$ such that each $Y_i$ is 
the translate of an abelian subvariety of $A$, and 
\[
  Y(\dQ) = \bigcup Y_i(\dQ) .
\]
\end{theo}

There is a map $j:\symmetric^d X\to J=\jacobian X$ defined by 
$\{x_1,\dots,x_d\}\mapsto [x_1+\cdots + x_d - d(0)]$. The fibers are $\dP^n$ 
for varying $n$. Applying Faltings' theorem to the image of the map $j$, we 
get $(j(\symmetric^d X))(\dQ) = \bigcup Y_i(\dQ)$, 
whence 
\begin{align*}
  (\symmetric^d X)(\dQ) 
    &= \bigcup_{n_i} \dP^{n_i}(\dQ) \cup \bigcup j^{-1}(Y_i(\dQ)) \\
    &= \bigcup_{n_i} \dP^{n_i}(\dQ) \cup \bigcup_{\dim Y_i\geqslant 1} j^{-1}(Y_i(\dQ)) \cup \bigcup_{\dim Y_i=0} j^{-1}(Y_i(\dQ)) .
\end{align*}
Since $\bigcup \dP^{n_1}(\dQ)$ is definitely infinite and 
$\bigcup_{\dim Y_i\geqslant 1} j^{-1}(Y_i(\dQ))$ could be infinite, we will 
count the quantity $\bigcup_{\dim Y_0=0} j^{-1}(Y_i(\dQ))$. The following 
theorem proved in \cite{hs91} is useful. 

\begin{theo}[Harris-Silverman]
Let $X$ be a nice curve over $\dC$. If $\symmetric^2 X$ contains an elliptic 
curve, then $X$ is either hyperelliptic or bielliptic. 
\end{theo}

Here, a \emph{bielliptic curve} is a double cover of an elliptic curve. If 
$X$ is the hyperelliptic $y^2=f$, we get $\dP^1\subset \symmetric^2 X$, and 
$\{(x,\sqrt{f(x)}),(x,-\sqrt{f(x)}):x\in \dQ\}\subset (\symmetric^2 X)(\dQ)$. 

\begin{defi}
The \emph{special set} $\cS(V)$ of a $\dQ$-variety $V$ is the Zariski closure 
of the union of the images of all nonconstant maps $f:G\to V$, where $G$ ranges 
over group varieties defined over $\overline\dQ$. 
\end{defi}

We will try to count $(\symmetric^d X)(\dQ)\smallsetminus \cS(\symmetric^d X)$. 
When $d=1$, we have the following theorem proved in \cite{c85}. 

\begin{theo}[Coleman]
Fix $g\geqslant 2$ and a prime $p$. There is an effectively computable bound 
$N(g,p)$ such that if $X$ is a nice curve over $\dQ$ of genus $g$ with good 
redution at $p$ and with $g>\rank J(\dQ)$, then $\# X(\dQ)\leqslant N(g,p)$. 
\end{theo}

If $p>2 g$, then $\# X(\dQ)\leqslant \# X(\dF_p)+(2 g-2)$. In that case, Stoll 
improved the bound to $\# X(\dF_p)+2r$. If $p>2$, Stoll improved it further 
to $\# X(\dF_p) + \lfloor\frac{2r}{p-2}\rfloor$. 

When $d\geqslant 2$, some things are known. In 1993, Klassen counted points on 
$\symmetric^d X$ away from a divisor of dimension $d-1$, where $X$ has gonality 
$>d$. Here, the \emph{gonality} of a curve $X$ is the minimal degree of a map 
$X\to \dP^1$. 

Baker-Bhargava-Wetherell explicitely found all points on 
$(\symmetric^2 X)(\dQ)$ for $X$ hyperelliptic. 

In 2009, Siksek removed the gonality hypothesis from Klassen's result. 

\begin{theo}[Park]
Let $d\geqslant 1$, $p$ be a prime, and $g\geqslant 2$. Then there exists an 
effectively computable bound $N(g,d,p)$ such that for every nice curve $X$ over 
$\dQ$ of genus $g$ with good reduction at $p$ with $\rank J\leqslant g-d$, 
satisfying (*), then 
\[
  \#\left(\symmetric^d(\dQ)\smallsetminus \cS(\symmetric^d X)\right) \leqslant N(g,d,p) .
\]
\end{theo}

If $\rank J\leqslant 1$, the hypothesis (*) is unnecessary. It is a rather 
technical hypothesis that we won't go into here. 





\subsection{Some applications}

\begin{prop}[Park]
We can take $N(3,3,2)=1539$ for any degree-$7$ odd hyperelliptic curve $X$ 
such that $\rank J\leqslant 1$, with good reduction at $2$. 
\end{prop}

This bound, while not fantastic, is far better than the one from Faltings' 
theorem. 

\begin{prop}[Park]
A positive proportion of hyperelliptic curves with $X$ genus $g\geqslant 3$ 
have no points of $\deg\leqslant 2$-points in $\symmetric^2 X$ outside of the 
special set of $\symmetric^2 X$. 
\end{prop}

For these curves, $(\symmetric^2 X)(\dQ)$ is parameterized by $\dP^1$. 





\subsection{Chabauty's method}

For a more in-depth introduction, see \autoref{sec:poonen-iv}. Let $X/\dQ$ be 
a nice curve with rank $r<g$ and good reduction at $p$. For 
$\omega\in \Gamma(J_{\dQ_p},\Omega^1)$, there is a unique group homomorphism 
$\eta_\omega:J(\dQ_p) \to \dQ_p$ sending $x$ to $\int_0^x\omega$ when this 
integral is defined. It is known that there exists $\omega$ such that 
$\eta_\omega(J(\dQ))=0$. 

On each residue disk $U$, there is a coordinate system in which 
\[
  \omega|_U = \sum w_i(t_1,\dots,t_g)\, \mathrm{d} t_i .
\]
Restricting to the curve $X|_U$, we get 
$\omega|_{X\cap U} = w(t)\, \mathrm{d}t$. Then 
\[
  \#\{x\in U:\eta_\omega(x)=0\} \geqslant \#(X(\dQ)\cap U) .
\]
Consider the restriction $\eta_\omega:(\symmetric^d X)(\dQ_p) \to \dQ_p$, 
given by 
\begin{align*}
  \{x_1,\dots,x_d\} 
    &\mapsto \int_0^{[x_1+\cdots + x_d-d(0)]} \omega \\
    &= \int_0^{[x_1-0]} \omega + \int_0^{[x_2 + \cdots + x_d-(d-1)(0)]}\omega \\
    &= \int_0^{[x_1-0]}\omega + \cdots + \int_0^{[x_d-0]}\omega \\
    &= I(t_1) + \cdots + I(t_d) .
\end{align*}
We can estimate the zeros of this map in terms of the power series $I$. We have 
to assume $r+d\leqslant g$. From this we get $d$ independent power series in 
$K_{x_i}\llbracket t_i\rrbracket$, where $[K_{x_i}:\dQ_p]\leqslant d$, 
vanishing on $(\symmetric^d X)(\dQ)$. 

There is a theory of generalized Newton polygons, partially done by Rabinoff 
and Bernstein. 





% !TEX root = sms.tex

\section{Topological and algebra-geometric methods over function fields II}\label{sec:ellenberg-ii}
\thanksauthor{Jordan Ellenberg}





d


% !TEX root = sms.tex

\section{Future perspectives}
\thanksauthor{Manjul Bhargava}





d


% !TEX root = sms.tex

\section{Exercises}





[list of TAs]





\subsection{Some cohomological computations for the representation \texorpdfstring{$V=\symmetric_2(n)\oplus \symmetric_2(n)$}{V=Sym2(n)+Sym2(n)} of \texorpdfstring{$G=\speciallinear_n$}{G=SLn} and \texorpdfstring{$H=\speciallinear_n/\boldsymbol\mu_2$}{H=SL2/mu2}}

Suppose $G$ is a reductive group with a representation $V$ over a field $k$. 
et $V\gq G=\spectrum k[V]^G$ denote the canonicl quotient. Let 
$f\in (V\gq G)(k)$ be a rational invariant and suppose $G(k^s)$ acts 
transitively on $V_f(k^s)$ with abelian stabilizers. In Gross' talk, we 
learned two obstructions for the existence of a rational element $v\in V(k)$ 
with invariant $f$. In this worksheet, we will make some computations regarding 
these obstructions when the representation $V$ is the space 
$\symmetric_2(n)\oplus\symmetric_2(n)$ of pairs of symmetric bilinear forms and 
when the reductive group is either $\speciallinear_n$ or 
$\speciallinear_n/\boldsymbol\mu_2$ with the action given by 
$g\cdot (A,B)=(\transpose g A g, \transpose g B g)$. 


\subsubsection{Warm up}

Consider the conjugation action of $G=\generallinear(W)$ on $V=\End(W)$. One 
can obtain invariants by taking the coefficients $c_1,\dots,c_n$ of the 
characteristic polynomial. 

\begin{enonce*}[remark]{Exercises}
Show that the ring of polynomial invariants $k[V]^G=k[c_1,\cdots,c_n]$ via the 
following steps (or however you want to). 
\begin{enumerate}
  \item Show that for any $c_1,\dots,c_n$ in $k^s$, there exists some 
    $T\in V(k^s)$ with characteristic polynomial 
    \[
      \det(x\cdot 1-T) = x^{2n+1} + c_1 x^{2n} + \cdots + c_{2n+1} .
    \]
    This shows that there is no relation among the invariants. 
  \item Show that for any $c_1,\dots,c_n$ in $k^s$ such that 
    $f(x)=x^{2n+1} + c_1 x^{2n} + \cdots + c_{2n+1}$ has no repeated roots and 
    for any $T,T'\in V_f(k^s)$, there exists some $g\in G(k^s)$ such that 
    $g T g^{-1} = T'$. This shows that there are no other invariants. 
\end{enumerate}
\end{enonce*}

Next we consider the conjugation action of the subgroup $H=\speciallinear(W)$ 
on $V=\End(W)$. Let $T\in V(k)$ be a regular semisimple operator, that is its 
characteristic polynomial $f(x)$ has no repeated factors. Let $L=k[x]/f(x)$ be 
the associated $k$-vector space of dimension $n$. 

\begin{enonce*}[remark]{Exercises}
\begin{enumerate}
  \item Show that the stabilizer $H_T$ of $T$ is isomorphic to 
    $(\weilres_{L/k}\dG_\multiplicative)^{\norm=1}$, the kernel of the norm 
    map $\weilres_{L/k}\dG_\multiplicative\to\dG_\multiplicative$. 
  \item Show that $\h^1(k,H_T)  \simeq k^\times/\norm(L^\times)$ by taking 
    Galois cohomology of the short exact sequence 
    \[
      1 \to H_f \to \weilres_{L/k}\dG_\multiplicative \to \dG_\multiplicative \to 1 .
    \]
    Note Shapiro's lemma implies that $\h^1(k,\weilres_{L/K} M) = \h^1(L,M)$ 
    for any $\galois(L^s/L)$-module $M$. 
  \item Using the same idea, show that 
    $\h^1\left(k,(\weilres_{L/k}\boldsymbol\mu_2)^{\norm=1}\right) \simeq (L^\times/2)^{\norm=\square}$. 
\end{enumerate}
\end{enonce*}










\bibliographystyle{alpha}
\bibliography{sms}

\end{document}
