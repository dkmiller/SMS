\documentclass[english,letterpaper]{smfart}

\usepackage[francais,english]{babel}
\usepackage{amsmath,amssymb,extarrows,mathrsfs,mathtools,smfthm,sms,stmaryrd}
\usepackage[hyperfootnotes=false]{hyperref}
\usepackage[all]{xy}

\SetSymbolFont{stmry}{bold}{U}{stmry}{m}{n}
\NumberTheoremsIn{subsection}

\author{Henri Darmon}
  \address{McGill University}
  \email{darmon@math.mcgill.ca}
  \urladdr{http://www.math.mcgill.ca/darmon/}
\author{Andrew Granville}
  \address{University of Montreal}
  \email{andrew@dms.umontreal.ca}
  \urladdr{http://www.dms.umontreal.ca/~andrew/}
  
\title{Counting arithmetic objects}
\date{June 23 -- July 4, 2014}

\begin{document}
\frontmatter

\begin{abstract}
This is a collection of talks from the summer school ``Counting arithmetic 
objects'' at the Centre de Recherches Math\'ematiques, cosponsored by the 
MSRI. Talks covered various aspects of the emerging theory of ``arithmetic 
statistics.'' 
\end{abstract}

\subjclass{11G05,11N35, 11N37, 11N45}
\keywords{class groups, elliptic curves, sieves}
\thanks{Notes taken by Daniel Miller}

\maketitle
\tableofcontents
\mainmatter





% !TEX root = sms.tex

\section{Introduction and perspective}\label{sec:bhargava-i}
\thanksauthor{Manjul Bhargava}





\subsection{Motivation}

The main question we are interested in is: given a class $\cC$ of objects ``of 
arithmetic interest,'' how many objects are there in $\cC$, up to isomorphism, 
having bounded invariants? 

\begin{enonce}[remark]{Example}
The following are the main examples we're interested in: 
\begin{center}
\begin{tabular}{c|c}
$\cC$ & invariant \\ \hline
number fields of given degree & discriminant \\
class group elements of number fields of given degree & '' \\
rational points on curves & height \\
elliptic curves weighted by rank & '' \\
$n$-Selmer elements of Jacobians of curves & '' 
\end{tabular}
\end{center}
All of these will be defined precisely later on. 
\end{enonce}

Given such a class of objects of arithmetic interest, how they are distributed 
(asymptotically) with respect to their basic invariants? Beyond the cases of 
degree $2$ number fields and genus $0$ curves, little was known at the 
beginning of the 20th century. 





\subsection{Strategy}

Direct methods of counting arithmetic objects generally fail except in the 
``easy'' cases of degree $2$ number fields and genus $0$ curves. The modern 
approach uses representation theory. We try to find a map 
\[
  \cC/\simeq \hookrightarrow G(\dZ)\backslash V(\dZ) ,
\]
where $G$ is an algebraic group and $V$ is a representation of $G$, both 
defined over $\dZ$. More precisely, one finds such a map that sends the 
invariants of objects in $\cC$ to the ring of fundamental polynomial 
invariants of the action of $G$ on $V$. Good choices of such maps often come 
from algebraic geometry, but we have to work out the theory over $\dZ$. 

\begin{enonce}[remark]{Example}[Gauss]
In his \emph{Disquisitiones}, Gauss constructs a map 
\[
  \left\{\begin{array}{c}\text{ideal classes of (orders } \\ \text{in) quadratic fields with} \\ \text{non-square discriminant}\end{array}\right\}\bigr/\simeq 
  \iso 
  \speciallinear_2(\dZ)\bigl\backslash \left\{\begin{array}{c}\text{integer binary quadratic} \\ \text{forms }a x^2+b x y+c y^2\end{array} \right\} 
\]
The map sends the discriminant of such an ideal class to $b^2-4 a c$ (the 
discriminant of the quadratic form, which is the unique polynomial invariant 
of quadratic forms). If $I=\langle \alpha,\beta\rangle$, then 
the corresponding quadratic form is $\norm(\alpha x+\beta x)/\norm(I)$. 
\end{enonce}

\begin{enonce}[remark]{Example}[Levi, Delone-Faddeev] % 20s, 40s [Granville talk]
Recall that a \emph{cubic ring} is a (commutative, unital) ring structure 
on $\dZ^3$. They constructed a map 
\[
  \{\text{cubic rings}\}/\simeq \iso \generallinear_2(\dZ)\bigl\backslash\left\{\begin{array}{c}\text{integer binary cubic forms} \\ ax^3+b x^2 y+c x y^2+d y^3\end{array}\right\} ,
\]
which ``preserves discriminant.'' The discriminant of a binary cubic form is 
$b c^2-4 a c^3 - 4 b^3 d - 27 a^2 d^2 + 18 a b c d$. If 
$R=\langle 1,\alpha,\beta\rangle$ is a cubic ring, the associated binary cubic 
form is 
$\sqrt{\discriminant(\alpha x+\beta y)/\discriminant(R)} = [R:\dZ[\alpha x+\beta y]]$. 
See \cite{df64} for Delone-Faddeev's original approach. Bhargava's thesis 
\cite{b01} and Wood's thesis \cite{w09} develop things further. See 
\autoref{sec:granville-i} for more details. 
\end{enonce}

\begin{enonce}[remark]{Example} % [Wood II]
There are similar maps 
\[
  \{\text{quartic (resp.~quintic) fields}\}/\sim \to \{\text{representations\ldots}\} .
\]
These are also discriminant-preserving (in all these cases, the discriminant on 
the right is the unique polynomial invariant). See \autoref{sec:wood-ii} for 
more details. 
\end{enonce}

\begin{enonce}[remark]{Example}[Birch, Swinnerton-Dyer] % Shankar
There is a map 
\[
  \left\{\begin{array}{c}\sigma\in E(\dQ)/2:E\text{ of the } \\ \text{form }y^2=x^3+A x+B\end{array}\right\} \hookrightarrow\generallinear_2(\dQ)\backslash \left\{\begin{array}{c}\text{integer binary} \\ \text{quartic forms}\end{array}\right\} .
\]
The program \texttt{mwrank} created by Cremona uses this. In fact, this map 
factors through the $2$-Selmer group $\selmer_2(E)$. Write $E_{A,B}$ for the 
elliptic curve $y^2=x^3+A x+B$. This map sends $E_{A,B}$ to the fundamental 
invariants $I,J$ (of degree $2$, $3$ respectively) of binary quartics. See 
\autoref{sec:shankar-i} and \autoref{sec:shankar-ii} for more. 
\end{enonce}

\begin{enonce}[remark]{Example}[Cremona-Fisher-Stoll]
For $n\in \{3,4,5\}$, there is a map 
\[
  \{\sigma \in E(\dQ)/n:E\text{ of the form }E_{A,B}\} \hookrightarrow G(\dZ)\backslash V(\dZ) ,
\]
where $A,B$ are send to the fundamental invariants $I,J$ on the right-hand 
side. 
\end{enonce}

\begin{enonce}[remark]{Example} % [Gross I, Poonen IV]
There is a non-injective, but still useful map 
\[
  \left\{\begin{array}{c}\text{rational points on odd hyperelliptic} \\ \text{curves }y^2=x^{2 g+1} + a_1 x^{2 g} + \cdots + a_{2g+1} \end{array} \right\} \to \specialorthogonal_{2 g+1}(\dZ) \backslash \symmetric^2(\dZ^{2 g+1}) .
\]
This sends the $a_i$ to the invariants on the right-hand side. See 
\autoref{sec:gross-i} and \autoref{sec:poonen-iv} for more. 
\end{enonce}

\begin{enonce}[remark]{Example} % [Gross II, Ho] -- also see Jack Thorne, [Wood III]
There is a (non-injective) map 
\[
  \left\{\begin{array}{c}\text{rational points on even hyperelliptic} \\ \text{curves } z^2=a_0 x^{2 g+2} + \cdots + a_{2 g+2} y^{2 g+2} \end{array} \right\} \to \text{product of SL's} \backslash\cdots ,
\]
sending the $a_i$ to invariants. See \autoref{sec:gross-ii}, 
\autoref{sec:ho}, \autoref{sec:thorne}, and \autoref{sec:wood-iii} for more. 
\end{enonce}

Once we've found our map from arithmetic objects to orbits, the question 
becomes: how many orbits of $G(\dZ)$ on $V(\dZ)$ are there having bounded 
invariants? Gauss worked this out for binary quadratic forms. Let $h(D)$ be 
the number of $\speciallinear_2(\dZ)$-orbits of integer binary quadratic forms of 
discriminant $D$. 

\begin{theo}[Gauss, Lipschitz, Mertens]
\[
  \sum_{0<-D<X} h(D) \sim \frac{\pi}{18} X^{3/2} .
\]
\end{theo}
\begin{proof}
Gauss shows that every integer binary quadratic form $a x^2+b x y+c y^2$ 
with $D=b^2-4 a c<0$ has a unique $\speciallinear_2(\dZ)$-equivalent form 
satisfying $|b|<a\leqslant c$ or $0<b=a\leqslant c$. 

We apply geometry of numbers to the counting problem 
\[
  \sum_{0<-D<X} h(D) \sim \# \{(a,b,c):0<4 a c-b^2<X\text{ and }|b|<a\leqslant c\} .
\]
Gauss's conjecture tells us that the number of points on the right is 
asymptotically the volume of the region. Proving this is tricky! In fact, the 
conjecture is false in general. If we consider the region 
$R=\{(a,b,c):0\leqslant 4 a c-b^2<X\text{ and }|b|\leqslant a \leqslant c\}$, 
then $R\cap \dZ^3$ contains infinitely points. 
\end{proof}

One can attack the lattice-point counting problem explicitly by writing the 
count as a triple sum $\sum_{a,b,c}$, approximating the sum by a triple 
integral, and keeping track of error terms. This is how Lipschitz and Mertens 
proved the result. Working through this is a good exercise. Davenport developed 
some general principles for bounded regions. He used the principle to reprove 
Gauss' count of binary quadratics, and extended the argument to a count of 
binary cubic forms. This requires knowing explicit inequalities for the region. 

A third approach uses zeta functions (or more generally $L$-functions). 
% [Granville, Taniguchi, Thorne]
Siegel first applied this to binary quaratic forms. Goldfeld-Hoffstein, 
Shintani, and Datskovsky extended these methods. See 
\autoref{sec:granville-ii}, \autoref{sec:taniguchi}, and 
\autoref{sec:thorne} for details. 

% [Shankar]
There is a hybrid method: average over a compact continuum of fundamental 
domains. It doesn't need explicit inequalities, but still uses elementary 
geometry of numbers. The method does adapt to situations where there are more 
than one invariant. For examples, it works on all the above examples. In 
particular, it gives a count of quartic and quintic fields, boundedness of 
average rank of elliptic curves, and produces lots of hyperelliptic curves with 
few rational points. See \autoref{sec:shankar-i} and \autoref{sec:shankar-ii} 
for details. 

What if we replace $\dQ$ with another base field, like a number field or 
function field? Over a function field, one can use algebro-geometric and 
topological methods. Boundedness of average rank was proved by de Jong. 
Ellenberg has proved many other results of this type. Also, the ``hybrid 
method'' works over an arbitrary global field, as is worked out in 
\autoref{sec:wang-ii}. 





% !TEX root = sms.tex

\section{Algebraic groups, representation theory, and invariant theory}
\thanksauthor{Eyal Goren}





This lecture will consist mostly of a review of the basic terminology, as 
well as a little bit of Galois cohomology. An ``official version'' of the 
notes can be found online at 
\url{http://www.math.mcgill.ca/goren/AlgberaicGroups.SMS2014.pdf}. 





\subsection{Algebraic groups}

For us, a \emph{linear algebraic group} is a Zariski-closed subgroup of 
$\generallinear_N(\bar k)$ for some integer $N\geqslant 1$, where $k$ is a 
fixed field of characteristic zero, and $\bar k$ is the algebraic closure of 
$k$. Good general references for linear algebraic groups are the books 
\cite{b91,gw09,h75,s09}. 

\begin{example}
The main example of a linear algebraic group is $G=\generallinear_N$. It 
contains several standard subgroups:
\begin{align*}
  B &= \begin{pmatrix} \ast & \cdots & \ast \\ & \ddots & \vdots \\ & & \ast \end{pmatrix} && \text{``standard Borel''} \\
  U &= \begin{pmatrix} 1 & \cdots & \ast \\ & \ddots & \vdots \\ & & 1\end{pmatrix} && \text{``unipotent radical of $B$''} \\
  T &= \begin{pmatrix} \ast \\ & \ddots \\ & & \ast \end{pmatrix} && \text{``maximal torus''}
\end{align*}
\end{example}

\begin{example}
Let $q$ be a symmetric bilinear form corresponding to the matrix 
$(q_{i j})_{1\leqslant i,j\leqslant N}$. Let 
\[
  \specialorthogonal(q) = \{g\in \generallinear_N: g q \transpose g = q\text{ and }\det g=1\} .
\]
If $q$ is the form 
$q(x_1,\dots,x_{2n+1}) = \frac 1 2 (x_1 x_{2n+1} + x_2 x_{2n} + \cdots + x_{n+1}^2$, 
then a maximal torus consists ``anti-triangular'' matrices with 
$(t_1,\dots,t_n,1,t_n^{-1},\dots,t_1^{-1})$. 
\end{example}

An affine group $G$ is determined by its coordinate ring $\bar k[G]$. The 
group operations $m:G\times G\to G$, $i:G\to G$, $e:1\to G$ correspond via 
the Yoneda lemma to $m^\ast:\bar k[G]\to \bar k[G]\otimes \bar k[G]$, 
$i^\ast:\bar k[G] \to \bar k[G]$, $e^\ast:\bar k[G] \to \bar k$. These give 
$\bar k[G]$ the structure of a Hopf algebra. 

A \emph{homomorphism} $f:G\to H$ of algebraic is a morphism of varieties that 
respects the group structures. It corresponds to a ring 
$f^\ast:\bar k[H] \to \bar k[G]$ that respects the comultiplication.

A \emph{character} of $G$ is a homomorphism 
$f:G\to \generallinear_1 = \dG_\multiplicative$. This corresponds to a 
Hopf-algebra homomorphism $\phi:\bar k[t^{\pm 1}] \to \bar k[G]$. If we let $f$ 
be the image of $t$ under this map, then the fact that $\phi$ respects 
comultiplication tells us that $m^\ast(f) = f\otimes f$. We call such 
elements \emph{grouplike}. Let $\character^\ast(G)$ be the group of characters 
of $G$; we have seen that $\character^\ast(G)$ is in bijection with the set 
of grouplike elements of $\bar k[G]$. 

If $g\in G$, put $\inner g:G\to G$ for the action of $g$ by inner 
automorphisms, i.e.\ $\inner g(x) = g x g^{-1}$. So we have a homomorphism 
$\inner:G\to \automorphism G$. 





\subsection{Non-abelian cohomology}

Let $G$ be a topological group acting continuously on a discrete group $M$. 
Define 
\begin{align*}
  \h^0(G,M) &= M^G = \{m\in M:g m=m\text{ for all }g\in G\} \\
  \h^1(G,M) &= \{\zeta:G\to M\text{ such that }\zeta(a b) = \zeta(a)\cdot a \zeta(b)\} / \sim 
\end{align*}
where $\zeta\sim \xi$ if there exists $m\in M$ such that 
$\zeta(a) = m^{-1} \xi(a) \cdot a m$ for all $a\in G$. The set $\h^0(G,M)$ is 
naturally a group, but $\h^1(G,M)$ is only a pointed set. If 
$0 \to A \to B \to C \to 0$ is an exact sequence of $G$-groups, we get a long 
exact sequence 
\[
  0 \to A^G \to B^G \to C^G \xrightarrow\delta \h^1(G,A) \to \h^1(G,B) \to \h^1(G,C) \to \h^2(G,A) \to \cdots 
\]
with the last map only existing if $A$ is central in $B$. A good source for all 
of this is \cite{s79}. 

We define $\delta$ directly. Given $c\in C^G$, lift $c$ to $b\in B$, and let 
$\zeta=\delta(c)$ be $\zeta(g)=b^{-1} \cdot g b$. One can check that the class 
of $\zeta$ in $\h^1(G,A)$ is well-defined. 





\subsection{Forms}

Suppose $G$ is defined over $k$. A \emph{$k$-form} of $G$ is an algebraic group 
$H$ over $k$ together with an isomorphism $f:G_{\bar k} \iso H_{\bar k}$. Let 
$\Gamma=\galois(\bar k/k)$. For all $\sigma\in G$, we have 
$\sigma f:G_{\bar k}\iso H_{\bar k}$. Then 
$f^{-1}\circ \sigma f\in \automorphism_{\bar k}(G)$. An easy exercise in the 
definitions shows that this is a cocycle. In fact, we have 

\begin{theo}
There is a natural isomorphism of pointed sets 
\[
  \{\text{$k$-forms of $G$}\}/\sim \iso \h^1(\Gamma,\automorphism_{\bar k} G) .
\]
\end{theo}

If $H$ corresponds to $\zeta:G\to M$, then $H(k)=G(\bar k)^\Gamma$, where 
$\Gamma$ now acts by $\tau \cdot g = \zeta(\tau)(\tau(g))$. 

\begin{example}[compact forms]
Let $\Gamma=\galois(\dC/\dR)=\langle c\rangle$. Let 
$\theta(g) = \transpose{\bar g}^{-1}$ be the \emph{Cartan involution}. The 
cocycle $\zeta$ given by $\zeta(c)=\theta$ corresponds to the real form 
$\unitary_N$ of $\generallinear_N(\dC)$. It is defined by 
$\unitary_N(\dR) = \{g\in \generallinear_N(\dC):\theta g=g\}$. 
\end{example}

\begin{theo}
Any (connected) reductive algebraic group $G$ over $\dR$ has a unique compact 
form. 
\end{theo}

All of the groups $\generallinear_n$, $\speciallinear_n$, $\specialorthogonal_n$, 
$\symplectic_{2 n}$,\ldots are reductive. 

\begin{example}
If $G=\dG_\multiplicative$, the compact form is 
$T(\dR)=\{z\in \dC^\times:z \bar z=1\}$. We have 
\[
  T\simeq \specialorthogonal(2) = \left\{\begin{pmatrix} a & b \\ -b & a \end{pmatrix} : a^2+b^2=1\right\} ,
\]
via $\begin{pmatrix} a & b\\ -b & a \end{pmatrix} \mapsto a+b i$. 
\end{example}

If $M$ is a $\Gamma$-module, we often put $\h^i(k,M) = \h^i(\Gamma,M)$. 
Also, if $G$ is an algebraic group defined over $k$, we put 
$\h^i(k,G) = \h^i(\Gamma,G(\bar k))$. 

\begin{example}
Start with the exact sequence 
$1 \to \dG_\multiplicative \to \generallinear_N \to \projectivegenerallinear_N \to 1$. 
The long exact sequence in cohomology is 
\[
  1 \to k^\times \to \generallinear_N(k) \to \projectivegenerallinear_N(\bar k)^\Gamma \to \h^1(k,\dG_\multiplicative) \to \h^1(k,\generallinear_N) \to \h^2(k,\dG_\multiplicative) .
\]
The famous \emph{Hilbert Theorem 90} tells us that 
$\h^1(k,\generallinear_N) = 1$ for all $N\geqslant 1$, so we get 
\[
  \projectivegenerallinear_N(\bar k)^\Gamma = \projectivegenerallinear_N(k) = \generallinear_N(k)/\dG_\multiplicative(k) .
\]
Moreover, $\h^1(k,\projectivegenerallinear_N)\hookrightarrow \h^2(k,\dG_\multiplicative)$. We call 
$\brauer(k)=\h^2(k,\dG_\multiplicative)$ the \emph{Brauer group} of $k$. 
Since 
$\projectivegenerallinear_N(\bar k) = \automorphism_{\bar k\text{-}\mathsf{Alg}}(M_N(\bar k))$, 
we see that $\h^1(k,\projectivegenerallinear_N)$ classifies $k$-forms of 
$M_N(\bar k)$, i.e.~central simple algebras over $K$ of rank $N^2$. 
\end{example}





\subsection{Jordan decomposition}

Any $g\in \generallinear_N(\bar k)$ has a unique decomposition 
$g=g_\simple g_\unipotent$, where $g_\simple$ is simple (i.e.~diagonalizable), 
$g_\unipotent$ is unipotent, and 
$g_\simple g_\unipotent = g_\unipotent g_\simple$. One has 
$(g_\unipotent - 1)^N=0$. For example, in the two-dimensional case, a matrix 
$\begin{pmatrix} t_1 & u \\ & t_2\end{pmatrix}$ is already diagonalizable if 
$t_1\ne t_2$, or $u=0$. If neither of those occur, we write it as 
\[
  \begin{pmatrix} t & u \\ & t \end{pmatrix} = \begin{pmatrix} t \\ & t\end{pmatrix} \begin{pmatrix} 1 & u/t \\ & 1 \end{pmatrix} .
\]
The Jordan decomposition enjoys very strong rigidity properties. Namely, 
if $g\in G\subset \generallinear_N$, then also $g_\simple\in G$ and 
$g_\unipotent \in G$. If $f:G\to H$ is a homomorpism of algebraic groups, then 
we have $f(g_\simple) = f(g)_\simple$ and $f(g_\unipotent) = f(g)_\unipotent$. 





\subsection{Tori}

A \emph{torus} $T$ is a form of $\dG_\multiplicative^N$ for some $N$. Tori of 
rank $N$ over $k$ are classified by 
$\h^1(\Gamma,\automorphism_{\bar k}(\dG_\multiplicative^N)) = \hom(\Gamma,\generallinear_N(\dZ))/\text{conj}$. 

If $T$ is a orus, then its group of characters 
$\character^\ast(T_{\bar k}) \simeq \character^\ast(\dG_\multiplicative^N) = \hom(\dG_\multiplicative^N,\dG_\multiplicative) = \dZ^N$ 
has a continuous action of $\Gamma$, via 
$\sigma \chi(g) = \sigma(\chi(\sigma^{-1}(g)))$. Tori are completely classified 
by this action. 

\begin{theo}
The functor $\character^\ast$ induces an anti-equivalence of categories 
\[
  \{\text{tori over }k\} \iso \{\text{finite free $\dZ$-modules with continuous $\Gamma$-action}\} .
\]
\end{theo}

A linear action of a torus $T$, namely $f:T\to \generallinear_N$, is 
simultaneously diagonalizable (every element of $T$ is semi-simple). This 
follows from rigidity properties of the Jordan Decomposition. 

All maximal tori in a linear algebraic group $G$ are conjugate. The common 
dimension of these tori is called the \emph{rank} of $G$, and written 
$\rank_{\bar k}(G)$. 

\begin{example}
The standard torus of diagonal matrices $T\subset\generallinear_N$ is 
maximal, so $\rank(\generallinear_N) = N$. 
\end{example}





\subsection{Solvable groups}

An algebraic group $G$ is called \emph{solvable} if it is solvable ``in the 
usual sense.'' In other words, there exists a filtration 
$1=G_0\subset G_1\subset \cdots \subset G_l = G$ such that each 
$G_i$ a normal algebraic subgroup of $G_{i+1}$, and each 
$G_{i+1}/G_i$ is abelian. The standard Borel $B\subset \generallinear_N$ is 
solvable. In fact, $B$ is a maximal solvable subgroup. 

\begin{theo}[Kolchin-Lie]
If $G\subset \generallinear_N$ is solvable, then $G$ can be conjugated 
into the standard Borel $B\subset \generallinear_N$. 
\end{theo}
\begin{proof}
One uses the fact that if $G$ acts on a projective space, then it has a fixed 
point. The standard representation $G\to \generallinear_N$ gives an action of 
$G$ on $\dP^{N-1}$, and a fixed point for $G$ in $\dP^{N-1}$ gives a line fixed 
by the action of $G$. 
\end{proof}

\begin{theo}[Borel]
If a solvable group $G$ acts on a proper variety, then $G$ has a fixed point. 
\end{theo}

For a group $G$, let $\rad(G)$ be the maximal connected normal solvable 
subgroup of $G$, and let $\rad_\unipotent(G)$ be the maximal connected normal 
unipotent subgroup of $G$. We say that $G$ is \emph{semisimple} if 
$\rad(G)=1$, and \emph{reductive} if $\rad_\unipotent(G)=1$. Clearly 
semisimple groups are reductive. The groups 
$\speciallinear_n$, $\specialorthogonal_n$, $\symplectic_n$ are semisimple 
and $\generallinear_n$, $\generalsymplectic_{2 n}$, $\generalspin_n$ are 
reductive. 

For any $G$, the quotient $G/\rad_\unipotent(G)$ is reductive, and 
$G/\rad(G)$ is semisimple. If $G$ is reductive, then $G/Z(G)$ is 
semisimple. A \emph{Levi subgroup} of a group $G$ is a subgroup $H$ such 
that $G=H\ltimes \rad_\unipotent (G)$. Such an $H$ will be a maximal 
reductive subgroup of $G$. 

A maximal connected solvable subgroup of $G$ is called a \emph{Borel subgroup}. 
The group $B$ of upper-triangular matrices is a borel subgroup of 
$\generallinear(n)$. Every torus is contained in a Borel subgroup, and if $G$ 
is a reductive group, then all Borel subgroups of $G$ are conjugate. 

A group $G$ over $\dC$ is reductive if and only if every representation 
$\rho:G\to \generallinear_N$ is semi-simple (a direct sum of irreducible 
representations). Alternatively, the ring $\dC[\rho(G)]$ should be semi-simple. 

\begin{example}
The group $\simeq U_1 = \begin{pmatrix} 1 & \ast \\ & 1 \end{pmatrix}$ 
is not reductive. 
\end{example}





\subsection{Parabolic subgroups}

A subgroup $P$ of a connected algebraic group $G$ is called \emph{parabolic} if 
the quotient $G/P$ is projective. The basic theorem is that $P$ is parabolic if 
and only if $P$ contains a Borel subgroup $B$. So in $\generallinear_N$, a 
subgroup is parabolic if it contains a conjugate of the subgroup of 
upper-triangular matrices. 

\begin{example}
Let $k$ be an algebraically closed field. Recall that a \emph{flag} in 
$k^n$ is a collection of subspaces 
$F=(0\subsetneq F_1\subsetneq \cdots \subsetneq F_a=k^n)$. The \emph{type} of 
$F$ is $\boldsymbol d=(\dim F_i)_i$. The space of type $\boldsymbol d$ is a 
projective variety $\flag_{\boldsymbol d}$ on which $\generallinear_n$ acts 
transitively. Let $P$ be a stabilizer of a flag. Then 
$G/P\simeq \flag_{\boldsymbol d}$ and $P$ is parabolic. For example, if 
$F_i$ is the span of $\{e_1,\dots,e_{d_i}\}$, then 
\[
  P = \begin{pmatrix} 1_{h_1} \end{pmatrix} [finish]
\]
\end{example}

[finish]





% !TEX root = sms.tex

\section{Basic algebraic number theory}\label{sec:3}
\thanksauthor{Eknath Ghate}





Essentially, this lecture will try to cover two semester-long courses 
(algebraic number theory and class field theory) in an hour. Hopefully, we'll 
focus on the theory of Hilbert class fields, and later on complex 
multiplication. 





\subsection{Number fields}

A \emph{number field} $K$ is a finite field extension of $\dQ$. An important 
invariant of $K$ is its \emph{class group}, $\class_K=I_K/P_K$, where 
$I_K$ is the group of all fractional ideals in $K$ and $P_K$ is the group of 
principal fractional ideals. 

\begin{theo}
The group $\class_K$ is finite. 
\end{theo}

So we can define the \emph{class number} $h_K$ of $K$ to be the cardinality 
of $\class_K$. The class number $h_K=1$ if and only if $\cO_K$ is a principal 
ideal domain. The famous \emph{Dirichlet unit theorem} says that $\cO_K^\times$ 
is a finitely-generated abelian group, and gives a formula for 
$\rank\cO_K^\times$ in terms of the number of real and complex places of $K$. 

We would like to relate $h_K$ with quadratic forms. Consider towers of 
fields 
\[\xymatrix@=0.5cm{
  & L \\
  F \ar@{-}[ur] & & K \ar@{-}[ul] \\
  & \dQ \ar@{-}[ul]^-2 \ar@{-}[ur]_-3 \ar@{-}[uu]^-{S_3} 
}\]
where $L/F$ is an unramified (cubic) extension. Then counting the number of 
cubic fields $K$ that are nowhere ramified with $|\discriminant K|<X$ is 
equivalent to summing $\# h_3(F)$ for $|\discriminant F|<X$. The average of 
$h_3(F)$ is $\frac 4 3$ for $F$ real quadratic, and $2$ for $F$ imaginary 
quadratic. By the ``average'' we mean, for example, 
\[
  \sum_{0<|\discriminant F|<X} h_3(F) \sim \frac 4 3 X 
\]
in the case of real quadratic $F$. The point here is that averaging class 
numbers is equivalent to counting certain types of fields. 





\subsection{Starting point of class field theory}

Class field theory is, in general, the study of abelian extensions of a field 
$k$. Let $K$ be a number field, and let $H/K$ be the maximal unramified 
abelian extension of $K$. One calls $H$ the \emph{Hilbert class field} of 
$K$. It is known that $H$ is a number field. Moreover, the \emph{Artin map} 
$I_K \to \galois(H/K)$ determined by $\fp\mapsto \frobenius_\fp = (\fp,H/K)$ 
induces an isomorphism $\class_K\iso \galois(H/K)$. Recall that the 
\emph{Frobenius element} $\frobenius_\fp\in \galois(H/K)$ is characterized by 
$\frobenius_\fp(x)\equiv x^{N_{K/\dQ}(\fp)}\pmod\fP$ for all $\fP\mid \fp$ in 
$H$. 

It is easy to show that $\fp$ splits compltely in $H$ if and only if $\fp$ is 
principal. Somewhat harder is the principal ideal theorem: 

\begin{theo}
Every ideal of $K$ becomes principal in $H$. 
\end{theo}
\begin{proof}
Let $G=\galois(K^\mathrm{ur}/K)$, and let $G_1\subset G$ correspond to the 
extension $H/K$. We have a commutative diagram: 
\[\xymatrix{
  \class_H \ar[r]^-{\artin} 
    & G_1^\abelian \\
  \class_K \ar[u] \ar[r]^-{\artin}
    & G^\abelian \ar[u]_-V 
}\]
where $V$ is the \emph{transfer map}. Since $G_1$ corresponds to $H$, 
$V=0$, whence $\class_K\to \class_H$ is the trivial map. The result 
follows. 
\end{proof}

If $K=\dQ$, then $H=\dQ$. 

Assume from now on that $K=\dQ(\sqrt d)$ is an imaginary quadratic field 
($d<0$). There are nine possible values of $d<0$ for which 
$h_{\dQ(\sqrt d)}=1$, namely 
\[
  -1,-2,-3,-7,-11,19,-43,-67,-163 .
\]
It is known that $h_K\to \infty$ as $d\to -\infty$. 

\begin{enonce}[remark]{Example}
Let $K=\dQ(\sqrt{-23})$. Then $H/K$ is cyclic of degree $3$. It is known 
that $H=K(\alpha)$, where $\alpha^3-\alpha+1=0$. 
\end{enonce}





\subsection{Complex multiplication}

One might ask if for any number field $K$, there is an explicit way of 
finding $\alpha\in \overline\dQ$ such that $H=K(\alpha)$. The theory of complex 
multiplication describes how to do this explicitely whenever $K$ is 
imaginary quadratic. The main theorem is the following: 

\begin{theo}
Let $E$ be an elliptic curve with CM by $\cO_K$. Then 
$H=K(j(E))$. 
\end{theo}

Recall that $E$ has \emph{complex multiplication} by $\cO_K$ if 
$\End E \simeq \cO_K$. If $E+y^2=4 x^3-g_2 X-g_3$, then its 
\emph{$j$-invariant} is given by 
\[
  j(E) = 1728 \frac{g_2^3}{g_2^3 - g_3^2} .
\]
Alternatively, if $E=\dC/\langle 1,\tau\rangle$, then 
$j(E) = \frac 1 q + 744 + 196884 q^2 + \cdots$, where $q=e^{2\pi i \tau}$. 

The rest of this lecture will be a sketch of a proof of this theorem. 





\subsection{Elliptic curves from fractional ideals}

Let $\cE_\dC(K)$ be the set of isomorphism classes of elliptic curves over 
$\dC$ with complex multiplication by $\cO_K$. There is a bijection 
$\class_K \iso \cE_\dC(K)$. Given $\fa\subset \cO_K$, we have a canonical 
embedding $\fa\hookrightarrow \dC$. Send $[\fa]$ to the elliptic curve 
$\dC/\fa$. This gives us a simply transitive action of $\cO_K$ on 
$\cE_\dC(K)$ via $[\fa]\cdot (\dC/\fb) = \dC/(\fa^{-1} \fb)$. 





\subsection{Fields of definition}

It is known that CM elliptic curves have rational models. In other words, 
the natural map $\cE_{\overline\dQ}(K) \to \cE_\dC(K)$ induced by an 
embedding $\overline\dQ\hookrightarrow\dC$ is a bijection. We know this 
because $E$ is \emph{always} defined over $\dQ(j(E))$. But when $E$ is CM, 
$j(E)\in \overline\dQ$, so $E$ is defined over $\overline\dQ$. Indeed, if 
$\sigma\in \automorphism\dC$, then $j(E^\sigma)$ also has CM by $K$. So the 
orbit of $E$ under $\automorphism(\dC)$ lies in $\cE_\dC(K)$, a finite set. 
This tells us that $\cE_{\overline\dQ}(K) \twoheadrightarrow \cE_\dC(K)$. We 
leave injectivity as an exercise. 

Because of this, we will write $\cE(K)$ instead of $\cE_{\overline\dQ}(K)$ or 
$\cE_\dC(K)$. 





\subsection{Towards \texorpdfstring{$H$}{H}}

Fix $E\in \cE(K)$. For each $\sigma\in \galois(\overline\dQ/K)$, there is a 
unique $[\fa]\in \class_K$ such that $E^\sigma = [\fa]\cdot E$. We define a map 
$F:\galois(\overline\dQ/K) \to \class_K$ by $F(\sigma) = [\fa]$. 

\begin{prop}
1. $F$ does not depend on the choice of $E$. 

2. $F$ is a homomorphism. 
\end{prop}
\begin{proof}
1. This is subtle. 

2. Let $\sigma,\tau\in \galois(\overline\dQ/K)$. Then 
\begin{align*}
  F(\sigma\tau) E 
    &\simeq E^{\sigma\tau} \\
    &= (E^\sigma)^\tau && \text{part 1}\\
    &= (F(\sigma) E)^\tau \\
    &= F(\tau)(F(\sigma) E) \\
    &= (F(\sigma)F(\tau)) E && \text{$\class_K$ is abelian}
\end{align*}
\end{proof}

Let $L$ be the fixed field of $\ker(F)$. We claim that 
$L=K(j(E))$. Indeed, 
\begin{align*}
  \galois(\overline\dQ/L)
    &=\{\sigma\in \galois(\overline\dQ/K):E^\sigma \simeq F(\sigma) E \simeq E\} \\
    &= \{\sigma:j(E)^\sigma = j(E)\} .
\end{align*}

Note that $F$ is an injection $\galois(L/K)\hookrightarrow \class_K$. So 
$L/K$ is abelian. We will show that it is unramified of the right degree. 

Let $\fm$ be the conductor of $L/K$. It is the greatest common divisor of all 
$\fm\subset \cO_K$ such that $K_{\fm,1}\subset \ker(\artin_{L/K})$, where 
$K_{\fm,1} = \langle (\alpha):\alpha\simeq 1\pmod\fm\rangle$. One checks that 
the composite 
\[\xymatrix{
  I_K^\fm \ar[r]^-{\artin} 
    & \galois(L/K) \ar[r]^-F 
    & \class_K 
}\]
is the ``identity map'' $\fa\mapsto [\fa]$. It follows that $F$ is surjective. 
So $F:\galois(L/K)\iso \class_K$, and $L=K(j(E))$. 

If $F(((\alpha),L/K)) = 1$ then $(\alpha)\in I_K^\fm$ is principal. Indeed, it 
suffices to show $((\alpha),L/K)=1$, and this follows from the injectivity of $F$. 
Since $\alpha$ was arbitrary, $\fm=1$, so $L/K$ is unramified. The rest is a 
simple dimension argument. 

\begin{theo}
If $E$ is CM, then $j(E)$ is an algebraic integer. 
\end{theo}

As a corollary, one has the surprizing fact that $e^{\pi\sqrt{163}}$ is 
very close to an integer. The main theorem of complex multiplication has an 
analogue for ray class fields. One gets 
$\operatorname{RCF}(\fm) = K(j(E),h(E[\fm]))$, where $\fm\subset \cO_K$ and $h$ 
is the Weber function. 





% !TEX root = sms.tex

\section{Basics of binary quadratic forms and Gauss composition}\label{sec:granville-i}
\thanksauthor{Andrew Granville}

An official Beamer version of these notes can be found online at 
\url{http://www.crm.umontreal.ca/sms/2014/pdf/AlgTalkSlides.pdf}. 










\backmatter
\bibliographystyle{smfalpha}
\bibliography{sms}

\end{document}
