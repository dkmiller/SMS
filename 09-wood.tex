% !TEX root = sms.tex

\section{Cubic rings}\label{sec:wood-i}
\thanksauthor{Melanie Matchett-Wood}





Recall that the goal of this conference is ``counting arithmetic 
objects.'' The first sort of objects we might try to count are number fields. 
To do this, we would like a parameterization of all number fields of a given 
degree. For degree $2$, this is easy. All quadratic fields are of the form 
$\dQ(\sqrt D)$, where $D$ is a square-free integer. 
In general, a degree $n$ number field is of the form $\dQ(\theta)$. Let $f$ 
be the minimal polynomial of $\theta$; it will be of the form 
$f(x)=x^n+a_{n-1} x^{n-1} + \cdots + a_0\in \dZ[x]$. The problem is: determining 
whether two monic irreducible polynomials yield the same field is quite 
difficult. 

Rather than counting number fields, we will try to count their maximal orders 
instead. 





\subsection{Some definitions}

\begin{defi}
A \emph{rank-$n$ ring} is a (commutative, unital) ring $R$ such that 
$R\simeq \dZ^n$ as a $\dZ$-module. 
\end{defi}

For $n=2,3,4,5$, we call these rings quadratic, cubic, quartic, and quintic. 
Typical cubic rings are maximal orders $R=\cO_K$ in cubic number fields $K$. 
But non-maximal orders in cubic number fields are also cubic rings. For 
example, $\dZ[3\sqrt[3] 2]\subset \dQ(2^{1/3})$ is a perfectly good cubic 
ring, with $\dZ$-basis $\{1,3\sqrt[3] 2,9\sqrt[3] 4\}$. A still more 
pathological example is $\dZ[X]/X^3$, which has $\dZ$-basis $\{1,X,X^2\}$. If 
$R$ is any quadratic ring, then $\dZ\times R$ is another ``pathological'' cubic 
ring. 

In the end, we'll count number fields by counting their maximal orders, which 
we will count by including them into a larger class of rank-$n$ rings. 

\begin{defi}
Let $R$ be a rank-$n$ ring. The \emph{trace map} $\trace:R\to\dZ$ is 
defined by $\trace(r) = \trace(\cdot r:R\to R)$. 
\end{defi}

If $\alpha_1,\dots,\alpha_n$ is a $\dZ$-basis of $R$, then the 
\emph{discriminant} of $R$ is 
$\discriminant(R) = \det(\trace(\alpha_i \alpha_j)_{i,j})$. 





\subsection{Quadratic rings}

We know that quadratic rings (up to isomorphism) are in bijection with 
$\{D\in \dZ:d\equiv 0,1\pmod 4\}$. Given a quadratic ring $R$, the 
corresponding integer is $D=\discriminant(R)$. (It is an old theorem that 
$\discriminant(R)$ is a square modulo $4$.) Given such an integer $D$, 
the corresponding ring is 
\[
  \dZ[\tau]/\left(\tau^2 - D\tau + \frac{D^2-D}{4}\right)
\]
If, for example $D=0$, the corresponding ring is $\dZ[\tau]/\tau^2$. 

From our parameterization of quadratic rings, it is easy to count them! 





\subsection{Cubic rings}

Let $R$ be a cubic ring with $\dZ$-basis $\{1,W,T\}$. A key invariant is 
$W T = q+r W + s T$ for some $q,r,s\in \dZ$. Choose a new 
$\dZ$-basis $\{1,\omega,\theta\}$ with $\omega=W-s$ and $\theta=T-r$. In our 
new basis, $\omega\theta = n$ for some $n\in \dZ$. We call such a basis 
\emph{normalized}. Every basis of $R/\dZ$ has a unique normalized lift to 
$R$. 

We also write $\omega^2=m-b \omega + a\theta$ and 
$\theta^2 = \ell-d\omega+c\theta$. The fact that multiplication on $R$ is 
associative forces some conditions on $\{m,b,a,\ell,d,c\}$. The 
associativity relations are exactly: 
\begin{align*}
  n &= - a d \\
  \ell &= - b c \\
  m &= - a c .
\end{align*}
This induces a bijection between isomorphism classes of cubic rings with 
choice basis of $R/\dZ$ and quadruples $(a,b,c,d)\in \dZ^4$. Clearly 
$\generallinear_2(\dZ)$ acts on cubic rings with basis of $R/\dZ$; orbits of 
this action are isomorphism classes of cubic rings. The only thing here that 
is mysterious is the $\generallinear_2(\dZ)$-action on $\dZ^4$. 

If we write $f(x,y) = a x^3 + b x^2 y + c xy^2 + d y^3$ and 
$g\in \generallinear_2(\dZ)$, then 
\[
  (g f)(x,y) = \frac{1}{\det(g)} f\left(\begin{pmatrix} x & y \end{pmatrix} \cdot g\right) .
\]
In other words, if we identify $\dZ^4$ with the space of cubic forms in 
two variables, then the action of $\generallinear_2(\dZ)$ is the natural one 
(with a twist by $\det^{-1}$). 
This parameterization of cubic rings goes back to \cite{df64}. It was also 
used in \cite{dh69}, and saw its first modern formulation in 
\cite{ggs02}. 

\begin{example}
The cubic ring corresponding to $(a,b,c,d)=(0,0,0,0)$ is pretty pathological -- 
namely $\dZ[\omega,\theta]/(\omega,\theta)^2$. 
\end{example}

[\ldots wasn't able to take notes to the end\ldots]




