% !TEX root = sms.tex

\section{Geometric properties of curves}
\thanksauthor{Henri Darmon}




Here we treat those properties of curves which hold over an arbitrary field. 
Later on, in \autoref{sec:7}, we will specialize to number fields. 





\subsection{Motivation}

Throughout, $k$ is a field. For simplicity we assume $k$ has characteristic 
zero. 

\begin{defi}
A \emph{curve} over a field $k$ is a smooth geometrically connected variety of 
dimension one over $k$. 
\end{defi}

Concretely, we think of equations like 
\begin{align*}
  1 &= x^2 - D y^2 \\
  y^2 &= x^3 + a x+b \\
  z^n &= x^n + y^n .
\end{align*}
The difference between the general definition and these concrete 
examples should be seen as analogous to the difference between the notion of 
an ``abstract vector space'' and concrete examples $\dR^n$. 

The key tool for passing from an abstract curve to a concrete representation is 
the \emph{Riemann-Roch Theorem}. Let $X$ be a proper curve. Zariski-open 
subsets of $X$ are of the form $U=X\smallsetminus \{p_1,\dots,p_s\}$, where 
the $p_i\in X(\bar k)$ and $\{p_1,\dots,p_s\}$ is stable under the action of 
$G_k=\galois(\bar k/k)$. We have a sheaf $\sO=\sO_X$ of ``regular functions'' 
on $X$. For $U\subset X$, the ring $\sO(U)$ consists of all regular functions 
$U\to \dA^1$. 

Our goal is to understand $\sO(U)$ as a ring. Ideally, we would like to 
write $\sO(U)=k[f_1,\dots,f_n]/(p_1,\dots,p_m)$. In other words, we want 
sections $f_1,\dots,f_n\in \sO(U)$ such that $(f_1,\dots,f_n)$ induces an 
embedding $U\hookrightarrow \dA^n$. We certainly can't do this with $U=X$, 
because $\sO(X)=k$. If $k=\dC$, this fact is known as Liouville's theorem, but 
it holds for arbitrary $k$. 





\subsection{Crude form of Riemann-Roch}

We assume there is a point $\infty\in X(k)$. Put 
$U=X\smallsetminus \{\infty\}$. Define $\sO(U;n\infty)$ to be the set of 
functions $f\in \sO(U)$ such that $v_\infty(f)\geqslant -n$. This gives us a 
filtration $\sO(X)\subset \sO(U;\infty)\subset \sO(U;2\infty)\subset \cdots$ 
and $\bigcup \sO(U,n\infty) = \sO(U)$. Moreover, 
each successive quotient is at most one-dimensional, so 
$\dim \sO(U,n\infty) \leqslant n+1$. The Riemann-Roch Theorem gives a lower 
bound for $\dim \sO(U,n\infty)$. 

\begin{theo}[Riemann-Roch; crude form]
Let $X$ be a proper curve over $k$ and $\infty\in X(k)$. Then there is an 
integer $g\geqslant 0$, depending only on $X$, such that 
\[
  \dim \sO(U;n\infty) \geqslant n+1-g 
\]
with equality if $n\gg 0$. 
\end{theo}
\begin{proof}[Idea of proof]
Choose a local parameter $t$ at $\infty$. Define the \emph{principal part} 
$\principalpart_\infty:\sO(U;n\infty) \to t^{-n} k[t] / k[t]$ in the obvious 
way. What are the obstructions to producing $f$ with given principal part at 
$\infty$? The only obstruction comes from the Residue Theorem stated below. As 
a corollary, if $\omega\in \Omega^1(X)$ is a global regular differential, and 
if $f\in \sO(U)$, then $\residue_\infty(f\omega) = 0$. So global regular 
differentials provide obstructions to constructing $f$. Define 
$\residue_\infty:t^{-n} k[t]/k[t] \to \Omega^1(X)^\vee$ by 
$\residue_\infty(f)(\omega) = \residue_\infty(f\omega)$. We have a 5-term 
sequence 
\begin{equation*}
  0 \to k \to \sO(U;n\infty) \xrightarrow{\principalpart_\infty} t^{-n} k[t]/t \xrightarrow{\residue_\infty} \Omega^1(X)^\vee \to \Omega^1(X;-n\infty)^\vee \to 0.
\end{equation*}
The last term needs explanation. Define $\Omega^1(X;-n\infty)$ to be the set of 
$\omega\in \Omega^1(X)$ such that $v_\infty(\omega) \geqslant n$. The inclusion 
$\Omega^1(X;-n\infty)\hookrightarrow \Omega^1(X)$ induces the surjection in the 
sequence. 

One can check that the sequence is exact. For $n\gg 0$, 
$\Omega^1(X;-n\infty)=0$. We have in fact proved the ``more precise form'' 
below. 
\end{proof}

\begin{theo}[residue theorem]
If $\omega$ is a meromorphic differential on $X$, then 
$\sum_{x\in X} \residue_x(\omega) = 0$. 
\end{theo}

Let's recall what this means At each $x\in X(\bar k)$, choose a 
uniformizing parameter $t$ at $x$. One can write locally 
$\omega = (a_{-m} t^{-m} + \cdots) dt$; put $\residue_x(\omega) = a_{-1}$. 
Surprizingly, this does not depend on our choice of $t$. 

\begin{theo}[Riemann-Roch; more precise form]
Let $g=\dim \Omega^1(X)$. Then 
$\dim \sO(U;n\infty) - \dim\Omega^1(X;-n\infty) = n+1 - g$. 
\end{theo}

We call the integer $g$ the \emph{genus} of $X$. 





\subsection{Some vocabulary}

A \emph{divisor} of $X$ is a formal finite linear combination of points in 
$X(\bar k)$ with integer coefficients. So a typical divisor looks like 
\[
  D = \sum_{x\in X(\bar k)} n_x\cdot x ,
\]
where each $n_x\in \dZ$ and $n_x=0$ for all but finitely many $x$. Write 
$\divisors(X_{\bar k})$ for the (abelian group) of divisors on $X$, and 
write $\divisors(X)=\divisors(X_{\bar k})^{G_k}$. 

Recall the field of \emph{rational functions} on $X$ is 
$k(X)=\varinjlim_U \sO(U)$. Define $\divisor:k(X)^\times \to \divisors(X)$ 
by 
\[
  \divisor(f) = \sum_{x\in X(\bar k)} v_x(f)\cdot x .
\]
Divisors of the form $\divisor(f)$ are called \emph{principal divisors}. 
If $D_1,D_2$ are divisors, write $D_1\geqslant D_2$ if 
$n_x(D_1)\geqslant n_x(D_2)$ for all $x\in X(\bar k)$. For an arbitrary 
divisor $D$, define 
\[
  \sL(D) = \{f\in k(X):\divisor(f)\geqslant - D\} .
\]
Our space $\sO(U;n\infty)$ earlier is just $\sL(n\infty)$. Define the space of 
\emph{meromorphic divisors} by 
$\Omega_\meromorphic^1(X)=\varinjlim_U \Omega^1(U)$; this is a one-dimensional 
$k(X)$-vector space. Choose nonzero $\omega\in \Omega_\meromorphic^1(X)$. We 
call $K=\divisor(\omega)$ the \emph{canonical divisor}. It does not depend on 
the choice of $\omega$. 

It turns out that $\Omega^1(X)$ can be identified with $\sL(K)$. Indeed, 
we have a natural isomorphism $\sL(K) \iso \Omega^1(X)$ defined by 
$f\mapsto f\omega$. 

Finally, if $D$ is a divisor, one often writes $\ell(D)$ for 
$\dim \sL(D)$. We can now state the final form of the Riemann-Roch Theorem. 

\begin{theo}[Riemann-Roch]
For all divisors $D\in \divisors(X)$, we have 
$\ell(D)-\ell(K-D) = \deg D + 1-g$. 
\end{theo}





\subsection{Consequences of Riemann-Roch}

We could set $D=0$. Then the theorem specializes to 
$1-\ell(K)=1-g$, so $\ell(K)=g$. Since $\ell(K)=\dim \Omega^1(X)$, this 
recovers our definition of the genus of $X$. 

We could set $D=K$. Then the theorem tells us that 
$\ell(K)-1 = \deg K+1-g$. We already know $\ell(K)=g$, so 
$g-1=\deg K+1-g$, so $\deg K=2g - 2$. In other words, the number of zeros of 
a non-zero $\omega\in \Omega^1(X)$ is $2 g-2$. 

\begin{enonce}[remark]{Example}[$g=0$]
If $X(k)\ne \varnothing$, choose a point $\infty\in X(k)$. Then Riemann-Roch 
says $\ell(n\infty) = n+1$. In particular, $\sL(\infty) = k\oplus k t$, 
$\sL(2\infty) = k\oplus k t\oplus k t^2$, and in egneral 
$\sL(n\infty) = k\oplus \cdots \oplus k t^n$. So 
$\sO(U)=k[t]$, whence $U\simeq \dA^1$ and $X\simeq \dP^1$. 
Even if $X(k)=\varnothing$, $X$ has a rational divisor of degree $2$, namely 
$-K$. It must of be of the form $-K=p+p'$ for $p,p'$ conjugates in a quadratic 
extension of $k(X)$. We know that $\sL(-K)=k\oplus k u\oplus k v$, and that 
$\sL(-2 K)$ is spanned by $\{1,u,v,u v,v^2,u^2\}$. But 
$\ell(-2 K)=5$, so we must have a linear relation 
$a+b u+c v+d u v+e v^2 + f u^2 + 0$. In particular, all curves of genus zero 
are conics. 
\end{enonce}

\begin{enonce}[remark]{Example}[elliptic curves]
A curve of genus $1$ with chosen $\infty\in X(k)$ is called an \emph{elliptic 
curve}. Since $\dim \Omega^1(X)=1$, there is (up to homothety) a unique 
non-vanishing regular divisor $\omega\in \Omega^1(X)$. This gives an 
isomorphism $\sO_X\iso \Omega_X^1$. Apply Riemann-Roch to the spaces 
$\sL(n\cdot \infty)$. We get 
\begin{center}
\begin{tabular}{c|cl}
$n$ & $\ell(n\cdot \infty)$ & generators of $\sL(n\cdot\infty)$ \\ \hline
1 & 1 & $\{1\}$ \\
2 & 2 & $\{1,x\}$ \\
3 & 3 & $\{1,x,y\}$ \\
4 & 4 & $\{1,x,y,x^2\}$ \\
5 & 5 & $\{1,x,y,x^2,x y\}$ \\
6 & 6 & $\{1,x,y,x^2,x y,y^2, x^3\}$
\end{tabular}
\end{center}
Since $\ell(6\cdot \infty)=6$, we must have $y^2-x^3\in \sL(5\infty)$, 
so we get a relation 
\[
  y^2 + a_1 x y + a_3 y = x^3 + a_2 x^2 + a_4 x + a_6 .
\]
Now elementary algebra gets rid of $a_1$ if $2$ is invertible, and reduces 
further to an equation 
\[
  y^2 = x^3 + a x+b 
\]
if $3$ is also invertible. 
\end{enonce}

If $X$ is a curve of genus one with $X(k)=\varnothing$, the simplest case is 
when $X$ has a rational divisor of degree $2$. This is true for elements of 
$\selmer_2(E)$. We have 

\begin{theo}
$X$ has an equation of the form 
\[
  y^2 = a x^4 + b x^3 + c x^2 + d x+e .
\]
\end{theo}
\begin{proof}
Our rational divisor of degree $2$ is of the form $p+p'$. We have 
\begin{align*}
  \sL(p+p') &= k\oplus k x \\
  \sL(2 p+2 p') &= k\oplus k x\oplus k x^2\oplus k y \\
  \sL(4 p+4 p') &= \ldots 
\end{align*}
The rest is easy. 
\end{proof}




