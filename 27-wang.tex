% !TEX root = sms.tex

\section{Pencils of quadrics: the geometry}
\thanksauthor{Jerry Wang}





Pencils of quadrics have shown up many times (though not under that name) in 
this summer school. I will explain some of the geometry of quadrics. 





\subsection{Notation}

Let $k$ be a perfect field of characteristic not $2$. Let $\sL$ be a 
rational generic pencil of quadrics in $\dP^{2n+1}$. Such an $\sL$ will be 
of the form 
\[
  \{x Q-y Q_2:(x:y)\in \dP^1\}
\]
where $Q_1,Q_2\subset \dP^{2n+1}$ are quadrics. ``Rational'' means the $Q_i$ 
are defined over $k$. ``Generic'' means the binary form 
$f(x,y) = (-1)^{n+1}\det(x Q_1-y Q_2)$ has no repeated factors, 
i.e.~$\discriminant(f)\ne 0$. Alternatively, $C:z^2=f(x,y)$ should be a smooth 
hyperelliptic curve of genus $n$. The \emph{base locus} $B=Q_1\cap Q_2$ will 
be smooth of dimension $2 n-1$. 

Let $F$ be the variety of maximal linear subspaces of $B$. That is: 
\[
  F=\{X\simeq \dP^{n-1}:X\subset B\} .
\]
For example, when $n=1$, we have two quadrics $Q_1,Q_2\subset \dP^3$. We have 
$F=B=Q_1\cap Q_2$, a genus one curve. So over $\bar k$, $F$ is isomorphic to an 
elliptic curve. This is basically the construction Bhargava used to study 
Selmer elements of elliptic curves. 

If $n=2$, $B=Q_1\cap Q_2$ is a degree $4$ three-fold. The variety $F$ turns 
out to be an abelian surface (over $\bar k$). 

This apparent pattern holds. 

\begin{theo}[Reid, Donagi, Desale-Ramanan]
Over $\bar k$, the variety $F\simeq \jacobian C$; an abelian variety of 
dimension $n$. 
\end{theo}

To obtain arithmetic information about $C$, we need a result that works over 
an arbitrary (possibly non algebraically closed) base field. 

\begin{theo}
The variety $F$ is a $J=\jacobian(C)$-torsor. Moreover, there is an algebraic 
group structure on $G=J\sqcup F\sqcup J^1\sqcup F$ compatible with that of 
$J$, and for which $G/J\simeq \dZ/4$. 
\end{theo}


Here $J^1=\picard^1(C)$, the moduli space of degree-$1$ line bundles on $C$. 

For $n=1$, $Q_1,Q_2\subset \dP^3$, we had $F=Q_1\cap Q_2$. The curve $C$ is 
defined by $z^2=\det(x Q_1-y Q_2)$. The curve $\det(x Q_1-y Q_2)$ is cut out 
by a binary quartic form. There is a canonical isomorphism $H^1=C$, so we have 
a group structure on $G=E\sqcup F\sqcup F\sqcup F$. Multiplication by $2$ on 
$G$ gives a map $2:C\to E$; this is a $2$-cover of $E$. This is the cover used 
in the study of 2-Selmer groups of elliptic curves. Multiplication by $4$ gives 
a map $4:F\to E$; this is the $4$-cover used in the study of 4-Selmer groups of 
elliptic curves. 

For general $n\geqslant 2$, there are a couple cases. 

Case 1: $C(k)\ne\varnothing$. Choose $\infty\in C(k)$. Put 
$F[2]_\infty = \{X\in F:X+X=(\infty)\}$; this is a $J[2]$-torsor. So we get an 
element of $\h^1(k, J[2])$. There are two subcases 
corresponding to whether or not $\infty$ is a Weierstrass point. If $C$ has a 
Weierstrass point, we get all torsors of $J[2]$ in this way. When $\infty$ is a 
non-Weierstrass point, we don't get all of $\h^1(k,J[2])$, but we do get 
``enough'' points in $\h^1(k,J[2])$, namely the entire kernel of 
$\gamma:\h^1(k,J[2]) \to \h^1(k,?)$ as in \autoref{sec:gross-i}. 

Case 2: the map $2:F\to J^1$ is a $2$-cover of $J^1$. If $k$ is a global field 
and $C$ is everywhere locally soluble, we get all locally soluble 2-covers of 
$J^1$ using this method. 




