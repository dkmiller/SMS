% !TEX root = sms.tex

\section{Arithmetic invariant theory and hyperelliptic curves I}\label{sec:gross-i}
\thanksauthor{Benedict Gross}





Let $k$ be a field, $G$ a reductive group over $k$, and 
$G\to \generallinear(V)$ a representation of $G$. The ring 
$\symmetric^\bullet(V^\vee)$ contains a subring 
$\symmetric^\bullet(V^\vee)^G$ of invariant polynomials. An important theorem 
is that $\symmetric^\bullet(V^\vee)^G$ is a finitely generated $k$-algebra. 
Write $V\gq G$ for the variety 
$\spectrum\left(\symmetric^\bullet(V^\vee)^G\right)$; this comes with a 
canonical ``projection'' $\pi:V\to V\gq G$. 





\subsection{First examples and results}

\begin{enonce}[remark]{Example}
Consider $G=\generallinear(W)$ and $V=\mathfrak{gl}(W)=\End W$, with the 
adjoint action of $G$ on $V$. It is known that 
$\mathfrak{gl}(W)\gq \generallinear(W)$ is an affine space with coordinates the 
``coefficients of the symmetric polynomial.'' 
\end{enonce}

More generally, if $G$ is a reductive group and $\frakg=\lie G$ under the 
adjoint representation, then $\frakg\gq G$ is affine. That is, Chevalley 
proved that the adjoint representation of a reductive group is coregular. 
Winberg generalized this even further. If $\theta:G\to G$ is an 
automorphism of order $m$ and $\frakg=\bigoplus_a \frakg(a)$, then 
the action of $G^\theta$ on each $\frakg(a)$ is coregular. 

Suppose $f\in (V\gq G)(k)$. Let $V_f$ be the fiber in $V$ of $\pi$ over $f$. 
Then $V_f(k)$ is a (possibly empty) union of $G(k)$-orbits. 

\begin{enonce}[remark]{Example}
When $G=\generallinear(n)$ and $V=\mathfrak{gl}(n)$, then $V_f$ is all linear 
$T$ with fixed characteristic polynomial $f$. The set $V_f(k)$ is always 
nonempty. Indeed, let $L=k[x]/f$ and $\theta:L\to L$ be ``multiplication 
by $x$.'' Then $\theta$ is a $k$-linear transformation with characteristic 
polynomial $f$. Choosing an isomorphism $L\simeq k^n$ gives an element in 
$V_f(k)$. Roughly, ``every polynomial is the characteristic polynomial of a 
map defined over the base field.'' 

If the discriminant $\Delta(f)\ne 0$, there 
is a single orbit of $G(k)$ on $V_f(k)$. For any $T\in V_f(k)$, 
the stabilizer $G_T$ is isomorphic to the Weil restriction 
$\Pi_{L/k}\dG_\multiplicative$; a maximal torus in $\generallinear(V)$ if $L$ 
is \'etale. 
If $f(x)=x^k$, then $V_f$ is known as the \emph{nilpotent cone}; the orbit 
we constructed is the \emph{regular nilpotent}. 
\end{enonce}

\begin{enonce}[remark]{Example}
Consider the action of $\speciallinear(W)$ on 
$\mathfrak{sl}(W)=\End(W)^{\trace=0}$. The ring of invariants is freely 
generated by all but the constant term of the characteristic polynomial, so 
$\mathfrak{sl}(n)\gq \speciallinear(n) \simeq \dA^{n-1}$. If $\Delta(f)\ne 0$, 
then orits in $V_f(k)$ are in bijection with $k^\times / \norm(L^\times)$. 

If for example $\dim W=2$ and $f(x)=x^2+1$, then the orbit space 
$V_f(k)/G_f(k)$ is in bijection with $\dQ^\times/\norm(\dQ(i)^\times)$; a 
(huge) abelian 2-group. 
\end{enonce}

For $\generallinear(n)$ and $\speciallinear(n)$, we obtained that 
$V_f(k)$ was a torsor over $\h^1(k,G_f)$. But this used 
$\h^1(k,\speciallinear_n) = \h^1(k,\generallinear_n) = 0$. 





\subsection{Principles of arithmetic invariant theory}

\begin{enonce}{Principle}
Assume $V_f(k)\ni v$, and that $G(k^s)$ acts transitively on 
$V_f(k^s)$. Then the set of $G(k)$-orbits on $V_f(k)$ is in bijection with the 
kernel of the map of pointed sets 
$\h^1(k,G_v) \to \h^1(k,G)$. 
\end{enonce}
\begin{proof}
Say $v'\in V_f(k)\subset V_f(k^s)$, write $v'=g(v)$ for some $g\in G(k^s)$. 
Send the orbit of $v'$ to the class in $\h^1(k,G_v)$ of the cocycle 
$\sigma\mapsto c_\sigma = g^{-1} \circ g^\sigma$. Checking that this is a 
bijection is easy. 
\end{proof}

If $G$ is one of $\generallinear(n)$, $\speciallinear(n)$, $\symplectic(n)$, 
then $\h^1(k,G)=0$, so $V_f(k)/G(k) = \h^1(k,G_v)$. 

\begin{enonce}[remark]{Example}
Let $W$ be a split orthogonal space over $k$ of dimension $n=2 g+1$. So $W$ 
is a direct sum of $g$ hyperbolic planes and a single copy of $k$. Let 
$G=\specialorthogonal(W)$. For example, if 
$g=1$, then $G\iso \projectivegenerallinear(2)\iso \specialorthogonal_3$. 
Let $V=\mathfrak{so}(W)$; the space of trace-zero self-adjoint operators on 
$W$. Then 
$\symmetric^\bullet(W^\vee)^G=k[c_2,\dots,c_{2g+1}]$, freely generated on 
the coefficients of the characteristic polynomial. It's a bit more 
difficult to show that the fibers of 
$\mathfrak{so}(W) \to \mathfrak{so}(W)\gq \specialorthogonal(W)$ are all 
nonempty. Given $f$ in the quotient, define as before $L=k[x]/f$. This is a 
$k$-algebra of rank $2 g+1$. Let $\langle\lambda,\mu\rangle$ be the coefficient 
$x^{2 g}$ in $\lambda\mu$; also $\trace_{L/k}(\lambda\mu/f'(x))$. The 
operator ``multiplication by $x$'' on $L$ is self-adjoint with characteristic 
polynomial $f$. A bit of work shows that this gives an element of $V_f(k)$. 
\end{enonce}

In fact, for $V=\mathfrak{so}(n)$, $G=\specialorthogonal(n)$, the map 
$V(k) \to V\gq G$ has a standard section known as the Kostant section. 

Let's compute stabilizers. For $f$ with $\Delta(f)\ne 0$, it is easy to see 
that $G_v=\specialorthogonal(W)\cap L^\times$. This is 
$L^\times[2]^{\norm=1}$. As a group scheme, this is 
$\ker(\Pi_{L/k}\boldsymbol\mu_2 \xrightarrow{\norm{}}\boldsymbol\mu_2)$. 
An easy computation of Galois cohomology shows that 
$\h^1(k,G_v) = (L^\times/2)^{\norm=0}$. In this case, $\h^1(k, G_v)$ is also 
$\h^1(k,J[2])$, where $J$ is the Jacobian of $y^2=f(x)$. 

In general, our first principle is not very useful, because the map 
$\gamma:\h^1(k,G_v) \to \h^1(k,G)$ can be pretty complicated, and is not easy to 
pin down explicitly. 

\begin{enonce}{Principle}
For any $c\in \h^1(k,G)$, there is a twisted group $G^c$ over $k$ and twisted 
representation $V^c$ over $k$. The fiber of $\gamma$ over $c$ is the set of 
orbits of $G^c(k)$ in $V_f^c(k)$. 
\end{enonce}

Recall our map 
$\h^1(k,G_v) = \h^1(k,J[2]) \xrightarrow\gamma \h^1(k,\specialorthogonal(W))$, 
where $J$ is the Jacobian of $y^2=f$. The 2-Selmer group $\selmer_2(J)$ is a 
subset of $\h^1(k,J[2])$, and in \cite{bg13}, Bhargava and I showed that 
$\selmer2(J) = \ker(\gamma)$. In general, when is $V_f$ empty for all 
$G^c$?

\begin{enonce}{Principle}
Assume $G(k^s)$ acts transitively on $V_f(k^s)$ and $G_v(k^s)$ is abelian. 
\begin{enumerate}
  \item If the class of $d_f$ is non-trivial in $\h^2(k,G_f)$, there is no 
    $k$-rational point in any fiber. 
  \item If $d_f=0$, then the fiber is nontrivial for some pure inner form $G^c$. 
\end{enumerate}
\end{enonce}
\begin{proof}
Take $v\in V_f(k^s)$. Then $\sigma_v = \prescript{\sigma}{}{f}\circ f$ is also 
in $V_f(k^s)$. Define $\theta_\sigma:G_{c_v} \to G_v$ by 
$\alpha\mapsto g_\sigma \alpha g_\sigma^{-1}$. This map is an isomorphism that 
does not depend on $v$. Since 
$\theta_{\sigma\tau} = \theta_\sigma\circ\prescript{\sigma}{}{\theta_\tau}$, 
this descends $G_v$ to a group $G_f$ defined over $k$. 
\end{proof}

So if $G_v(k^s)$ is abelian and $f\in (V\gq G)(k)$, there is a ``stabilizer'' 
$G_f$ of $f$ even if $V_f(k)=\varnothing$. 

\begin{enonce}[remark]{Example}
Let $G=\speciallinear(W)$, where $\dim W=2 g+2$. Let 
$V=\symmetric^2(W^\vee)\oplus \symmetric^2(W^\vee)$. We can think of 
$V$ as the space of pairs $v=(A,B)$ of symmetric matrices. The ring of 
invariants is freely generated by the coefficients of the bilinear form 
$f(x,y) = (-1)^{g+1}\det(x A-y B)$. Assume $\Delta(f)\ne 0$. Then put 
$G_f=(\Pi_{L/k}\boldsymbol\mu_2)^{\norm=1}$, where $L/k$ is the extension 
constructed earlier. If we put $f(x,1) = f_0 \cdot g(x)$, then 
$L=k[x]/g$. What is the class $d_f\in \h^2(k,G_f)$? This cohomology group 
has a subgroup $k^\times / k^\times \norm(L^\times)$. The class $d_f$ is just 
the class of $f_0$ in $k^\times / k^\times \norm(L^\times)\subset \h^2(k,G_f)$. 
\end{enonce}

For example, if $k=\dR$, $g=0$, $f_0=-1$, $f=-x^2-y^2$, and $g=x^2+1$, then there 
are no orbits. 




