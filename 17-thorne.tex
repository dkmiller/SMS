% !TEX root = sms.tex

\section{Zeta function methods}\label{sec:thorne}
\thanksauthor{Frank Thorne}





\subsection{Motivation}

The question is: what are zeta functions good for? Let $N_3^\pm(X)$ be the 
number of cubic fields $K$ with $0<\pm\discriminant(K)<X$. Define 
\begin{align*}
  C^\pm &= \begin{cases} 1 & + \\ 3 & - \end{cases} \\
  K^\pm &= \begin{cases} 1 & + \\ \sqrt 3 & -\end{cases}
\end{align*}

\begin{theo}
The following holds: 
\[
  N_3^\pm(X)=C^\pm \frac{1}{12\zeta(3)}X + K^\pm \frac{4\zeta(1/3)}{5\Gamma(2/3)^3\zeta(5/3)} X^{5/6} + O(X^{2/3+\epsilon}) .
\]
\end{theo}

We would like to understand how zeta functions can be used to provide such 
good error terms. 





\subsection{Definitions}

As we have done before, put 
\begin{align*}
  V(\dZ) &= \{a u^3 + b u^2 v + c u v^2 + d v^3:a,b,c,d\in \dZ\} \\
  \widehat V(\dZ) &= \{\cdots : 3\mid b,c\} .
\end{align*}
The group $\generallinear_2(\dZ)$ acts on both of these via 
\[
  (\gamma\cdot f)(u,v) = \frac{1}{\det\gamma}f\left(\begin{pmatrix} u & v\end{pmatrix}\cdot \gamma\right) .
\]
\begin{theo}
There is a natural bijection 
\[
  \generallinear_2(\dZ)\backslash V(\dZ) \iso \{\text{cubic rings}\}/\sim .
\]
\end{theo}

\begin{defi}[Shintani]
Put 
\begin{align*}
  \xi^\pm(s) 
    &= \sum_{x\in \generallinear_2(\dZ)\backslash V^\pm(\dZ)} \frac{1}{\#\stabilizer(x)}|\discriminant(x)|^{-s} \\
    &= \sum_{\substack{R\text{ cubic ring} \\ 
  \pm \discriminant(R)>0}} \frac{1}{\#\automorphism(R)} |\discriminant(R)|^{-s} . 
\end{align*}
\end{defi}

\begin{theo}[Shintani]
The functions $\xi^\pm(s)$ have analytic continuation to $\dC$ except for 
poles at $s=1,\frac 5 6$, explicit residue formulas at these poles, and a 
functional equation 
\[
  \begin{pmatrix} \xi^+(1-s) \\ \xi^-(1-s) \end{pmatrix} = \Gamma\left(s-\frac 1 6\right)\Gamma(s)^2 \Gamma\left(s+\frac 1 6\right) \frac{3^{6 s-2}}{2\pi^{4 s}} \begin{pmatrix} \sin(2\pi s) & \sin(\pi s) \\ 3\sin(\pi s) & \sin(2\pi s)\end{pmatrix} \begin{pmatrix} \widehat \xi^+(s) \\ \widehat\xi^-(s) \end{pmatrix} 
\]
where $\widehat\xi^\pm$ are defined in terms of $\widehat V(\dZ)$ instead of 
$V(\dZ)$. 
\end{theo}
This was proved in \cite{s72}. 





\subsection{How analytic number theorists count}

We'll see explicitly how analytic properties of $\xi^\pm$ translate into 
asymptotic estimates for the number of cubic rings. 

\begin{enonce}{Principle}[Perron's formula]
Given any Dirichlet series $B(s)=\sum_{n\geqslant 1} b(n) n^{-s}$ which is 
absolutely convergent for $\Re s=2$, then 
\[
  \sum_{n\leqslant X} b(n) = \frac{1}{2\pi i}\int_{2-i\infty}^{2+i\infty} B(s) X^s \, \frac{\mathrm{d}s}{s} .
\]
\end{enonce}

For the $b(n)$ completely arbitrary, this is not very helpful. The idea is: for 
specific $b(n)$, shift the contour of this integral. For example, if we are 
trying to count integers less than $X$, apply Davenport's lemma to conclude 
that there are $X+O(1)$ integers between $0$ and $X$. We define 
\[
  \zeta(s) = \sum_{n\geqslant 1} n^{-s} .
\]
By Perron's formula, we get 
\[
  \sum_{1\leqslant n<X} 1 = \frac{1}{2\pi i} \int_{2-i\infty}^{2+i\infty} \zeta(s) X^s\frac{\mathrm{d}s}{s} .
\]
See if you can spot the mistake in the following computation: 
\begin{align*}\tag{$\ast$}\label{eq:diverge}
  \sum_{n\leqslant X} 1 
    &= \residue_{s=1} + \residue_{s=0}\left(\zeta(s)X^s\frac{\mathrm{d}s}{s}\right) + \frac{1}{2\pi i}\int_{-1-i\infty}^{1+i\infty} \zeta(s) X^s\frac{\mathrm{d}s}{s} \\
    &= X+\zeta(0) + (\text{error}) .
\end{align*}
Does the integral in \eqref{eq:diverge} converse? For this, we need some bounds 
on $\zeta$. 

\begin{prop}
If $\sigma<0$, we have $\zeta(-\sigma+i t) \ll (1+|t|)^{1/2+\sigma}$. 
\end{prop}

\begin{enonce}{Exercise}
Use the functional equation and Stirling's approximation to prove the 
Proposition. 
\end{enonce}

Inside the critical strip, controlling the behavior of $\zeta$ is a large 
problem. But outside $\{0<\Re s<1\}$, things are relatively straightforward. 

We can use the above Proposition to show that the integral appearing in 
\eqref{eq:diverge} diverges. 





\subsection{The Landau method}

A big principle in analytic number theory is that it is important to work with 
``smooth sums.'' 

\begin{prop}
Let $b:\dN\to \dC$ and $B(s)=\sum b(n) n^{-s}$. Then 
\begin{align*}
  \sum_{n<X} b(n)\left(1-\frac n X\right) &= \frac{1}{2\pi i} \int_{2-i\infty}^{2+i\infty} B(s) \frac{X^s}{s(s+1)}\, \mathrm{d}s \\
  \frac 1 2\sum_{n<X}\left(1-\frac n X\right)^2 &= \frac{1}{2\pi i}\int_{2-i\infty}^{2+i\infty} B(s) \frac{X^s}{s(s+1)(s+2)}\, \mathrm{d} s.
\end{align*}
\end{prop}

\begin{enonce}{Exercise}
Prove and generalize the Proposition. 
\end{enonce}

These are all special cases of Mellin inversion. Using the Proposition, we 
continue our counting of integers: 
\[
  \sum_{n<X} (X-n) = \frac{X^2}{2} - X + \frac{1}{2\pi i} \int_{-1-i\infty}^{1+i\infty} \zeta(s) \frac{X^{1+s}}{s(s+1)}\, \mathrm{d} x .
\]
The integral is $O(X^{1/2+\epsilon})$. Moreover, 
\[
  \sum_{n<X+Y} (X+Y-n) = \frac{X+Y}{2} - (X+Y) + O((X+Y)^{1/2+\epsilon}) .
\]
Subtract the first equation from the second, to conclude that 
\[
  \sum_{n<X} 1 = X+O(X^{1/4+\epsilon}) .
\]
A more careful analysis of the integrals gets better error terms. 





\subsection{Why does the zeta function have such good analytic properties?}

\begin{defi}[Shintani]
The \emph{global zeta function} is, for ``nice test functions'' 
$f:V_\dR\to \dC$, 
\[
  Z(f,s) = \int_{\generallinear_2(\dR)/\generallinear_2(\dR)} |\det g|^{2 s} \left(\sum_{x\in V(\dZ)\smallsetminus S} f(g x)\right)\, \mathrm{d} g ,
\]
where $S=\{x\in V(\dZ):\discriminant(x)=0\}$. 
\end{defi}

\begin{prop}
We have the following decomposition:
\[
  Z(f,s) = \frac{1}{4\pi} \xi^+(s) \int_{V^+(\dR)} |\discriminant(x)|^{s-1} f(x)\, \mathrm{d}x + \frac{\xi_-(s)}{2\pi} \int_{V^-(\dR)} |\discriminant(x)|^{s-1} f(x)\,\mathrm{d} x .
\]
\end{prop}

Recall from Bhargava's lectures that the number of irreducible 
$\generallinear_2(\dZ)$-orbits in $G(\dZ)$ with $|\discriminant|<X$ is 
\[
  C \frac{\displaystyle\int_\cF\#\{x\in g B\cap V(\dZ)^\mathrm{irr}:|\discriminant(x)|<X\}\,\mathrm{d}g}{\displaystyle\int_B |\discriminant(b)|^{-1}\, \mathrm{d} v} .
\]
The sieve gives the estimate 
\[
  N_\mathrm{max}^\pm(X) = \sum_{q\geqslant 1} \mu(q) N^\pm(q,x) 
\]
where $N^\pm_\mathrm{max}(X)$ is the number of maximal cubic rings $R$ with 
$0<\pm \discriminant(R)<X$ and 
$N^\pm(q,X)$ is the number of cubic rings ``nonmaximal at $q$.'' Define the 
$q$-nonmaximal zeta functions: 
\[
  \xi_q^\pm(s) = \sum_{x\in \generallinear_2(\dZ)\backslash V^\pm(\dZ)} \frac{1}{\#\stabilizer(x)} \Phi_q(x) |\discriminant(x)|^{-s} ,
\]
where $\Phi_q(x)$ is the characteristic function of the set of cubic forms 
nonmaximal at $q$. We get 
\[
  \widehat\xi_q^\pm(s) = q^{8 s-8} \sum_{x\in \generallinear_2(\dZ)\backslash \widehat V_\dZ} \frac{1}{\#\stabilizer(x)}\widehat\Phi_q(x) |\discriminant(x)|^{-s} ,
\]
where 
\[
  \widehat\Phi_q(x) = \sum_{y\in V(\dZ/q^2)} \Phi_q(y) \exp\left(\frac{2\pi i[x,y]}{q^2}\right) ,
\]
and 
\[
  [x,y] = x_r y_1 - \frac 1 3 x_3 y_2 + \frac 1 3 x_2 y_3 - x_1 y_4 .
\]




