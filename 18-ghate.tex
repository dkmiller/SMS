% !TEX root = sms.tex

\section{Counting Artin representations and modular forms of weight one}
\thanksauthor{Eknath Ghate}





We'll start by spending a little bit of time explaining what modular forms are. 





\subsection{Brief introduction to modular forms}

Let $f = \sum_{n\geqslant 1} a_n q^n$ be a primitive cusp form of weight $1$, 
level $N\geqslant 1$, and character 
$\varepsilon:(\dZ/N)^\times \to \dC^\times$. Then $f$ is a holomorphic function 
$f:\fH\to \dC$, where $\fH=\{z\in \dC:\Im z>0\}$ is the upper-half plane. The 
function $f$ transforms by 
\[
  f(\gamma z) = \varepsilon(d) (c z+d) f(z) ,
\]
for all matrices $\begin{pmatrix} a & b \\ c & d \end{pmatrix}\in \Gamma_0(N)$. 
Moreover, $f$ is ``holomorphic at cusps.'' 

Deligne and Serre proved that to asuch a modular form is attached a continuous 
Galois representation $\rho_f:G_\dQ \to \generallinear_2(\dC)$ such that 
$\trace \rho_f(\frobenius_\ell) = a_\ell$ and 
$\det\rho_f(\frobenius_\ell) = \varepsilon(\ell)$ for all primes $\ell\nmid N$. 
Since $G_\dQ$ is compact and totally disconnected, the image of $\rho_f$ is 
finite. Once we projectivize, the image of 
$\widetilde\rho_f:G_\dQ\to \projectivegenerallinear_2(\dC)$ is one of 
\begin{align*}
  D_m && \text{dihedral group of order }2m \\
  A_4 && \text{tetrahedral} \\
  S_4 && \text{octahedral} \\
  A_5 && \text{icosahedral}
\end{align*}
We call the representations with projective image $A_4,S_4,A_5$ exotic. They 
are quite rare, with the first one occurring when $N=800$. 

\begin{conjecture}
The number of exotic forms of prime level $N$ is $O(N^\epsilon)$. 
\end{conjecture}

If the level $N$ is prime, then only octahedral or icosahedral images occur. 
In this talk we will restrict to counting octahedral forms of prime level. Here 
is recent progress on the conjecture:
\begin{center}
\begin{tabular}{c|c}
person & bound \\ \hline
Duke & $O(N^{7/8+\epsilon})$ \\
Wang & $O(N^{5/6+\epsilon})$ \\
M-V & $O(N^{4/5+\epsilon})$ \\
Ganguly & $O(N^{3/4+\epsilon})$ \\
Ellenberg & $O(N^{2/3+\epsilon})$
\end{tabular}
\end{center}

\begin{theo}[Bhargava-Ghate]
Let $N_\mathrm{oct}^\mathrm{prime}(X)$ be the number of octohedral forms of 
prime level $<X$. Then 
\[
  N_\mathrm{oct}^\mathrm{prime}(X) = O(X/\log X) .
\]
\end{theo}
This is proven in \cite{bg09}. So on average, the number of octahedral forms 
of prime level is bounded by a constant. 

\begin{proof}
The idea is to ``count forms by counting forms.'' That is, we use the fact 
that octahedral modular forms of weight one correspond to Galois 
representations, which in turn correspond go quartic number fields, which 
come from quartic forms. 

Step 1. It is enough to count (linear) Galois representations. The Artin 
conjecture says that there is a bijection between octahedral forms and 
isomorphism classes of $\rho:G_\dQ \to \generallinear_2(\dC)$ with 
$\rho(G_\dQ)\simeq S_4$ and $\det\rho(c)=-1$. The direction $\Rightarrow$ was 
proved in \cite{ds74}, while $\Leftarrow$ is the Langlands-Tunnell theorem 
proved in \cite{l80,t81}.

One uses Serre's conjecture (proved in full generality in 
\cite{kw09-i,kw09-ii}) to prove the Artin conjecture. Choose 
$\cO\subset \dC$, the ring of integers in a number field such that 
$\rho(G_\dQ)\subset \generallinear_2(\dC)$. We get a commutative diagram 
\[\xymatrix{
  G_\dQ \ar[r]^-\rho \ar[dr]_-{\bar\rho} 
    & \generallinear_2(\cO) \ar[d] \\ 
  & \generallinear_2(\overline{\dF_p}) .
}\]
Serre's conjecture predicts that if $\bar\rho$ is odd and irreducible, then 
$\bar\rho\sim \bar\rho_g$ for a modular form $g$, where 
$g\in S_1(\Gamma_0(N),4)$. Write $g=\sum b_n q^n$. Then 
$a_\ell\equiv b_\ell\pmod p$ for all $\ell\nmid N$. Since this works for 
infinitely many $p$, there exists $g$ such that $a_\ell=b_\ell$. Thus 
$\rho$ is modular. 

Step 2. It is enough to count projective Galois representations. There is 
a surjection from isomorphism of odd $\rho:G_\dQ\to \generallinear_2(\dC)$ 
with $\widetilde\rho(G_\dQ)\simeq S_4$, to isomorphism classes of 
$\widetilde\rho:G_\dQ\to \projectivegenerallinear_2(\dC)$ with 
$\widetilde\rho(G_\dQ)\simeq S_4$ and $\widetilde\rho(c)\ne 1$. Surjectivity 
follows from $\h^2(G_\dQ,\dC^\times)=0$. The map is not injective: if 
$\chi$ is any character, Tate proved that 
$\rho\otimes\chi\mapsto\widetilde\rho$. But we can control the map when $N$ is 
square-free. If $p\| N$, then 
\[
  \rho|_{I_p} \sim \begin{pmatrix} \varepsilon_p \\ & 1 \end{pmatrix} .
\]
Choose $\chi$ with $\chi^{12}=1$, and apply this with $\rho\otimes\chi$ 
instead of $\rho$. The new character is $(\varepsilon\chi^2)^{12}$. When 
$N=p$, there are only two such $\chi$ so that $\rho\otimes\chi$ map to the 
same $\widetilde\rho$. We get that $\varepsilon=\varepsilon_p$ is odd. Some 
technical manipulations yield that $\rho\mapsto \widetilde\rho$ is 
2-to-1 when the level $N$ is prime. 

Step 3. It is enough to count quartic fields with Galois closure having group 
$S_4$. A projective representation 
$\widetilde\rho:G_\dQ \to \projectivegenerallinear_2(\dC)$ with 
$\widetilde\rho(G_\dQ)\simeq S-4$ and $\widetilde\rho(c)\ne 1$ cuts out a field 
$K$ over $\dQ$ that is not totally real. 

There is a key technical problem here. We want to count modular forms by level, 
but we usually count number fields by discriminant. Let $N$ be the level of 
$f$. If $K$ is the field corresponding to $f$ and $D=\discriminant(K)$, then 
we might have $N\ne D_K$. However, a prime $p\mid N_f$ if and only if 
$p\mid D_K$. Note that if $p\geqslant 5$, then either $p\| N_f$ or 
$p^2\| N_f$. However, it is possible for $p^3\| D_K$. Consider the following 
ramification table in the octahedral case (for minimal forms): 
\begin{center}
\begin{tabular}{c|c|c|c|c|c}
$I_p$ & $G_p$ & ram.~in $K$ & $D_K$ & $N_f$ & $p\equiv$ \\ \hline
$(12)$     & $I_p$       & $1^2 11$ & $p$ & $p$ \\
$(12)$     & $(12),(34)$ & $1^2 2$  & $p$ & $p^2$ \\
$(12)(34)$ & $I_p$       & $1^2 1^{\underline 2}$ & $p^2$ & $p$\\
$(13)(24)$ & $(1234)$    & $2^2$    & $p^2$ & $p$ \\
$(12)(34)$ & $V_4$       & $2^2$    & $p^2$ & $p^2$ \\
$(12)(34)$ & $(12)(34)$  & $1^2 1$  & $p^2$ & $p^2$ \\
$(123)$    & $I_p$       & $1^3 1$  & $p^2$ & $p$   & $1\pmod 3$ \\
$(123)$    & $S_3$       & $1^3 1$  & $p^2$ & $p^2$ & $2\pmod 3$ \\
$(1234)$   & $I_p$       & $1^4 $   & $p^3$ & $p$   & $1\pmod 4$ \\
$(1234)$   & $D_4$       & $1^4$    & $p^3$ & $p^2$ & $3\pmod 4$ 
\end{tabular}
\end{center}
[\ldots some notation I don't understand\ldots]. 
Five times, the power of $p$ in $D_K$ is at most the power of $p$ in 
$N_f$. The other five possibilities, this fails. For $4/5$ of the time we can 
still control things. The other possibility is $p\equiv 1\pmod 3$. 

We use some facts about quartic fields. Consider field extensions 
$E-\supset K_6\supset K_3\supset \dQ$ corresponding to the inclusions 
[\ldots missed this part\ldots]. 

The extension $K_6/K_3$ has Galois group $S_4$ if and only if 
\begin{enumerate}
  \item $K_3/\dQ$ has Galois group $S_3$ 
  \item $\norm_\dQ(\discriminant(K_6/K_3))=n^2$ for $n$ square-free 
  \item $\discriminant(K_4) = \discriminant(k_3) n^2$ 
  \item $K_6/K_3$ ramifies if and only if $p=1^4$ or 
    $1^2 1^2$ or $2^2$. 
\end{enumerate} 
We use the following theorem of Serre to simplify the table: 
\begin{center}
\begin{tabular}{c|c|c|c|c}
Ram.~in $K_4$ & $|\discriminant(K_4)|$ & level & $|\discriminant(K_3)|$ & $n$ \\ \hline
$1^2 11$ & $p$ & $p$ & $p$ & $1$ \\
$1^4$ & $p^3$ & $p$ & $p$ & $p$ 
\end{tabular}
\end{center}

Step 4. Use Bhargava's counting results in the quartic case, as well as some 
sieve methods. Recall, from \cite{b04} that isomorphism classes of maximal
$S_4$-quartic orders are in bijection with 
$\generallinear_2(\dZ)\times \generallinear_3(\dZ)$-classes of pairs of 
irreducible ternary quadratic forms $(A,B)$. From \cite{b05}, we the number 
$N_4(X)$ of number fields of $S_4$-quartic fields of $|\discriminant|<X$ is 
$O(X)$. A technical modification shows that the number $N_4^\mathrm{prime}(X)$ 
of $S_4$-quartic fields of prime level $<X$ is $O(X/\log X)$. As a 
corollary, 
\[
  \sum_{\substack{0<|\discriminant(K_3)|<X \\ \text{prime}}} h_2^\ast(K_3) \leqslant C\frac{X}{\log X} .
\]
We can finally count the number of octahedral of prime level $<X$ modular 
forms.  
\end{proof}

\begin{theo}[Serre]
In prime level, the discriminant of $K_4$ is either $p^3$ or $-p$. 
\end{theo}




