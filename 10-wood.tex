% !TEX root = sms.tex

\section{Quartic and quintic rings}\label{sec:wood-ii}
\thanksauthor{Melanie Matchett-Wood}





The plan is to first review a bit of the theory of cubic rings. We'll spend 
most of the time on quartic rings, then give a brief treatment of quintic 
rings. 





\subsection{Cubic rings revisited}

We are interested in passing from cubic rings to forms geometrically. 
For simplicity, we assume $R=\cO_k$ is the maximal order in a cubic number 
field. Consider the affine scheme $\spectrum(\cO_k)$; we want to embed this 
into $\dP_\dZ^1$. In general, a map $X\to \dP_\dZ^1$ is determined by a line 
bundle $\sL$ on $X$ together with two global sections that generate $\sL$. 
For $\spectrum(\cO_k)$, this consists of an ideal $\fa\subset \cO_k$ together 
with two elements generating $\fa$. We choose the inverse different 
$\fD^{-1}\subset \cO_k$, and two elements of $\fD^{-1}$ having trace zero as 
our global sections. 
It turns out that the image of $\spectrum(\cO_k) \to \dP_\dZ^1$ is the zero-set 
of a binary cubic form. 

If $\cO_k$ is the maximal order in a number field $k$ with $[k:\dQ]=n$, the 
same construction embeds $\spectrum(\cO_k)$ into $\dP_\dZ^{n-2}$. 
It turns out that for any maximal order, a basis of 
$\ker(\trace:\fD^{-1} \to \dZ)$ will always generate $\fD^{-1}$ as a 
$\cO$-module. 





\subsection{Quartic rings}

Start as we did with cubic rings. Given a quartic ring $Q$, write down a 
$\dZ$-basis $\{1,\alpha_1,\alpha_2,\alpha_3\}$ for $Q$. Just as with cubic 
rings, we can encode the multiplication of $Q$ by  
\[
  \alpha_i \alpha_j = c_{i j}^0 + \sum_{k=1}^3 c_{i j}^k \alpha_k .
\]
We could ``shift'' some of the basis elements in order to make some of the 
$c_{i j}^k=0$. Associativity gives us some conditions on the 
$c_{i j}^k$. We're left with a ``moduli scheme'' for quartic rings with 
basis, but the polynomial relations between the $c_{i j}^k$ are complicated 
enough that a direct, explicit approach, does not get you very far. 

In the paper \cite{wy92}, Write and Yukie showed that quartic fields are 
parameterized by pairs of ternary and quadratic forms, moduli the natural 
action of $\generallinear_2(\dQ)\times \generallinear_3(\dQ)$. Unfortunately 
this approach isn't very useful either. More recently, in the paper 
\cite{b04}, Bhargava realized that the problem needed to be understood over 
$\dZ$, and that  cubic resolvent are essential. 

We start with cubic resolvent fields $K/\dQ$ with $[K:\dQ]=4$. Let 
$\widetilde K$ be the Galois closure of $K$; we assume 
$\galois(\widetilde K/\dQ)\simeq S_4$. The group $S_4$ has a canonical 
subgroup $D_4$ of index $3$, consisting of permutations which are 
symmetries of the square 
\[\xymatrix{
  1 \ar@{-}[r] \ar@{-}[d] 
    & 2 \ar@{-}[d] \\
  4 \ar@{-}[r] 
    & 3
}\]
Galois theory gives us a subfield $K_3\subset \widetilde K$ such that 
$[K_3:\dQ]=3$; this is the \emph{cubic resolvent} of $K$. For 
$k\in K$, let $k^{(1)},k^{(2)},k^{(3)},k^{(4)}$ be the Galois conjugates of 
$K$. Then $k^{(1)} k^{(3)} + k^{(2)} k^{(4)}$ is an element of $K_3$. Put 
$\phi_{4,3}(k) = k^{(1)} k^{(3)} + k^{(2)} k^{(4)}$; this is a 
discriminant-preserving map $K\to K_3$. 

Next we define the cubic resolvent of a ring. Let $Q$ be a quartic ring, which 
for simplicity we assume is an order in a $S_4$-quartic field $K$. 

\begin{defi}
A \emph{cubic resolvent ring} of $Q$ is a cubic ring $R$ in the resolvent 
field $K_3$ such that 
\begin{enumerate}
  \item $\discriminant(R)=\discriminant(Q)$
  \item for all $q\in Q$, $\phi_{4,3}(q)\in R$. 
\end{enumerate}
\end{defi}
We have a quadratic map $\phi_{4,3}:Q\to R$. It descends to a map 
$\phi_{4,3}:Q/\dZ \to R/\dZ$; here the quotients are taken as 
$\dZ$-modules, not as rings. The $\dZ$ in the quotient is the 
sub-$\dZ$-module generated by the multiplicative unit in the ring. 

\begin{exercise}
Show that $\phi_{4,3}$ descends to a map $Q/\dZ\to R/\dZ$. 
\end{exercise}

An element of $Q/\dZ$ can be written as $\ell\alpha_1+m \alpha_2 + n \alpha_3$ 
for $\ell,m,n\in \dZ$. It's image under $\phi_{4,3}$ will be a linear combination 
of $\omega$ and $\theta$, i.e. 
\[
  \phi_{4,3}(\ell \alpha_1 + m\alpha_2 + n\alpha_3) = A(\ell,m,n) \omega + B(\ell,m,n)\theta .
\]
The functions $A$ and $B$ are ternary quadratic forms with coefficients in 
$\dZ$. 

It is not clear whether every quartic ring even has a cubic resolvent. Even if 
it does, how many? Luckily, every quartic ring does have a cubic resolvent, but 
this resolvent is not necessarily unique. But maximal quartic rings (which 
include maximal quartic orders) have a unique cubic resolvent. 

We want to parameterize isomorphism classes of pairs $(Q,R)$, where $Q$ is a 
quartic ring and $R$ is a cubic resolvent of $Q$. The main result of 
\cite{b04} is that these are in bijection with 
$\generallinear_2(\dZ)\times \generallinear_3(\dZ)$-classes of pairs of ternary 
quadratic forms. This bijection preserves discriminants, and allows you to 
detect prime splitting, automorphism groups, \ldots as the parameterization of 
cubic rings. 

To be completely explicit, a \emph{ternary quadratic form} is of the form 
\[
  A(\ell,m,n) = a_{1 1} \ell^2 + a_{1 2} \ell m + a_{1 3} \ell n + \cdots + a_{3 3} n^2 .
\]
The $\generallinear_2$ and $\generallinear_3$ action can be written down 
explicitly. We identify the form $A$ with a $3\times 3$ matrix 
\[
  \begin{pmatrix} a_{11} & a_{12}/2 & a_{13}/2 \\ a_{12}/2 & a_{22} & a_{23}/2 \\ a_{13}/2 & a_{23}/2 & a_{33} \end{pmatrix} 
\]
and similarly for $B$. The quantity $4\det(A x+B y)$ is a binary cubic 
form with coefficients in $\dZ$. The corresponding cubic ring is the cubic 
resolvent ring in the pair $(Q,R)$. What we haven't done is describe how to 
construct the quartic ring $Q$ from $A$ and $B$. 





\subsection{Geometric perspective}

Write $A=a_{11} x^2+a_{1 2} x y + a_{1 3} x z + \cdots$ and similarly for $B$. 
These cut out a subscheme of $\dP^2$. Over $\dQ$, we would expect this subscheme 
to have four points (over $\overline\dQ$). The Galois conjugates of any such 
point $p$ are among the four intersection points, so $p$ is defined over a 
(at most) quartic extension of $\dQ$. 

Let's do this over $\spectrum \dZ$. A good reference is \cite{w11}. The two 
forms $A,B$ over $\dZ$ cut out a subscheme $V_{A,B}$ of $\dP_\dZ^2$. The 
functions on $V_{A,B}$ recover the quartic ring corresponding to $A,B$. 

If $\cO$ is the maximal order in a quartic field $K$, then using the inverse 
different we can embed $\spectrum(\cO)\hookrightarrow \dP_\dZ^2$. From 
\cite{ce96}, such subschemes of $\dP_\dZ^2$ are cut out by pairs of ternary 
quadratic forms. 





\subsection{Quintic rings}

In \cite{b08}, it is proved that there is a bijection between isomorphism 
classes of pairs $(R,S)$, where $R$ is a quintic ring and $S$ is a 
sextic resolvent of $R$, and $\generallinear_4(\dZ)\times \speciallinear_5(\dZ)$-orbits 
of quadruples of $5\times 5$ skew-symmetric matrices (alternatively, quinary 
alternating forms). 

We'll briefly describe the sextic resolvent of a quintic ring. At the level of 
Galois groups of fields, we're looking for an index $6$ subgroup of 
$S_5$. Make a pentagon out of $\{1,2,3,4,5\}$. There are $6$ ways to put them 
into two disjoint $5$-cycles. The permutations that fix this decomposition into 
$2$-cycles gives us such a subgroup. 

The resolvent associated to $(R,S)$ is a map $\Lambda^2 S^\vee \to R^\vee$. 
From a geometric perspective: over $\dQ$, in $\dP_{\overline\dQ}^3$, we have 
$A=A_1 x+A_2 y+A_3 z+A_4 w$, where $A_1$ is an alternating $5\times t$ matrix. 
The matrix $A$ has five skew-symmetric $4\times 4$ minors. The determinant of 
such a minor is a a square, so the Pfaffian $\sqrt{\det(\text{minor})}$ is a 
quadratic form in $4$ variables. So from $A$ we get forms $Q_1,\dots,Q_5$. 
These cut out a subvariety of $\dP_{\overline\dQ}^3$, which is the analogue 
of the ring $R$. 

We should expect the quadratic forms $Q_1,\dots,Q_5$ to have no common zeros! 
But when a $5$-tuple of quadratic forms come from Pfaffians of an alternating 
matrix, the subscheme they cut out is non-empty. 

We would have $\spectrum(\cO)\subset \dP_\dZ^3$. There is a geometric analogue 
of this in \cite{be77} which shows that such a subvariety of $\dP^3$ has to be 
cut out by $5$ quadratic forms which come from Pfaffians of an alternating 
matrix. 



