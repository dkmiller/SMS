% !TEX root = sms.tex

\section{Selmer groups and heuristics II}\label{sec:poonen-iii}
\thanksauthor{Bjorn Poonen}





The main references for this lecture are \cite{bklpr13} and 
\emph{Boundedness of rank}, by Derek Garton, Jennifer Park, Jon Voight, 
Melanie Matchett Wood. 

In \autoref{sec:poonen-ii}, we gave a conjecture predicting the average size of 
$\selmer_n E$ as $E$ ranges over elliptic curves over $\dQ$. At the end, we 
passed to the $p^\infty$-component: there is an exact sequence 
\begin{equation*}\tag{$\sequence_E$}\label{eq:seq-E}
\xymatrix{
  0 \ar[r] 
    & E(\dQ)\otimes (\dQ_p/\dZ_p) \ar[r] 
    & \selmer_{p^\infty}(E) \ar[r] 
    & \sha(E)[p^\infty] \ar[r] 
    & 0 .
}
\end{equation*}
We are going to try to predict the distribution of this sequence. 





\subsection{The orthogonal Grassmannian}

Recall that for a ring $A$, $A$-valued points $\grassmannian_{n,2n}(A)$ of 
the $(n,2n)$-Grassmannian consists of locally free rank-$n$ $A$-submodules 
$Z\subset A^{2n}$ such that $Z$ is a direct summand. There is a subfunctor 
$\orthogonalgrassmannian_n\subset \grassmannian_{n,2n}$, for which 
$\orthogonalgrassmannian_n(A)$ is the set of $Z\in \grassmannian_{n,2n}(A)$ 
such that $Q|_Z=0$. 

Both the functors $\grassmannian_{n,2n}$ and $\orthogonalgrassmannian_n$ are 
represented by smooth projective schemes over $\dZ$. The orthogonal 
Grassmannian $\orthogonalgrassmannian_n$ has two connected components. For 
any field $k$, $Z,Z'\in \orthogonalgrassmannian_n(k)$ are in the same component 
if and only if $\dim(Z\cap Z')\equiv n\pmod 2$. 

The fact that $\orthogonalgrassmannian_n$ is smooth tells us that the fibers in 
each level of the inverse system 
\[
  \cdots \to \orthogonalgrassmannian_n(\dZ/p^{e+1}) \to \orthogonalgrassmannian_n(\dZ/p^e) \to \orthogonalgrassmannian_n(\dZ/p^{e-1}) \to \cdots 
\]
have constant cardinality. We get a ``uniform'' probability measure on 
$\orthogonalgrassmannian_n(\dZ_p)=\varprojlim\orthogonalgrassmannian_n(\dZ/p^e)$ 
by taking the inverse limit of the uniform measure on each 
$\orthogonalgrassmannian_n(\dZ/p^e)$. 





\subsection{The model}

Start with $V=\dZ_p^{2 n}$. Fix $W=\dZ_p^n\times 0\subset V$; this is maximal 
isotropic. Choose a random $Z\in \orthogonalgrassmannian_n(\dZ_p)$. Form the 
sequence 
\begin{equation*}\tag{$\sequence_Z$}\label{eq:seq-Z}
\xymatrix@=0.4cm{
  0 \ar[r] 
    & (Z\cap W)\otimes (\dQ_p/\dZ_p) \ar[r] 
    & (Z\otimes \dQ_p/\dZ_p)\cap (W\otimes \dQ_p/\dZ_p) \ar[r] 
    & \textnormal{cokernel} \ar[r] 
    & 0 .
}
\end{equation*}
So from a random $Z\in \orthogonalgrassmannian_n(\dZ_p)$ we get a random 
sequence of co-finitely generated $\dZ_p$-modules. The following theorem is 
proved in \cite{bklpr13}. 

\begin{theo}
The limit as $n\to \infty$ of the distribution of $\sequence_V$ exists.
\end{theo}

\begin{conjecture}
The limit as $n\to \infty$ of the distribution of the $\sequence_V$ is the 
distribution of $\sequence_E$ for $E\in \cE$. That is, if $\cS$ is a short 
exact sequence of $\dZ_p$--modules, then 
\[
  \lim_{n\to \infty}\probability_{Z\in \orthogonalgrassmannian_n(\dZ_p)}(\sequence_Z\simeq \cS) = \probability_{E\in \cE}(\sequence_E\simeq \cS) .
\]
\end{conjecture}





\subsection{Consequences for rank}

We have $(Z\cap W)\otimes \dQ_p/\dZ_p = (\dQ_p/\dZ_p)^r$, where 
$r=\dim_{\dQ_p}(Z\otimes \dQ_p \cap W\otimes \dQ_p)$. Outside a (measure zero) 
lower-dimensional locus, the rank is $0$ for $Z$ in one component of 
$\orthogonalgrassmannian$, and $1$ for $Z$ in the other component. 

\begin{coro}
The conjecture implies that $50\%$ of elliptic curves have rank $0$, 
$50\%$ have rank $1$, and $0\%$ have rank $\geqslant 2$. 
\end{coro}





\subsection{Consequences for \texorpdfstring{$\sha$}{Sha}}

Write $\sequence_Z$ as $0\to R \to S \to T \to 0$. 

\begin{prop}
In $\sequence_Z$, $R$ is the maximal divisible subgroup of $S$ and $T$ is 
finite (for each $n$). 
\end{prop}

\begin{coro}
The conjecture implies that $\sha(E)[p^\infty]$ is finite for 
$100\%$ of elliptic curves over $\dQ$. 
\end{coro}

There are three distributions on the set of isomorphism classes of finite 
abelian $p$-groups, each conjectured to be the distribution of 
$\sha(E)[p^\infty]$ for $E\in \cE$ of rank $r$. 

\begin{prediction}
This is due to Delaunay \cite{d01}. As $E$ ranges over elliptic curves over 
$\dQ$ with rank $r$, the distribution of $\sha(E)[p^\infty]$ is 
\[
  \probability(G) = \frac{\#G^{1-r}}{\#\automorphism(G,[\cdot])}\prod_{i=r+1}^\infty (1-p^{1-2i}) .
\]
\end{prediction}

\begin{prediction}
This is due to \cite{bklpr13}. Choose 
$Z\in \orthogonalgrassmannian_n(\dZ_p)$ such that $\rank(Z\cap W)=r$. Form 
$0 \to R\to S \to T\to 0$, and take 
\[
  \lim_{n\to \infty} (\textnormal{distribution of }T) .
\]
Then $\sha(E)[p^\infty]$ is distributed according to this limit. 
\end{prediction}

\begin{prediction}
This is also due to \cite{bklpr13}. Choose random $A\in M_{2n+r}(\dZ_p)$ such 
that $\rank A=2n$ and $\transpose A=-A$. We have an exact sequence 
\[\xymatrix{
  \dZ_p^{2n+r} \ar[r]^-A 
    & \dZ_p^{2n+r} \ar[r] 
    & \coker A \ar[r] 
    & 0 .
}\]
Then $\sha(E)[p^\infty]$ is distributed as 
\[
  \lim_{n\to\infty} (\textnormal{distribution of }(\coker A)_\mathrm{tors}) .
\]
This is similar to the Friedman-Washington heuristic for class groups, 
which chooses a random $A\in M_n(\dZ_p)$, and takes 
\[
  \lim_{n\to \infty} (\textnormal{distribution of }\coker A) .
\]
They prove that this distribution is identical to the Cohen-Lenstra 
distribution for $\class(k)[p^\infty]$, as $k$ ranges over imaginary quadratic 
fields. 
\end{prediction}

The following is one of the main theorems of \cite{bklpr13}. 

\begin{theo}
For each $r\geqslant 0$, the three distributions above coincide. 
\end{theo}





\subsection{Heuristics for boundedness of \texorpdfstring{$\rank E$}{rk E}}

Consider an elliptic curve $E$ with conductor / discriminant / height $N$. 
Choose a random $A\in M_n(\dZ)$ such that $\transpose A=-A$, with entries 
of size $X=X(N)$. The function $X$ will be determined later. Then 
\begin{itemize}
  \item $(\coker A)_\mathrm{tors}$ models $\sha(E)$ 
  \item $\rank_\dZ(\coker A)$ models $\rank E$
\end{itemize}

The following has not been previously written up. For each $r\geqslant 1$, 
$\probability(\rank E\geqslant r)$ should be modeled by 
$\probability(\rank A\leqslant n-r)$. It is a ``probable theorem'' that 
\[
  \probability(\rank A\leqslant n-r) \sim \frac{X^{n(n-r)/2}}{X^{n(n-1)/2}} \sim \frac{1}{(X^{n/2})^{r-1}} .
\]
where $n\equiv r\pmod 2$. (The analogous problem for symmetric integer matrices 
is an honest theorem proved in \cite{ek95}.) 

There is a heuristic to suggest $X^{n/2}\sim N^{1/24}$. When $r=2$, random 
matrix theory already suggests an answer. Consider $E$ with the sign in the 
functional equation being $+1$ (so $E$ should have even rank). Define 
\[
  \# \sha_0 = \begin{cases} \# \sha & \rank E=0 \\ 0 & \text{otherwise} \end{cases}
\]
Part of the Birch and Swinnerton-Dyer conjecture says that 
\[
  L(E,1) = \frac{\# \sha_0 \Omega \prod c_p}{\# E(\dQ)_\mathrm{tors}^2} .
\]
(If $\rank E>0$, then both sides should be zero.) Solving for 
$\sqrt{\#\sha_0}$ (and throwing out $\#E(\dQ)_\mathrm{tors}$, the Tamagawa 
numbers, and $L(E,1)$) we get 
\[
  \sqrt{\#\sha_0} \sim O(\Omega^{-1/2}) ,
\]
and $\Omega\sim N^{-1/12}$. Thus 
\[
  X^{-n/2} \sim \probability(\rank E\geqslant 2) = \probability(\sqrt{\#\sha_0} = 0) \sim N^{-1/24}
\]
so we conclude that 
\[
  \probability(\rank E\geqslant r) \sim N^{(1-r)/24} .
\]

There are $N^{5/6}$ elliptic curves of height $\leqslant N$. If $r-1>20$, we 
would expect finitely many elliptic curves of rank $\geqslant r$. 

\begin{prediction}
$\rank E\leqslant 21$, with finitely many exceptions. 
\end{prediction}

In particular, there is a global bound for the rank of elliptic curves over 
$\dQ$. 




