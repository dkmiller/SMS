% !TEX root = sms.tex

\section{Rational points on curves}
\thanksauthor{Michael Stoll}





The goal is to discuss how one can show that a hyperelliptic curve has no 
rational points. 





\subsection{Hyperelliptic curves}

Let $k$ be a field of characteristic not $2$. 

\begin{defi}
A \emph{hyperelliptic curve} over $k$ is the smooth projective curve associated 
to an affine curve of the form $y^2=f(x)$ with $f\in k[x]$ squarefree (and 
$\deg f\geqslant 5$). 
\end{defi}

We will write $C:y^2=f(x)$; $C$ means the projective curve associated to 
the subvariety $V(y^2-f)$ of the affine plane. We can define the projective 
curve $C$ as follows: homogenize $f$ as $f(x)=F(x,1)$ with 
$F\in k[x,z]$ squarefree and homogeneous of even degree. Then 
$y^2=F(x,z)$ describes $C\subset \dP_{1,g+1,1}^2$ when $\deg F=2 g+2$. 

There are points at infinity. If 
$f(x)=f_{2 g+2} x^{2 g+2} + \cdots + f_1 x + f_0$, then 
$F(x,z) = f_{2 g+2} x^{2 g+2} + \cdots + f_0 z^{2 g+2}$. If $\deg f$ is odd, 
there is just one point $\infty=(1:0:0)$ at infinity. If $\deg f$ is even, 
then there are two points at infinity: $\infty_{\pm s} = (1:\pm s:0)$, where 
$f_{2 g+2}=s^2$. If $f_{2 g+2}$ is not a square in $k$, these points at 
infinity will not be defined over $k$. 

It is known that $C$ has genus $g$. The set of $k$-rational points of $C$ is 
\[
  C(k) = \{(\xi,\eta)\in k^2 : \eta^2 = f(\xi)\} \cup 
  \begin{cases} 
    \{\infty\} & \deg f\text{ odd} \\ 
    \{\infty_s,\infty_{-s}\} & \deg f\text{ even, $f_{2 g+2}$ a square} \\ 
    \varnothing & \text{otherwise} 
  \end{cases}
\]

The condition $\deg f\geqslant 5$ implies $g\geqslant 2$. Faltings' theorem 
tells us that $C(\dQ)$ is finite. Our motivating problem is: determine 
$C(\dQ)$ explicitly for given $C$. This is wide open at the present. 
Heuristically, we expect $100\%$ of hyperelliptic curves of genus $g$ to have 
no rational points. This needs some explanation. For a hyperelliptic curve 
$C:y^2=f$ defined over $\dQ$, we can assume $f\in \dZ[x]$. We can then order 
hyperelliptic curves by the height of $f$. 





\subsection{Local solubility}

Suppose we have some set $A$ which we want to prove is empty. One way to 
do this is to construct a map $A\to B$ and show that $B=\varnothing$. If, for 
example $A=C(\dQ)$, then for each place $v$ of $\dQ$ we have a natural 
injection $C(\dQ)\hookrightarrow C(\dQ_v)$. If $C(\dQ_v)=\varnothing$ for 
some $v$, then $C(\dQ)=\varnothing$. 

\begin{defi}
A curve $C$ is said to be \emph{everywhere locally soluble} if 
$C(\dR)\ne\varnothing$ and $C(\dQ_p)\ne\varnothing$ for all primes $p$. 
\end{defi}

Equivalently, $C$ is everywhere locally soluble if $C(\dA)\ne\varnothing$. 

\begin{theo}
If $C$ is not everywhere locally soluble, then $C(\dQ)=\varnothing$. 
\end{theo}

\begin{example}
Consider $C:y^2-x^6-x^2-17$. Then $C(\dR)=\varnothing$. 
\end{example}

\begin{example}
The curve $C:y^2=-x^6-3 x^5+4 x^4+2 x^3 +4 x^2 - 3 x-1$ has 
$C(\dQ_{11})=\varnothing$, because $C(\dF_{11})=\varnothing$. 
\end{example}

As a general principle, local questions are computable, whereas global 
questions are very hard. In our case, we have the following result. 

\begin{prop}
There is an algorithm that decides if a hyperelliptic curve over $\dQ$ is 
everywhere locally soluble or not. 
\end{prop}
\begin{proof}[Sketch of proof]
We need to take care of two problems: 
\begin{itemize}
  \item there are infinitely many places of $\dQ$  
  \item for each place $v$, the field $\dQ_v$ is uncountable 
\end{itemize}
It is easy to check whether $C(\dR)=\varnothing$. This is the case if and only 
if $f$ has no real roots, which happens exactly when $f$ is strictly positive 
or strictly negative. This is easy to decide. 
For fixed $p$, ``$C(\dQ_p)=\varnothing$'' reduces to a question modulo $p^n$ 
via Hensel's lemma. So for any completion $v$ of $\dQ$, it is a finite problem 
to check whether $C(\dQ_v)=\varnothing$. For $p$ sufficiently large, 
$p\nmid\discriminant(f)$, so we have $C(\dQ_p)\ne\varnothing$ via a theorem of 
Weil, namely $\#C(\dF_p)\geqslant p+1-2 g\sqrt p$. For 
$p\nmid\discriminant(f)$, the curve $C$ is smooth over $\dF_p$, so points in 
$C(\dF_p)$ lift to points in $C(\dQ_p)$. 
\end{proof}

A curve which is everywhere locally soluble does not necessarily have a 
solution in $\dQ$. For fixed $g\geqslant 2$, the everywhere locally soluble 
curves of genus $g$ have a density $\delta_g$. For example 
$\delta_2\approx 0.84$ and $\delta_g\to 1$ as $g\to \infty$. So we expect 
$100\%$ of hyperelliptic curves to have no rational points, but also for 
$100\%$ of hyperelliptic curves to be everywhere locally soluble. 





\subsection{Descent}

Consider $C:y^2=f(x)$ with $f(x)=f_1(x) f_2(x)$, with 
$\deg f_1,\deg f_2$ even and $f_1,f_2\in \dZ[x]$. Let $P=(\xi,\eta)\in C(\dQ)$. 
Then $f_1(\xi)\ne 0$, $f_2(\xi)\ne0$, or both. There exists a unique squarefree 
$d\in \dZ$, together with $\eta_1,\eta_2\in \dQ$, such that 
$f_1(\xi) = d\eta_1^2$, $f_2(\xi) = d\eta_2^2$. More geometrically, $P$ lifts 
to a rational point on $D_d:d y_1^2 = f_1(x),d y2^2=f_2(x)$ with 
$\pi_d:D_d\to C$ defined by $(x,y_1,y_2)\mapsto (x,d y_1 y_2)$. So we have 
reduced the problem of finding rational points on $C$ to that of finding 
rational points on the family $\{D_d:d\in \dZ\}$ of curves. 

If $p\mid d$, then $\overline\xi$ a common root of 
$\overline{f_1},\overline{f_2}\in \dF_p[x]$ implies 
$p\mid \resultant(f_1,f_2)\ne 0$. This is possible for only finitely many 
values of $d$. Let $T$ be the set of possible $d$. For each $d\in T$, we can 
use an algorithm to check if $D_d$ is everywhere locally soluble. If none of 
them are, then $C(\dQ)=\varnothing$. 

\begin{example}
Let $C:y^2=f_1 f_2$, where 
\begin{align*}
  f_1 &= -x^2-x-1 \\
  f_2 &= x^4 + x^3 + x^2 + x+2 .
\end{align*}
As an exercise, check that $C$ is everywhere locally soluble. We have 
$\resultant(f_1,f_2)=\pm 19$, so we can set $T=\{\pm 1,\pm 19\}$. If 
$d<0$, then $D_d(\dR)=\varnothing$. If $d\equiv 1\pmod 3$, then 
$D_d(\dF_3)=\varnothing$. It follows that $C(\dQ)=\varnothing$. 
\end{example}

This approach can be generalized to unramified coverings $\pi:D\to C$ that are 
Galois over $\overline\dQ$. 





\subsection{The (fake) 2-Selmer set}

Recall our strategy for showing that a set is empty. There is a more 
sophisticated version. Instead of looking at maps $A\to B$, look at 
commutative diagrams 
\[\xymatrix{
  A \ar[r] \ar[d] 
    & B \ar[d]^-\beta \\
  C \ar[r]^-\gamma 
    & D .
}\]
If $\image(\beta)\cap \image(\gamma)=\varnothing$, then $A=\varnothing$. We 
will construct such a diagram with $A=C(\dQ)$. 

Let $L=\dQ[x]/f$ and $L_v=L\otimes \dQ_v$. Write $T$ for the image of $x$ in 
$L$ and each $L_v$. Define 
\[
  H = \left\{\alpha\in L^\times/(L^\times)^2 \dQ^\times:N_{L/\dQ}(\alpha) = \leadingcoefficient(f)\cdot (\text{square})\text{ in }\dQ^\times\right\}
\]
where $\leadingcoefficient(f)$ is the leading coefficient of $f$. Similarly 
define $H_v$ for each place $v$. There are natural maps $\rho_v:H\to H_v$. 
Define $\delta:C(\dQ) \to H$ by 
\begin{align*}
  (\xi,\eta) &\mapsto (\xi-T) (L^\times)^2 \dQ^\times && \text{if }\eta\ne 0 \\
  (\xi,0) &\mapsto (\xi-T-f_1(T))(L^\times)^2 \dQ^\times && \text{if }f(x)=(x-\xi)f_1(x) \\
  \infty_{\pm s} &\mapsto (L^\times)^2 \dQ^\times .
\end{align*} 
Similarly define $\delta_v:C(\dQ_v)\to H_v$. We have a commutative diagram 
\[\xymatrix{
  C(\dQ) \ar[r]^-\delta \ar[d] 
    & H \ar[d]^-{(\rho_v)_v} \\
  C(\dA) \ar[r]^-{\prod \delta_v} 
    & \prod H_v .
}\]

\begin{defi}
The \emph{(fake) 2-Selmer set} of $C$ is 
\[
  \selmer_2^\mathrm{fake}(C) = \{\alpha\in H:\rho_v(\alpha)\in \image(\delta_v)\text{ for all }v\} .
\]
\end{defi}

If $\selmer_2^\mathrm{fake}(C)=\varnothing$, then $C(\dQ)=\varnothing$. 
There is a ``2-Selmer set'' $\selmer_2(C)$ with a map 
$\selmer_2(C) \to \selmer_2^\mathrm{fake}(C)$ that is surjective. It is 
bijective for some curves, but is usually 2-to-1. 

The set $\selmer_2^\mathrm{fake}(C)$ can be computed. Let $S$ be the set of 
places of $L$ dividing one of 
$\{2,\infty,\discriminant(f),\leadingcoefficient(f)\}$. Then 
\[
  \selmer_2^\mathrm{fake}(C)\subset H_S = \{\alpha (L^\times)^2\dQ^\times:v_\fp(\alpha)\text{ even for all }\fp\notin S\} .
\]
The set $H_S$ is finite by standard arguments. 

In \cite{bg13}, it is shown that the (upper) density of genus $g$ 
with $\selmer_2^\mathrm{fake}(C)\ne\varnothing$ is $o(2^{-g})$. 

\begin{example}[Bruin, Stoll]
Of the $\sim 200000$ isomorphism classes of genus $2$ curves of height 
$\leqslant 3$, all but $\sim 1500$ either 
\begin{itemize}
  \item have a rational point, 
  \item fail to be everywhere locally soluble, or 
  \item have empty 2-Selmer set .
\end{itemize}
\end{example}




