% !TEX root = sms.tex

\section{Basics of binary quadratic forms and Gauss composition}\label{sec:granville-i}
\thanksauthor{Andrew Granville}

An official Beamer version of these notes can be found online at 
\url{http://www.crm.umontreal.ca/sms/2014/pdf/AlgTalkSlides.pdf}. These notes 
are essentially a transcription of the official version. 





\subsection{Sums of two squares}

We start with a very old theorem, but proving it using some basic geometry of 
numbers. 

\begin{theo}
Any prime $p\equiv 1\pmod 4$ can be written as the sum of two squares. 
\end{theo}
\begin{proof}
Since $p\equiv 1\pmod 4$, there is $i\in \dZ$ such that $i^2\equiv -1\pmod p$. 
Consider the set of integers 
\[
  \{m+n i:0\leqslant m,n\leqslant \lfloor \sqrt p\rfloor\} .
\]
This set has cardinality $(\lfloor\sqrt p\rfloor+1)^2>p$, so by the pigeonhole 
principle, two are congruent modulo $p$; say 
\[
  m+n i\equiv M+N i\pmod p
\]
where $0\leqslant m,n,M,N\leqslant \lfloor\sqrt p\rfloor$ and 
$(m,n)\ne (M,N)$. Let $r=m-M$ and $s=N-n$ so that 
\[
  r\equiv i s\pmod p 
\]
where $|r|,|s|\leqslant \lfloor \sqrt p\rfloor < \sqrt p$, and $r$ and $s$ are 
both nonzero. Now 
\[
  r^2+s^2 \equiv (i s)^2 +s^2 = s^2(i^2+1) \equiv 0\pmod p
\]
and $0<r^2+s^2 < \sqrt p^2+\sqrt p^2=2 p$. The only multiple of $p$ between 
$0$ and $2 p$ is $p$, hence $r^2+s^2=p$. 
\end{proof}

It is also well-known that a product of sums of squares is a sum of squares, 
via the identity 
\[
  (a^2+b^2)(c^2+e^2) = (a c+ b e)^2 + (a e-b c)^2 .
\]
This has the easy generalization 
\[
  (a^2+d b^2)(c^2+ d e^2) = (a c+d b e)^2 + d(a e- b c)^2 .
\]
Gauss's perspective on this was as follows. A \emph{binary quadratic form} is 
a polynomial of the form $f(x,y) = a x^2 + b x y + c y^2$. If we choose 
$f(x,y) = x^2+ d y^2$, then 
\[
  f(a,b)f(c,e) = f(a c+d b e,a e-b c) .
\]
The latter values in $f$, namely $a c+d b e$ and $a e-b c$, are bilinear 
forms in $a,b,c,e$. Does this generalize to other ``multiplications''? 






\subsection{Pell's equation}

Pell asked whether there are integer solutions $x,y$ to 
\[
  x^2 - d y^2 = 1 .
\]
Solutions always exist, and can be found using the continued fraction expansion 
of $\sqrt d$. This was certainly known by Brahmagupta in 628 A.D.~and probably 
by Archimedes much earlier. To solve his ``Cattle Problem,'' one needs to find 
a solution to 
\[
  x^2 - 4729494 y^2 = 1.
\]
The smallest solution has about $2\cdot 10^6$ digits! 


\begin{theo}
Let $d\geqslant 2$ be a non-square integer. Then there exist 
$x,y\in \dZ$ such that $y\ne 0$ and 
\[
  x^2 - d y^2 = 1. 
\]
If $(x,y)$ is the smallest positive solution, then all others are given by 
\[
  x_n + y_n \sqrt d = (x+y\sqrt d)^n .
\]
\end{theo}

A more interesting problem is to look for solutions to 
$x^2-d y^2 = \pm 4$. Especially with a ``$-$,'' this is very difficult, though 
there has great recent progress by Fouvry and Kluners. 

\begin{theo}
Any quadratic irrational real number has an eventually periodic continued 
fraction. 
\end{theo}

Here are some examples:
\begin{center}
\begin{tabular}{l|l}
$d$ & expansion of $\sqrt d$ \\ \hline
2 & $[1,\overline 2]$ \\
3 & $[1,\overline{1,2}]$ \\
5 & $[2,\overline 4]$ \\
6 & $[2,\overline{2,4}]$ \\
7 & $[2,\overline{1,1,1,4}]$ \\
8 & $[2,\overline{1,4}]$ \\
10 & $[3,\overline 6]$ \\
11 & $[3,\overline{3,6}]$ \\
12 & $[3,\overline{2,6}]$ \\
13 & $[3,\overline{1,1,1,1,6}]$ 
\end{tabular}
\end{center}
If $p_k/q_k$ are the convergents for $\sqrt d$, then 
$p_{n-1}^2 - d q_{n-1}^2 = (-1)^n$. Here are some examples of how for out one 
has to go to obtain a solution to Pell's equation:
\begin{center}
\begin{tabular}{c|ll}
$d$ & expansion of $\sqrt d$ & Pell's equation \\ \hline 
$2$ & $[1,\overline 2]$ & $1^2 - 2\cdot 1^2 = -1$ \\
$3$ & $[1,\overline{1,2}]$ & $2^2 - 3\cdot 1^2 = 1$ \\
$6$ & $[2,\overline{2,4}]$ & $5^2 - 6\cdot 2^2 = 1$ \\
$7$ & $[2,\overline{1,1,1,4}]$ & $8^2 - 7\cdot 3^2 = 1$ \\
$13$ & $[3,\overline{1,1,1,1,6}]$ & $18^2 - 13\cdot 5^2 = -1$ \\
$19$ & $[4,\overline{2,1,3,1,2,8}]$ & $170^2 - 19\cdot 39^2 = 1$ \\
$22$ & $[4,\overline{1,2,4,2,1,8}]$ & $197^2 - 22\cdot 42^2 = 1$ \\
$31$ & $[5,\overline{1,1,3,5,3,1,1,10}]$ & $1520^2 - 31\cdot 273^2 = 1$ \\
$43$ & $[6,\overline{1,1,3,1,5,1,3,1,1,12}]$ & $3482^2 - 43\cdot 531^2 = 1$ \\
$46$ & $[6,\overline{1,3,1,1,2,6,2,1,1,3,1,12}]$ & $24335^2 - 46\cdot 3588^2 = 1$ \\
$76$ & $[8,\overline{1,2,1,1,5,4,5,1,1,2,1,16}]$ & $57799^2 - 76\cdot 6630^2 = 1$
\end{tabular}
\end{center}
It is clear that the period of the continued fraction expansion grows quite 
quickly. Here is a table of the first few largest periods:
\begin{center}
\begin{tabular}{c|l}
$n$ & equation \\ \hline
16 & $2143295^2 - 94\cdot 221064^2 = 1$ \\
16 & $4620799^2 - 124\cdot 414960 2 = 1$ \\
16 & $2588599^2 - 133\cdot 224460 2 = 1$ \\
18 & $77563250^2 - 139\cdot 6578829 2 = 1$ \\
20 & $1728148040^2 - 151\cdot 140634693^2 = 1$ \\
22 & $1700902565^2 - 166\cdot 132015642^2 = 1$ \\
26 & $278354373650^2 - 211\cdot 19162705353^2 = 1$ \\
26 & $695359189925^2 - 214\cdot 47533775646^2 = 1$ \\
26 & $5883392537695^2 - 301\cdot 339113108232^2 = 1$ \\
34 & $2785589801443970^2 - 331\cdot 153109862634573^2 = 1$ \\
37 & $44042445696821418^2 - 421\cdot 2146497463530785^2 = -1$ \\
40 & $84056091546952933775^2 - 526\cdot 3665019757324295532^2 = 1$ \\
42 & $181124355061630786130^2 - 571\cdot 7579818350628982587^2 = 1$
\end{tabular}
\end{center}
The length of the continued fractions here are around $2\sqrt d$, and the 
size of the fundamental solutions is around $10^{\sqrt d}$. It is believed that 
the smallest solution is of size $C^{\sqrt d}$ for some $C$, but very little 
has been proved. Understanding the distribution of sizes of the smallest 
solutions to Pell's equation is an outstanding open question in number theory. 

There is one last fact we need on Pell's equation. Let 
$\epsilon_d=x_1+y_1\sqrt d$ be the smallest solution with $x_1,y_1\in \dN$ to 
\[
  x^2 - d y^2 = 1 .
\]
If $x^2 - d y^2=n$ and $x,y\geqslant 0$, put 
$\alpha = x+y\sqrt d>\sqrt n$. If 
$\sqrt n\epsilon_d^k \leqslant \alpha < \sqrt n \epsilon_d^{k+1}$, let 
$\beta=\alpha \epsilon_d^{-k} = u+v\sqrt d$. Then 
$\sqrt n\leqslant \beta < \sqrt n\epsilon_d$, with $u,v\geqslant 1$ and 
$u^2 - d v^2 = n$. 





\subsection{Binary quadratic forms}

Let $f(x,y)=a x^2 + b x y + c y^2$ be a binary quadratic form. The basic 
question is: what integers are represented by $f$? Namely, for what $N$ are 
there coprime $m,n$ so that $N=f(m,n)$? We may assume $\gcd(a,b,c)=1$. 
Completing the square, we obtain 
\[
  4 a N = (2 a m+b n)^2  d n^2
\]
where the discriminant $d=b^2 -4 a c$, so $d$ is congruent to either $0$ or 
$1$ modulo $4$. If $d<0$, the right side can only take positive values\ldots 
things are easier when $d>0$. If $a>0$, the form is positive definite. 





\subsection{Structure of ideals}

\subsection{Composition}

\subsection{Prehomogeneous vector spaces}

\subsection{Invariants of number fields}


