% !TEX root = sms.tex

\section{Basic algebraic number theory}\label{sec:3}
\thanksauthor{Eknath Ghate}





Essentially, this lecture will try to cover two semester-long courses 
(algebraic number theory and class field theory) in an hour. Hopefully, we'll 
focus on the theory of Hilbert class fields, and later on complex 
multiplication. 





\subsection{Number fields}

A \emph{number field} $K$ is a finite field extension of $\dQ$. An important 
invariant of $K$ is its \emph{class group}, $\class_K=I_K/P_K$, where 
$I_K$ is the group of all fractional ideals in $K$ and $P_K$ is the group of 
principal fractional ideals. 

\begin{theo}
The group $\class_K$ is finite. 
\end{theo}

So we can define the \emph{class number} $h_K$ of $K$ to be the cardinality 
of $\class_K$. The class number $h_K=1$ if and only if $\cO_K$ is a principal 
ideal domain. The famous \emph{Dirichlet unit theorem} says that $\cO_K^\times$ 
is a finitely-generated abelian group, and gives a formula for 
$\rank\cO_K^\times$ in terms of the number of real and complex places of $K$. 

We would like to relate $h_K$ with quadratic forms. Consider towers of 
fields 
\[\xymatrix@=0.5cm{
  & L \\
  F \ar@{-}[ur] & & K \ar@{-}[ul] \\
  & \dQ \ar@{-}[ul]^-2 \ar@{-}[ur]_-3 \ar@{-}[uu]^-{S_3} 
}\]
where $L/F$ is an unramified (cubic) extension. Then counting the number of 
cubic fields $K$ that are nowhere ramified with $|\discriminant K|<X$ is 
equivalent to summing $\# h_3(F)$ for $|\discriminant F|<X$. The average of 
$h_3(F)$ is $\frac 4 3$ for $F$ real quadratic, and $2$ for $F$ imaginary 
quadratic. By the ``average'' we mean, for example, 
\[
  \sum_{0<|\discriminant F|<X} h_3(F) \sim \frac 4 3 X 
\]
in the case of real quadratic $F$. The point here is that averaging class 
numbers is equivalent to counting certain types of fields. 





\subsection{Starting point of class field theory}

Class field theory is, in general, the study of abelian extensions of a field 
$k$. Let $K$ be a number field, and let $H/K$ be the maximal unramified 
abelian extension of $K$. One calls $H$ the \emph{Hilbert class field} of 
$K$. It is known that $H$ is a number field. Moreover, the \emph{Artin map} 
$I_K \to \galois(H/K)$ determined by $\fp\mapsto \frobenius_\fp = (\fp,H/K)$ 
induces an isomorphism $\class_K\iso \galois(H/K)$. Recall that the 
\emph{Frobenius element} $\frobenius_\fp\in \galois(H/K)$ is characterized by 
$\frobenius_\fp(x)\equiv x^{N_{K/\dQ}(\fp)}\pmod\fP$ for all $\fP\mid \fp$ in 
$H$. 

It is easy to show that $\fp$ splits compltely in $H$ if and only if $\fp$ is 
principal. Somewhat harder is the principal ideal theorem: 

\begin{theo}
Every ideal of $K$ becomes principal in $H$. 
\end{theo}
\begin{proof}
Let $G=\galois(K^\mathrm{ur}/K)$, and let $G_1\subset G$ correspond to the 
extension $H/K$. We have a commutative diagram: 
\[\xymatrix{
  \class_H \ar[r]^-{\artin} 
    & G_1^\abelian \\
  \class_K \ar[u] \ar[r]^-{\artin}
    & G^\abelian \ar[u]_-V 
}\]
where $V$ is the \emph{transfer map}. Since $G_1$ corresponds to $H$, 
$V=0$, whence $\class_K\to \class_H$ is the trivial map. The result 
follows. 
\end{proof}

If $K=\dQ$, then $H=\dQ$. 

Assume from now on that $K=\dQ(\sqrt d)$ is an imaginary quadratic field 
($d<0$). There are nine possible values of $d<0$ for which 
$h_{\dQ(\sqrt d)}=1$, namely 
\[
  -1,-2,-3,-7,-11,19,-43,-67,-163 .
\]
It is known that $h_K\to \infty$ as $d\to -\infty$. 

\begin{enonce}[remark]{Example}
Let $K=\dQ(\sqrt{-23})$. Then $H/K$ is cyclic of degree $3$. It is known 
that $H=K(\alpha)$, where $\alpha^3-\alpha+1=0$. 
\end{enonce}





\subsection{Complex multiplication}

One might ask if for any number field $K$, there is an explicit way of 
finding $\alpha\in \overline\dQ$ such that $H=K(\alpha)$. The theory of complex 
multiplication describes how to do this explicitely whenever $K$ is 
imaginary quadratic. The main theorem is the following: 

\begin{theo}
Let $E$ be an elliptic curve with CM by $\cO_K$. Then 
$H=K(j(E))$. 
\end{theo}

Recall that $E$ has \emph{complex multiplication} by $\cO_K$ if 
$\End E \simeq \cO_K$. If $E+y^2=4 x^3-g_2 X-g_3$, then its 
\emph{$j$-invariant} is given by 
\[
  j(E) = 1728 \frac{g_2^3}{g_2^3 - g_3^2} .
\]
Alternatively, if $E=\dC/\langle 1,\tau\rangle$, then 
$j(E) = \frac 1 q + 744 + 196884 q^2 + \cdots$, where $q=e^{2\pi i \tau}$. 

The rest of this lecture will be a sketch of a proof of this theorem. 





\subsection{Elliptic curves from fractional ideals}

Let $\cE_\dC(K)$ be the set of isomorphism classes of elliptic curves over 
$\dC$ with complex multiplication by $\cO_K$. There is a bijection 
$\class_K \iso \cE_\dC(K)$. Given $\fa\subset \cO_K$, we have a canonical 
embedding $\fa\hookrightarrow \dC$. Send $[\fa]$ to the elliptic curve 
$\dC/\fa$. This gives us a simply transitive action of $\cO_K$ on 
$\cE_\dC(K)$ via $[\fa]\cdot (\dC/\fb) = \dC/(\fa^{-1} \fb)$. 





\subsection{Fields of definition}

It is known that CM elliptic curves have rational models. In other words, 
the natural map $\cE_{\overline\dQ}(K) \to \cE_\dC(K)$ induced by an 
embedding $\overline\dQ\hookrightarrow\dC$ is a bijection. We know this 
because $E$ is \emph{always} defined over $\dQ(j(E))$. But when $E$ is CM, 
$j(E)\in \overline\dQ$, so $E$ is defined over $\overline\dQ$. Indeed, if 
$\sigma\in \automorphism\dC$, then $j(E^\sigma)$ also has CM by $K$. So the 
orbit of $E$ under $\automorphism(\dC)$ lies in $\cE_\dC(K)$, a finite set. 
This tells us that $\cE_{\overline\dQ}(K) \twoheadrightarrow \cE_\dC(K)$. We 
leave injectivity as an exercise. 

Because of this, we will write $\cE(K)$ instead of $\cE_{\overline\dQ}(K)$ or 
$\cE_\dC(K)$. 





\subsection{Towards \texorpdfstring{$H$}{H}}

Fix $E\in \cE(K)$. For each $\sigma\in \galois(\overline\dQ/K)$, there is a 
unique $[\fa]\in \class_K$ such that $E^\sigma = [\fa]\cdot E$. We define a map 
$F:\galois(\overline\dQ/K) \to \class_K$ by $F(\sigma) = [\fa]$. 

\begin{prop}
1. $F$ does not depend on the choice of $E$. 

2. $F$ is a homomorphism. 
\end{prop}
\begin{proof}
1. This is subtle. 

2. Let $\sigma,\tau\in \galois(\overline\dQ/K)$. Then 
\begin{align*}
  F(\sigma\tau) E 
    &\simeq E^{\sigma\tau} \\
    &= (E^\sigma)^\tau && \text{part 1}\\
    &= (F(\sigma) E)^\tau \\
    &= F(\tau)(F(\sigma) E) \\
    &= (F(\sigma)F(\tau)) E && \text{$\class_K$ is abelian}
\end{align*}
\end{proof}

Let $L$ be the fixed field of $\ker(F)$. We claim that 
$L=K(j(E))$. Indeed, 
\begin{align*}
  \galois(\overline\dQ/L)
    &=\{\sigma\in \galois(\overline\dQ/K):E^\sigma \simeq F(\sigma) E \simeq E\} \\
    &= \{\sigma:j(E)^\sigma = j(E)\} .
\end{align*}

Note that $F$ is an injection $\galois(L/K)\hookrightarrow \class_K$. So 
$L/K$ is abelian. We will show that it is unramified of the right degree. 

Let $\fm$ be the conductor of $L/K$. It is the greatest common divisor of all 
$\fm\subset \cO_K$ such that $K_{\fm,1}\subset \ker(\artin_{L/K})$, where 
$K_{\fm,1} = \langle (\alpha):\alpha\simeq 1\pmod\fm\rangle$. One checks that 
the composite 
\[\xymatrix{
  I_K^\fm \ar[r]^-{\artin} 
    & \galois(L/K) \ar[r]^-F 
    & \class_K 
}\]
is the ``identity map'' $\fa\mapsto [\fa]$. It follows that $F$ is surjective. 
So $F:\galois(L/K)\iso \class_K$, and $L=K(j(E))$. 

If $F(((\alpha),L/K)) = 1$ then $(\alpha)\in I_K^\fm$ is principal. Indeed, it 
suffices to show $((\alpha),L/K)=1$, and this follows from the injectivity of $F$. 
Since $\alpha$ was arbitrary, $\fm=1$, so $L/K$ is unramified. The rest is a 
simple dimension argument. 

\begin{theo}
If $E$ is CM, then $j(E)$ is an algebraic integer. 
\end{theo}

As a corollary, one has the surprizing fact that $e^{\pi\sqrt{163}}$ is 
very close to an integer. The main theorem of complex multiplication has an 
analogue for ray class fields. One gets 
$\operatorname{RCF}(\fm) = K(j(E),h(E[\fm]))$, where $\fm\subset \cO_K$ and $h$ 
is the Weber function. 




