% !TEX root = sms.tex

\section{Future perspectives}
\thanksauthor{Manjul Bhargava}





\subsection{Future directions / things to think about after the workshop}

We now have a general technique to count (nondegenerate, a.k.a.~irreducible) 
orbits having bounded invariants in a representation $G(\dZ)$ acting on 
$V(\dZ)$, for a representation $(G,V)$ defined over $\dZ$. Our first goal 
is thus to find interesting representations! Of particular interest would be 
representations where the orbits correspond to objects or arithmetic / 
geometric / topological interest. 

In the future, it would be nice to parameterize rings of rank $n>5$ 
(i.e.~parameterize $n$ points in $\dP^{n-2}$). For example, three points in 
$\dP^1$ are parameterized by binary cubic forms. Four points in 
$\dP^2$ are parameterized by pencils of conics, i.e.~pairs of ternary 
quadratic forms, which we know correspond to quartic rings. 
Five points in $\dP^3$ arise as quadruple of $5\times 5$ symmetric matrices via 
the Pfaffians of their linear combinations. This representation of 
$\speciallinear(4)\times \speciallinear(5)$ parameterizes quintic rings. 
In general, we would like to understand $n$ points in $\dP^{n-2}$ in terms of 
forms. Even for $n=6$, we do not know how to do this (in terms of 
prehomogeneous vector spaces or something similar). 

It would also be nice to parameterize $n$-Selmer elements of elliptic curves 
for $n>5$. Geometrically, we want to parameterize maps of genus one curves $C$ 
to $\dP^{n-1}$ of degree $n$. For example, maps $C\to \dP^1$ of degree $2$ 
ramify at four points, so they correspond to binary quartic forms. Maps 
$C\to \dP^2$ are plane cubics, parameterized by ternary cubic forms. Maps 
$C\to \dP^3$ are the intersection of two quadrics, hence parameterized by 
pairs $2\otimes \symmetric^2(4)$. Finally, maps $C\to \dP^4$ are the 
intersection of five quadrics, which come from $5\otimes \bigwedge^2 5$ 
(quintuples of $5\times 5$ skew-symmetric matrices). 

All these constructions were (in some form) known in the 19th century, though 
not over $\dZ$. If we added yet another variable, we would go from points, to 
genus one curves, to K3 surfaces. See \cite{bhk13} for work along these lines. 

We could look for analogous maps for objects involving surfaces or 
higher-dimensional varieties. Also, we could try to count objects parameterized 
by non-coregular surfaces, as in \cite{by13}. Aside from a few ``baby steps,'' 
essentially nothing is known. 

Parameterizing rings of rank $\leqslant 5$ allowed us to parameterize number 
fields. We could add more structure and try to parameterize fields with special 
Galois groups. For example, we don't know how to count $A_4$-quartic, 
$A_5$-quintic, or $D_5$-quintic fields with bounded discriminant. Some work 
along these lines is in \cite{bs13}, in which the action of 
$\specialorthogonal(x^2+x y+y^2)$ on a 2-dimensional space is used to 
parameterize $C_3$-cubic rings. The subspace of pairs of ternary quadratic 
forms 
\[
  \left(\begin{pmatrix} 0 & 0 & 0 \\ 0 & \ast & \ast \\ 0 & \ast & \ast \end{pmatrix}, \ast \right) 
\]
is preserved by a parabolic in $\speciallinear(2)\times \speciallinear(3)$. 
This group parameterizes $D_4$-quartic orders. 

We could try to parameterize $n$-Selmer elements in families of 
elliptic curves with extra structures (e.g.~marked rational points). We know 
how to parameterize 2 rational points, but have no idea how to do $k>2$ 
rational points. 

Similarly, we could try to parameterize $n$-Selmer group / set elements of 
higher genus curves, or $k$-tuples of elements of the this type with given 
invariants. (This is needed to get the $k$-th moments.) For example, 
$2\times 2\times 2$ cubes up to $\speciallinear_2(\dZ)$-action parameterize 
pairs of ideal classes in a cubic ring. Kevin Wilson has done some work along 
these lines. 

There are lists of coregular representations. One way of proving new theorems 
is to look through these lists and try to find interesting geometric 
interpretations of these representations. 

Families of modular forms could also be attacked via orbit-spaces of 
representations. 

Is there a systematic way of constructing ``good'' representations for a given 
arithmetic object? Coregular representations always seem to work. There is 
great work of Jack Thorne \cite{t13} in this direction, using Vinberg theory. 
Also there is upcoming work of Ho / Sam. 

Currently, we have no way of showing that a given group action on a 
unirational variety can't parameterize something interesting. Prehomogeneous 
vector spaces and coregular spaces have \emph{not} all been classified. The 
classification in \cite{sk77} is only of irreducible reductive reduced 
prehomogeneous vector spaces. Even their list is not finite -- it contains 
some infinite families. There are already known interesting 
(a.k.a.~parameterize something interesting) prehomogeneous 
vector spaces that are non-reductive or non-reduced, for examples. It would be 
very nice if there were a complete classification of prehomogeneous vector 
spaces. 

It's important to further develop counting techniques. Right now, we have 
two methods: geometry of number and zeta functions. So far, zeta-function 
methods have not worked on higher-dimensional representations. Hopefully, there 
is a way of unifying these (or just using both) to strengthen results. The 
adelic counting method described in \cite{p13} can at least explain some of the 
cancellations that we saw in \autoref{sec:shankar-ii}. 

We could try to connect with other problems in other areas. For example, 
Miller has maps from families of knots with given invariants into certain 
orbit spaces. Also, we could to connect counting techniques as in 
\autoref{sec:wang-ii} with topological techniques as in 
\autoref{sec:ellenberg-ii}. Chabauty methods as in \autoref{sec:poonen-iv} 
and \autoref{sec:park} could also be strengthened. Finally, we should try to 
develop better heuristics for all of the above, along with arithmetic 
justifications for the heuristics. Ellenberg and Venjakob have results in this 
direction for imaginary quadratic fields. Finally, Skinner and Urban's work 
\cite{su14} on $p$-adic $L$-functions vis-\`a-vis the Iwasawa main conjecture 
for $\generallinear(2)$ is also relevant. 





\subsection{Applications to the Birch and Swinnerton-Dyer conjecture}

This is partially an advertisement for the November workshop ``Counting 
arithmetic objects (ranks of elliptic curves)'' 
(\url{http://www.crm.umontreal.ca/act/theme/theme_2014_2_en/counting_e.php}) at 
the Centre de Recherches Math\'ematiques. 
Here are some examples of the sort of results that will be covered. 

\begin{theo}[Bhargava, Shankar]
A positive proportion of elliptic curves over $\dQ$ have rank $0$. 
\end{theo}
\begin{proof}
Use Dokchitser-Dokchitser to show existence of curves with odd (resp.~even) 
$p$-Selmer rank. From $\average(\#\selmer_5)=6$, we conclude that the average 
rank of an elliptic curve is $\leqslant 1.05$. This could be achieved if 
$95\%$ have rank $1$ and $5\%$ have rank $2$. This situation cannot happen if 
$p$-Selmer rank parities are well-distributed, as they are (often enough to 
rule out this scenario). This reduces the average rank bound to 
$\leqslant 0.885$. 
\end{proof}

\begin{theo}[Bhargava, Skinner]
A positive proportion of elliptic curves have rank $1$. 
\end{theo}
\begin{proof}
Dokchitser-Dokchitser implies a lot of elliptic curves have $5$-Selmer rank 
$5$. We want to show that these actually have rank $1$. Use work of Skinner 
\cite{s14} on Heegner points. 
\end{proof}

\begin{theo}[Bhargava, Skinner, Zhang]
At least $>66.48\%$ elliptic curves over $\dQ$ satisfy the rank part of the BSD 
conjecture. They also satisfy the $p$-part of BSD for all $p\ne 2$. 
\end{theo}

\begin{coro}
Most elliptic curves over $\dQ$ have finite Tate-Shafarevich group. 
\end{coro}
\begin{proof}
This follows from work of Kolyvagin. 
\end{proof}

Clearly, the standard assumption ``$p\ne 2$'' at the start of most papers 
needs to be removed. 




