% !TEX root = sms.tex

\section{Binary quartic forms: bounded average rank of elliptic curves}
\thanksauthor{Arul Shankar}





\subsection{Introduction}

Recall that every elliptic curve over $\dQ$ can be written as 
$y^2=x^3+A x+B$ for $A,B\in \dQ$. 

\begin{theo}[Mordell]
The abelian group $E(\dQ)$ is finitely generated. 
\end{theo}

So we can write $E(\dQ)=T\oplus \dZ^r$, where $T$ is a finite abelian group, 
and $r=\rank E$ is the \emph{rank} of $E$. We are going to study the average 
rank of elliptic curves. To do this, we need to order elliptic curves in some 
way. 

The elliptic curve $E_{A<B}:y^2=x^3+A x+B$ is isomorphic to 
$E_{n^4 A,n^6 B}:y^2=x^3+n^4 A x+n^6 B$ for all $n\in \dQ^\times$. So we can 
assume $A,B\in \dZ$. If we assume that $p^4\mid A\Rightarrow p^6\nmid B$, then 
the $A,B$ are unique. Thus there is a bijection between isomorphism classes of 
elliptic curves over $\dQ$ and 
\[
  \cE = \{E_{A,B}:(A,B)\in \dZ^2\text{ and }p^4\mid A\Rightarrow p^6\nmid B\} .
\]
We could also look at subfamilies of $\cE$ cut out by (possibly infinitely 
many) congruence conditions. We could also look at ``thin'' families consisting 
of all quadratic twists of some elliptic curve. 





\subsection{Ordering elliptic curves}

To talk about averages, we need to order elliptic curves in some way. The most 
obvious invariants to use are the discriminant and conductor. We have 
\[
  \discriminant(E_{A,B}) = \Delta(E_{A,B}) = 4 A^3 - 27 B^2 .
\]
The problem with ordering elliptic curves by discriminant is that we don't know 
how to count the number of elliptic curves with discriminant bounded by $X$. 
Essentially, the region $\{(x,y)\in \dR^2:4 x^2 - 27 y^3<X\}$ is noncompact, 
which makes the counting problem very hard. What is easier is to order elliptic 
curves by (naive) height: 
\[
  H(E_{A,B}) = \max\{|4 A^3|, 27 B^2\} .
\]

Let $f$ be a function on elliptic curves. The \emph{average} of $f$ is 
\[
  \average(f) = \lim_{X\to \infty} \frac{\sum_{H(E)<X} f(E)}{\sum_{H(E)<X} 1} .
\]
It's not hard to show that $\average(\# T)=1$. That is, on average an elliptic 
curve has no nontrivial torsion. This follows from Hilbert irreducibility. 

Our question is: what can we say about the average rank? 

\begin{enonce}{Conjecture}[Goldfeld, Katz-Sarnak]
$\average(\rank) = \frac 1 2$. Moreover, $50\%$ of elliptic curves have rank 
$0$ and $50\%$ have rank $1$. 
\end{enonce}

The conjecture was originally made with elliptic curves ordered by conductor, 
but ordering by conductor, height and discriminant are all expected to yield 
the same average. 

Given an elliptic curve $E$, we can define an $L$-function which we denote 
$L_E(s)$. The \emph{completed $L$-function} $L_E^\ast(s)$ satisfies a 
functional equation 
\[
  L_E^\ast(s) = \omega(E) L_E^\ast(1-s), 
\]
where $\omega(E)$, the \emph{root number} of $E$, is $\pm 1$. The 
\emph{analytic rank} of $E$ is the order of vanishing of $L_E$ at 
$s=1/2$. The \emph{Birch and Swinnerton-Dyer conjecture} predicts that the 
analytic rank and algebraic rank of $E$ are the same. But the analytic 
rank of $E$ is also the analytic rank of $L_E^\ast$. So the BSD conjecture 
implies that $\rank(E)$ is even if and only if $\omega(E)=1$. It's expected 
that $\omega(E)$ is equidistributed, i.e.~half of elliptic curves have 
$\omega(E)=0$ and half have $\omega(E)=1$. Assuming BSD, it would follow 
that half of elliptic curves have even rank, and half have odd rank. We also 
expect the rank of an elliptic curve to be ``as small as it can get away 
with,'' which would force $100\%$ of elliptic curves with $\omega(E)=0$ to 
have rank zero, and $100\%$ of elliptic curves with $\omega(E)=1$ to have rank 
one. 

In \cite{ks99}, Katz and Sarnak studied the family of all $L$-functions of 
elliptic curves. Assuming GRH and BSD, the we have the following bounds on 
$\average(\rank)$:
\begin{center}
\begin{tabular}{c|c}
\cite{b92} & $\leqslant 2.14$ \\
\cite{h04} & $\leqslant 2$ \\
\cite{y06} & $\leqslant 1.79$
\end{tabular}
\end{center}

More recently, we have the following theorem proven in \cite{j02}. 

\begin{theo}[de Jong]
For the family of all elliptic curves over $\dF_q(t)$, the average rank is 
bounded above by $\frac 7 6+\epsilon(q)$, where $\epsilon(q)\to 0$ as 
$q\to \infty$. 
\end{theo}

The method is to bound $\average(\# \selmer_3)$. 





\subsection{Selmer groups}

The fundamental idea is that from the canonical short exact sequence 
\[
  0 \to E(\dQ)/p \to \selmer_p(E) \to \sha(E)[p] \to 0 ,
\]
we get a bound on $\rank E$ in terms of $\selmer_p(E)$. Note that 
$\#(E(\dQ)/p) \geqslant p^{\rank E}$. 

First, we want a parameterization of 2-Selmer elements of elliptic curves. 
More generally, if $\sigma\in \selmer_p(E)$. then we can think of $\sigma$ as a 
locally soluble $p$-covering of $E$. Such a covering is a twist of 
$[p]:E\to E$. It will be a genus-one curve $C$ isomorphic to $E$ over 
$\overline\dQ$, along with $C\to E$ such that the following diagram commutes:
\[\xymatrix{
  C \ar[d]^-\wr \ar[dr] \\
  E \ar[r]^-{[p]} 
    & E .
}\]
See \autoref{sec:7} for more details. The covering $C$ is \emph{soluble} if 
$C(\dQ)\ne \varnothing$, and it is \emph{locally soluble} if 
$C(\dQ_v)\ne\varnothing$ for all places $v$ of $\dQ$. Locally soluble 
$p$-coverings of $E$ are in natural bijection with $\selmer_p(E)$, and 
soluble $p$-coverings are in bijection with $E(\dQ)/p$. For the rest of this 
lecture, we will concentrate on $p=2$. 

It turns out that a locally soluble $2$-covering of $E$ yields a binary quartic 
form over $\dQ$. Conversely, a binary quartic form gives a $2$-cover. 

Let $V$ be the space of binary quartic forms. The group $\generallinear(2)$ 
acts on $V$ via 
\[
  (\gamma\cdot f)(x,y) = \frac{1}{(\det\gamma)^2}f\left(\begin{pmatrix} x & y \end{pmatrix} \cdot \gamma\right) .
\]
The center $Z(\generallinear_2)$ acts trivially, so the action descends to 
one of $\projectivegenerallinear(2)$ on $V$. The ring of invariants is 
freely generated by two elements $I,J$, which have degree $2$ and $3$ 
respectively in the coefficients of the form. 

\begin{theo}[Birch-Swinnerton-Dyer, Cremona-Fisher-Stoll]
There is a bijection between 2-Selmer elements and the quotient 
$\projectivegenerallinear_2(\dQ)\backslash V(\dQ)^\mathrm{ls}$, where 
$V(\dQ)^\mathrm{ls}$ is the subset of locally soluble forms, in which 
$(A,B)$ corresponds to $I=-3\cdot 2^6 A$ and $J=-272\cdot 2^6 B$, and 
$A(f)=-I(f)/3\cdot 2^4$ and $B(f) = -J(f)/27\cdot 2^6$. 
\end{theo}

We will write this as 
$\selmer_2(E_{A,B}) = \projectivegenerallinear_2(\dQ)\backslash V(\dQ)_{A,B}^\mathrm{ls}$. 
The identity element of $\selmer_2(E_{A,B})$ corresponds to the orbit of 
binary quadratic forms with a rational linear factor. 

We would like a parameterization of 2-Selmer elements that involves binary 
quartic forms with integral coefficients instead of just rational coefficients. 

\begin{lemm}[Birch, Swinnerton-Dyer]
If $f\in V(\dQ_p)$, then $A(f),B(f)\in \dZ_p$. If $f$ is $\dQ_p$-solvable, then 
$\projectivegenerallinear_2(\dQ_p)\cdot f\cap V(\dZ_p)\ne\varnothing$. 
\end{lemm}

\begin{theo}
If $f\in V(\dQ)$, then $A(f),B(f)\in \dZ$. If $f$ is locally soluble, then 
$\projectivegenerallinear_2(\dQ)\cdot f\cap V(\dZ)\ne\varnothing$. 
\end{theo}
\begin{proof}
For each prime, find $\gamma_p\in\projectivegenerallinear_2(\dQ_p)$ so that 
$\gamma_p\cdot f\in V(\dZ_p)$. Since $\projectivegenerallinear_2$ has class 
number $1$, there exists $\gamma\in \projectivegenerallinear_2(\dQ)$ so that 
$\gamma\cdot f\in V(\dZ_p)$ for all $p$, hence $\gamma\cdot f\in V(\dZ)$. 
\end{proof}

\begin{theo}[Birch, Swinnerton-Dyer and Cremona, Fisher, Stoll]
The set $\selmer_2(E_{A,B})$ is naturally in bijection with 
$\projectivegenerallinear_2(\dQ) \backslash V(\dZ)_{A,B}^\mathrm{ls}$. 
\end{theo}

Define the \emph{height} of a binary quartic form to be 
\[
  H(f) = \max\{4|A(f)|^3, 27 B(f)^2\} .
\]
So the goal is to count $\projectivegenerallinear_2(\dQ)$-equivalence classes 
of integral, locally soluble binary quartic forms with height bounded by $X$. 





\subsection{First step}

The goal is to count $\projectivegenerallinear_2(\dZ)$-orbits of 
$V(\dZ)_{H<X}^\mathrm{irr}$. The method is extremely similar to how Bhargava 
counted binary cubic forms. First we construct a fundamental domain $\cF_X$ for 
the action of $\projectivegenerallinear_2(\dZ)$ on $V(\dR)_{H<X}$. Next we 
estimate $\#\{\cF_X\cap V(\dZ)\}$ using averaging. 

We begin by finding a fundamental set for 
$\projectivegenerallinear_2(\dR)\backslash V(\dR)$. Over any field $k$ in 
which $6$ is invertible, 
$\projectivegenerallinear_2(k)\backslash V(k)_{A,B}^{k-\mathrm{sol}}$ is 
in bijection with $E_{A,B}(k)/2$. Given $A,B\in \dR$, the set 
$\projectivegenerallinear_2(\dR)\backslash V(\dR)_{A,B}^\mathrm{ls}$ is in 
bijection with $E_{A,B}(\dR)/2$. But the group of $\dR$-valued points of an 
elliptic curves is easy to understand. It is $\dZ/2\times S^1$ or $S^1$, 
depending on whether the discriminant is positive or negative. It follows that 
$E_{A,B}(\dR)/2$ has either $1$ or $2$ elements as $\Delta(E_{A,B})<0$ or 
$\Delta(E_{A,B})>0$. 

If $\Delta(E_{A,B})<0$, then 
$\projectivegenerallinear_2(\dR)\backslash V(\dR)_{A,B}$ is a singleton, and 
any $f$ in the set will have exactly $2$ real roots. If 
$\Delta(E_{A,B})>0$, there are two orbits, one consisting of forms with two 
real roots, and one consisting of forms with positive-definite binary quartic 
forms. Define 
\begin{align*}
  V(\dR)^{(0)} &= \{f\in V(\dR)\text{ with 4 real roots}\} \\
  V(\dR)^{(1)} &= \{f\in V(\dR)\text{ with 2 real roots}\} \\
  V(\dR)^{(2+)} &= \{f\in V(\dR)\text{ positive definite}\} .
\end{align*}
A fundamental set for $\projectivegenerallinear_2(\dR)\backslash V(\dR)^{(i)}$ 
for $i\in \{0,1,2+\}$ is $\{f\text{ having invariants }A,B\}$. We obtain 
\[
  \projectivegenerallinear_2(\dR)\backslash V(\dR)^{(i)}_{H<1} = R_1^{(i)} ,
\]
with, for example, 
\[
  R_1^{(0)} = \{x^3 y + A x y^3 + B y^4,(A,B)\in \dR^2,\Delta(E_{A,B})>0,H(E_{A,B})<1\} .
\]
For general height, we scale: 
\[
  \projectivegenerallinear_2(\dR)\backslash V(\dR)_{H<X}^{(i)} = X^{1/6} R_1^{(i)} = R_X^{(i)} .
\]

So $R_X^{(i)}$ is a fundamental domain for the action of 
$\projectivegenerallinear_2(\dR)$ on $V(\dR)_{H<X}^{(i)}$. We want a 
fundamental domain for the action of $\projectivegenerallinear_2(\dZ)$. Choose 
$\cF=\projectivegenerallinear_2(\dZ)\backslash \projectivegenerallinear_2(\dR)$; 
then $\cF\cdot R_X^{(i)}$ is an $n_i$-fold cover of 
$\projectivegenerallinear_2(\dZ)\backslash V(\dR)_X^{(i)}$. It turns out that 
$n_1=2$ and $n_0 = n_{2+} = 4$. 

It follows that 
\[
  \#\projectivegenerallinear_2(\dZ)\backslash V(\dZ)_{H<X}^{(i),\mathrm{irr}} = \frac{1}{n_i} \#(\cF\cdot R_X^{(i)}\cap V(\dZ)^\mathrm{irr}) .
\]
As with binary cubic forms, we average: 
\begin{align*}
  \#\projectivegenerallinear_2(\dZ)\backslash V(\dZ)_{H<X}^{(i),\mathrm{irr}} 
    &= \frac{1}{n_i \volume(G_0)} \int_{G_0} \#\left(\cF gR_X^{(i)} \cap V(\dZ)^\mathrm{irr}\right)\, \mathrm{d} g \\
    &= \frac{1}{n_i\volume(G_0)} \int_\cF \#\left(g G_0 R_X^{(i)}\cap V(\dZ)^\mathrm{irr}\right)\, \mathrm{d} g \\
    &= \frac{1}{n_i \volume(G_0)} \int_\cF \volume\left(g G_0 R_X^{(i)}\right)\, \mathrm{d} g + O(X^{3/4}) \\
    &= \frac{1}{n_i\volume(G_0)} \int_{G_0} \volume\left(\cF g R_X^{(1)}\right)\, \mathrm{d} g + O(X^{3/4}) \\
    &= \frac{1}{n_i} \volume\left(\cF R_X^{(i)}\right) + O(X^{3/4}) .
\end{align*}

We can summarize all of this in the following theorem. 

\begin{theo}[Bhargava, Shankar]
\[
  \#\left(\projectivegenerallinear_2(\dZ)\backslash V(\dZ)_{H<X}^{(i),\mathrm{irr}}\right) = \frac{1}{n_i} \volume\left(\cF R_X^{(i)}\right) + O(X^{3/4}) .
\]
\end{theo}

As an easy corollary, the average rank of elliptic curves is bounded. 
We can compute this as  
\[
  \frac{1}{n_i} \volume\left(\cF R_X^{(i)}\right) = \frac{1}{n_i} |J|\volume(\cF) \volume\left(R_X^{(i)}\right) + O(X^{5/6}) 
\]
So the number of elliptic curves with height $<X$ is some constant multiple of 
$X^{5/6}$. Thus $\average(\#\selmer_2)$ is bounded, whence $\average(\rank)$ is 
bounded. 

In \autoref{sec:shankar-ii}, we'll derive an explicit bound for 
$\rank(\average)$. 




