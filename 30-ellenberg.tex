% !TEX root = sms.tex

\section{Topological and algebro-geometric methods over function fields I}\label{sec:ellenberg-i}
\thanksauthor{Jordan Ellenberg}





I will give a ``sales pitch'' for thinking about these problems in the context 
of global function fields. The idea is that the main problems can be approached 
more geometrically. Some problems in arithmetic statistics are much easier in 
the function field context. 





\subsection{Motivating examples}

\begin{enonce}{Question}
How many integers are there between $N$ and $2 N$?
\end{enonce}

See \autoref{sec:granville-ii} for an interesting (and sophisticated) approach 
to this question. 

\begin{enonce}{Question}
How many squarefree integers are there between $N$ and $2 N$?
\end{enonce}

Call this number $\squarefree(N)$. To be squarefree is to be indivisible by 
$4,9,25,49,\ldots$, i.e.~not divisible by $p^2$ for any prime $p$. One might 
expect ``being indivisible by $p^2$'' to be independent for distinct $p$, so 
\begin{align*}
  \squarefree(N) 
    &\sim N\cdot\left(1-\frac 1 4\right)\left(1-\frac 1 9\right) \cdots \\
    &= N\cdot \prod_p \left(1-p^{-2}\right)^{-1} \\
    &= \zeta(2)^{-1} N .
\end{align*}
So $\lim_{N\to \infty}\frac{\squarefree{N}}{N} = \zeta(2)^{-1}$. This is a 
common phenomenon in arithmetic statistics -- some kind of behavior 
asymptotically occurs an $L$-value percent of the time. 

First, let's understand how this problem looks over more general global fields. 

\begin{defi}
A \emph{global field} is either 
\begin{itemize}
  \item A number field, i.e.~a finite extension of $\dQ$. 
  \item The function field of a curve over a finite field $\dF_q$. 
    (Equivalently, a field isomorphic to a finite extension of $\dF_q(t)$.)
\end{itemize}
\end{defi}

We will be quite loose in identifying a curve over $\dF_q$ and its function 
field, because there is an (anti-)equivalence of categories between smooth 
proper geometrically irreducible (aka ``nice'') curves over $\dF_q$ and 
field extensions of $\dF_q$ of transcendence degree $1$. 





\subsection{The analogy between number fields and function fields}

For a number field $K$, there is a unique embedding $\dQ\hookrightarrow K$. But 
there might be \emph{many} ways to embed $\dF_q(t)$ into a global field $K$. 
For example, $\dF_q(t^{17})\subset\dF_q(t)$ is a ``non-standard'' embedding of 
$\dF_q(t)$ into $\dF_q(t)$. So global function fields do not ``come with'' the 
structure of an extension of $\dF_q(t)$. This phenomenon lies behind the fact 
that Mordell-Weil ranks are unbounded over function fields. (See examples of 
Ulmer.) In this lecture we'll mainly talk about $\dF_q(t)$. 

What is the function-field analogue of counting squarefree integers in a box? 
One problem is that $\dQ$ has only one ``nice'' subring, whereas $\dF_q(t)$ has 
lot of ``nice'' subrings. We'll use the following analogy:
\begin{center}
\begin{tabular}{c|c}
number fields & function fields \\ \hline
$\dQ$ & $\dF_q(t)$ \\
$\dZ$ & $\dF_q[t]$ \\
$|\cdot|:\dZ\to \dR$ & $|f|_\infty = q^{\deg f}$ \\
$[N, 2 N]=\{n\in \dN:|n|\sim N\}$ & set of monic polys with $|f|= N=q^n$ \\
$\#(\dN\cap [N,2 N])\sim N$ & $\#($monic polys with $|f|=N)=N$ \\
of these, $\sim \zeta_\dZ(2)^{-1} N$ are squarefree & $\sim \zeta_{\dF_q[t]}(2)^{-1}$ are squarefree
\end{tabular}
\end{center}

That is, the limiting proportion of squarefree monic polynomials in 
$\dF_q[t]$ is 
\[
  \prod_p \left(1-|p|^{-2}\right) = 1-q^{-1} .
\]
as $p$ ranges over monic irreducible polynomials in $\dF_q[t]$. In fact, the 
number of squarefree monic polynomials of degree $n$ in $\dF_q[t]$ is exactly 
$q^n-q^{n-1}$ for all $n\geqslant 2$, and $q$ for $n=1$. So we have a 
power-saving result with \emph{much} better error term over function fields. So 
we shouldn't think of there being an analogy between any particular number field 
and any particular function field. Rather, there is an analogy between the 
\emph{class} of number fields and the \emph{class} of function fields. 





\subsection{Geometric picture}

What is geometric about what we've done? We introduce yet another function 
field, $\dC(t)$. We can once again think about the set of monic squarefree 
polynomials of degree $n$ in $\dC[t]$. This set is not just a set -- it is a 
\emph{space} (namely an algebraic variety). The space of monic squarefree 
polynomials of degree $n$ is called the (unordered) \emph{configuration space} 
of $\dC$, denoted $\configuration^n \dC$. It parameterizes $n$-tuples of 
distinct points in $\dC$, up to permutation. This isomorphism is given by 
$f\mapsto \{\text{roots of $f$}\}$. The inverse sends an $n$-tuple 
$(z_1,\dots,z_n)$ to the polynomial $f(t)=(t-z_1)\cdots(t-z_n)$. 

We are morally constrained to think of this configuration space not just as 
a complex manifold, but as a scheme over $\spectrum\dZ$. Namely, there is a 
scheme $\configuration^n \dA^1$ over $\spectrum\dZ$ such that 
\[
  (\configuration^n\dA^1)(K) = \{\text{monic squarefree polynomials of degree $n$ in $K[t]$}\}
\]
for any field $K$. In fact, this has a simple description. Namely, the moduli 
space of \emph{all} monic polynomials of degree $n$ is $\dA^n$. A polynomial $f$ 
is squarefree if and only if the discriminant $\Delta(f)$ is nonzero, where 
$\Delta$ is a polynomial in the coefficients of $f$. For example, 
\[
  \Delta(t^2+a_1 t+a_2) = a_1^2 - 4 a_2 .
\]
So $\configuration^n\dA^1$ is $\dA^n\smallsetminus V(\Delta)$. Note: we would 
get a different space if we parameterized ordered $n$-tuples numbers, where 
we care about ordering. We'll call that $\pureconfiguration^n$, the 
\emph{pure configuration space}. The group $S_n$ acts on $\pureconfiguration^n$ 
by permuting the $n$-tuples, and the quotient 
$\pureconfiguration^n/S_n$ is $\configuration_n$. Note that 
$\pureconfiguration^n\dA^1=\dA^n\smallsetminus \bigcup_{i\ne j} V(z_i-z_j)$, 
where $z_1,\dots,z_n$ are the the coordinates of $\dA^n$. 

The set of monic squarefree polynomials in $\dF_q[t]$ of degree $n$ is just 
$\configuration^n\dA^1(\dF_q)$. So our counting problem is: what is 
$\#\configuration^n\dA^1(\dF_q)$? We saw that the answer is $q^n-q^{n-1}$. 

What if we only cared about what happens as $q\to \infty$? For example, what is 
the probability that a degree-$n$ polynomial over $\dF_q[t]$ is squarefree? We 
had been fixing $q$ and letting $n\to \infty$. A simpler question is fixing $n$ 
and letting $q\to \infty$. As $q\to \infty$, 
\[
  \lim_{q\to\infty}\frac{\#\configuration^n\dA^1(\dF_q)}{q^n} 
    = \lim_{q\to \infty}\frac{\#(\dA^n\smallsetminus V(\Delta))(\dF_q)}{q^n} 
    = 1 .
\]





\subsection{M\"obius functions}

\begin{enonce}{Question}
What is the average of the M\"obius function?
\end{enonce}

Recall the \emph{M\"obius function} $\mu$ is the arithmetic function defined by 
\[
  \mu(n) = \begin{cases} 0 & n\text{ is squarefree} \\ 1 & n\text{ the product of an even number of distinct primes} \\ -1 & n\text{ the product of an odd number of distinct primes} \end{cases}
\]
Earlier, we computed that the expected value of $\mu^2$ is 
$\expected(\mu^2)=\zeta(2)^{-1}$. How does this look for function fields? We 
can define the M\"obius function of a polynomial in exactly the same way. But 
over function fields, we have the beautiful 

\begin{enonce}{Fact}
\[
  \mu(f) = (-1)^{\deg f} \left(\frac{\Delta(f)}{q}\right) 
\]
where $\left(\frac{\cdot}{q}\right)$ is the Legendre symbol. 
\end{enonce}

Note that $(-1)^n\mu(f)+1$ is the number of square roots of $\Delta(f)$ in 
$\dF_q$, where $n=\deg f$. Let's make a variety geometrizing this problem. 
Define $Y_n$ to be the space parameterizing pairs $(f,y)$, where $f$ is a monic 
squarefree degree $n$ polynomial, and $y$ is a square root of $\Delta(f)$. Then 
\[
  \#Y_n(\dF_q) = \sum_{f:\deg f=n} \left((-1)^n \mu(f) + 1\right)
\]
So $\#Y_n(\dF_q) - q^n = (-1)^n \sum_f \mu(f)$. We expect 
\[
  \# Y_n(\dF_q) = q^n + o(q^n) .
\]
Why? There is a map $Y_n\to \dA^n$ given by $(f,y)\mapsto f$. This is a 
double branched at the vanishing locus of $\Delta$. In general, we expect an 
$n$-dimensional variety to have approximately $q^n$ points over $\dF_q$. But 
this expectation only is valid when the variety is irreducible. So we think 
$\# Y_n(\dF_q)\sim q^n$ because we think $Y_n$ is irreducible. Indeed, the 
Weil conjectures guarantee that if $Y_n$ is geometrically irreducible, then 
\[
  \lim_{q\to \infty} \frac{\#Y_n(\dF_q)}{q^n} = 1 ,
\]
so the limit as $q\to \infty$ of the average of the M\"obius function is zero. 

But how do we \emph{actually know} that $Y_n$ is irreducible? What if 
$\Delta(a_1,\dots,a_n)$ were actually $G^2$ for some other polynomial $G$? 
Then $\mu(f)$ would be $(-1)^n$ for \emph{all} $f$ of degree $n$. I argue that 
the underlying idea here is a computation of monodromy. 





\subsection{Monodromy}

Recall the $S_n$-Galois cover 
$\pureconfiguration^n \twoheadrightarrow \configuration^n$. The normal subgroup 
$A_n\subset S_n$ corresponds to an intermediate (degree-2) Galois cover 
$U_n\to \configuration^n$. In fact, the following diagram is Cartesian with 
the left arrow being an \'etale cover with group $\dZ/2$:
\[\xymatrix{
  U_n \ar[r] \ar[d] 
    & Y_n \ar[d] \\
  \configuration^n{} \ar@{^{(}->}[r] 
    & \dA^n .
}\]
It is sufficient to show that $U_n$ is irreducible. A $\dZ/2$-cover of 
$\configuration^n$ is a map $\pi_1(\configuration^n)\to \dZ/2$, and the cover 
is irreducible if and only if this map is surjective. Whenever we have a 
``cover of moduli spaces'' $Y\to X$ of degree $n$, we have a map 
$\pi_1(X)\to S_n$. The image of this map is called the \emph{monodromy group} 
of the cover and $Y$ is irreducible if and only if the monodromy group is 
transitive. 

In \autoref{sec:ellenberg-ii}, we'll look at the idea that big monodromy 
implies ``averages are what you expect'' in the large $q$ regime. Sometimes, 
monodromy is not big, and its ``smallness'' can sometimes explain the failure 
of of heuristics. 





\subsection{Computational question}

This is related to the discussion of variation of Mordell-Weil ranks. As 
discussed above, there are $q^n-q^{n-1}$ squarefree onic polynomials of degree 
$n$ in $\dF_q[t]$. For each such $f(t)$, let 
\[
  C_f:y=f(t) 
\]
be the corresponding hyperelliptic curve. Its zeta function has the form 
\[
  \zeta(C_f,s) = \frac{P_f(q^{-s})}{(1-q^{-s})(1-q^{1-s})} ,
\]
where $P_f\in \dZ[X]$ has degree $2 g$, and all its roots have absolute value 
$q^{1/2}$. 

The question is: for how many $f$ does $P_f$ have $q^{1/2}$ as a root. Does 
this proportion look like $q^{\alpha n}$ for some $0<\alpha<1$?




