% !TEX root = sms.tex

\section{Rings associated to binary \texorpdfstring{$n$}{n}-ic forms, composition of \texorpdfstring{$2\times n\times n$}{2*n*n} boxes and class groups}\label{sec:wood-iii}
\thanksauthor{Melanie Matchett Wood}





We'll focus especially on the case of binary quartic forms, and the 
parameterization of ideal classes. Good references are 
\cite{n89,s01,s03,s05,w11-rings}. 






\subsection{The construction}

Let $f=a_0 x^n + a_1 x^{n-1} y + \cdots + a_n y^n$ be a binary $n$-ic form with 
the $a_i\in \dZ$. We will construct a rank $n$ ring $R_f$. 

First we'll give an explicit construction. 
If $a_0\ne 0$, consider the ring $\dQ[\theta]/f(\theta,1)$. This contains 
elements 
\begin{align*}
  1 \\
  a_0 \theta \\
  a_0 \theta^2 + a_1 \theta \\
  \ldots \\
  a_0 \theta^{n-1} + a_1 \theta^{n-2} + \cdots + a_{n-2} \theta .
\end{align*}
Let $R_f$ be the $\dZ$-module generated by these. It turns out that $R_f$ is 
closed under multiplication, so it is a rank-$n$ ring. For example, 
\begin{align*}
  (a_0\theta)^2 
    &= a_0^2 \theta^2 \\
    &= a_0(a_0 \theta^2 + a_1 \theta) - a_1(a_0 \theta) ,
\end{align*}
so $(a_0 \theta)^2\in R_f$. The same phenomenon occurs for all of the additive 
generators of $R_f$. 

A more highbrow construction is as follows. Inside $\dP_\dZ^1$ we have a 
closed subscheme $V_f$ cut out by $f=0$. The ring $R_f$ is just 
$\h^0(V_f,\sO)$, the ring of regular functions on $V_f$ (at least if not all 
the $a_i=0$). When $a_0\ne 0$, these functions are determined by their 
restriction to $\dA_\dZ^1\cap V_f$. The regular functions on $\dA_\dZ^1$ are 
polynomials in $\frac x y$, i.e.~$\h^0(\dA_\dZ^1,\sO) = \dZ[\frac x y]$. We 
have for example 
\[
  a_0 \frac x y = -\left(a_1 + a_2 \frac y x + \cdots \right) .
\]
We claim that the $\dZ$-span of 
$1,a_0 \frac x y,a_0(\frac x y)^2+a_1 \frac x y,\ldots$ is the whole ring of 
regular functions on $V_f$. This recovers our explicit definition of $R_f$. 

\begin{prop}
Let $f$ be a binary $n$-ic form. Then 
\begin{itemize}
  \item If $R_f$ is a domain, its fraction field is $\dQ[\theta]/f(\theta,1)$. 
  \item $\discriminant(R_f) = \discriminant(f)$ .
  \item $R_f$ is a domain if and only if $f$ is irreducible in $\dQ[x,y]$ 
  \item $R_f$ is a maximal order if and only if certain conditions modulo $p^2$  
    for every $p$ are satisfied. 
  \item If $R_f$ is maximal, then a prime $p$ splits in $R_f$ as $f(x,y)$ 
    factors modulo $p$. 
\end{itemize}
\end{prop}

The special case $a_0=1$ yields monogenic rings $\dZ[\alpha]$. The bad news is 
that for $n>3$ we do not obtain all rank $n$ rings (or all rank $n$ maximal 
orders). Heuristically, this works as follows. Given a form $f$ we can produce 
$V_f\subset \dP_\dZ^1$. A map to $\dP^1$ is determined by a line bundle with 
two generating global sections. In the maximal order case, this is an ideal 
class of the ring with two generating elements. If we think of 
$R_f\subset \dQ[\theta]/f(\theta,1)$, then the ideal class is 
$I_f=\langle 1,\theta\rangle$, which is a fractional ideal class unless 
$a_0=1$. 

The general story goes as follows. There is a bijection between binary $n$-ic 
forms up to $\generallinear_2(\dZ)$-equivalence and rank-$n$ rings with 
an ``ideal class'' satisfying some conditions on the ideal class. We don't 
get all rank $n$ rings because not all rings have ideal classes satisfying the 
conditions. For $n=2$, the conditions on the ideal class are trivial, so we 
get all quadratic rings and all ``ideal classes.'' (We put ideal class in 
quotation marks because we haven't defined such things for rings that are not 
integral domains.) 





\subsection{Geometric story of Gauss composition}

Start with a binary quadratic form $f(x,y)=a x^2 + b x y + c y^2$ for 
$a,b,c\in \dZ$. To this we can associate a quadratic ring with an ideal class. 
Geometrically, $\{f=0\}$ cuts out a subscheme $V_f$ of $\dP_\dZ^1$. The ring 
$\h^0(V_f,\sO)$ of regular functions on $V_f$ is the associated ring, and the 
invertible sheaf coming from $V_f \hookrightarrow \dP^1$ is the ideal class. 

We work with pathological rings (having zero-divisors and nilpotents) because 
we want to be able to talk about the behavior of a form when it is reduced 
modulo $p$. Even if things are well-behaved over $\dZ$, their reduction modulo 
$p$ can be pathological. 

We define ideal classes in general rings. Let $C$ be an order in a quadratic 
field. If $\fa,\fb$ are ideals in the same class, then $\fa= k \fb$ for some 
$k\in C\otimes \dQ$. The map $k:\fa \to \fb$ is an isomorphism of $C$-modules. 
Moreover, the converse holds: ideals that are isomorphic as $C$-modules are in 
the same ideal class. So we can replace the notion of an ``ideal class'' with 
an ``isomorphism class of modules.'' But certainly not all $C$-modules come 
from ideal classes. For example, $C^2$ is not an ideal class. However, all 
modules isomorphic to $\dZ^2$ as $\dZ$-modules are ideal classes. 

The reference for what follows is \cite{w11-gauss}. 

\begin{theo}
There is a bijection between 
$\generallinear_2(\dZ)\times \generallinear_1(\dZ)$-classes of 
$a x^2 + b x y+c y^2$ and isomorphism classes of $(C,M)$, where $C$ is a 
quadratic ring and $M$ is a $C$-module such that $M$ is $\simeq \dZ^2$ as a 
$\dZ$-module and $\trace:C\to \dZ$ is the same using multiplication on $C$ or 
$M$. 
\end{theo}

Recall that for $n=2$ we get all ideal classes. For $n=3$, the conditions 
on our ``ideal class'' force $I_f$ to be the inverse different. So we get 
precisely isomorphism classes of cubic rings. For $n>3$, the conditions on 
the ideal class become non-trivial. In particular, we don't get all maximal 
orders. For example, when $n=4$, we get exactly the quartic rings with a 
monogenic cubic resolvent. A good reference is \cite{w12}. 





\subsection{Parameterization of ideal classes of \texorpdfstring{$R_f$}{Rf}}

References here are \cite{b04-i,b04-ii,w14}. Let 
$A\in \dZ^2\otimes \dZ^n\otimes \dZ^n$, the space of ``pairs $(A_1,A_2)$ of 
$n\times n$ matrices with coefficients in $\dZ$.'' Put 
$f=\det(A)=\det(A_1 x_1 + A_2 x_2)$; this is a binary $n$-ic form. The group 
$G\subset \generallinear_2(\dZ) \times\generallinear_n(\dZ)$ consisting of 
$(g,h)$ with $\det(g)\det(h)=1$ acts on $\dZ^2\otimes \dZ^n\otimes \dZ^n$ by 
left and right multiplication on $(A_1,A_2)$: 
\[
  (g,h)\cdot (A_1,A_2) = (g A_1 h,g A_2 h) .
\]

\begin{theo}
For primitive, irreducible binary $n$-ic forms $f$ with coefficients in 
$\dZ$, there is a bijection between ideal classes of $R_f$ and 
$G$-classes of $A\in \dZ^2\otimes \dZ^n\otimes \dZ^n$ with $\det(A)=f$. 
\end{theo}

The ideal classes appearing in this theorem are not necessarily invertible. 
If we unravel this when $f$ is monic, the theorem generalizes the classical 
result parameterizing ideal classes of monogenic orders by conjugacy classes 
of matrices. We will not cover the proof of this. We will describe the 
map from forms to ideal classes by restricting to the case when $R_f$ is a 
maximal order. 

Instead of viewing $\dZ^2\otimes \dZ^n\otimes \dZ^n$ as pairs of 
$n\times n$ matrices, we can view it as $n$-tuples $(a_1,\dots,a_n)$ of 
$2\times n$ matrices. Given such a tuple, 
$a_1 y_1 + a_2 y_2 + \cdots + a_n$ is a $2\times n$ matrix. Let 
$g_1,\dots,g_{\binom 2 n}$ be the determinants of maximal minors of the matrix 
$a_1 y_1 + \cdots + a_n y_n$. Each $g_i$ is a quadratic form in $n$ variables. 
Let $V_{\boldsymbol g}\subset \dP_\dZ^{n-1}$ be the subscheme cut out by the 
$g_i$. If the $g_i$ were generic, we would expect 
$V_{\boldsymbol g} = \varnothing$. In our case, the ring of functions on 
$V_{\boldsymbol g}$ is $R_f$. The map 
$V_{\boldsymbol g}\hookrightarrow \dP_\dZ^{n-1}$ comes from a line bundle with 
$n$ global sections. The line bundle gives us an ideal class in $R_f$, which 
can be any ideal class in $R_f$. 

When $n=2$, there are three ways of ``slicing up'' an element of 
$\dZ^2\otimes \dZ^2\otimes \dZ^3$. The same quadratic form $f$ produces a ring 
$R_f$ with ideal classes $M_A$, $N_A$, $I_f$. It turns out that 
$M_A N_A = I_f^{n-3}$, where $I_f$ is the ``standard'' ideal coming from $f$. 
Since $n=2$, $M_A N_A I_f = 1$. 

Let $M$ be an ideal in the ring $R_f$, and let $N$ be an ideal such that 
$M N=I_f^{n-3}$. Then with the multiplication map 
$\dZ^n\otimes \dZ^n = M\otimes_\dZ N \to I_f^{n-3} = \dZ^n$, forget all but the 
last two coordinates. This gives us two $n\times n$ matrices, which define 
$M$ and $N$ to begin with. For $n=2$, this can all be done quite explicitly. 




