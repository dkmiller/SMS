% !TEX root = sms.tex

\section{Chabauty methods and hyperelliptic curves}\label{sec:poonen-iv}
\thanksauthor{Bjorn Poonen}





\subsection{Introduction}

For an elementary introduction to Chabauty's method, see \cite{mp12}. 

\begin{theo}[Faltings, 1983]
If $C$ is a curve of genus $\geqslant 2$ over $\dQ$, then $C(\dQ)$ is finite. 
\end{theo}

The proof of this is ``bad'' in the sense that it is highly ineffective. It 
does produce an upper bound on the number of rational points, but neither 
Faltings' or Vojta's later proof give horrible upper bounds for $\# C(\dQ)$, 
and give no upper bound on the height of those points. 

Much earlier Chabauty created a more effective method for attacking this 
problem for certain families of curves. 

\begin{theo}[Poonen, Stoll, 2013]
Let $C:y^2=f(x)$, where $\deg f=2 g+1$. 
\begin{enumerate}
  \item For each $g\geqslant 3$, the fraction of such $C$ satisfying 
    $C(\dQ)=\{\infty\}$ is positive. 
  \item This fraction tends to $1$ as $g\to \infty$. More precisely, it is 
    $\geqslant 1-(12 g+20)g^{-g}$. 
  \item Chabauty's method at the prime $2$ effectively determines $C(\dQ)$ for 
    such a fraction of curves. 
\end{enumerate}
\end{theo}
\begin{proof}
See \cite{ps13}. 
\end{proof}

Conjecturally, $100\%$ of such $C$ have $C(\dQ)=\{\infty\}$. This theorem is 
essentially the first case in which an effective version of Faltings' theorem 
was proven for a large class of curves. 





\subsection{Genus one}

Let $E$ be an elliptic curve over $\dC$. Then $E(\dC)\simeq \dC/\Lambda$ for 
some discrete subgroup $\Lambda\subset \dZ^2$ in $\dC$. The differential 
$d z$ is a well-defined holomorphic one-form on $\dC/\Lambda$, corresponding to 
an algebraic differential $\omega\in \Omega^1(E)$. If we had started with 
$\omega$, we can define $E(\dC)\to \dC/\Lambda$ by 
\[
  x\mapsto \int_0^x \omega .
\]
Changing the path $0\to x$ changes the integral by an element in the discrete 
lattice $\Lambda=\{\int_\gamma\omega:\gamma\in \pi_1 E(\dC)\}$. 





\subsection{Genus \texorpdfstring{$g$}{g}}

Let $C$ be a curve of genus $g$ over $\dC$. Then 
\[
  \{\textnormal{holomorphic 1-forms on $C$}\} = \Gamma(C,\Omega^1) .
\]
One definition of the genus of $C$ is that $h^0(\Omega^1) = g$, 
i.e.~$\Gamma(C,\Omega^1)$ is $g$-dimensional. Let $\omega_1,\dots,\omega_g$ be a 
basis. Fix $0\in C(\dC)$. Define a holomorphic map 
$i:C(\dC\to \dC^g/\Lambda$ by 
\[
  x\mapsto \left(\int_0^x \omega_1,\dots,\int_0^x \omega_g\right) .
\]
This is called the Abel-Jacobi map. If $z_i$ are the coordinates on 
$\dC^g/\Lambda$, we have $i^\ast \mathrm{d}z_j = \omega_j$. 

As mentioned in \autoref{sec:gross-ii}, Weil discovered an algebraic analogue 
of this. Let $C$ be a curve of genus $g$ over any field $k$, with chosen 
$0\in C(k)$. Then there exists a $g$-dimensional abelian variety $J$, the 
\emph{Jacobian} of $C$ such that 
\begin{itemize}
  \item For any field $L\supset k$, 
    \[
      J(L) \simeq \picard^0(C_L) = \divisors^0(C_L) / \textnormal{linear equivalence} .
    \]
  \item There is a morphism $i:C\to J$, $x\mapsto [x]-[0]$, which is an 
    embedding if $g\geqslant 1$. 
  \item The pullback map $i^\ast:\Gamma(J,\Omega^1) \to \Gamma(C,\Omega^1)$ is 
    an isomorphism. 
  \item If $k=\dC$, the analytic Abel-Jacobi map factors as 
    \[\xymatrix{
      C(\dC) \ar[r]^-i 
        & J(\dC) \ar[r]^-\sim 
        & \dC^g/\Lambda .
    }\]
\end{itemize}

One of the difficulties of dealing with curves of higher genus is that $C(k)$ 
is not naturally a group. But at least $C\hookrightarrow J$, and $J(k)$ is a 
group. 





\subsection{(Bad) real-analytic approach}

Let $C$ be a curve of genus $g\geqslant 2$ with $0\in C(\dQ)$. Let 
$i:C\hookrightarrow J$. First let's try a (bad) real-analytic approach 
to proving finiteness of $C(\dQ)$. We have a commutative diagram 
\[\xymatrix{
  C(\dQ) \ar[r] \ar[d] 
    & J(\dQ) \ar[d] \\
  C(\dR) \ar[r] 
    & J(\dR) .
}\]
The group $J(\dQ)$ is finitely generated by Mordell-Weil, let $r=\rank J(\dQ)$. 
Since $J$ is projective, $J(\dR)$ is a compact real Lie group, so 
$J(\dR) \simeq \dR^g/\dZ^g\oplus \text{finite}$. Note that 
$C(\dQ)=J(\dQ)\cap C(\dR)\subset J(\dR)$. But typically the group generated by 
some $x\in C(\dR)\subset J(\dR)$ is dense inside $J(\dR)$. Alternatively, the 
closure of $J(\dQ)$ in the classical topology is often open in $J(\dR)$. 





\subsection{\texorpdfstring{$p$}{p}-adic approach}

Chabauty suggested replacing $\dR$ with $\dQ_p$. Again we have a commutative 
diagram 
\[\xymatrix{
  C(\dQ) \ar[r] \ar[d] 
    & J(\dQ) \ar[d] \\
  C(\dQ_p) \ar[r] 
    & J(\dQ_p) .
}\]
We have $C(\dQ)\subset C(\dQ_p)\cap \overline{J(\dQ)}$, where here the closure 
is taken in the $p$-adic analytic topology. 

\begin{theo}[Chabauty 1941]
If $r<g$, then $C(\dQ_p)\cap \overline{J(\dQ)}$ is finite. In particular, 
$C(\dQ)$ is finite. 
\end{theo}

Often, the points in $C(\dQ_p)\cap \overline{J(\dQ)}$ can be approximated 
$p$-adically. 





\subsection{Structure of \texorpdfstring{$J(\dQ_p)$}{J(Qp)}}

For simplicity, assume $C$ has good reduction at $p$. So $C$ extends to a 
smooth proper curve over $\spectrum(\dZ_p)$. Weil's construction of the 
Jacobian works in great generality, so $J=\jacobian C$ also has good 
reduction, i.e.~extends to an abelian scheme over $\spectrum(\dZ_p)$. 

We can understand $J(\dQ_p)$ by looking at a reduction map. First note that by 
the ``valuative criterion for properness,'' $J(\dQ_p)=J(\dQ_p)$, so we have a 
homomorphism $J(\dZ_p) \to J(\dF_p)$, where $J(\dF_p)$ is a finite abelian 
group. Let $U$ be the kernel of this map. By smoothness, the map 
$J(\dZ_p) \to J(\dF_p)$ is surjective, so $J(\dZ_p)$ is a disjoint union of 
$J(\dF_p)$ copies of $U$. For suitable local coordinates $t_1,\dots,t_g$ at 
$0$, we get a $p$-adic analytic isomorphism 
$\boldsymbol t=(t_1,\dots,t_g):U\iso (p \dZ_p)^g$. We conclude that 
$J(\dZ_p)\iso \coprod_{J(\dF_p)} (p \dZ_p)^g$. 

We would like a $p$-adic analogue of the Abel-Jacobi map. For 
$\omega\in \Gamma(J,\Omega^1)$, there is a canonical homomorphism 
$\eta_\omega:J(\dQ_p) \to \dQ_p$ which we write 
\[
  x\mapsto \int_0^x \omega .
\]
If $\omega = \sum w_j(\boldsymbol t) \mathrm{d}t_j$ with 
$w_j\in \dZ_p\llbracket t_1,\dots,t_g\rrbracket$, just ``integrate formally'' 
and evaluate on $(p\dZ_p)^g$. This definition works for $x\in U$. If we require 
$\eta_\omega$ to be a homomorphism, there is a unique extension of 
$\eta_\omega$ from $U$ to $J(\dQ_p)$. If $x\notin U$, there exists some 
$n\geqslant 1$ such that $n x\in U$; then define 
\[
  \int_0^x \omega = \frac 1 n \int_0^{n\cdot x} \omega .
\]

Putting everything together, we get a $p$-adic analytic homomorphism 
$\log:J(\dQ_p) \to \dQ_p^g$ defined by 
\[
  \log x = \left(\int_0^x \omega_1,\dots,\int_0^x \omega_g\right) .
\]
This is a local diffeomorphism. 





\subsection{Consequences}

We know that $J(\dQ_p)\simeq \dZ_p^g\oplus \text{finite}$. If $r<g$, 
$\log J(\dQ)\subset \dQ_p^g$ will be contained in some hyperplane. Therefore 
there is some $0\ne \omega\in \Gamma(J_{\dQ_p},\Omega^1)$ such that 
$\eta_\omega|_{J(\dQ)}=0$. This gives an explicit way of finding a 
``smaller box'' inside $J(\dQ_p)$ in which $C(\dQ)$ fits. 

Just to recap, $C(\dQ_p)\cap \overline{J(\dQ)}$ is a subset of 
$C9\dQ_p)\cap \{\eta=0\}$, the set of zeros of $\int_0^x \omega$ on 
$C(\dQ_p)$. We can write $\omega=w(t)\, \mathrm{d}t$ with 
$w\in \dZ_p\llbracket t\rrbracket$. Then 
\[
  \eta = \int \omega \in \dQ_p\llbracket t\rrbracket .
\]
We have some control on the Newton polygon of $\eta$. If we write 
$\eta = \sum a_i t^i$, plot the points $(i,v_p(i))$. The lower-convex hull 
of these points is the Newton polygon, and from this we can understand the 
valuations of the zeros of $\eta$. In particular, the number of zeros can 
be controlled in terms of $g$. 





\subsection{Main result}

From \cite{bg13}, we know that $\average(\#\selmer_2 J)=3$. This wouldn't be 
sufficient, except that $\selmer_2 J$ carries ``one $2$-adic digit'' of 
information about $J(\dQ)$. We have a commutative diagram 
\[\xymatrix{
  C(\dQ) \ar@{^{(}->}[d] \ar@{^{(}->}[rr] 
    & & C(\dQ_p) \ar[d] \\
  J(\dQ) \ar@{^{(}->}[r] \ar@{->>}[d] 
    & \overline{J(\dQ)} \ar@{^{(}->}[r] \ar@{->>}[d] 
    & J(\dQ_p) \ar@{->>}[r]^-{\log} \ar@{->>}[d] 
    & \dZ_p^g \ar@{->>}[d] \ar@{.>}[dr]^-\rho \\
  J(\dQ)/p \ar@{->>}[r] \ar[dr] 
    & \overline{J(\dQ)}/p \ar[r] 
    & J(\dQ_p)/p \ar@{->>}[r] 
    & \dF_p^g \ar@{.>}[r] 
    & \dP^{g-1}(\dF_p) \\
  & \selmer_p J \ar[ur] 
}\]
We know that $J(\dQ)/p$ lives inside $\selmer_p J$ in $J(\dQ_p)/p$. We will 
concentrate on the case $p=2$. We can compare the image of $\selmer_p J$ and 
$\rho\log C(\dQ_p)$ inside $\dP^{g-1}(\dF_p)$. Bhargava and Gross proved an 
equidistribution result for Selmer elements. 




