% !TEX root = sms.tex

\section{Counting methods over global fields}\label{sec:wang-ii}
\thanksauthor{Jerry Wang}





In the previous lectures, we have seen how to parameterize objects of 
arithmetic interest by looking at orbits of group actions, and we have counted 
these orbits using analytic methods. In this lecture, we'll generalize the 
techniques to global fields. 





\subsection{Terminology}

A \emph{global field} K is one of the following:
\begin{itemize}
  \item number field (finite extension of $\dQ$)
  \item finite separable extension of $\dF_q(T)$
\end{itemize}
For global fields of finite characteristic, we are implicitly choosing a map 
from the corresponding curve to $\dP^1$. With that map, we can always define a 
ring of integers $\cO$, and a set $M_\infty$ of infinite places. Our underlying 
example will be the average size of $\selmer_2(E/K)$. 

Throughout, $K$ is a global field, not of characteristic $2$ or $3$. 





\subsection{Heights for global fields}

Let $K$ be a global field. An elliptic curve $E$ over $K$ can be written as 
$E:y^2=x^3+A x+B$ with $A,B\in K$. We think of $(A,B)\in \dP(4,6)$, where 
$\dP(4,6)=\dG_\multiplicative\backslash \dA^2\smallsetminus 0$ via the action 
$\alpha\cdot (A,B)=(\alpha^4A,\alpha^6 B)$. If $(A,B)\in \dA^2(K)$, define 
a fractional ideal 
\[
  I = \{\alpha\in K:\alpha\cdot (A,B)\in \dA^2(\cO)\} .
\]
Set 
\[
  H(A,B) = \norm(I)\prod_{v\in M_\infty} \max\left\{|A|_v^{1/4},|B|_v^{1/6}\right\} .
\]
A simple application of the product formula shows that this height is 
invariant under the action of $\dG_\multiplicative$. Unfortunately, the set 
$\dA^2(K)_{<X}=\{x\in \dA^2(K):H(x)<X\}$ might not be ``bounded.'' The solution 
is to construct a nice fundamental domain for 
$\dG_\multiplicative(K)\backslash S(K)$ (Here and elsewhere 
$S=\dA^2\smallsetminus 0$) so that 
$(\dG_\multiplicative(K)\backslash S(K))\cap S(K)_{<X}$ is bounded. 





\subsection{Orbit parameterization over \texorpdfstring{$K$}{K}}

There is a bijection between $\selmer_2(E)$ and the set of locally soluble 
orbits for the action of $G(K)$ on $V(K)$. Here $G=\projectivegenerallinear(2)$ 
and $V=\symmetric^4(2)$. If $K$ has characteristic not $2$ or $3$, everything 
works fine. 





\subsection{Locally soluble \texorpdfstring{$K$}{K}-orbits to integral orbits}

The key input over $\dQ$ is that if $v\in V(\dQ_p)^\mathrm{sol}$ with integral 
invariants, then there exists $g\in G(\dQ_p)$ such that $g v\in V(\dZ_p)$. 
Unlike our definition of heights, which used $h_\dQ=1$ and needed to be 
modified for general $K$, things here translate easily. 

\begin{lemm}
If $v\in V(K_\fp)^\mathrm{sol}$ with invariants in $\cO_\fp$, 
$\fp\notin M_\infty$, then there exists $g\in G(K_\fp)$ such that 
$g v\in V(\cO_\fp)$. 
\end{lemm}

Morally, replace $\dQ$ with $K$, $\dQ_p$ with $K_\fp$, and $\dZ_p$ with 
$\cO_\fp$. But we need to be careful: there is a fundamental difference in 
behavior between $\dZ$ and general $\cO_K$. 

Suppose $v\in V(K)^\mathrm{ls}$ with invariants in $\cO$. Then for all 
$\fp\notin M_\infty$, there exists $g_\fp\in G(K_\fp)$ such that 
$g_\fp v\in V(\cO_\fp)$. Put $g=(g_\fp)_{\fp\notin M_\infty}\in G(\dA_f)$, 
where $\dA_f$ is the ring of finite adeles. Inside $G(\dA_f)$ are two 
subgroups. One is $U=\prod_{\fp\notin M_\fp} G(\cO_\fp)$, the other is 
$G(K)$. If $K$ has trivial class group, the double quotient 
$U\backslash G(\dA_f)/G(K)$ will be trivial, but in general it is only finite. 
For number fields, this due to Borel \cite{b63}, and for function fields this is 
due to Conrad \cite{c12}. Put 
\[
  G(\dA_f) = \coprod_{\beta\in \class(G)} U \beta G(K) .
\]
There exists $g_\fp'\in G(\cO_\fp)$, $\beta\in \class(G)$, $h\in G(K)$ such 
that for all $\fp\notin M_\infty$ we have $G_\fp = g_\fp' \beta h$. Define 
\begin{align*}
  V_\beta &= V(K)\cap \beta^{-1}\Bigl(\prod_{\fp\notin M_\infty} V(\cO_\fp)\Bigr) \\
  G_\beta &= G(K)\cap \beta^{-1} U .
\end{align*}
Then the groups $V_\beta$ and $G_\beta$ are commensurable with 
$V(\cO)$ and $G(\cO)$. Since $\beta h v\in V(\cO_\fp)$ for all 
$\fp\notin M_\infty$, we have $h v\in V_\beta$. 

For any subgroup $G_0\subset G(K)$ and any $G_0$-invariant subset 
$V_0\subset V(K)$, $X>)$, let $N(V_0,G_0,X)$ be the number of irreducible 
$G_0$-orbits in $V_0$ of height $\leqslant X$, where each $G_0 v$ is weighted 
by 
\[
  \frac{1}{\#\stabilizer_{G_0}(v)} .
\]
Let $m:V(K)\to [0,1]$ be a $G_0$-invariant map defined by some congruence 
conditions (i.e.~$m=\prod_\fp m_\fp$). Let $N_m(V_0,G_0,X)$ be defined as 
$N(V_0,G_0,X)$, but weighted by 
\[
  \frac{m(v)}{\#\stabilizer_{G_0}(v)} .
\]

\begin{theo}
Define a weight $m'$ by 
\[
  m'(v)=\chi_{V(K)^\mathrm{ls}}(v) \frac{1}{\#\stabilizer_{G(K)}(v)}\left(\sum_{\beta\in \class(G)}\sum_{v_\beta\in G_\beta\backslash V_\beta\cap V(K)v} \frac{1}{\#\stabilizer_{G_\beta}(v_\beta)}\right)^{-1} .
\]
Then 
\[
  N(V(K)^\mathrm{ls},G(K),X) = \sum_{\beta\in \class(G)}N_{m'}(V_\beta,G_\beta,X) . 
\]
\end{theo}

It is nontrivial (but true) that $m' = \prod m'_\fp$. 





\subsection{Count integral orbits soluble at infinity}

Define $N_{m_\infty'}(V_\beta,G_\beta,X)$. Put 
$K_\infty = \prod_{v\in M_\infty} K_v$. Construct 
$G_\beta\backslash V(K_\infty)_{<X}$. Set 
\begin{align*}
  R(X) &= G(K_\infty) \backslash V(K_\infty)_{<X} \\
  \cF_X &= G_\beta\backslash G(K_\infty) 
\end{align*}
Then $\cF_\beta R(X)\twoheadrightarrow G_\beta\backslash V(K_\infty)_{<X}$. The 
fiber over $G_\beta v$ has size 
\[
  \frac{\#\stabilizer_{G(K_\infty)}(v)}{\#\stabilizer_{G_\beta}(v)} .
\]
We want 
\[
  N_{m_\infty}(V_\beta,G_\beta,X) = \int_{\cF_\beta R(X)} \frac{m_\infty(v)}{\#\stabilizer_{G(K_\infty)}(v)}\,\mathrm{d} v_{\infty,\beta} + \text{error} ,
\]
were we normalize our Haar measure by 
$v_{\infty,\beta}(V(K_\infty)/V_\beta)=1$. 
To continue, we need a version of Davenport's lemma for function fields. It is 
proved using Poisson summation. A more serious problem is the cusps. Without 
loss of generality assume $G_\beta - G(\cO)$ and $V_\beta=V(\cO)$. Reduction 
theory (worked out by Springer) tells us what $G(\cO)\backslash G(K_\infty)$ 
looks like for all local fields. There is still a ``$NAK$ decomposition.'' 

When $G=\projectivegenerallinear(2)$, we have 
\begin{align*}
  N &= \begin{pmatrix} 1 \\ \ast & 1 \end{pmatrix} \\
  A &= \left\{\begin{pmatrix} t^{-1} \\ & t\end{pmatrix}:t>\frac{\sqrt 3}{2}\right\} 
\end{align*}
Just as when $K=\dQ$, we cut off the cusps. Restrict the representation $V$ to 
the torus $A$: it decomposes as $V=\bigoplus_{\chi\in U_0} \chi$. For 
example, 
\[
  \symmetric^4(2) = \underbrace{\chi_{x^4}}_{t^{-4}} \oplus \underbrace{\chi_{x^3 y}}_{t^{-2}}\oplus \underbrace{\chi_{x^2 y^2}}_1 \oplus \underbrace{\chi_{x y^3}}_{t^2}\oplus \underbrace{\chi_{y^4}}_{t^4} .
\]
Describe reducibility using subsets of $U_0$. The cusp set  
$V(K_\infty)^\mathrm{cusp}\subset V(K_\infty)$ is defined by 
$|v(\chi)|<c_1$ for some $\chi\in U_0$, where $c_1$ is chosen so that if 
$v\in V(\cO)$, $|v(\chi)|<c_1$, then $v(\chi)=0$. 

There is a combinatorial condition on the characters of $A$ that implies 
\begin{enumerate}
  \item The number of irreducible elements of the cusp is small. 
  \item The volume of the cusp is small. 
\end{enumerate}
This condition only depends on the field $K$ through the torus $T$. For 
example, if the group is split over $\dQ$, the condition does not depend on the 
field at all. This has been worked out for all the representations we've seen 
so far. 

Finally, we need an estimation of reducibility. This is a purely local 
computation. 





\subsection{Solubility at finite primes}

We do this through the weight function $m_\fp'$. Again, this is purely 
local. 





\subsection{Uniformity estimate}

This is more-or-less local. The methods work over any global field. It has 
been worked out for $\selmer_n(E)$ with $n\in \{2,3,4,5\}$, and to count field 
extensions. 





\subsection{Compute local integrals}

This is (obviously) a local computation. 

The final result is: 
\[
  N(V(K)^\mathrm{ls},G(K),X) = \tau(G/K) \mu_\infty(X) \prod_{\fp\notin M_\infty} \mu_\fp .
\]

\begin{theo}
When all $E/K$ are ordered by height, then for $n\in \{2,3,4,5\}$, we have 
\[
  \average(\#\selmer_n E) = \sum_{d\mid n} d .
\]
\end{theo}




