% !TEX root = sms.tex

\section{Binary quartic forms: bounded average rank of elliptic curves II}\label{sec:shankar-ii}
\thanksauthor{Arul Shankar}





\subsection{Review}

Recall that if $E_{A,B}$ is the elliptic curve $y^2=x^3+A x+B$, then 
$\selmer_2(E_{A,B})$ is in bijection with the quotient 
$\projectivegenerallinear_2(\dQ)\backslash V(\dZ)_{(A,B)}^\mathrm{ls}$, where 
$V(\dZ)_{(A,B)}^\mathrm{ls}$ is the set of locally soluble integral binary 
quartic forms with invariants $A,B$. Even though the action of 
$\projectivegenerallinear_2(\dQ)$ does not preserve 
$V(\dZ)$, it still induces an equivalence relation, where $f\sim g$ whenever 
there is $\gamma\in \generallinear_2(\dZ)$ such that $\gamma \cdot f = g$. 
For example, the forms $p^4 x^4 + p^2 x y^3 + y^4$ and 
$x^4 + p^4 x y^3 p^4 y^4$ are $\projectivegenerallinear_2(\dQ)$-equivalent via 
the matrix $\begin{pmatrix} p^{-1} \\ & p \end{pmatrix}$. For asymptotics, we 
could restrict to irreducible quartic forms. We defined subsets 
$V(\dZ)_{(A,B)}^{(i)}$ of $V(\dZ)$; for definitions, see 
\autoref{sec:shankar-i}. We ended up with an estimate 
\[
  \#\left(\projectivegenerallinear_2(\dZ)\backslash V(\dZ)_{H<X}^{\mathrm{irr},(i)}\right) = \frac{1}{n_i} |J| \volume(\cF) \volume\left(R_X^{(i)}\right) + O(X^{3/4}) .
\]
In this lecture, we will prove the following theorem. 

\begin{theo}[Bhargava, Shankar]
$\average(\#\selmer_2) = 3$. 
\end{theo}

To do this, we will need to replace $V(\dZ)^\mathrm{irr}$ by 
$V(\dZ)^{\mathrm{irr},\mathrm{ls}}$, and replace 
$\projectivegenerallinear_2(\dZ)$-orbits by 
$\projectivegenerallinear_2(\dQ)$-equivalence classes. 





\subsection{Local solubility}

For each prime $p$, let $V(\dZ_p)^\mathrm{s}$ be the subset of $V(\dZ_p)$ 
consisting of soluble binary quartic forms. Let $V(\dZ)^{\mathrm{s}(p)}$ be the 
set of forms in $V(\dZ)$ whose image in $V(\dZ_p)$ is soluble. We start by 
computing 
\[
  \#\left(\projectivegenerallinear_2(\dZ)\backslash V(\dZ)_{H<X}^{\mathrm{irr},\mathrm{s}(p)}\right) = \frac{|J|}{n_i} \volume(\cF) \volume\left(R_X^{(i)}\right) \volume\left(V(\dZ_p)^\mathrm{s}\right) + O(X^{3/4}) .
\]
To do this for locally soluble forms, we will need to look at ``soluble at $p$ 
forms'' for all $p$. For that, we need a sieve. 

We want 
\[
  \#\left(\projectivegenerallinear_2(\dZ)\backslash V(\dZ)_{H<X}^{\mathrm{irr},p^2\mid \Delta}\right) = O(X^{5/6}/p^{1+\delta}) ,
\]
for any $\delta>0$. In fact, this is stronger than we need. All the proof 
requires is 
\[
  \sum_{p>M} \#\left(\projectivegenerallinear_2(\dZ)\backslash V(\dZ)_{H<X}^{\mathrm{irr},p^2\mid\Delta}\right) = O(X^{5/6}/f(M)) ,
\]
where $f(M)\to \infty$ as $M\to \infty$. 

Recall the (naive) height is $H(E_{A,B}) = \max\{4 |A|^3,27 B^2\}$. 
We want 
\[
  \#\{E_{A,B}:H(E_{A,B})<X\text{ and }p^2\mid \Delta(E_{A,B})\} = O\left(\frac{X^{5/6}}{p^{1+\delta}}\right),
\]
where $\delta>0$. When $p$ is large, map $E_{A,B}$ to the binary cubic form 
$x^3 + A x y^2 + B y^3$, which goes to 
$\generallinear_2(\dZ)\cdot (x^3+A x y^2 + B y^3)$. We have defined a map 
\[
  \varphi:U_1(\dZ) \to U(\dZ) \to \generallinear_2(\dZ)\backslash U(\dZ) ,
\] 
where $U_1(\dZ)$ is the space of elliptic curves and $U$ is the space of all 
binary cubic forms. This map is discriminant-preserving. 

\begin{theo}[Delone, Nagell, Siegel, Evertse, Akhtari]
The map $\varphi$ is at most 7-to-1 for elements with large enough 
discriminant. 
\end{theo}

This is very deep. 
As one application, a binary cubic form represents one at most seven times. 
It follows from the theorem that 
\[
  \#\left\{E_{A,B}:H(E_{A,B})<X\text{ and }p^2\mid \Delta(E_{A,B})\right\} = O\left(\frac{X}{p^2}\right) . 
\]
This isn't quite what we wanted because the estimate has $X$ instead of 
$X^{5/6}$. But for $p$ sufficiently large, it works. 

Suppose $p^2\mid \Delta(E_{A,B})$ for ``modulo $p$ reasons,'' i.e.~if $E_{A,B}$ 
has additive reduction. If $p>3$, then $A\equiv B\equiv 0\pmod p$. The bound in 
this case is 
\[
  O\left(\left(\frac{X^{1/3}}{p^2}+1\right)\left(\frac{X^{1/2}}{p}+1\right)\right) = O\left(\frac{X^{5/6}}{p^2} + \frac{X^{1/2}}{p}+1\right) .
\]
If $p\mid \Delta(E_{A,B})$ for ``modulo $p^2$ reasons,'' then fixing $A$ 
determines $B$ modulo $p^2$. In this case, the bound is 
\[
  O\left(X^{1/3} \cdot\left(\frac{X^{1/2}}{p}+1\right)\right) = O\left(X^{5/6}/p^2 + X^{1/3}\right) .
\]
Combining these estimates yields the uniform bound 
\[
  \#\{E:H(E)<X\text{ and }p^2\mid \Delta(E)\} = O\left(\frac{X^{5/6}}{p^{3/2}}\right) .
\]

Recall that there is a bijection between 
$\projectivegenerallinear_2(\dZ)\backslash V(\dZ)$ and the set of $(Q,C,x)$, 
where $Q$ is a quartic ring, $C$ is a cubic resolvent ring, and $x$ generates 
$C$ (?) The map $V(\dZ) \to \dZ^2\otimes \symmetric^2(\dZ^3)$ induces 
the map sending $(Q,C,x)$ to $(Q,C)$. On the side of forms, the map sends 
$a x^4 + b x^3 y + c x^2 y^2 + d x y^3 + e y^4$ to 
\[
  \begin{pmatrix} & & 1/2 \\ & -1 \\ 1/2 \end{pmatrix}, \begin{pmatrix} a & b/2 \\ b/2 & c & d/2 \\ & d/2 & e \end{pmatrix} .
\]
Our map yields the bound 
\[
  \#\left(\projectivegenerallinear_2(\dZ)\backslash V(\dZ)_{H<X}^{p^2\mid \Delta}\right) = O\left(\frac{X}{p^2}\right) .
\]
Combining everything, we get 
\begin{align*}
  \#\left(\projectivegenerallinear_2(\dZ)\backslash V(\dZ)_{H<X}^{p^2\mid \Delta,\mod p^2}\right) 
    &= O\left(\frac{X^{5/6}}{p^2} + X^{2/3}\right) \\
  \sum_{p<M} \#\left(\projectivegenerallinear_2(\dZ)\backslash V(\dZ)_{H<X}^{p^2\mid\Delta}\right) 
    &= O\left(\frac{X^{5/6}}{\log M}\right) .
\end{align*}





\subsection{Weights}

The problem is that a single $\projectivegenerallinear_2(\dQ)$-class in 
$V(\dZ)$ could break up into seven different 
$\projectivegenerallinear_2(\dZ)$-orbits. Given a form $f$, let 
\[
  B_f = \projectivegenerallinear_2(\dZ)\backslash \left(\projectivegenerallinear_2(\dQ)\cdot f\cap V(\dZ)\right) .
\]
For $f\in V(\dZ)$, we define 
\[
  W(f) = \begin{cases} 0 & \text{if $f$ is not locally soluble} \\ \left(\displaystyle\sum_{g\in B_f} \frac{\#\automorphism_\dQ(g)}{\automorphism_\dZ (g)}\right)^{-1} & \text{otherwise} \end{cases} 
\]
The weight of $f$ is a product of local weights. That is, define for each $p$ 
\[
  W_p(f) = \begin{cases} 0 & \text{if $f$ is not locally soluble} \\ \left(\displaystyle\sum_{g\in B_p(f)} \frac{\#\automorphism_{\dQ_p}(g)}{\automorphism_{\dZ_p} (g)}\right)^{-1} & \text{otherwise} \end{cases} 
\]
Then we have the following proposition (3.3 in my paper). 

\begin{prop}
For $f\in V(\dZ)$, we have $W(f) = \prod_p W_p(f)$. 
\end{prop}

We get the following formula: 
\[
  \#\left(\projectivegenerallinear_2(\dZ)\backslash V(\dZ)_{H<X}^{\mathrm{irr},W,(i)}\right) = \frac{|J|}{n_i} \volume(\cF) \volume\left(R_X^{(i)}\right) \prod_p \int_{V(\dZ_p)} W_p(f)\, \mathrm{d} f .
\]

\begin{prop}
\[
  \int_{V(\dZ_p)}W_p(f)\, \mathrm{d} f = |J|_p \volume(\projectivegenerallinear_2(\dZ_p)) \int_{\dZ_p^2} \frac{\#\left(E_{A,B}(\dQ_p)/2\right)}{\#E_{A,B}[2](\dQ_p)}\, \mathrm{d} (A,B) .
\]
\end{prop}

We also have 
\[
  |J|\volume\left(\projectivegenerallinear_2(\dZ)\backslash \projectivegenerallinear_2(\dR)\right) \int_{\{(A,B)\in\dR^2:H(E_{A,B})<X\}} \frac{\#(E_{A,B}(\dR)/2)}{\# E_{A,B}[2](\dR)}\, \mathrm{d}(A,B) .
\]

\begin{prop}[Brummer and Kramer]
\[
  \frac{\#(E(\dQ_v)/2)}{\# E[2](\dQ_v)} = \begin{cases} 1/2 & v=\infty \\ 2 & v=2 \\ 1 & \text{otherwise} \end{cases}
\]
\end{prop}

Thus the average of $\#\selmer_2)-1$ is the following limit: 
\[
  \lim_{X\to \infty} \frac{\displaystyle\volume(\projectivegenerallinear_2(\dZ)\backslash \projectivegenerallinear_2(\dR)) \prod_p \volume(\projectivegenerallinear_2(\dZ_p) + O(X^{3/2})) \int_{H(A,B)<X}\,\mathrm{d}(A,B)}{\displaystyle\int_{H<X} \mathrm{d}(A,B) + O(X^{1/2})} .
\]
The integrals cancel, and the product over all places in the numerator is the 
Tamagawa number $\tau(\projectivegenerallinear_2)=2$. It follows that 
\[
  \average(\#\selmer_2)=2+1=3 .
\]

When we generalize to $\selmer_n$ for $n\geqslant 3$, things get a bit more 
complicated, as in the following table: 
\begin{center}
\begin{tabular}{c|c|c|c|c}
$n$ & group & space & $\tau(G)$ & $\average(\#\selmer_n)$ \\ \hline
2 & $\projectivegenerallinear_2(\dZ)$ & $\symmetric^4(\dZ^2)$ & 2 & 3\\ 
3 & $\projectivegenerallinear_3(\dZ)$ & $\symmetric^3(\dZ^3)$ & 3 & 4\\ 
4 & qt.~of $\generallinear_2\times \generallinear_4$ & $\dZ^2\otimes\symmetric^2(\dZ^4)$ & 4 & 7\\
5 & qt.~of $\generallinear_5\times \generallinear_5$ & $\dZ^5\otimes \bigwedge^2 \dZ^5$ & 5 & 6
\end{tabular}
\end{center}




